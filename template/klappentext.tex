Im kommunikativen Miteinander stoßen Europas Bürger, die europäische Wirtschaft und auch die Politik schnell an sprachliche Grenzen. Moderne Sprachtechnologie besitzt das entscheidende Potenzial, Sprachgrenzen zu überwinden. 

Das vorliegende Weißbuch stellt den Stand der sprachtechnologischen Unterstützung für die deutsche Sprache dar und gehört zu einer Serie, die die vorhandenen Sprachressourcen und -technologien für 31 europäische Sprachen analysiert.

Die Analyse wurde von META-NET erstellt, einem von der Europäischen Kommission geförderten Spitzenforschungsnetzwerk. 
META-NET besteht aus 54 Forschungszentren aus 33 Ländern, die mit Interessensvertretern aus Wirtschaft, Verwaltung, NGOs, Sprachgemeinschaften und europäischen Universitäten zusammenarbeiten. Die Vision von META-NET ist hochqualitative Sprachtechnologie für alle Sprachen Europas, die die politische und wirtschaftliche Einheit in kultureller Vielfalt vollendet. 

%---------------------------------------------------------


Europa ist in den vergangenen 60 Jahren eine politisch-wirtschaftliche Einheit geworden. Kulturell und sprachlich ist der europäische Raum reich und vielfältig. Von Portugiesisch bis Polnisch, von Italienisch bis Isländisch -- im kommunikativen Miteinander stoßen Europas Bürger, die europäische Wirtschaft und auch die Politik schnell an sprachliche Grenzen. 
Moderne Sprachtechnologie und Sprachforschung besitzen das entscheidende Potenzial, Sprachgrenzen zu überwinden. 
Dieses Weißbuch gehört zu einer Serie, die Wissen über Sprachtechnologie und deren Potenzial vermitteln soll. Die derzeitige Verfügbarkeit und Nutzung von Sprachtechnologie in Europa variiert stark je nach Sprache. 

META-NET, ein von der Europäischen Kommission gefördertes Spitzenforschungsnetzwerk, hat die aktuellen Sprachressourcen und -technologien für 31 europäische Sprachen analysiert. Die Ergebnisse zeigen, dass bei allen Sprachen beträchtliche Defizite in der technologischen Unterstützung und signifikante Forschungslücken existieren. 

Eine ausführliche Expertenanalyse und Bewertung der aktuellen Situation dient dazu, die Wirksamkeit weiterer Forschungstätigkeiten zu maximieren.

META-NET besteht aus 54 Forschungszentren aus 33 Ländern, die mit Interessensvertretern aus Wirtschaft, Verwaltung, NGOs, Sprachgemeinschaften und europäischen Universitäten zusammenarbeiten. Gemeinsam mit diesen Gruppen entwickelt META-NET eine übergreifende Technologievision und eine strategische Forschungsagenda für das mehrsprachige Europa 2020.

Die Vision von META-NET ist hochqualitative Sprachtechnologie für alle Sprachen Europas, die die politische und wirtschaftliche Einheit in kultureller Vielfalt vollendet. Mithilfe dieser Technologie können die existierenden Schranken eingerissen und Brücken zwischen den Sprachen Europas gebaut werden. Hierfür müssen alle Interessengruppen aus Politik, Forschung, Wirtschaft und Gesellschaft ihre Kräfte bündeln.

