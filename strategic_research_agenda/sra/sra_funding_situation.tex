\ssection[Technologies for Europe's Languages: The Funding Situation]{Technologies for Europe's Languages:\newline The Funding Situation}
\label{european-funding-situation}

% FIXME: The following contains a collection of raw findings per language from the Language White Papers. The collection needs to be completed and the content still needs to be aggregated and integrated into the main text.

% FIXME: If we decide to keep this appendix we should write a short introduction to explain where this info is coming from.

\begin{multicols}{2}

\begin{small}
\subsection*{Basque}
\label{sec:basque}

Since 2000 up till today, the Spanish Government supported within the National Plan of Research and Technology several projects in the area of Multilingual Speech Technologies: TEHAM, AVIVAVOZ, and BUCEADOR. Their main purpose was to improve the quality of Speech Recognition, Speech Translation and Text to Speech Synthesis in all the official languages spoken in Spain: Basque, Galician, Catalan and Spanish.

\subsection*{Bulgarian}
\label{sec:bulgarian}

The first international and national funding supporting language technologies for Bulgarian began at the very beginning of 1990s. Over a short period of time financing for a number of research projects from European institutions was got: LaTeSLav (Language Processing Technologies for Slavic Languages, 1991-1994) aimed at developing a prototype of a grammar checker, BILEDITA (Bilingual Electronic Dictionaries and Intelligent Text Alignment, 1996-1998) for the development of bilingual electronic dictionaries, GLOSSER (Support of Second Language Acquisition and Learning from Aligned Corpora, 1996-1998) aimed at supporting foreign language training and others. The Multext-East (Multilingual Text Tools and Corpora for Central and Eastern European Languages, 1995-1997) extension of the previous Multext and EAGLES (European Commission’s Expert Advisory Group on Language Engineering Standards) EU projects provided the Bulgarian language resources in a standardised format with standard mark-up and annotation, and these resources were later expanded and upgraded in the ELAN (European Language Activity Network, 1998-1999), TELRI I in II (Trans European Language Resources Infrastructure, 1995-1998 / 1999-2001) and Concede (Consortium for Central European Dictionary Encoding, 1998-2000) projects. Bulgarian institutions are also involved in the CLARIN project (Common Language Resources and Technology Infrastructure). Other ongoing projects include those comprised by EUROPEANA aimed at developing the basic technologies and standards necessary to make knowledge on the Internet more widely available in the future.

\subsection*{Catalan}
\label{sec:catalan}

Machine translation, speech recognition, spelling and grammar checking research and industrial developments have been supported by different departments of the Generalitat de Catalunya for more than 20 years. The CREL – Centre de Referència en Enginyeria Lingüística, 1996-2000, managed by the Institut d’Estudis Catalans and with participants from the major Catalan Universities, was created with the specific aim of promoting the creation of resources and tools for the automatic processing of Catalan texts in a variety of applications. As regards the presence of the Catalan language in European infrastructures, in 2008 the Catalan Government signed an agreement with Universitat Pompeu Fabra, the national representative of the European project CLARIN in Spain, with the aim of building a Catalan demonstrator. In addition, the Biblioteca de Catalunya, Catalonia’s national library, is one of the partners in the EUROPEANA project. From 2000 up until the present day, the Spanish Government supported several projects in the area of multilingual speech technologies within the National Plan of Research and Technology, i.e., TEHAM, AVIVAVOZ, and BUCEADOR. The main purpose of these projects was to improve the quality of speech recognition, speech translation and text to speech synthesis in all the official languages spoken in Spain, i.e., Basque, Galician, Catalan and Spanish. In 2005, the Catalan Government launched a project to produce Language Resources for Speech Recognition and Speech Synthesis. As a result, Language Resources for telephone applications, in-car applications and microphone applications were produced. Later, the project TECNOPARLA (2007-2010) had as its objective the translation of speech between Spanish and Catalan. The project resulted on advances in all the component speech technologies, i.e. diarization, speech recognition, speech translation and text to speech synthesis.

%\subsection*{Croatian}
%\label{sec:croatian}
%
%n.\,a.

\subsection*{Czech}
\label{sec:czech}

Most of the government-originating funding programs are maintained by the Czech Science Foundation (GAÈR) and focused on basic research. Recently (2009), a new Technological Agency of the Czech Republic (Technologická agentura Èeské republiky, TAÈR) has been established, which shall focus on applied research. However, there are probably no LT-related projects funded by TAÈR yet. 

\subsection*{Danish}
\label{sec:danish}

The Research Council for the Technical Sciences had speech technology as one of the main themes in their strategy plan for 1998-2002, and they did support several language technology projects. The inter-disciplinary IT research programmes running from mid 1990s to 2003 have been able to support a few language technology related projects (as was the case for OntoQuery and SIABO), but have not had a specific aim to do so.
The Research Agency of the Danish Ministry for Science, Technology and Development decided in 2001 to promote language technology by funding a collaborative effort to enlarge the Danish language technology lexicon developed in the European PAROLE project. The resulting dictionary (STO) is now available through ELRA, and is being used for research and commercial purposes. It was the specific aim of the Research Agency to support the Danish language in the digital age, by granting funds for the development of basic language technology building blocks.
In parallel with the ESFRI initiative for research infrastructure, Denmark has started a roadmap process for research infrastructures. One of the research infrastructures to be supported will be a “Digital Humanities Laboratory” for the support of humanities research through the use of language technology.

\subsection*{Dutch (Netherlands, Flanders)}
\label{sec:dutch-neth-fland}

Activities for the Dutch language have been promoted and supported in several programmes and projects over the last one and a half decade. Thus a Dutch language spoken train information system was developed as a carrier for research in speech analysis and generation, in language analysis and generation, and in dialogue management in the OVIS programme in the late nineties. 

METAL produced a Dutch-French MT system for the ministries of Agriculture and Internal Affairs, and after the Dutch Language Union issued a call for the development of MT systems translating between Dutch on the one hand and English and French on the other in 1999, funded by public money, Systran developed such systems in the context of the NL-Translex project. In the IOP MMI (Innovation Research Programme on Man Machine Interaction) and CATCH programs language and speech technology have been used as tools for man machine interfaces and disclosing cultural heritage.

Most prominent in their focus on the Dutch language are the joint Netherlands-Flanders CGN and STEVIN programmes. These have yielded significant progress in the availability of basic resources (data and tools) for the Dutch language, some initial research and several end user applications. Though some of the results achieved in these projects can be exploited in industry and in academia (e.g. in the CLARIN-NL research infrastructure) the prospects for optimally exploiting these results in actual research and in industry further are grim. 

\subsection*{English}
\label{sec:english}

AKT (2000-2007), was a multi-million pound collaboration between five UK universities, with the aim of enhancing information and knowledge management in the age of the World Wide Web. The research conducted on the project formed an important contribution to the semantic web, in which the use of LT technologies played a central role. A follow-up project, ”EnAKTing the Unbounded Data Web: Challenges in Web Science”, is currently ongoing.
Two systems which support the user in creating new applications from existing libraries of processing components are the University of Sheffield’s GATE system, which has been under development for over 15 years, and the more recent UCompare system, which was developed as part of a collaboration between the Universities of Tokyo, Manchester and Colorado. Whilst current components in UCompare mainly deal with English, the library will be extended as part of META-NET to cover a number of different European languages.

%\subsection*{Estonian}
%\label{sec:estonian}
%
%n.\,a.

\subsection*{Finnish}
\label{sec:finnish}

Public support from TEKES has been an important source of funding for basic research especially through two large technology programs, USIX (User-Oriented Information Technology) 1999--2002 and FENIX (Interactive Computing) 2003--2007. Some of the core technologies identified in the program in UNIX were Finnish speech recognition, large data management and search interfaces.

NLP projects carried out within the FENIX technology program include FENIX 4M (Mobile and Multilingual Maintenance Man) and FinnONTO (Semantic Web Ontologies) at the University of Helsinki, New methods and applications in speech processing and Search-in-a-Box (University of Turku), Rich semantic media for personal and professional users (VTT Technical Research Centre of Finland) and Intelligent Web Services (Helsinki School of Science and Technology), StatHouse Semantics and Automatic content classification and ontologies (Seerco Ltd).  Recently, A joint project on speech synthesis between the University of Helsinki and Aalto University has been very successful in the new field of statistical parametric synthesis based on Hidden Markov Models and a new, physiologically grounded vocoding technology.

EU funded projects in Finland since the 1980’s include LR SIMPLE, LR PAROLE and EU MLIS 5008 LINGMACHINE. The Common Language Resources and Technologies Infrastructure (CLARIN) was funded by the Commission during 2008 - 2010, and the work within the initiative continues. The national part FINCLARIN is funded by the Ministry of Education and Culture.  HFST (Helsinki Finite State Transducer Technology), OMor (Open Source Morphologies), FinnWordNet, and FinnTreeBank are examples of currently ongoing projects.

\subsection*{French (France, Belgium, Switzerland, Canada)}
\label{sec:french-france-belg}

As a follow-up of the FRANCIL program of the AUPELF-UREF (Francophone Universities Association) from 1994 to 2000, the Techno-Langue program (2003-2005), supported by the Ministries of Research, Industry and Culture, included the development of Language Resources (corpus, lexica, dictionaries, etc.) for French and the organization of 8 evaluation campaigns, on topics such as Syntactic Parsing, Text Alignment, Machine Translation, Terminology Extraction, Information Retrieval (Question \& Answer), Broadcast News speech transcription (ESTER campaign), Text-to-Speech Synthesis and Spoken dialog. The ESTER campaign allowed producing, in 2004, 1,700 hours of Broadcast News speech in French, 100 hours of which have been transcribed, making it possible to develop Broadcast News transcription systems of sufficient quality and opening the feasibility of automatic video transcription and indexing for French.  Some of those activities are now continuing as individual projects supported by the ANR (ETAPE, PASSAGE, PORT MEDIA), which also organized the REPERE challenge on multimodal (text, speech and vision) person identification (2010-2013).

CNRS (the National Centre for Scientific Research) had several programs in that field along the years, and the French Ministry for Research settled in 2011 a Corpus Infrastructure in the Human and Social Sciences area, in connection with the EC CLARIN project.

Nowadays, OSEO supports the very large Quaero program gathering 31 industrial and academic partners (French and German) with a total budget of 200 M. and an amount of public funding of 99 M. over 5 years (2008-2013). Quaero addresses the development of around 30 technologies for various medias (speech, text, music, image, video) for the needs of 8 applications related to Multimedia and Multilingual Document processing (Digitization platform, Multimedia Entreprise Capture, Social impact media monitoring, Personalized video, Communication portals \& Digital Heritage, Online Translation Platform, Real Life Mobile Search and Multimedia search engines). All technologies are developed under a strict, regular quantitative evaluation scheme, while a special project produces the data that is necessary for developing and testing the technologies. Although the program mostly addresses the French language, some technologies will be developed for most of the 23 EU official languages. The Voxalead News online application developed by Dassault/Exalead in cooperation with LIMSI-CNRS, Vocapia Research and INRIA is a good example of a major technological achievement made possible within Quaero by gathering know-how in three different areas (search engines, speech processing and image processing). 

In Belgium, the Service de la langue de la Communaute francaise de Belgique has funded the development of terminological researches in the past and is officially in charge of coordinating the terminological activities (at the government level) in the Francophone part of Belgium. The OWIL (Observatoire du Traitement Informatique des Langues et de l’Inforoute) has centralized information on NLP research and activities for several years, but has stopped its activities in 2008.

There is currently no major LTR program in Switzerland.  The most relevant project may be the National Centre of Competence in Research on Interactive Multimodal Information Management (IM2), lead by IDIAP, where speech corpora have been collected, mainly in collaboration with the EC AMI and AMIDA projects. There used to be an “observatoire” for research, but the site is now inactive. 

Canada has a special agency for HLT: the LTRC (Language Technology Research Center) / CRTL (Centre de Recherche en Technologies Langagières). There are in Canada several academic teams working on HLT, but without any specific national program, apart from generic national research programs.

\subsection*{Galician}
\label{sec:galician}

The Centre for the Development of Industrial Technology (CDTI) is a Spanish public organisation, under the Ministry of Science and Innovation, whose objective is to help Spanish companies increase their technological profile. CDTI evaluates and finances R\&D projects through programmes such as CENIT and AVANZA.

The CENIT (National Strategic Consortiums for Technological Research) programme seeks to stimulate public and private-sector cooperation in R\&D. Information Technology and Communication is one of the programme’s priority areas. Projects in this area sometimes include research in Language Technologies.

The aim of the AVANZA Plan is to bring the Information Society to ordinary citizens, and to private and public sectors. Among their priorities is to facilitate the use of new technologies to elderly people and people with disabilities. User-friendly language technology tools offer the principal solution to satisfy this goal, for example by offering speech synthesis for the blind.

The Galician Regional Government supports research through the “Plan Galego de Investigación, Desenvolvemento e Innovación Tecnolóxica (PGIDIT)”. Language Technology is not a priority line, but along the years research groups from the universities and some companies have gotten grants for doing research and developments in LT.

\subsection*{German (Germany, Austria, Switzerland)}
\label{sec:germ-germ-austr}

COLLATE, funded by the BMBF from 2000 to 2006, was one of the first projects to address the issues of a language infrastructure and led to the creation of an information portal for the field (LT World). German and Austrian institutions are involved in the on-going European CLARIN project. Other on-going projects include EUROPEANA and THESEUS, a project co-funded by the Federal Ministry of Economics and Technology (BMWI) that aims to develop the basic technologies and standards needed to make knowledge on the Internet more widely available in the future.

In Austria, the Faculty of Computer Sciences at the University of Vienna is carrying out the JETCAT project on translation between Japanese and English, and an on-going project has been compiling the Austrian Academy Corpus since 2001. There are no dedicated LT programmes in Austria. Funding for LT-related topics typically comes from research programmes that have open topics, especially those that focus on the transfer of knowledge from academic research to industry (particularly via SMEs). Several of these programmes are administered by the Austrian Research Promotion Agency (FFG). The Vienna Science and Technology Fund (WWTF) is a fairly strong supporter of localised LT, especially for topics related to Vienna, such as synthesizing the speech of the Viennese dialect (or sociolect) and building MT systems to translate from Austrian German to Viennese and other dialects.

In Switzerland, interest in language technology began in the 1980s with strong involvement in the EUROTRA project. The Universities of Zurich and Geneva are currently working on several projects in the field of MT including MT between Standard German and Swiss German. Corpus-building projects include the collection of speech corpora by the National Centre of Competence in Research on Interactive Multi- modal Information Management and a project that collects SMS text messages in Swiss German. Generally speaking, Switzerland has a small LT sector, mainly because of limited funding opportunities. 

\subsection*{Greek}
\label{sec:greek}

A significant asset gained by previous funding initiatives is the establishment of a group of LT teams in major Research Centers and University labs as well as of LT aware teams in companies. Most of these teams are continuously active in the area through EU research activities and have acted positively in producing most of the LT resources and tools that are now available for Greek, covering the axes of text, speech and multimedia data processing. A new national all-areas R\&D funding initiative (SYNERGASIA) is currently unfolding, including some promising LT projects. Since this programme is built to foster academia-industry collaborations, it is foreseen that a good set of Greek language tools and LT enhanced systems and services will be available in the coming years.

\subsection*{Hungarian}
\label{sec:hungarian}

In recent years, the international trends of standardisation and uniformisation of existing resources have reached Hungary as well. Several projects started off with the objective of integration and interoperability, e. g. creating a unified Hungarian ontology, or harmonising the different coding systems of separately developed morphological analysers. In 2008, prominent Hungarian academic institutions and R\&D companies formed the Hungarian Platform for Speech and Language Technology, which aims to help sharing and integration of high quality knowledge accumulated in centres that worked in isolation beforehand; to work out detailed strategic and implementation plans and to help their subsequent implementation; to disseminate its analyses and proposals among the members of the IT sector; to represent the Hungarian interests at the international level; and to disseminate the achievements of the Platform among the potential users of the technology. Hungarian institutions have also been involved in the CLARIN project.

\subsection*{Icelandic}
\label{sec:icelandic}

In 2000, the Icelandic Government launched a special Language Technology Programme with the aim of supporting institutions and companies in creating basic resources for Icelandic language technology. This initiative resulted in several projects which have had profound influence on the field in Iceland. After the Language Technology Programme ended in 2004, researchers from three institutes (University of Iceland, Reykjavik University, and the Arni Magnusson Institute for Icelandic Studies), who had been involved in most of the projects funded by the programme, decided to join forces in a consortium called the Icelandic Centre for Language Technology (ICLT), in order to follow up on the tasks of the programme. Since 2005, the ICLT researchers have initiated several new projects which have been partly supported by the Icelandic Research Fund and the Icelandic Technical Development Fund. The most important product of these projects is the open source IceNLP package (IceTagger, IceParser and Lemmald) which is also available as an online service (http://nlp.cs.ru.is). In 2009, the ICLT received a relatively large three year Grant of Excellence from the Icelandic Research Fund to the project ‘Viable Language Technology beyond English – Icelandic as a test case’. Within that project, further basic LT resources for Icelandic are being developed.

\subsection*{Irish}
\label{sec:irish}

Ireland is home to a number of leading research institutions in the field of computational linguistics, and has a National Centre for Language Technology located at Dublin City University alongside the Centre for Next Generation Localisation, which works on using computational linguistics to bring added value to globalisation, localisation and information access. Both of these centres are well established and have strong links with industrial R\&D both locally and abroad, as well as strong ties with other European centres of excellence in the field. The machine translation research team based at the CNGL is among the largest and most successful in the world. 

% No info about funding situation.

\subsection*{Italian}
\label{sec:italian}

As for Italian, a considerable effort has been invested in Language Technologies research in Italy since 1997, when Human Language Technology (hence forth HLT) was designated a National research policy, with the launch of two three-year projects: TAL (National Infrastructure for Linguistic resources in the field of Automatic Treatment of Spoken and Written Natural Language), funded by the Italian government for about 1.75M Euros and LRCMM, devoted to mono and multilingual research in computational linguistics, funded for about 3M Euros. Since the two projects above were launched, only two small-size projects have been recently funded, i.e. MIUR-PARLI, for the harmonization of existing computational resources for Italian, and MIUR-PAISA, for the realization of a platform for learning Italian from annotated corpora. The majority of the production of language resources and technologies for Italian is the result of various EU-funded research projects and other initiatives. Italian groups are involved, often with coordination roles, in international networking projects, particularly at the European Level (for example in CLEF, the Cross Language Evaluation Forum, and in FLaReNet, a project fostering an international network for language resources). According to a recent META-NET survey, there are currently seven national projects running and six European projects coordinated by Italian partners.

%\subsection*{Latvian}
%\label{sec:latvian}
%
%n.\,a.

\subsection*{Lithuanian}
\label{sec:lithuanian}

Most of the initiative and commitment with regard to the functioning of the Lithuanian language within the information society and LT development originates on the national level. The year 2000 saw the launch of the first national programme of the Lithuanian language in the information society for the period of 2000–2006, coordinated by the State Commission of the Lithuanian Language. The Information Society Development Committee under the Ministry of Transport and Communications is responsible for the second phase of the program of the Lithuanian Language in the Information Society 2009-2013. The programme provides for the creation of an Internet portal with free access to all the available language resources and technologies, augmentation of the existing and newly created linguistic resources, improvement of the ASR and TTS technologies, new MT tools, improvement and development of semantic and syntactic analysis and search tools.

The Lithuanian Research Council has launched the first national program ”State and Nation: Heritage and Identity” that encompasses digitalization of intangible heritage (this program saw the implementation of the project called ”Development of a Lituanistic Digital Resource Metadata System out its Compatibility with CLARIN”).  Recently, The Lithuanian Council of Sciences also finances the programme for the Development of National Lituanistics 2009–2015, aimed at developing and promoting lituanistic research, helping meet the priority of lituanistic research, strengthening the input of lituanistic research data into the overall expansion of nation-wide humanistic, providing a scientific base for nurturing national self-consciousness and protecting lituanistic heritage.

\subsection*{Maltese}
\label{sec:maltese}

The government’s National IT Strategy 2008-10 included a number of objectives related to Maltese Language including (i) the development of online government in Maltese, (ii) creation of Maltese language tools, in collaboration with the University, and (iii) support for Maltese online communities.  Currently the language technology scene in Malta is under the influence of four main initiatives: 1.~A government-supported project partly funded by EU regional development funds is under way to bring speech technology within the reach of disabled persons. The project is currently focused on Maltese speech synthesis.  2.~Malta participates in METANET4U and is thus in receipt of significant EU funding aimed at the enhancement and distribution of resources and tools that are specifically for Maltese.  3.~The Maltese Language Resource Server (MLRS) has come to fruition and significant efforts are under way at University, through the Institute of Linguistics and the Department of Intelligent Computer Systems, to maintain and develop it.  4.~Finally, a new undergraduate programme in Human Language Technology is destined to be launched by the Institute of Linguistics in October 2011.  Besides these, a project to develop an electronic version of the Aquilina dictionary is currently in preparation.

\subsection*{Norwegian}
\label{sec:norwegian}

The Research Council of Norway has supported one language technology research program, namely KUNSTI (Kunnskapsutvikling for norsk språkteknologi). It was in part inspired by larger projects in other countries (e.g. the German project Verbmobil) and aimed to increase competence in language technology through basic research. KUNSTI aimed for R\&D to make spoken and written Norwegian in various forms (and to some extent Saami) accessible for computer processing. Twenty research projects of varying sizes were completed under the program, the largest two being in MT and speech processing.

It was an important achievement to establish the Language Technology Resource Collection for Norwegian – Språkbanken in 2010, after two decades of joint efforts between the Norwegian Language Council, the Research Council, commercial companies and the Norwegian research institutions. Språkbanken at the National Library is to function as an infrastructure for making Norwegian LRT available for research and commercial use, thus hopefully reducing the threshold for developing Norwegian LRT products. Sizeable LRT building projects (e.g. the INESS, NoTa-Oslo (Norsk Talespråkskorpus, the Oslo part), Norsk aviskorpus, WeSearch—Language technology for the web and SIRKUS) after KUNSTI have been financed through infrastructure programs (AVIT) or general ICT programs from the Research Council such as VERDIKT. As a part of building basic Norwegian LRT, Språkbanken signed a contract with Kaldera språkteknologi AS in 2011 to create wordnets for Norwegian.

%\subsection*{Polish}
%\label{sec:polish}
%
%n.\,a.

\subsection*{Portuguese (Portugal, Brazil)}
\label{sec:port-port-braz}

In the field of speech processing, it is worth noting the TECNOVOZ project, which started in 2006. This project was directed by INESC-ID and one of its major goals was to foster technology transfer to the business sector, having as partners companies like the public television RTP.  Portuguese and Brazilian institutions have been participating in the ongoing CLARIN project, aiming at establishing an integrated and interoperable European research infrastructure of language resources and technology.  In Portugal, funding for language technology comes mainly from the Ministry of Science, Technology and Higher Education, through the Foundation for Science and Technology (FCT). However, obtaining support for language technology projects is particularly difficult, if not impossible, because project proposals in this area are accepted and evaluated under the Electrical Engineering track in calls for project proposals, where they have to compete with hundreds of proposals on totally unrelated issues and face evaluation committees disconnected from the area and its research topics. On a par with FCT, the Fundação Calouste Gulbenkian occasionally funds some language technology projects.

In Brazil, the FAROL Project (2006-2010), developed and conducted by the Pontifical Catholic University of Rio Grande do Sul, aimed at reinforcing the cooperation links among teams in Brazil, promoting students and researchers interchange and better research quality in natural language processing.  In Brazil, funding for research, in general, and for language technology activities, in particular, is still limited and comes mainly from government agencies. The National Council for Scientific and Technological Development (CNPq), the Sao Paulo Research Foundation (FAPESP), the Coordination for Advancement of High Education Personnel (CAPES), and the Funding Agency for Studies and Projects (FINEP) are the four institutions that significantly support research in this country. Some of these agencies have provided also special joint university-industry funding programs. For instance, FAPESP and Microsoft Research recently formed a partnership to fund socially relevant projects in the state of Sao Paulo, which included, for instance, the PorSimples text simplification project in the area of language technology.

\subsection*{Romanian}
\label{sec:romanian}

Previous national programs have led to an initial impulse, but subsequent financial aid missing or not attractive enough has lead to a loss of interest from major ICT players and young researchers, formed by universities and the Academy. One of the programs of collaboration between industry and education that has a good impact and results in Romania is the MSDN Academic Alliance, offering students free access to different Microsoft technologies.

As for research programs, UAIC and RACAI have been involved in several national or international research programs, intended to develop existing or new language technologies. Many of the conducted projects were European funded. Some nationally funded projects also existed, such as: STAR (A System for Machine Translation for Romanian), SIR-RESDEC (Open Domain Question Answering System for Romanian and English), ROTEL (intelligent systems for the Semantic Web, based on the logic of ontologies and NLP), eDTLR (The Romanian Thesaurus Dictionary in electronic form), among others.

\subsection*{Serbian}
\label{sec:serbian}

The first interdisciplinary project entitled “Interactions between text and dictionaries” was conceived in 2002 as a joint project of the Departments of Serbian at the Faculty of Philology in Belgrade and the Faculty of Philosophy in Novi Sad, as well as the Faculty of Mathematics in Belgrade. In the scope of this project the first corpus of contemporary Serbian was developed and the development of an electronic morphological dictionary of Serbian following the so-called LADL format was initiated.  The project was later continued as a joint project of the Department of Serbian at the Faculty of Philology and the Faculty of Mathematics in the period from 2006 to 2010 under the name “A theoretical and methodological framework for the modernization of Serbian” and from 2011 to 2014 under the name “Serbian and its resources: theory, description and applications”. Within the scope of these projects the development of the electronic dictionary of simple words was finalised, and the development of the dictionary of compounds initiated, aligned French-Serbian and English-Serbian corpora of literary texts were developed, as well as local grammars for certain segments of Serbian (especially for named entities). Different software tools were also developed, among which special attention should be given to LeXimir, a workstation which enables integration and transformation of heterogeneous lexical resources.

Parallel with this research in the field of language, a project was funded within the social sciences field under the name “Fundamental cognitive processes and functions”, realised by the Department of Psychology at the Faculty of Philosophy in Belgrade. The aim of this project, among other things, was to investigate the possibility of the automatic annotation of texts based on an annotated corpus.  Speech synthesis and recognition is being realised at the Faculty of Technical Sciences of the University of Novi Sad through projects of technological development from 2005, namely “Development of speech technologies in Serbian and their application in Telekom Serbia” (2005-2007), “Man-machine speech communication” (2008-2010), “Development of dialogue systems for Serbian and other South-Slavic languages” (2011–2014). They provide support for different TTS and ASR applications and services including IVR systems, private branch exchanges, call centres, audio logging, track commercials, word spotter, etc.  In addition to national projects, Serbian scientific institutions have also taken part in various international projects related to the HLT field.

%\subsection*{Slovak}
%\label{sec:slovak}
%
%n.\,a.

\subsection*{Slovene}
\label{sec:slovene}

In general, it can be stated that in the last two decades language technology for Slovene was never supported by a consistently devised national funding scheme. The process of the development of HLT applications, tools and resources for Slovene has been therefore a mixture of international projects extending their scope from Western European languages to Central and Eastern Europe with a view toward the EU enlargement process, national research funding where speech interaction was the dominant research area, and the enthusiasm of individuals involved in LT or of larger groups working on the localization of free software such as Linux, OpenOffice, etc., to Slovene.  In the 2000s, the Alpineon software company, a spin-off from the Faculty of Electrical Engineering (University of Ljubljana) led a large consortium in the Voicetran project (2004-2008), the biggest national speech interaction project to date.  A new standardization effort concerning a morpho-syntactic tagging system, with origins in the MULTEXT-East project and used in the annotation of the FIDA line of corpora, and a newly developed syntactic annotation system was funded in the JOS project (Linguistic annotation of Slovene 2007--2009). The results of the project are now used in a large European Social Fund project “Communication of Slovene” (2008--2013), led by the Amebis software company, where a new tagger and parser is being developed along with an upgrade of the FidaPLUS corpus to more than 1 billion word Gigafida corpus.  Statistical data on national research funding shows that 18 national research projects were funded in the field of speech interaction from 1995 to 2010, 9 in the field of written language technologies and 3 in the digitization of (historical) resources. However, language technology as a field has never seen a more consistent national effort in the sense of building a LT language infrastructure for Slovene, exemplified by the German COLLATE project or TST Centrale for Dutch.

%\subsection*{Spanish}
%\label{sec:spanish}
%
%n.\,a.

%\subsection*{Swedish}
%\label{sec:swedish}
%
%n.\,a.
\end{small}

\end{multicols}

\clearpage

