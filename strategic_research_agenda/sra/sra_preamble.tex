%                                     MMMMMMMMM
%
%  MMA    MM   MMMMMM  MMMMMMM   MM    MMMMMMMM   MMA   MM  MMMMMMM MMMMMMM
%  MMMA AMMM   MM        MM     MMMM              MMMM  MM  MM        MM
%  MM MMM MM   MMMMMM    MM    IM  MI   MMMMMMM   MM MMxMM  MMMMMM    MM
%  MM  M  MM   MM        MM   .MMMMMM.            MM  MMMM  MM        MM
%  MM     MM   MMMMMM    MM   MM    MM            MM   MMM  MMMMMMM   MM
%
%
%          - META-NET Strategic Research Agenda | English Metadata -
% 
% ----------------------------------------------------------------------------

\usepackage{polyglossia}
\setotherlanguages{english}

\title{Strategic Research Agenda for Multilingual Europe 2020 --- ~}

\spineTitle{The META-NET Strategic Research Agenda for Multilingual Europe 2020}

\subtitle{~ --- ~}

\author{
  {\footnotesize edited by the}\\
  META Technology Council
%  Aljoscha Burchardt \\
%  Georg Rehm \\
%  Hans Uszkoreit \\
%  \footnotesize{(editors)}
}

\authoraffiliation{
  {\footnotesize edited by the}\\
  META Technology Council
%  Aljoscha Burchardt~ {\small DFKI}\\
%  Georg Rehm~ {\small DFKI}\\
%  Hans Uszkoreit~ {\small DFKI, Universität des Saarlandes}\\
%  \footnotesize{(editors)}\\
}

\editors{
  % Georg Rehm, Hans Uszkoreit\\(editors)
}

% Text in left column on backside of the cover
\SpineLText{%
}

% Text in right column on backside of the cover
\SpineRText{%
}

% Quotes from VIPs on backside of the cover
\quotes{% 
  In everyday communication, Europe’s citizens, business
  partners and politicians are inevitably confronted with language
  barriers. Language technology has the potential to overcome these
  barriers and to provide innovative interfaces to technologies and
  knowledge. This document presents a Strategic Research Agenda for
  Multilingual Europe 2020.
  %
  The agenda was prepared by META-NET, a European Network of
  Excellence. META-NET consists of 60 research centres in 34
  countries, who cooperate with stakeholders from economy, government
  agencies, research organisations, non-governmental organisations,
  language communities and European universities. META-NET’s vision is
  high-quality language technology for all European languages.\\\\
  %
  ``The research carried out in the area of language technology is of
  utmost importance for the consolidation of Portuguese as a language
  of global communication in the information society.''\\
  \textcolor{grey2}{--- Dr.~Pedro Passos Coelho (Prime-Minister of
  Portugal)}\\[2mm]
  %
  ``It is imperative that language technologies for Slovene are
  developed systematically if we want Slovene to flourish also in the
  future digital world.''\\
  \textcolor{grey2}{--- Dr. Danilo Türk (President of the Republic of
  Slovenia)}\\[2mm]
  %
  ``For such small languages like Latvian keeping up with the ever
  increasing pace of time and technological development is
  crucial. The only way to ensure future existence of our language is
  to provide its users with equal opportunities as the users of larger
  languages enjoy. Therefore being on the forefront of modern
  technologies is our opportunity.''\\ \textcolor{grey2}{--- Valdis
  Dombrovskis (Prime Minister of Latvia)}\\[2mm]
  %
  ``Europe's inherent multilingualism and our scientific expertise are
  the perfect prerequisites for significantly advancing the challenge
  that language technology poses. META-NET opens up new opportunities
  for the development of ubiquitous multilingual technologies.''\\
  \textcolor{grey2}{--- Prof. Dr. Annette Schavan (German Minister of
  Education and Research)} }

% Funding notice
\FundingNotice{%
  Strategic Research Agenda for Multilingual Europe 2020\\
  META Technology Council (eds.)\\
  Version~1.0~(December 1, 2012)\\

  \begin{minipage}[t]{0.75\linewidth}
  The project META-NET (T4ME, Grant Agreement 249\,119) has been partially
  funded with support from the European Commission. This publication
  reflects the views only of the authors, and the Commission cannot be
  held responsible for any use which may be made of the information
  contained therein.\\
  \end{minipage}
}

% LTInfoBox
\LTInfoBox{
  \sffamily
  \begin{tabular}{p{\linewidth}}

  \rowcolor{orange2}\vspace{.15cm}\centerline{{\textcolor{white}{\Large\textbf{What are the major topics?}}}}\\

  \rowcolor{orange2}{\vspace*{-6mm}\begin{itemize}
    \item \emph{Information access and management.} Example: Information retrieval.
    \item \emph{Communication between humans and between humans and
        machines.} Example: Spoken dialogue system.
    \item \emph{Translation of spoken and written content.} Example: Document translation.
    \end{itemize}}\\[-2mm]

  \rowcolor{orange1} \vspace{.15cm}\centerline{{\textcolor{white}{\Large\textbf{What are common Language Technology applications?}}}}\\[-2mm]
  \rowcolor{orange1} 
  Language technologies include: spelling and grammar checkers; web search; voice dialing; interactive dialogue systems (e.\,g., phone banking or train reservation systems); interactive assistants such as Apple's Siri or Google's voice search; crosslingual search in digital libraries (e.\,g., Europeana); term extraction; speech synthesis for navigation systems; recommender systems for online shops; automatic content summarisation; and machine translation systems such as Google Translate and Microsoft's Bing Translator.
  \vspace*{3mm} \\ 

  \rowcolor{orange2}\vspace{.10cm}\centerline{{\textcolor{white}{\Large\textbf{What is Language Technology?}}}}\\[-2mm]
  \rowcolor{orange2} Language technologies are technologies for automatically analysing and generating the most complex information medium in our world, \emph{human language}, in both its spoken and written forms (as well as sign language). These technologies are developed by experts involved in linguistics, computer science, computational linguistics and related disciplines \cite{jurafsky-martin01,manning-schuetze1,lt-world1,lt-survey1}.
  \vspace*{3mm} \\
  \end{tabular}
  \normalfont
}
