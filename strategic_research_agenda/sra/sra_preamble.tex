%                                     MMMMMMMMM
%                                                                             
%  MMO    MM   MMMMMM  MMMMMMM   MM    MMMMMMMM   MMD   MM  MMMMMMM MMMMMMM   
%  MMM   MMM   MM        MM     ?MMM              MMM$  MM  MM         MM     
%  MMMM 7MMM   MM        MM     MM8M    MMMMMMM   MMMMD MM  MM         MM     
%  MM MMMMMM   MMMMMM    MM    MM  MM             MM MMDMM  MMMMMM     MM     
%  MM  MM MM   MM        MM    MMMMMM             MM  MMMM  MM         MM     
%  MM     MM   MMMMMM    MM   MM    MM            MM   MMM  MMMMMMM    MM
%
%
%          - META-NET Strategic Research Agenda | English Metadata -
% 
% ----------------------------------------------------------------------------

\usepackage{polyglossia}
\setotherlanguages{english}

\title{Strategic Research Agenda for Multilingual Europe 2020 --- ~}

\spineTitle{The META-NET Strategic Research Agenda for Multilingual Europe 2020}

\subtitle{~ --- ~}

\author{
  Aljoscha Burchardt \\
  Georg Rehm \\
  Hans Uszkoreit \\
  \footnotesize{(editors)}
}

\authoraffiliation{
  Aljoscha Burchardt~ {\small DFKI}\\
  Georg Rehm~ {\small DFKI}\\
  Hans Uszkoreit~ {\small DFKI, Universität des Saarlandes}\\
  \footnotesize{(editors)}
}

\editors{
  % Georg Rehm, Hans Uszkoreit\\(editors)
}

% Text in left column on backside of the cover
\SpineLText{%
}

% Text in right column on backside of the cover
\SpineRText{%
}

% Quotes from VIPs on backside of the cover
\quotes{% 
  In everyday communication, Europe’s citizens, business
  partners and politicians are inevitably confronted with language
  barriers. Language technology has the potential to overcome these
  barriers and to provide innovative interfaces to technologies and
  knowledge. This document presents a Strategic Research Agenda for
  Multilingual Europe 2020.
  %
  The paper was prepared by META-NET, a Network of Excellence
  funded by the European Commission. META-NET consists of 60 research
  centres in 34 countries, who cooperate with stakeholders from
  economy, government agencies, research organisations,
  non-governmental organisations, language communities and European
  universities. META-NET’s vision is high-quality language technology
  for all European languages.\\\\
  %
  ``Conserving [Lithuanian] for future generations is a responsibility
  of the whole of the European Union. How we proceed with developing
  information technology will pretty much determine the future of the
  Lithuanian language.''\\ \textcolor{grey2}{--- Andrius Kubilius
  (Prime Minister of the Republic of Lithuania)}\\[1mm]
  %
  ``It is imperative that language technologies for Slovene are
  developed systematically if we want Slovene to flourish also in the
  future digital world.''\\
  \textcolor{grey2}{--- Dr. Danilo Türk (President of the Republic of
  Slovenia)}\\[1mm]
  %
  ``For such small languages like Latvian keeping up with the ever
  increasing pace of time and technological development is
  crucial. The only way to ensure future existence of our language is
  to provide its users with equal opportunities as the users of larger
  languages enjoy. Therefore being on the forefront of modern
  technologies is our opportunity.''\\ \textcolor{grey2}{--- Valdis
  Dombrovskis (Prime Minister of Latvia)}\\[1mm]
  %
  ``Europe's inherent multilingualism and our scientific expertise are
  the perfect prerequisites for significantly advancing the challenge
  that language technology poses. META-NET opens up new opportunities
  for the development of ubiquitous multilingual technologies.''\\
  \textcolor{grey2}{--- Prof. Dr. Annette Schavan (German Minister of
  Education and Research)} }

% Funding notice left column
\FundingLNotice{
\ 
}

% Funding notice right column
\FundingRNotice{\textcolor{black}{The development of this Strategic
Research Agenda has been funded by the Seventh Framework Programme of
the European Commission under the contract T4ME (Grant Agreement
249\,119).}}
