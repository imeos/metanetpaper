%                                     MMMMMMMMM
%
%  MMA    MM   MMMMMM  MMMMMMM   MM    MMMMMMMM   MMA   MM  MMMMMMM MMMMMMM
%  MMMA AMMM   MM        MM     MMMM              MMMM  MM  MM        MM
%  MM MMM MM   MMMMMM    MM    IM  MI   MMMMMMM   MM MMxMM  MMMMMM    MM
%  MM  M  MM   MM        MM   .MMMMMM.            MM  MMMM  MM        MM
%  MM     MM   MMMMMM    MM   MM    MM            MM   MMM  MMMMMMM   MM
%
%
%          - META-NET Strategic Research Agenda | English Metadata -
% 
% ----------------------------------------------------------------------------

\usepackage{polyglossia}
\setotherlanguages{english}

\title{Strategic Research Agenda for Multilingual Europe 2020 --- ~}

\spineTitle{The META-NET Strategic Research Agenda for Multilingual Europe 2020}

\subtitle{~ --- ~}

\author{
  {\footnotesize edited by the}\\
  META Technology Council
%  Aljoscha Burchardt \\
%  Georg Rehm \\
%  Hans Uszkoreit \\
%  \footnotesize{(editors)}
}

\authoraffiliation{
  {\footnotesize edited by the}\\
  META Technology Council
%  Aljoscha Burchardt~ {\small DFKI}\\
%  Georg Rehm~ {\small DFKI}\\
%  Hans Uszkoreit~ {\small DFKI, Universität des Saarlandes}\\
%  \footnotesize{(editors)}\\
}

\editors{
  % Georg Rehm, Hans Uszkoreit\\(editors)
}

% Text in left column on backside of the cover
\SpineLText{%
}

% Text in right column on backside of the cover
\SpineRText{%
}

% Quotes from VIPs on backside of the cover
\quotes{% 
  In everyday communication, Europe’s citizens, business
  partners and politicians are inevitably confronted with language
  barriers. Language technology has the potential to overcome these
  barriers and to provide innovative interfaces to technologies and
  knowledge. This document presents a Strategic Research Agenda for
  Multilingual Europe 2020.
  %
  The paper was prepared by META-NET, a Network of Excellence
  funded by the European Commission. META-NET consists of 60 research
  centres in 34 countries, who cooperate with stakeholders from
  economy, government agencies, research organisations,
  non-governmental organisations, language communities and European
  universities. META-NET’s vision is high-quality language technology
  for all European languages.\\\\
  %
  ``The research carried out in the area of language technology is of
  utmost importance for the consolidation of Portuguese as a language
  of global communication in the information society.''\\
  \textcolor{grey2}{--- Dr.~Pedro Passos Coelho (Prime-Minister of
  Portugal)}\\[2mm]
  %
  ``It is imperative that language technologies for Slovene are
  developed systematically if we want Slovene to flourish also in the
  future digital world.''\\
  \textcolor{grey2}{--- Dr. Danilo Türk (President of the Republic of
  Slovenia)}\\[2mm]
  %
  ``For such small languages like Latvian keeping up with the ever
  increasing pace of time and technological development is
  crucial. The only way to ensure future existence of our language is
  to provide its users with equal opportunities as the users of larger
  languages enjoy. Therefore being on the forefront of modern
  technologies is our opportunity.''\\ \textcolor{grey2}{--- Valdis
  Dombrovskis (Prime Minister of Latvia)}\\[2mm]
  %
  ``Europe's inherent multilingualism and our scientific expertise are
  the perfect prerequisites for significantly advancing the challenge
  that language technology poses. META-NET opens up new opportunities
  for the development of ubiquitous multilingual technologies.''\\
  \textcolor{grey2}{--- Prof. Dr. Annette Schavan (German Minister of
  Education and Research)} }

% Funding notice
\FundingNotice{%
  This is Version~1.0~(\today) of the Strategic Research
  Agenda for Multilingual Europe 2020.  The development of this
  document has been funded by the Seventh Framework Programme of the
  European Commission under the contract T4ME (Grant Agreement
  249\,119).
  % Version~0.9~(\today) for public discussion -- Please send feedback
  % to this SRA to georg.rehm@meta-net.eu with the subject line
  % ``META-NET SRA: feedback'' or participate in our online discussion
  % forum at \url{http://www.meta-net.eu/forum.}
}

% LTInfoBox
\LTInfoBox{
  \begin{tabular}{p{.90\linewidth}}
  \rowcolor{orange2}\vspace{.10cm}\centerline{{\textcolor{white}{\Large  What is Language Technology?}}}\\[-2mm]
  \rowcolor{orange2} Language technologies are specialised information technologies for processing automatically the most complex information medium in our world: \emph{human languages} -- in both modalities (spoken and written language) and also in both directions (analysis and generation of language). 

  Language technologies are developed by experts involved in computer
  science, linguistics, computational linguistics and related
  disciplines. References: \cite{jurafsky-martin01,manning-schuetze1,lt-world1,lt-survey1}
  \vspace*{3mm} \\

  \rowcolor{orange1} \vspace{.15cm}\centerline{{\textcolor{white}{\Large  What are common Language Technology applications?}}}\\[-2mm]
  \rowcolor{orange1} Spell and grammar checking in text processing applications and editing tools; 
  web search; voice dialing; interactive dialogue systems (from banking over the phone to train reservation systems to Apple's Siri); cross-lingual search in digital libraries (such as, e.\,g., Europeana); synthethic voices for navigation systems; recommender systems for online shops; machine translation systems such as Google Translate, etc.
  \vspace*{3mm} \\ 

  \rowcolor{orange2} \vspace{.15cm}\centerline{{\textcolor{white}{\Large  What are the major topics?}}}
  \center
  \begin{tabular}{p{.30\linewidth} p{.30\linewidth}p{.30\linewidth}}
  \rowcolor{yellow} \vspace{.001cm} {\Large Information} & \vspace{.001cm} {\Large Communication} & \vspace{.001cm}  {\Large Translation} \\[-1.2mm]
  \rowcolor{yellow} \small Access and management & \small Human-human; human-machine & \small Spoken and written \\[.4em]
  \rowcolor{yellow} Example: & Example: & Example: \\[-1.2mm]
  \rowcolor{yellow} Information retrieval & Spoken dialogue system & Document translation\\[2mm]
  \end{tabular}
  \vspace*{2mm} \\

  \end{tabular}
}
