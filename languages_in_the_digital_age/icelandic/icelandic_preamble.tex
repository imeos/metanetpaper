%                                     MMMMMMMMM        
%                                                                             
%  MMO    MM   MMMMMM  MMMMMMM   MM    MMMMMMMM   MMD   MM  MMMMMMM MMMMMMM   
%  MMM   MMM   MM        MM     ?MMM              MMM$  MM  MM         MM     
%  MMMM 7MMM   MM        MM     MM8M    MMMMMMM   MMMMD MM  MM         MM     
%  MM MMMMMM   MMMMMM    MM    MM  MM             MM MMDMM  MMMMMM     MM     
%  MM  MM MM   MM        MM    MMMMMM             MM  MMMM  MM         MM     
%  MM     MM   MMMMMM    MM   MM    MM            MM   MMM  MMMMMMM    MM
%
%
%          - META-NET Language Whitepaper | Icelandic Metadata -
% 
% ----------------------------------------------------------------------------

\usepackage[icelandic, english]{babel}


\title{Íslensk\ \ \ \ \ \ \ tunga á stafrænni öld --- The Icelandic Language in the Digital Age}

\spineTitle{The Icelandic Language in the Digital Age --- Íslensk tunga á stafrænni öld}

\subtitle{White Paper Series --- Hvítbókaröð}

\author{
  Eiríkur Rögnvaldsson\\
  Kristín M. Jóhannsdóttir\\
  Sigrún Helgadóttir\\
  Steinþór Steingrímsson
}

\authoraffiliation{
  Eiríkur Rögnvaldsson~ {\small Háskóla Íslands}\\
  Kristín M. Jóhannsdóttir~ {\small Háskóla Íslands}\\
  Sigrún Helgadóttir~ {\small Árnastofnun} \\
  Steinþór Steingrímsson~ {\small Háskóla Íslands}
}

\editors{
  Georg Rehm, Hans Uszkoreit\\(ritstjórar, \textcolor{grey1}{editors})
}

% Text in left column on backside of the cover
\SpineLText{\selectlanguage{english}%
  In everyday communication, Europe’s citizens, business partners and politicians are inevitably confronted with language barriers.  
  Language technology has the potential to overcome these barriers and to provide innovative interfaces to technologies and knowledge. 
  This white paper presents the state of language technology support for the Icelandic language. 
  It is part of a series that analyzes the available language resources and technologies for 31~European languages. 
  The analysis was carried out by META-NET, a Network of Excellence funded by the European Commission.
  META-NET consists of 54 research centres in 33 countries, who cooperate with stakeholders from economy, government agencies, research organisations and others. %, non-governmental organisations, language communities and European universities. 
  META-NET’s vision is high-quality language technology for all European languages. 
}

% Text in right column on backside of the cover
\SpineRText{\selectlanguage{icelandic}%
  Evrópubúar standa sífellt frammi fyrir tungumálaþröskuldum í daglegum samskiptum sín á milli -- í viðskiptum, stjórnmálum og hversdagslífinu. Máltækni getur nýst til að yfirstíga þá þröskulda og skapa nýjan aðgang að tækni og þekkingu. Þessi hvítbók kynnir stöðu máltækni\-stuðnings við íslensku. Hún er hluti af ritröð þar sem gerð er grein fyrir tiltækum málföngum og máltækni fyrir 31 Evróputungumál. Að greiningunni stendur META-NET, sem er öndvegisnet fjármagnað af Evrópusambandinu. Innan META-NET eru 54 rannsóknarsetur í 33 löndum sem starfa með hagsmunaaðilum úr viðskiptalífinu, opinberum stofnunum, rannsóknarsetrum og ýmsum öðrum. Framtíðarsýn META-NET er að til verði hágæða máltækni fyrir öll evrópsk tungumál.}
\vspace{-5mm}
% Quotes from VIPs on backside of the cover
\quotes{%
\selectlanguage{icelandic}„Máltækni er afar mikilvægur stuðningur við alls kyns málrannsóknir og styrkur við þá viðleitni að íslensk tunga verði notuð á öllum sviðum þjóðfélagsins í samræmi við opinbera íslenska málstefnu.“ \\
 \textcolor{grey2}{--- Dr. Guðrún Kvaran (prófessor, formaður Íslenskrar málnefndar)}\\[3mm]
%\selectlanguage{english}Language technology is an essential tool in a variety of linguistic research, and supports the official Icelandic policy of promoting the national language in all aspects of communication. \\
 %\textcolor{grey2}{--- Prof. Dr. Guðrún Kvaran (Chair of the Icelandic Language Council)}[3mm]
%  \selectlanguage{icelandic}Þetta er frábært yfirlit um núverandi stöðu máltækni í Evrópu. Þetta er ákall um aðgerðir, beint til áhrifamanna í löndum sem vilja að landsmenn taki þátt í 21. öldinni á jafnréttisgrundvelli við þá sem eiga ensku að móðurmáli.\\
%\textcolor{grey2}{--- Helga Waage (stofnandi og tæknilegur framkvæmdastjóri Mobilitus)}\\[3mm]
  \selectlanguage{english}„This is an excellent overview of the current state of language technology in Europe. This is a call to action for decision makers in countries that want their citizens to participate in the 21st century on equal footing with native English speakers.“ \\
\textcolor{grey2}{--- Helga Waage (co-founder and CTO of Mobilitus)}\\[3mm]
% \selectlanguage{icelandic}Að geta tekið þátt í stafrænu samfélagi án þess að afsala sér eigin tungumáli eru sjálfsögð mannréttindi sem einungis þróun máltækni getur tryggt. Þetta, ásamt greinagóðri úttekt á veikri stöðu íslenskrar máltækni gagnvart öðrum Evrópumálum, er ákall skýrslunnar um víðtækar aðgerðir á þessu sviði.\\
 % \textcolor{grey2}{--- Hannes Högni Vilhjálmsson, Ph.D. (dósent í tölvunarfræði við Háskólann í Reykjavík)}\\[3mm]
 % \selectlanguage{english}To be able to participate in the digital society without giving up one's language is a human right which can only be secured by the development of language technology. This, together with a comprehensive overview of the weak status of language technology for Icelandic compared to other European languages, is the basis for the report's call to action in this field. \\
 % \textcolor{grey2}{--- Hannes Högni Vilhjálmsson, Ph.D (Associate Professor of Computer Science, Reykjavik University)}
}

% Funding notice left column
\FundingLNotice{\selectlanguage{icelandic}
  Höfundar þessa rits þakka höfundum hvítbókar um þýsku fyrir leyfi til að endurnýta almenna kafla úr verki þeirra \cite{lwpgerman}.\vfill
  \bigskip
  Gerð þessarar hvítbókar var kostuð af Sjöundu rammaáætlun Evrópusambandsins og Stefnumótunaráætlun Evrópusambandsins í upplýsinga- og samskiptatækni samkvæmt samningum við T4ME (styrksamningur 249119), CESAR (styrksamningur 271022), META\-NET4U (styrksamningur 270893) og META-NORD (styrksamningur 270899).
}

% Funding notice right column
\FundingRNotice{\selectlanguage{english}
  The authors of this document are grateful to the authors of the White Paper on German for permission to re-use selected language-independent materials from their document \cite{lwpgerman}. \vfill
  \bigskip
  The development of this white paper has been funded by the Seventh Framework Programme and the ICT Policy Support Programme of the European Commission under the contracts T4ME (Grant Agreement 249119), CESAR (Grant Agreement 271022), META\-NET4U (Grant Agreement 270893) and META-NORD (Grant Agreement 270899).
}
