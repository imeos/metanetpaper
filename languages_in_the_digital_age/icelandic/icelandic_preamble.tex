%                                     MMMMMMMMM                                         
%                                                                             
%  MMO    MM   MMMMMM  MMMMMMM   MM    MMMMMMMM   MMD   MM  MMMMMMM MMMMMMM   
%  MMM   MMM   MM        MM     ?MMM              MMM$  MM  MM         MM     
%  MMMM 7MMM   MM        MM     MM8M    MMMMMMM   MMMMD MM  MM         MM     
%  MM MMMMMM   MMMMMM    MM    MM  MM             MM MMDMM  MMMMMM     MM     
%  MM  MM MM   MM        MM    MMMMMM             MM  MMMM  MM         MM     
%  MM     MM   MMMMMM    MM   MM    MM            MM   MMM  MMMMMMM    MM
%
%
%          - META-NET Language Whitepaper | Icelandic Metadata -
% 
% ----------------------------------------------------------------------------

\usepackage[icelandic, english]{babel}


\title{Íslensk tunga á stafrænni öld --- The Icelandic Language in the Digital Age}

\subtitle{White Paper Series --- Hvítbókaröð}

\author{
  Eiríkur Rögnvaldsson\\
  Kristín M. Jóhannsdóttir\\
  Sigrún Helgadóttir\\
  Steinþór Steingrímsson
}

\authoraffiliation{
  Eiríkur Rögnvaldsson~ {\small Háskóla Íslands}\\
  Kristín M. Jóhannsdóttir~ {\small Háskóla Íslands}\\
  Sigrún Helgadóttir~ {\small Árnastofnun} \\
  Steinþór Steingrímsson~ {\small Háskóla Íslands}
}

\editors{
  Georg Rehm, Hans Uszkoreit\\(Herausgeber, \textcolor{grey1}{editors})
}

% Text in left column on backside of the cover
\SpineLText{\selectlanguage{english}%
  In everyday communication, Europe’s citizens, business partners and politicians are inevitably confronted with language barriers.  
  Language technology has the potential to overcome these barriers and to provide innovative interfaces to technologies and knowledge. 
  This white paper presents the state of language technology support for the German language. 
  It is part of a series that analyzes the available language resources and technologies for 31~European languages. 
  The analysis was carried out by META-NET, a Network of Excellence funded by the European Commission.
  META-NET consists of 54 research centres in 33 countries, who cooperate with stakeholders from economy, government agencies, research organisations, non-governmental organisations, language communities and European universities. 
  META-NET’s vision is high-quality language technology for all European languages. 
}

% Text in right column on backside of the cover
\SpineRText{\selectlanguage{icelandic}%
  Evrópubúar – viðskiptafélagar, stjórnmálamenn og almennir borgarar– standa sífellt frammi fyrir tungumálaþröskuldum í daglegum samskiptum sín á milli. Máltækni getur nýst til að yfirstíga þá þröskulda og skapa nýjan aðgang að tækni og þekkingu. Þessi hvítbók kynnir stöðu máltæknistuðnings við íslensku. Hún er hluti af ritröð þar sem gerð er grein fyrir tiltækum málföngum og máltækni fyrir 31 Evróputungumál. Að greiningunni stendur META-NET, sem er öndvegisnet fjármagnað af Evrópusambandinu. Innan META-NET eru 54 rannsóknarmiðstöðvar í 33 löndum sem starfa með hagsmunaaðilum úr viðskiptalífinu, opinberum stofnunum, rannsóknarstofnunum, frjálsum félagasamtökum, málsamfélögum og evrópskum háskólum. Framtíðarsýn META-NET felst í uppbyggingu hágæða máltækni fyrir öll evrópsk tungumál.}

% Quotes from VIPs on backside of the cover
\quotes{%
  Excepteur sint occaecat cupidatat non proident, sunt in culpa qui officia deserunt mollit anim id est laborum. Lorem ipsum dolor sit amet, consectetur adipisicing elit, sed do eiusmod tempor labore et dolore magna aliqua. \\
  \textcolor{grey2}{--- Prof. Dr. John Doe (Member of the European Parliament and VIP)}\\[3mm]
  Excepteur sint occaecat cupidatat non proident, sunt in culpa qui officia deserunt mollit anim id est laborum. Lorem ipsum dolor sit amet, consectetur adipisicing elit, sed do eiusmod tempor labore et dolore magna aliqua. \\
  \textcolor{grey2}{--- Dr. Jane Doe (Member of the European Parliament and VIP)}
}

% Funding notice left column
\FundingLNotice{\selectlanguage{icelandic}
  Höfundar þessa rits þakka höfundum hvítbókar um þýsku fyrir leyfi til að endurnýta almenna kafla úr verki þeirra. \cite{lwpgerman}
  \bigskip
  Gerð þessarar hvítbókar var kostuð af Sjöundu rammaáætlun Evrópusambandsins og Stefnumótunaráætlun Evrópusambandsins í upplýsinga- og samskiptatækni samkvæmt samningum við T4ME (samningur 249119), CESAR (samningur 271022), META\-NET4U (samningur 270893) og META-NORD (samningur 270899).
}

% Funding notice right column
\FundingRNotice{\selectlanguage{english}
  The authors of this document are grateful to the authors of the White Paper on German for permission to re-use selected language-independent materials from their document. \cite{lwpgerman} 
  \bigskip
  The development of this white paper has been funded by the Seventh Framework Programme and the ICT Policy Support Programme of the European Commission under the contracts T4ME (Grant Agreement 249119), CESAR (Grant Agreement 271022), META\-NET4U (Grant Agreement 270893) and META-NORD (Grant Agreement 270899).
}
