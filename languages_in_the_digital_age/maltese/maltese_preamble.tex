%                                     MMMMMMMMM                                         
%                                                                             
%  MMO    MM   MMMMMM  MMMMMMM   MM    MMMMMMMM   MMD   MM  MMMMMMM MMMMMMM   
%  MMM   MMM   MM        MM     ?MMM              MMM$  MM  MM         MM     
%  MMMM 7MMM   MM        MM     MM8M    MMMMMMM   MMMMD MM  MM         MM     
%  MM MMMMMM   MMMMMM    MM    MM  MM             MM MMDMM  MMMMMM     MM     
%  MM  MM MM   MM        MM    MMMMMM             MM  MMMM  MM         MM     
%  MM     MM   MMMMMM    MM   MM    MM            MM   MMM  MMMMMMM    MM
%
%
%          - META-NET Language Whitepaper | Maltese Metadata -
% 
% ----------------------------------------------------------------------------

\usepackage{polyglossia}
\usepackage{covington}
\setotherlanguages{maltese,english}


\title{Il-Lingwa Maltija\ \ \ \ \ \ fil-Era\ \ \ \ \ Diġitali --- The Maltese Language\ \ \ \ in the\ \ \ \ \ Digital Age}

\subtitle{White Paper Series --- Serje ta’ White Papers}

\author{
  Mike Rosner\\
  Jan Joachimsen
}

\authoraffiliation{
  Mike Rosner~ {\small University of Malta}\\
  Jan Joachimsen~ {\small University of Malta}  
}

\editors{
  Georg Rehm, Hans Uszkoreit\\(Edituri, \textcolor{grey1}{editors})
}

% Text in left column on backside of the cover
\SpineLText{\selectlanguage{english}%
  In everyday communication, Europe’s citizens, business partners and politicians are inevitably confronted with language barriers.  
  Language technology has the potential to overcome these barriers and to provide innovative interfaces to technologies and knowledge. 
  This white paper presents the state of language technology support for the German language. 
  It is part of a series that analyzes the available language resources and technologies for 31~European languages. 
  The analysis was carried out by META-NET, a Network of Excellence funded by the European Commission.
  META-NET consists of 54 research centres in 33 countries, who cooperate with stakeholders from economy, government agencies, research organisations, non-governmental organisations, language communities and European universities. 
  META-NET’s vision is high-quality language technology for all European languages. 
}

% Text in right column on backside of the cover
\SpineRText{\selectlanguage{maltese}%
Fil-kommunikazzjoni tal-ħajja ta' kuljum, iċ-ċittadini tal-Ewropa, imsieħba fin-negozju u politikanti huma kkonfrontati inevitabbilment minn ostakli lingwistiċi. Teknoloġija lingwistika għandha l-potenzjal sabiex tagħleb dawn l-ostakli u biex tipprovdi interfaces innovativi għat-teknoloġiji u l-għarfien. Din il-Whitepaper tippreżenta l-istat tal-appoġġ ta' teknoloġija lingwistika għal-lingwa Maltija. Hija parti minn serje li janalizza r-riżorsi u t-teknoloġiji lingwistiċi li huma disponnibli għal-31 lingwa Ewropeja. L-analiżi ġiet imwettqa minn META-NET, netwerk ta' eċċenlenza ffinanzjati mill-Kummissjoni Ewropeja. META-NET tikkonsisti minn 54 ċentri ta' riċerka fi 33 pajjiż li jikkowoperaw ma' partijiet interessati mill-ekonomija, mill-aġenziji tal-gvern, organizzazzjonijiet ta' riċerka, organizzazzjonijiet mhux governattivi, komunitajiet tal-lingwa u l-universitajiet Ewropej. Il-viżjoni ta' META-NET hija waħda ta' teknoloġija lingwistika ta' kwalità għolja għal-lingwi kollha Ewropej.
}

% Quotes from VIPs on backside of the cover
\quotes{%
  Excepteur sint occaecat cupidatat non proident, sunt in culpa qui officia deserunt mollit anim id est laborum. Lorem ipsum dolor sit amet, consectetur adipisicing elit, sed do eiusmod tempor labore et dolore magna aliqua. \\
  \textcolor{grey2}{--- Prof. Dr. John Doe (Member of the European Parliament and VIP)}\\[3mm]
  Excepteur sint occaecat cupidatat non proident, sunt in culpa qui officia deserunt mollit anim id est laborum. Lorem ipsum dolor sit amet, consectetur adipisicing elit, sed do eiusmod tempor labore et dolore magna aliqua. \\
  \textcolor{grey2}{--- Dr. Jane Doe (Member of the European Parliament and VIP)}
}

% Funding notice left column
\FundingLNotice{\selectlanguage{maltese} 
  Il-pubblikatur huwa grat lejn l-awturi tal-white paper Ġermaniża\cite{lwpgerman} għall-permess li jirriproduċi materjal mill-white paper imsemmija.  Grazzi lil Ritianne Stanyer u Roberta Abela għat-translazzjoni ta' dan id-dokument.
  \bigskip L-iżvilupp ta’ din il-white paper ġie ffinanzjat mis-Seba' Programm ta' Qafas u l-Programm ta' Appoġġ għall-Politika dwar l-ICT tal-Kummissjoni Ewropea taħt il-kuntratti T4ME (Grant Agreement 249119), CESAR (Grant Agreement 271022), METANET4U (Grant Agreement 270893) u META-NORD (Grant Agreement 270899).
}

% Funding notice right column
\FundingRNotice{\selectlanguage{english} The authors of this document
  are grateful to the authors of the White Paper on German\cite{lwpgerman} for
  permission to re-use selected language-independent materials from
  their document. Thanks to Ritianne Stanyer and Roberta Abela for the translation of this document.
  \bigskip
  The development of this white paper has been funded by the Seventh
  Framework Programme and the ICT Policy Support Programme of the
  European Commission under the contracts T4ME (Grant Agreement
  249119), CESAR (Grant Agreement 271022), METANET4U (Grant Agreement
  270893) and META-NORD (Grant Agreement 270899).
}
