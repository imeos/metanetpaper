%                                     MMMMMMMMM                                         
%                                                                             
%  MMO    MM   MMMMMM  MMMMMMM   MM    MMMMMMMM   MMD   MM  MMMMMMM MMMMMMM   
%  MMM   MMM   MM        MM     ?MMM              MMM$  MM  MM         MM     
%  MMMM 7MMM   MM        MM     MM8M    MMMMMMM   MMMMD MM  MM         MM     
%  MM MMMMMM   MMMMMM    MM    MM  MM             MM MMDMM  MMMMMM     MM     
%  MM  MM MM   MM        MM    MMMMMM             MM  MMMM  MM         MM     
%  MM     MM   MMMMMM    MM   MM    MM            MM   MMM  MMMMMMM    MM
%
%
%          - META-NET Language Whitepaper | Maltese Metadata -
% 
% ----------------------------------------------------------------------------

%\usepackage[maltese, english]{babel}
\usepackage{polyglossia}
\setotherlanguages{maltese,english}


\title{Il-Lingwa Maltija Fl-Era Diġitali --- The Maltese Language in the Digital Age}

\spineTitle{The Maltese Language in the Digital Age --- Il-Lingwa Maltija Fl-Era Diġitali}

\subtitle{White Paper Series --- Serje ta’ White Papers}

\author{
  Mike Rosner \\
  Jan Joachimsen  
}

\authoraffiliation{
  Mike Rosner~ {\small University of Malta}\\
  Jan Joachimsen~ {\small University of Malta} 
}

\editors{
  Georg Rehm, Hans Uszkoreit\\(edituri, \textcolor{grey1}{editors})
}

% Text in left column on backside of the cover
\SpineLText{\selectlanguage{english}% In everyday communication,
  Europe’s citizens, business partners and politicians are inevitably
  confronted with language barriers.  Language technology has the
  potential to overcome these barriers and to provide innovative
  interfaces to technologies and knowledge.  This white paper presents
  the state of language technology support for the Maltese language.
  It is part of a series that analyses the available language
  resources and technologies for 30~European languages.  The analysis
  was carried out by META-NET, a Network of Excellence funded by the
  European Commission.  META-NET consists of 54 research centres in 33
  countries, who cooperate with stakeholders from economy, government
  agencies, research organisations, non-governmental organisations,
  language communities and European universities.  META-NET’s vision
  is high-quality language technology for all European languages.  }

% Text in right column on backside of the cover
\SpineRText{\selectlanguage{maltese}% Fil-komunikazzjoni ta’ kuljum,
   iċ-ċittadini tal-Ewropa, sħab tan-negozju u l-politikanti huma
   inevitabbilment jiffaċċjaw ostakli lingwistiċi.  It-Teknoloġija
   Lingwistika għandha l-potenzjal biex tegħleb dawn l-ostakoli u biex
   tipprovdi \emph{interfaces} innovattivi għal teknoloġiji u
   għarfien.  Dan \emph{white paper} tippreżenta l-istat tal-appoġġ
   tat-teknoloġija lingwistika għall-lingwa Maltija.  Huwa parti minn
   sensiela li tanalizza r-riżorsi tal-lingwa disponibbli u
   teknoloġiji għal 30~lingwi Ewropej.  L-analiżi saret minn META-NET,
   Netwerk ta’ Eċċellenza ffinanzjati mill-Kummissjoni Ewropea.
   META-NET tikkonsisti minn 54 ċentru ta’ riċerka fi 33 pajjiż, li
   jikkoperaw ma’ partijiet interessati mill-ekonomija, l-aġenziji
   tal-gvern, organizzazzjonijiet ta’ riċerka, %u oħrajn
   organizzazzjonijiet mhux governattivi, komunitajiet tal-lingwa u
   l-universitajiet Ewropej.  Il-viżjoni ta’ META-NET hija lingwa
   teknoloġija ta’ kwalità għolja għall-lingwi kollha Ewropej.}
   \vspace{-5mm}
% Quotes from VIPs on backside of the cover
\quotes{
\selectlanguage{maltese}%
„Lejn mappa tal-Malti fl-era diġitali.“ \\
 \textcolor{grey2}{--- Prof. Ray Fabri (Ċermen tal-Institut tal-Lingwistika, Università ta' Malta)}\\[3mm]
  \selectlanguage{english}%
``Mapping out the territory of Maltese in a digital age." \\
\textcolor{grey2}{--- Prof. Ray Fabri (Chairman of the Institute of Linguistics, University of Malta)}
}

% Funding notice left column
\FundingLNotice{\selectlanguage{maltese}\vskip2mm
  L-awturi ta’ dan id-dokument huma grati lejn l-awturi
  tal-\emph{White Paper} Ġermaniż għall-permess biex jużaw mill-ġdid
  materjali magħżulin mid-dokument tagħhom \cite{lwpgerman} li huma
  indipendenti mill-lingwa. 
  
  \bigskip
  Grazzi lil Ritianne Stanyer u Roberta Abela għat-translazzjoni ta’
  dan id-dokument. L-awturi huma obbligati lejn il-Professur Ray
  Fabri, li l-artiklu iċċitat fil-biblijografija bħala Fabri (2011a),
  kien l-ispirazzjoni għal ħafna mill-kontenut u ħafna mill-eżempji
  f’din id-dokument.
  
  \bigskip
  \begin{spacing}{1.2}
  L-iżvilupp ta’ din il-white paper ġie ffinanzjat mis-Seba’ Programm
ta’ Qafas u l-Programm ta’ Appoġġ għall-Politika dwar l-ICT
tal-Kummissjoni Ewropea taħt il-kuntratti T4ME (Grant Agreement
249119), CESAR (Grant Agreement 271022), METANET4U (Grant Agreement
270893) u META-NORD (Grant Agreement 270899).\end{spacing}}

% Funding notice right column
\FundingRNotice{\selectlanguage{english}\vskip2mm
  The authors of this document are grateful to the authors of the
  White Paper on German for permission to re-use selected
  language-independent materials from their document
  \cite{lwpgerman}. 
  
  \bigskip
  Thanks to Ritianne Stanyer and Roberta Abela for the translation of
  this document. The authors are indebted to Prof. Ray Fabri, whose
  article, cited in the bibliography as Fabri (2011a), was the
  inspiration for much of the content and many of the examples in the
  document.
  
  \bigskip
  \begin{spacing}{1.2}
  The development of this white paper has been funded by the Seventh
Framework Programme and the ICT Policy Support Programme of the
European Commission under the contracts T4ME (Grant Agreement 249119),
CESAR (Grant Agreement 271022), META\-NET4U (Grant Agreement 270893)
and META-NORD (Grant Agreement 270899).\end{spacing}}
