%------------------------------------
% META-NET Whitepaper series
%------------------------------------

\documentclass[11pt]{article}
\usepackage{hyperref}

\begin{document}

  \title{Il-Lingwi fis-Soċjetà Ewropea tal-Informazzjoni}
  
  \author{
    Dr. Aljoscha Burchardt, DFKI \and
    Prof. Dr. Markus Egg, Humboldt-Universität zu Berlin \and
    Kathrin Eichler, DFKI \and
    Dr. Brigitte Krenn, ÖFAI \and
    Prof. Dr. Annette Leßmöllmann, Hochschule Darmstadt \and
    Dr. Georg Rehm, DFKI \and
    Prof. Dr. Manfred Stede, Universität Potsdam \and
    Prof. Dr. Hans Uszkoreit, Universität des Saarlandes and DFKI \and
    Prof. Dr. Martin Volk, Universität Zürich \and
    Mike Rosner, University of Malta \and
    Jan Joachimsen, University of Malta
  }
  
  \maketitle
  
  \tableofcontents
  \clearpage
  
  \ssection*{Preface}
  Il-White Paper hija għall-edukaturi, ġurnalisti, politikanti, kommunitajiet ta’ lingwi u oħrajn, li jixtiequ jistabbilixxu Ewropa tabilħaqq multilingwali. \\\\
  Din hija parti minn serje White Papers li jippromwovu għarfien dwar it-teknoloġija lingwistika u l-potenzjal tagħha. Id-disponibiltà u l-użu tat-teknoloġija lingwistika fl-Ewropa tvarja bejn lingwa u oħra. Konsegwentement, l-azzjonijiet li huma meħtieġa sabiex jiġu appoġġjati r-riċerka u l-iżvilupp tat-teknoloġiji lingwistiċi ivarjaw ukoll f’kull lingwa.  L-azzjonijiet meħtieġa jiddependu fuq bosta fatturi, bħal kumplessità ta’ lingwa partikolari u d-daqs tal-komunità tagħha.\\\\
  META-NET, Netwerk ta’ Eċċellenza tal-Kummissjoni Ewropea, wettaq analiżi dwar ir-riżorsi u t-teknoloġiji lingwistiċi kurrenti. Din l-analiżi kienet ibbażata fuq 23 lingwa uffiċjali Ewropea, kif ukoll lingwi reġjonali oħra importanti fl-Ewropa. Ir-riżultati ta’ dan l-analiżi jissuġġerixxu li hemm bosta nuqqasijiet fir-riċerka għal kull lingwa. Analiżi aktar dettaljata u esperta u assessjar tas-sitwazzjoni kurrenti għandha tgħin sabiex timmassimizza l-impatt ta’ riċerka addizzjonali u timminimizza xi riskji.\\\\
  META-NET tikkonsisti f’44 ċentru ta’ riċerka minn 31 pajjiż li qed jaħdmu ma’ partijiet interessati minn negozji kummerċjali, aġenziji governattivi, industriji, organizzazzjonijiet ta’ riċerka, kumpaniji ta’ software, fornituri ta’ teknoloġija u universitajiet Ewropej. Flimkien, dawn qed joħolqu viżjoni teknoloġika komuni filwaqt li jiżviluppaw aġenda ta’ riċerka strateġika li turi kif applikazzjonijiet teknoloġiċi lingwistiċi jistgħu jindirizzaw xi nuqqasijiet ta’ riċerka sal-2020. 
  
  \ssection*{Kuntatt}
  META-NET\\
  DFKI Projektbüro Berlin\\
  Alt-Moabit 91c\\
  10559 Berlin\\
  il-Ġermanja\\\\
  \url{office@meta-net.eu}\\
  \url{http://www.meta-net.eu}
  
  \ssection*{Rikonoxximenti}
  Il-pubblikatur huwa grat lejn l-awturi tal-white paper Ġermaniża għall-permess li jirriproduċi materjal mill-white paper imsemmija. 
  
  \ssection*{Sommarju Eżekuttiv}
    

\end{document}
