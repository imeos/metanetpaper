%                                     MMMMMMMMM                                         
%                                                                             
%  MMO    MM   MMMMMM  MMMMMMM   MM    MMMMMMMM   MMD   MM  MMMMMMM MMMMMMM   
%  MMM   MMM   MM        MM     ?MMM              MMM$  MM  MM         MM     
%  MMMM 7MMM   MM        MM     MM8M    MMMMMMM   MMMMD MM  MM         MM     
%  MM MMMMMM   MMMMMM    MM    MM  MM             MM MMDMM  MMMMMM     MM     
%  MM  MM MM   MM        MM    MMMMMM             MM  MMMM  MM         MM     
%  MM     MM   MMMMMM    MM   MM    MM            MM   MMM  MMMMMMM    MM
%
%
%          - META-NET Language Whitepaper | Danish Metadata -
% 
% ----------------------------------------------------------------------------

\usepackage{booktabs}
\usepackage{longtable}
\usepackage{tabulary}
\usepackage{tabularx}
\usepackage{rotating}
\usepackage{makecell}
\usepackage{multirow}
\usepackage{colortbl}
\usepackage{multicol,framed,lipsum}
\usepackage{polyglossia}
\setotherlanguages{danish,english}


\title{Det danske sprog i den digitale  tidsalder --- The Danish Language in the Digital Age}

\spineTitle{The Danish Language in the Digital Age --- Det danske sprog i den digitale  tidsalder }

\subtitle{White Paper Series --- Hvidbogsserie}

\author{
  Bolette Sandford Pedersen  \\
  J\"{u}rgen Wedekind  \\
  Steen B\o hm-Andersen \\
  Peter Juel Henrichsen\\
  Sanne Hoffensetz-Andresen\\
  Sabine Kirchmeier-Andersen\\
  Jens Otto Kj\ae rum\\
  Louise Bie Larsen\\
  Bente Maegaard\\
  Sanni Nimb\\
  Jens-Erik Rasmussen\\
  Peter Revsbech\\
  Hanne Erdman Thomsen
}

\authoraffiliation{
  \mbox{Bolette S.\ Pedersen~{\small K\o benhavns Universitet}}\\
  \mbox{J\"{u}rgen Wedekind~{\small K\o benhavns Universitet}}\\
  \mbox{Steen B\o hm-Andersen~{\small Ankiro}}\\ 
  \mbox{Peter J.\ Henrichsen~{\small Copenhagen Business School}}\\
  \mbox{Sanne Hoffensetz-Andresen~{\small ordbogen.com}}\\
  \mbox{Sabine Kirchmeier-Andersen~{\small Dansk Sprogn\ae vn}}\\
  \mbox{Jens Otto Kj\ae rum~{\small Prolog Development Center}}\\
  \mbox{Louise Bie Larsen~{\small Ankiro}}\\
  \mbox{Bente Maegaard~{\small K\o benhavns Universitet}}\\
  \mbox{Sanni Nimb~{\small Det Danske Sprog- og Litteraturselskab}}\\
  \mbox{Jens-Erik Rasmussen~{\small Mikro V\ae rkstedet}}\\
  \mbox{Peter Revsbech~{\small ordbogen.com}}\\
  \mbox{Hanne E.\ Thomsen~{\small Copenhagen Business School}}
}

\editors{
  Georg Rehm, Hans Uszkoreit\\(udgivere, \textcolor{grey1}{editors})
}

% Text in left column on backside of the cover
\SpineLText{\selectlanguage{english}%
  In everyday communication, Europe’s citizens, business partners and politicians are inevitably confronted with language barriers.  
  Language technology has the potential to overcome these barriers and to provide innovative interfaces to technologies and knowledge. 
  This white paper presents the state of language technology support for the Danish language. 
  It is part of a series that analyzes the available language resources and technologies for 30~European languages. 
  The analysis was carried out by META-NET, a Network of Excellence funded by the European Commission.
  META-NET consists of 54 research centres in 33 countries, who cooperate with stakeholders from economy, government agencies, research organisations, non-governmental organisations, language communities and European universities. 
  META-NET’s vision is high-quality language technology for all European languages. 
}

% Text in right column on backside of the cover
\SpineRText{\selectlanguage{danish}%
 N\aa r vi kommunikerer i hverdagen, bliver Europas borgere,
forretnings\-partnere og politikere helt uundg\aa eligt konfronteret med
sprogbarrierer.  Sprogteknologien har potentiale til at nedbryde disse
barrierer og levere helt nye gr\ae nseflader til teknologi og viden.
Denne hvidbog pr\ae senterer status for sprogteknologisk st\o tte til det
danske sprog. Den er en del af en serie, som analyserer de
sprogresurser og -teknologier der er tilg\ae ngelige for 30 EU-sprog.
Analysen blev udf\o rt af META-NET, et Network of Excellence som er
finansieret af EU-Kommissionen.  META-NET best\aa r af 54
forskningscentre i 33 lande der samarbejder med interessenter fra
\o konomien, regeringer, forskningsinstitutioner, ikke-statslige
organisationer, sprogf\ae llesskaber og europ\ae iske universiteter.
META-NETs vision er sprogteknologi af h\o j kvalitet for alle europ\ae iske
sprog.
}

% Quotes from VIPs on backside of the cover
\quotes{%
The symbiosis of language and technology is in rapid growth today.
Being able to use, understand, and communicate with the technology
through our local languages imposes high demands on Danish research and
development in language technology.\\
\textcolor{grey2}{--- Kim Escherich (IBM Executive Innovation Architect, Sensor Solutions)}\\[3mm]
Hvis vi har ambitioner om at bruge det danske sprog i fremtidens
teknologiske univers, skal der g\o res en indsats nu for at fastholde
ekspertise og udbygge den viden vi har. Det viser META-NET rapporten
med stor tydelighed. Ellers risikerer vi at kun folk der taler flydende
engelsk, vil \mbox{f\aa} gl\ae de af de nye generationer af web-, tele-
og robotteknologi der er \mbox{p\aa} vej.\\
\textcolor{grey2}{--- Sabine Kirchmeier-Andersen (Direkt\o r for Dansk Sprogn\ae vn)}\\[3mm]
%The symbiosis of language and technology is in rapid growth today.
%Being able to use, understand, and communicate with the technology
%through our local languages imposes high demands on Danish research and
%development in language technology.
%\textcolor{grey2}{--- Kim Escherich (IBM Executive Innovation Architect, Sensor Solutions)}
}

% Funding notice left column
\FundingLNotice{\selectlanguage{danish}\vskip2mm
  Forfatterne af dette dokument er forfatterne til sprograpporten for tysk taknemmelige for tilladelsen til at genbruge udvalgt sproguafh\ae ngigt materiale fra deres dokument \cite{lwpgerman}.\\
 
  Udarbejdelsen af denne sprograpport er blevet finansieret af EU's 7.\ 
  rammeprogram og ICT Policy Support Programme under kontrakterne T4ME
  (kontrakt nr.\  249119), CESAR (kontrakt nr. 271022), METANET4U
  (kontrakt nr.\ 270893) og META-NORD
  (kontrakt nr.\  270899). Vi takker Lina Henriksen og Sussi Olsen for den danske overs\ae ttelse af den engelske vsersion.}

% Funding notice right column
\FundingRNotice{\selectlanguage{english}\vskip2mm
  The authors of this document are grateful to the authors of the white paper on German for permission to re-use selected language-independent materials from their document \cite{lwpgerman}. \\
 
  The development of this white paper has been funded by the Seventh
  Framework Programme and the ICT Policy Support Programme of the
  European Commission under the contracts T4ME (Grant Agreement 249119),
  CESAR (Grant Agreement 271022), METANET4U (Grant Agreement 270893)
  and META-NORD (Grant Agreement 270899). We thank Lina Henriksen and Sussi Olsen for the Danish translation of the English version.
}

\hyphenation{glo-balt del-tagelse}
