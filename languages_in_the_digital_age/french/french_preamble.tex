%                                     MMMMMMMMM
%
%  MMA    MM   MMMMMM  MMMMMMM   MM    MMMMMMMM   MMA   MM  MMMMMMM MMMMMMM
%  MMMA AMMM   MM        MM     MMMM              MMMM  MM  MM        MM
%  MM MMM MM   MMMMMM    MM    IM  MI   MMMMMMM   MM MMxMM  MMMMMM    MM
%  MM  M  MM   MM        MM   .MMMMMM.            MM  MMMM  MM        MM
%  MM     MM   MMMMMM    MM   MM    MM            MM   MMM  MMMMMMM   MM
%
%
%          - META-NET Language Whitepaper | French Metadata -
% 
% ----------------------------------------------------------------------------

\usepackage{polyglossia}
\setotherlanguages{french,english}

\title{La langue française  \ \  \ \ \ \  à l{\mbox '}Ère du numérique --- The French Language in the Digital Age}

\spineTitle{The French Language in the Digital Age --- La langue française à l' Ère du numérique}

\subtitle{White Paper Series --- Collection de Livres Blancs}

\author{
  Joseph Mariani \\
  Patrick Paroubek \\
  Gil Francopoulo \\
  Aurélien Max \\
  François Yvon \\
  Pierre Zweigenbaum \\
}
\authoraffiliation{
  Joseph Mariani {\tiny IMMI-CNRS \& LIMSI-CNRS}\\
  Patrick Paroubek {\tiny LIMSI-CNRS}\\
  Gil Francopoulo {\tiny IMMI-CNRS \& TAGMATICA}\\
  Aurélien Max {\tiny LIMSI-CNRS \& U. Paris Sud 11}\\
  François Yvon {\tiny LIMSI-CNRS \& U. Paris Sud 11}\\
  Pierre Zweigenbaum {\tiny LIMSI-CNRS}\\
}

\editors{
  Georg Rehm, Hans Uszkoreit\\(\'Editeurs, \textcolor{grey1}{editors})
}

% Text in left column on backside of the cover
\SpineLText{\selectlanguage{english}%
 In everyday communication, Europe’s citizens, business partners and politicians are inevitably confronted with language barriers. Language technology has the potential to overcome these barriers and to provide innovative interfaces to technologies and knowledge. This white paper presents the state of language technology support for the French language. It is part of a series that analyses the available language resources and technologies for 30 European languages. The analysis was carried out by META-NET, a Network of Excellence funded by the European Commission. META-NET consists of 54 research centres in 33 countries, who cooperate with stakeholders from economy, government agencies, research organisations, non-governmental organisations, language communities and European universities. META-NET’s vision is high-quality language technology for all European languages.
}

% Text in right column on backside of the cover
\SpineRText{\selectlanguage{french}%
Chaque jour en Europe, les citoyens, les partenaires commerciaux et les hommes politiques sont inévitablement confrontés aux barrières linguistiques. Les Technologies de la Langue ont la capacité de renverser ces barrières et de fournir des interfaces innovantes pour l'accès aux connaissances. Ce Livre Blanc présente l'état des Technologies de la Langue pour la langue française. Il fait partie d'une collection qui analyse les ressources et les technologies existantes pour 30 langues européennes. Cette analyse a été conduite dans le cadre de  META-NET, un Réseau d'Excellence soutenu par la Commission Européenne. META-NET est constitué de 54 centres de recherche dans 33 pays, qui collaborent avec les acteurs de l'économie, les agences gouvernementales, les organismes de recherche, les organisations non-gouvernementales et les universités européennes. La vision de META-NET est celle de l'existence de technologies de la langue de haute qualité au service de toutes les langues européennes.
}

% Quotes from VIPs on backside of the cover
\quotes{%
{\small
«\,Le réseau d'excellence META-NET apporte une contribution
inestimable à l'élaboration d'une véritable stratégie européenne en faveur du
multilinguisme, qui s'appuie sur les technologies existantes tout en
encourageant le développement de technologies futures. Cette démarche va dans
le sens d'une meilleure compréhension entre les citoyens et les administrations
communautaires, et facilite la prise en compte de la diversité linguistique, au
niveau national comme au niveau régional, par exemple dans les territoires de
l'Outre-mer français.\,»\\
%
% The META-NET
% Network of Excellence provides an invaluable contribution to the development of
% a genuine European strategy in support to multilingualism, based on existing
% technologies while encouraging the development of new innovative technologies.
% This approach aims at a better understanding between citizens and community
% administrations, and will facilitate the recognition of linguistic diversity,
% at the national and regional levels, including in the overseas French territories.
%
 \textcolor{grey2}{--- Xavier North (Délégué Général à la Langue
 Française et aux Langues de France, DGLFLF)% (10 janvier 2012 –
                                % January 10, 2012)
}
}
}

% Funding notice left column
\FundingLNotice{\selectlanguage{french}\vskip2mm
Les auteurs de ce document remercient les auteurs du Livre Blanc sur
l'allemand de leur avoir permis de réutiliser des éléments génériques
de leur document \cite{lwpgerman}.

  \bigskip
\begin{spacing}{1.2}
  La production de ce Livre Blanc a été financée par le septième
  Programme-Cadre et le Programme d'appui stratégique en Technologies de
  l’Information et de la Communication (TIC) de la Commission Européenne
  dans le cadre des contrats T4ME (convention d’aide 249\,119), CESAR
  (convention d’aide 271\,022), METANET4U (convention d’aide 270\,893) et
  META-NORD (convention d’aide 270\,899).
\end{spacing}
}

% Funding notice right column
\FundingRNotice{\selectlanguage{english}\vskip2mm
  The authors of this document
  are grateful to the authors of the White Paper on the German language for
  permission to re-use selected language-independent materials from
  their document \cite{lwpgerman}.
  
  \bigskip
\begin{spacing}{1.2}
  The development of this White Paper has been funded by the Seventh
  Framework Programme and the ICT Policy Support Programme of the
  European Commission under the contracts T4ME (Grant Agreement
  249\,119), CESAR (Grant Agreement 271\,022), METANET4U (Grant Agreement
  270\,893) and META-NORD (Grant Agreement 270\,899).
\end{spacing}
}
