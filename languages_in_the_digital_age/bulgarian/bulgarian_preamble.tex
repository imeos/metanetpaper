%                                     MMMMMMMMM                                         
%                                                                             
%  MMO    MM   MMMMMM  MMMMMMM   MM    MMMMMMMM   MMD   MM  MMMMMMM MMMMMMM   
%  MMM   MMM   MM        MM     ?MMM              MMM$  MM  MM         MM     
%  MMMM 7MMM   MM        MM     MM8M    MMMMMMM   MMMMD MM  MM         MM     
%  MM MMMMMM   MMMMMM    MM    MM  MM             MM MMDMM  MMMMMM     MM     
%  MM  MM MM   MM        MM    MMMMMM             MM  MMMM  MM         MM     
%  MM     MM   MMMMMM    MM   MM    MM            MM   MMM  MMMMMMM    MM
%
%
%          - META-NET Language Whitepaper | Bulgarian Metadata -
% 
% ----------------------------------------------------------------------------

\usepackage{polyglossia}
\setotherlanguages{bulgarian,english}


\title{Българският език в дигиталната епоха --- The Bulgarian
  Language\ \ \ \ in the\ \ \ \ \ Digital Age}

\subtitle{White Paper Series --- Серия Бели книги}

\author{
  Diana Blagoeva \\
  Svetla Koeva \\
  Vladko~Murdarov
  % Names in Bulgarian
  % Диана Благоева~  {\small Българска академия на науките}\\
  % Светла Коева~  {\small Българска академия на науките}\\
  % Владко Мурдаров~  {\small Българска академия на науките}\\
}

\authoraffiliation{
  Diana Blagoeva~  {\small Bulgarian Academy of Sciences}\\
  Svetla Koeva~ {\small Bulgarian Academy of Sciences}\\
  Vladko~Murdarov~  {\small Bulgarian~Academy~of~Sciences}
  % Names in Bulgarian
  % Диана Благоева~  {\small Българска академия на науките}\\
  % Светла Коева~  {\small Българска академия на науките}\\
  % Владко Мурдаров~  {\small Българска академия на науките}\\
}

\editors{
  Georg Rehm, Hans Uszkoreit\\(редактори, \textcolor{grey1}{editors})
  % Names in Bulgarian
  %  Георг Рем, Ханс Уцкорайт\\(редактори \textcolor{grey1}{editors})
} 

% Text in left column on backside of the cover
\SpineLText{\selectlanguage{english}%
  In everyday communication, Europe’s citizens, business partners and politicians are inevitably confronted with language barriers.  
  Language technology has the potential to overcome these barriers and to provide innovative interfaces to technologies and knowledge. 
  This white paper presents the state of language technology support for the German language. 
  It is part of a series that analyzes the available language resources and technologies for 31~European languages. 
  The analysis was carried out by META-NET, a Network of Excellence funded by the European Commission.
  META-NET consists of 54 research centres in 33 countries, who cooperate with stakeholders from economy, government agencies, research organisations, non-governmental organisations, language communities and European universities. 
  META-NET’s vision is high-quality language technology for all European languages. 
}

% Text in right column on backside of the cover
\SpineRText{\selectlanguage{bulgarian}%
Европейските граждани, бизнес партньори и политици в ежедневното си общуване неизбежно се сблъс\-кват с езикови бариери. Езиковите технологии имат потенциала да преодолеят тези бариери и да осигурят иновативна кореспонденция между технологията и знанието. Настоящата Бяла книга представя равнището на развитие на езиковите технологии за български език. Документът е част от серията Бели книги, анализиращи съществуващите езикови ресурси и технологии за 31 европейски езика. Изследването е осъществено от META-NET, мрежа за върхови постижения, финансирана от Европейската комисия. META-NET се състои от 54 изследователски центъра от 33 страни, които си сътрудничат с бизнеса, правителствени агенции, други научни организации, езикови общности и университети. Визията на META-NET е високотехнологични езикови ресурси за всички европейски езици.}

% Quotes from VIPs on backside of the cover
\quotes{%
 ``META-NET работи за преодоляването на езиковите бариери при ежедневното
общуване на европейските граждани,  бизнес партньори и политици." \\
\textcolor{grey2}{--- Емил Стоянов (член на Европейския парламент)}\\[3mm]
``Езиковите технологии могат
да допринесат съществено за запазването на
културното и езиково
многообразие в Европа." \\
\textcolor{grey2}{--- Запрян Козлуджов (ректор на Пловдивкия университет)}
}

% Funding notice left column
\FundingLNotice{\selectlanguage{bulgarian} Авторите на този документ
  благодарят сърдечно на авторите на Бялата книга за немски за
  предоставената възможност да използват избрани езиково независими
  части \cite{lwpgerman}.

  \bigskip
  Разработката на настоящата Бяла книга е финансирана по Седма рамкова
  програма и Програма ICT Policy Support Programme на Европейската
  комисия, договори T4ME (договор за финансиране 249119), CESAR
  (договор за финансиране 271022), METANET4U (договор за финансиране
  270893) и META-NORD (договор за финансиране 270899).
}

% Funding notice right column
\FundingRNotice{\selectlanguage{english} The authors of this document
  are grateful to the authors of the White Paper on German for
  permission to re-use selected language-independent materials from
  their document \cite{lwpgerman}.

  \bigskip
  The development of this white paper has been funded by the Seventh
  Framework Programme and the ICT Policy Support Programme of the
  European Commission under the contracts T4ME (Grant Agreement
  249119), CESAR (Grant Agreement 271022), METANET4U (Grant Agreement
  270893) and META-NORD (Grant Agreement 270899).
}
