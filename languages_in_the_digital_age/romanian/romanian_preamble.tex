%                                     MMMMMMMMM                                         
%                                                                             
%  MMO    MM   MMMMMM  MMMMMMM   MM    MMMMMMMM   MMD   MM  MMMMMMM MMMMMMM   
%  MMM   MMM   MM        MM     ?MMM              MMM$  MM  MM         MM     
%  MMMM 7MMM   MM        MM     MM8M    MMMMMMM   MMMMD MM  MM         MM     
%  MM MMMMMM   MMMMMM    MM    MM  MM             MM MMDMM  MMMMMM     MM     
%  MM  MM MM   MM        MM    MMMMMM             MM  MMMM  MM         MM     
%  MM     MM   MMMMMM    MM   MM    MM            MM   MMM  MMMMMMM    MM
%
%
%          - META-NET Language Whitepaper | Romanian Metadata -
% 
% ----------------------------------------------------------------------------

\usepackage{covington}
\usepackage{polyglossia}
\setotherlanguages{romanian,english}


\title{Limba română în era digitală --- The Romanian Language in the Digital Age}
\spineTitle{The Romanian Language in the Digital Age --- Limba română în era digitală}

\subtitle{White Paper Series --- Seria de studii}

\author{
  Diana Trandabăț \\
  Elena Irimia \\
  Verginica Barbu Mititelu \\
  Dan Cristea \\
  Dan Tufiș
}
\authoraffiliation{
  Diana Trandabăț\\{\small University “A. I. Cuza” of Iași,\\ Romanian Academy, Institute for Computer Science\\}
  Elena~ Irimia\\{\small Romanian~ Acad.~Research~ Inst.~for~ AI\\}
  Verginica~ Mititelu\\{\small Romanian~Acad.~Res.~Inst.~for~AI\\}
  Dan Cristea\\{\small University “A. I. Cuza” of Iași,\\ Romanian Academy, Institute for Computer Science\\}
  Dan Tufiș\\{\small Romanian Academy Research Inst. for~ AI\\}
}
\editors{
  Georg Rehm, Hans Uszkoreit\\(editori, \textcolor{grey1}{editors})
}

% Text in left column on backside of the cover
\SpineLText{\selectlanguage{english}%
  In everyday communication, Europe’s citizens, business partners and politicians are inevitably confronted with language barriers.  
  Language technology has the potential to overcome these barriers and to provide innovative interfaces to technologies and knowledge. 
  This white paper presents the state of language technology support for the Romanian language. 
  It is part of a series that analyzes the available language resources and technologies for 30~European languages. 
  The analysis was carried out by META-NET, a Network of Excellence funded by the European Commission.
  META-NET consists of 54 research centres in 33 countries, who cooperate with stakeholders from economy, government agencies, research organisations, non-governmental organisations, language communities and European universities. 
  META-NET’s vision is high-quality language technology for all European languages. 
}

% Text in right column on backside of the cover
\SpineRText{\selectlanguage{romanian}%
  Cetățenii, partenerii de afaceri și politicienii europeni se confruntă în mod inevitabil în comunicarea de zi cu zi cu bariere lingvistice. Tehnologiile limbajului au potențialul de a depăși aceste bariere și de a oferi interfețe inovative pentru noile tehnologii și cunoștințe. Acest studiu prezintă situația sprijinului acordat  tehnologiilor limbajului pentru limba română. El face parte dintr-o serie care analizează resursele și tehnologiile lingvistice disponibile pentru 30 de limbi europene. Analiza a fost efectuată de către META-NET, o rețea de excelență finanțată de Comisia Europeană. META-NET este formată din 54 de centre de cercetare din 33 de ţări, care colaborează cu persoane cheie din economie, agenții guvernamentale, institute de cercetare, organizații non-guvernamentale, comunități lingvistice și universități europene. Viziunea META-NET este de a oferi tehnologii ale limbajului de înaltă calitate pentru toate limbile europene.
}

% Quotes from VIPs on backside of the cover
\quotes{%
  „Scriem un mesaj pe telefonul mobil şi nici măcar nu suntem conştienţi de tehnologia care anticipează ce cuvinte vrem să scriem. Suntem obişnuiţi să trăim într-o lume în care dispozitivul GPS ne poate arăta drumul spre casă spunându-ne când să facem la stângă sau la dreapta. Numeroase tehnologii dezvoltate în ultimii ani au implicaţii foarte concrete în viaţa de zi cu zi a cetăţenilor din Uniunea Europeană. Tehnologiile lingvistice reprezintă un element central al Uniunii Europene întrucât limbile înseşi ocupă un loc central în modul de funcţionare a UE". \\
  \textcolor{grey2}{--- Leonard Orban (fost Comisar European pentru Multilingvism)}%\\[3mm]
% „Europa Unită se confruntă cu o aparentă contradicţie dintre nevoia de largă comunicare internaţională şi imperativul păstrării culturilor şi limbilor naţionale. Aici apar probleme serioase impuse de traducerea dintr-o limbă în alta, învăţarea cel puţin a unei limbi de circulaţie internaţională şi a punerii în concordanţă a tehnologiilor şi metodelor de traducere. Importanţa folosirii noilor tehnologii este evidentă dacă menţionăm doar că o combinaţie de cate două a celor 23 de limbi oficiale este de 253 de perechi de limbi". \\
  %\textcolor{grey2}{--- Acad. Ionel Haiduc (Preşedintele Academiei Romane)}
}

% Funding notice left column
\FundingLNotice{\selectlanguage{romanian}
%FIXME translation needed
% The authors of this document
%  are grateful to the authors of the White Paper on German for
%  permission to re-use selected language-independent materials from
%  their document
Autorii acestui document sunt recunoscători autorilor studiului pentru limba germană, care le-au permis să (re)folosească în prezentul document anumite materiale independente de limbă \cite{lwpgerman}.\vfill\bigskip
Acest studiu a fost finanțat prin Programul Cadru nr. 7 și prin Programul de sprijinire a politicii în domeniul Tehnologiilor Informației și Comunicațiilor (ICT Policy Support Programme) al
Comisiei Europene prin proiectele T4ME (contract nr. 249119), CESAR (contract nr. 271022), METANET4U (contract nr. 270893) și META-NORD (contract nr. 270899).
}

% Funding notice right column
\FundingRNotice{\selectlanguage{english} The authors of this document
  are grateful to the authors of the White Paper on German for
  permission to re-use selected language-independent materials from
  their document \cite{lwpgerman}.\vfill
  \bigskip
  The development of this white paper has been funded by the Seventh
  Framework Programme and the ICT Policy Support Programme of the
  European Commission under the contracts T4ME (Grant Agreement
  249119), CESAR (Grant Agreement 271022), METANET4U (Grant Agreement
  270893) and META-NORD (Grant Agreement 270899).
}
