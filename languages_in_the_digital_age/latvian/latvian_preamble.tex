%                                     MMMMMMMMM
%
%  MMA    MM   MMMMMM  MMMMMMM   MM    MMMMMMMM   MMA   MM  MMMMMMM MMMMMMM
%  MMMA AMMM   MM        MM     MMMM              MMMM  MM  MM        MM
%  MM MMM MM   MMMMMM    MM    IM  MI   MMMMMMM   MM MMxMM  MMMMMM    MM
%  MM  M  MM   MM        MM   .MMMMMM.            MM  MMMM  MM        MM
%  MM     MM   MMMMMM    MM   MM    MM            MM   MMM  MMMMMMM   MM
%
%
%          - META-NET Language Whitepaper | Latvian Metadata -
% 
% -------------------------------------------------------------------------

\usepackage{polyglossia}
\setotherlanguages{latvian,english}
\hyphenation{Da-līb-or-ga-ni-zā-ci-jas starp-tau-tis-kas}


\title{Latviešu valoda digitālajā laikmetā --- The \mbox{Latvian} Language in the Digital Age}
% Title for the spine of the cover
\spineTitle{The Latvian Language in the Digital Age --- Latviešu valoda digitālajā laikmetā}

\subtitle{White Paper Series --- Balto grāmatu sērija}

\author{
  Inguna Skadiņa \\
  Andrejs Veisbergs \\
  Andrejs Vasiļjevs \\
  Tatjana Gornostaja \\
  Iveta Keiša \\
  Alda Rudzīte  
}
\authoraffiliation{
  Inguna Skadiņa~ {\small Tilde}\\
  Andrejs Veisbergs~ {\small Latvijas Universitāte}\\
  Andrejs Vasiļjevs~ {\small Tilde}\\
  Tatjana Gornostaja~ {\small Tilde}\\
  Iveta Keiša~ {\small Tilde}\\
  Alda Rudzīte~ {\small Tilde}\\
}
\editors{
  \textcolor{grey1}{Georg Rehm, Hans Uszkoreit}\\
  (\textcolor{grey1}{editors}, redaktori)
}

% Text in left column on backside of the cover
\SpineLText{\selectlanguage{english}%
  \begin{spacing}{1.1}
  In everyday communication, Europe’s citizens, business partners and politicians are inevitably confronted with language barriers.  
  Language technology has the potential to overcome these barriers and to provide innovative interfaces to technologies and knowledge. 
  This white paper presents the state of language technology support for the Latvian language. 
  It is part of a series that analyses the available language resources and technologies for 30~European languages. 
  The analysis was carried out by META-NET, a Network of Excellence funded by the European Commission.
  META-NET consists of 54 research centres in 33 countries, who cooperate with stakeholders from economy, government agencies, research organisations, non-governmental organisations, language communities and European universities. 
  META-NET’s vision is high-quality language technology for all European languages. 
\end{spacing}
}

% Text in right column on backside of the cover
\SpineRText{\selectlanguage{latvian}%
  \begin{spacing}{1.1}
Ikdienas saziņā Eiropas iedzīvotājiem, darījumu part-neriem un politiķiem neizbēgami jārēķinās ar valodu barjeru radīto apgrūtinājumu.
Valodu tehnoloģijas var pārvarēt šīs barjeras un sniegt jaunas iespējas, kā izmantot tehnoloģijas un piekļūt zināšanām.
Šī baltā grāmata ir veltīta latviešu valodas tehnoloģiju atbalstam.
Tā ir daļa no sērijas, kas analizē pieejamos valodu resursus un tehnoloģijas 30~Eiropas valodām.
Analīzi veica Eiropas Komisijas finansēts izcilības tīkls META-NET, ko veido 54~pētniecības centri 33~valstīs.
Tie sadarbojas ar nozares pārstāvjiem, valsts aģentūrām, pētniecības iestādēm, nevalstiskajām organizācijām, valodu kopienām un Eiropas universitātēm.
META-NET mērķis ir nodrošināt augstas kvalitātes valodu tehnoloģijas visām Eiropas valodām.
\end{spacing}
}

% Quotes from VIPs on backside of the cover
\quotes{
\vspace{-7mm}
  \begin{spacing}{1.1}
``Diversity of cultures, traditions and languages is one of the most important treasures of Europe and it is our duty to preserve this heritage for generations to come. For such small languages like Latvian keeping up with the ever increasing pace of time and technological development is crucial. The only way to ensure the future existence of our language is to provide its users with equal opportunities as the users of larger languages enjoy. Therefore being at the forefront of modern technologies is our opportunity."\\
\textcolor{grey2}{--- Valdis Dombrovskis (Prime Minister of Latvia)}\\[3mm]
``This overview of the status of research on automated processing in the Latvian language contributes to the overall picture of EU capacity in a fast growing communication technologies area with vast potential for practical applications, much needed to further its policy of multilingualism and the growth of an information society."\\
\textcolor{grey2}{--- Dr. Vaira Vīķe-Freiberga (President of Latvia 1999-2007)}
\end{spacing}
}

% Funding notice left column
\FundingLNotice{\selectlanguage{latvian}\vskip2mm
 Šī dokumenta autori pateicas vācu valodas baltās grāmatas autoriem par atļauju atkārtoti izmantot daļu sava dokumenta materiālu, kas neskar konkrēto valodu \cite{Meta100}.
 
 \bigskip
 Autori pateicas Daigai Deksnei, Kārlim Gobam un Raivim Skadiņam par vērtīgajiem ierosinājumiem, \textit{Tildes} Lokalizācijas un dokumentācijas daļas tulkotājiem, īpaši Elitai Kalniņai, par sākotnējo dokumenta tulkojumu, Aivaram Bērziņam, Ievai Dātavai, Evitai Korņējevai, Indrai Sāmītei un Lindai Staužai par neizsīkstošu palīdzību dokumenta galīgās versijas sagatavošanā.
 
 \bigskip
  Šīs baltās grāmatas sagatavošanu finansiāli atbalstīja Eiropas Komisijas Septītā pamatprogramma un IKT politikas atbalsta programma saskaņā ar līgumiem T4ME (dotācijas nolīgums 249\,119), CESAR (dotācijas nolīgums 271\,022), \mbox{METANET4U} (dotācijas nolīgums 270\,893) un META-NORD (dotācijas nolīgums 270\,899).
}

% Funding notice right column
\FundingRNotice{\selectlanguage{english}\vskip2mm
   The authors of this document are grateful to the authors of the white paper on German for permission to re-use selected language-independent materials from their document.\cite{Meta100}
   
 \bigskip
  The authors would like to thank Daiga Deksne, Kārlis Goba and Raivis Skadiņš for their valuable comments and contributions, translators of Tilde's Localization and Documentation department, especially Elita Kalniņa for initial translation, and Aivars Bērziņš, Ieva Dātava, Evita Korņējeva, Indra Sāmīte, and Linda Stauža for their support in editing and finalization process of this document.

  \bigskip  
  The development of this White Paper has been funded by the Seventh
  Framework Programme and the ICT Policy Support Programme of the
  European Commission under the contracts T4ME (Grant Agreement
  249\,119), CESAR (Grant Agreement 271\,022), METANET4U (Grant Agreement
  270\,893) and META-NORD (Grant Agreement 270\,899).}
