%                                     MMMMMMMMM                                         
%                                                                             
%  MMO    MM   MMMMMM  MMMMMMM   MM    MMMMMMMM   MMD   MM  MMMMMMM MMMMMMM   
%  MMM   MMM   MM        MM     ?MMM              MMM$  MM  MM         MM     
%  MMMM 7MMM   MM        MM     MM8M    MMMMMMM   MMMMD MM  MM         MM     
%  MM MMMMMM   MMMMMM    MM    MM  MM             MM MMDMM  MMMMMM     MM     
%  MM  MM MM   MM        MM    MMMMMM             MM  MMMM  MM         MM     
%  MM     MM   MMMMMM    MM   MM    MM            MM   MMM  MMMMMMM    MM
%
%
%          - META-NET Language Whitepaper | Polish Metadata -
% 
% ----------------------------------------------------------------------------

\usepackage[polish, english]{babel}


\title{Język polski w erze cyfrowej --- The Polish Language in the Digital Age}

\subtitle{White Paper Series --- Seria raportów}

\author{
  Marcin Miłkowski
 }

\authoraffiliation{
  Marcin Miłkowski~ {\small Instytut Podstaw Informatyki PAN}
 }

\editors{
  Georg Rehm, Hans Uszkoreit\\(redakcja, \textcolor{grey1}{editors})\\
  Anna Cichosz (przekład na polski, \textcolor{grey1}{Polish translation}) 
}

% Text in left column on backside of the cover
\SpineLText{\selectlanguage{english}%
\small{
  In everyday communication, Europe’s citizens, business partners and politicians are inevitably confronted with language barriers.  
  Language technology has the potential to overcome these barriers and to provide innovative interfaces to technologies and knowledge. 
  This white paper presents the state of language technology support for the Polish language. 
  It is part of a series that analyzes the available language resources and technologies for 31~European languages. 
  The analysis was carried out by META-NET, a Network of Excellence funded by the European Commission.
  META-NET consists of 54 research centres in 33 countries, who cooperate with stakeholders from economy, government agencies, research organisations and others.
 %, non-governmental organisations, language communities and European universities. 
  META-NET’s vision is high-quality language technology for all European languages. 
}
}

% Text in right column on backside of the cover
\SpineRText{
\selectlanguage{polish}
\small{W życiu codziennym europejscy obywatele, przedsiębiorcy i politycy nieuchronnie napotykają bariery językowe. Technologie językowe dają możliwość pokonania tych barier i mogą posłużyć do stworzenia innowacyjnych interfejsów umożliwiających obsługę urządzeń i dostęp do wiedzy. Niniejszy raport przedstawia poziom technologii językowych w języku polskim. Należy do serii, która analizuje dostępne zasoby językowe i technologie w 31 językach europejskich. Analiza ta została przeprowadzona przez META-NET, sieć doskonałości finansowaną przez Komisję Europejską.  Na META-NET składają się 54 ośrodki naukowe w 33 krajach, które współpracują z podmiotami gospodarczymi, agencjami rządowymi, instytucjami badawczymi i innymi. Wizją sieci META-NET jest tworzenie wysokiej jakości technologii językowych dla wszystkich języków europejskich.}
}
\vspace{-5mm}

% Quotes from VIPs on backside of the cover
\quotes{
%\selectlanguage{polish}„Przetwarzanie języka naturalnego to coraz ważniejsza gałąź gospodarki opartej na wiedzy.” \\
% \textcolor{grey2}{--- anonimowy prezes}\\[3mm]
\selectlanguage{english}“I consider the scientific challenges that linguistic engineering faces as extremely promising and intellectually attractive intellectually research area, where apparently different methodologies and tools of linguistics and computer science meet. Language technologies, as a result of this research, will have a growing influence on capabilities and communication models of the contemporary world as well as on the way human natural languages, such as the Polish language, take part in this process. The text data analysis, speech synthesis and speech recognition, machine translation and text summarisation are more and more present in our everyday life. For their presence to be rational and functional, for it to serve the needs of the economy, as well as the social and cultural life well, further large-scale work in this area is needed..“ \\ \textcolor{grey2}{--- Prof. Michał Kleiber (President of the Polish Academy of Sciences)}
% \selectlanguage{polish} Wyzwania naukowe stojące przed inżynierią lingwistyczną uważam za niezwykle obiecujący i  atrakcyjny intelektualnie obszar badawczy, na którym spotykają się  odległe pozornie narzędzia i metodologie lingwistyki i informatyki.  Technologie przetwarzania języka, jako rezultat tych badań, będą miały coraz większy wpływ na możliwości i modele komunikacyjne współczesnego świata oraz na udział w tych procesach języków narodowych, a więc także języka polskiego. Analiza danych tekstowych, synteza mowy i jej rozpoznawanie, automatyczne tłumaczenie i streszczanie  są coraz powszechniej obecne w naszym życiu. Aby ta obecność  była racjonalna i funkcjonalna, aby  dobrze służyła gospodarce, życiu społecznemu i kulturalnemu, potrzebne są, na szeroką skalę prowadzone, dalsze prace w tym zakresie.
}

% Funding notice left column
\FundingLNotice{\selectlanguage{polish}
Autor tego opracowania dziękuje autorom raportu dotyczącego języka niemieckiego za zgodę na wykorzystanie materiałów niezależnych od języka \cite{lwpgerman}.\vfill
  \bigskip
Opracowanie niniejszego raportu zostało sfinansowane w ramach siódmego programu ramowego oraz programu na rzecz wspierania polityki w zakresie technologii informacyjnych i komunikacyjnych Komisji Europejskiej w ramach umów T4ME (grant 249119), CESAR (grant 271022), METANET4U (grant 270893) i META-NORD (grant 270899).
}

% Funding notice right column
\FundingRNotice{\selectlanguage{english}
  The authors of this document are grateful to the authors of the White Paper on German for permission to re-use selected language-independent materials from their document \cite{lwpgerman}. \vfill
  \bigskip
  The development of this white paper has been funded by the Seventh Framework Programme and the ICT Policy Support Programme of the European Commission under the contracts T4ME (Grant Agreement 249119), CESAR (Grant Agreement 271022), META\-NET4U (Grant Agreement 270893) and META-NORD (Grant Agreement 270899).
}

