%                                     MMMMMMMMM
%
%  MMA    MM   MMMMMM  MMMMMMM   MM    MMMMMMMM   MMA   MM  MMMMMMM MMMMMMM
%  MMMA AMMM   MM        MM     MMMM              MMMM  MM  MM        MM
%  MM MMM MM   MMMMMM    MM    IM  MI   MMMMMMM   MM MMxMM  MMMMMM    MM
%  MM  M  MM   MM        MM   .MMMMMM.            MM  MMMM  MM        MM
%  MM     MM   MMMMMM    MM   MM    MM            MM   MMM  MMMMMMM   MM
%
%
%          - META-NET Language Whitepaper | Polish Metadata -
% 
% ----------------------------------------------------------------------------

\usepackage[polish, english]{babel}


\title{Język polski \ \ \ \ \ \ w erze \ \ \ \ \ \ cyfrowej --- The Polish Language in the Digital Age}

% Title for the spine of the cover
\spineTitle{The Polish Language in the Digital Age --- Język polski w erze cyfrowej}

\subtitle{White Paper Series --- Seria raportów}

\author{
  Marcin Miłkowski
 }

\authoraffiliation{
  Marcin~Miłkowski\\ {\small Instytut Podstaw Informatyki PAN}
 }

\editors{
  Georg Rehm, Hans Uszkoreit\\ (redakcja, \textcolor{grey1}{editors}) 
}

% Text in left column on backside of the cover
\SpineLText{\selectlanguage{english}%
\small{%
  In everyday communication, Europe’s citizens, business partners and politicians are inevitably confronted with language barriers.  
  Language technology has the potential to overcome these barriers and to provide innovative interfaces to technologies and knowledge. 
  This white paper presents the state of language technology support for the Polish language. 
  It is part of a series that analyzes the available language resources and technologies for 30~European languages. 
  The analysis was carried out by META-NET, a Network of Excellence funded by the European Commission.
  META-NET consists of 54 research centres in 33 countries, who cooperate with stakeholders from economy, government agencies, research organisations and others.
 %, non-governmental organisations, language communities and European universities. 
  META-NET’s vision is high-quality language technology for all European languages. 
}
}

% Text in right column on backside of the cover
\SpineRText{%
  \selectlanguage{polish}%
  \small{W życiu codziennym europejscy obywatele, przedsiębiorcy i politycy nieuchronnie napotykają bariery językowe. Technologie językowe dają możliwość pokonania tych barier i mogą posłużyć do stworzenia innowacyjnych interfejsów umożliwiających obsługę urządzeń i dostęp do wiedzy. Niniejszy raport przedstawia poziom technologii językowych w języku polskim. Należy do serii, która analizuje dostępne zasoby językowe i technologie w 31 językach europejskich. Analiza ta została przeprowadzona przez META-NET, sieć doskonałości finansowaną przez Komisję Europejską.  Na META-NET składają się 54 ośrodki naukowe w 33 krajach, które współpracują z podmiotami gospodarczymi, agencjami rządowymi, instytucjami badawczymi i innymi. Wizją sieci META-NET jest tworzenie wysokiej jakości technologii językowych dla wszystkich języków europejskich.}
}
\vspace{-5mm}

% Quotes from VIPs on backside of the cover
\quotes{
\selectlanguage{english}“I consider the scientific challenges that linguistic engineering faces as extremely promising and intellectually attractive research area, where apparently different methodologies and tools of linguistics and computer science meet. Language technologies, as a result of this research, will have a growing influence on capabilities and communication models of the contemporary world as well as on the way human natural languages, such as the Polish language, take part in this process. The text data analysis, speech synthesis and speech recognition, machine translation and text summarisation are more and more present in our everyday life. For their presence to be rational and functional, for it to serve the needs of the economy, as well as the social and cultural life well, further large-scale work in this area is needed.“ \\ \textcolor{grey2}{--- Prof. Michał Kleiber (President of the Polish Academy of Sciences)}
}

% Funding notice left column
\FundingLNotice{\selectlanguage{polish}
  Autor tego opracowania dziękuje autorom raportu dotyczącego języka niemieckiego za zgodę na wykorzystanie materiałów niezależnych od języka \cite{lwpgerman}.

  \medskip
  Przekład na język polski: Anna Cichosz.

  \medskip
  Opracowanie niniejszego raportu zostało sfinansowane w ramach siódmego programu ramowego oraz programu na rzecz wspierania polityki w zakresie technologii informacyjnych i komunikacyjnych Komisji Europejskiej w ramach umów T4ME (grant 249\,119), CESAR (grant 271\,022), METANET4U (grant 270\,893) i META-NORD (grant 270\,899).
}

% Funding notice right column
\FundingRNotice{\selectlanguage{english}
  The author of this document is grateful to the authors of the White Paper on German for permission to re-use selected language-independent materials from their document \cite{lwpgerman}.

  \medskip
  Polish translation: Anna Cichosz

  \medskip
  The development of this white paper has been funded by the Seventh Framework Programme and the ICT Policy Support Programme of the European Commission under the contracts T4ME (Grant Agreement 249\,119), CESAR (Grant Agreement 271\,022), META\-NET4U (Grant Agreement 270\,893) and META-NORD (Grant Agreement 270\,899).
}
