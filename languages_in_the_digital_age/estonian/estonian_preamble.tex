%                                     MMMMMMMMM                                         
%                                                                             
%  MMO    MM   MMMMMM  MMMMMMM   MM    MMMMMMMM   MMD   MM  MMMMMMM MMMMMMM   
%  MMM   MMM   MM        MM     ?MMM              MMM$  MM  MM         MM     
%  MMMM 7MMM   MM        MM     MM8M    MMMMMMM   MMMMD MM  MM         MM     
%  MM MMMMMM   MMMMMM    MM    MM  MM             MM MMDMM  MMMMMM     MM     
%  MM  MM MM   MM        MM    MMMMMM             MM  MMMM  MM         MM     
%  MM     MM   MMMMMM    MM   MM    MM            MM   MMM  MMMMMMM    MM
%
%
%          - META-NET Language Whitepaper | Estonian Metadata -
% 
% ----------------------------------------------------------------------------

\usepackage{polyglossia}
\setotherlanguages{estonian,german,english}


\title{Eesti \ \ \ \ \ \ \ \ \ \ \ \ \ \ keel digiajastul --- The Estonian Language in the Digital Age}

\subtitle{White Paper Series --- Valge raamatu sari}

\author{
  Krista Liin\\
  Kadri Muischnek\\
  Kaili Müürisep\\
  Kadri Vider
}
\authoraffiliation{
  Krista Liin~ {\small Tartu Ülikool}\\
  Kadri Muischnek~ {\small Tartu Ülikool}\\
  Kaili Müürisep~ {\small Tartu Ülikool}\\
  Kadri Vider~ {\small Tartu Ülikool}
}
\editors{
  Georg Rehm, Hans Uszkoreit\\(toimetajad, \textcolor{grey1}{editors})
}

% Text in left column on backside of the cover
\SpineLText{\selectlanguage{english}%
  In everyday communication, Europe’s citizens, business partners and politicians are inevitably confronted with language barriers.  
  Language technology has the potential to overcome these barriers and to provide innovative interfaces to technologies and knowledge. 
  This white paper presents the state of language technology support for the German language. 
  It is part of a series that analyzes the available language resources and technologies for 31~European languages. 
  The analysis was carried out by META-NET, a Network of Excellence funded by the European Commission.
  META-NET consists of 54 research centres in 33 countries, who cooperate with stakeholders from economy, government agencies, research organisations, non-governmental organisations, language communities and European universities. 
  META-NET’s vision is high-quality language technology for all European languages. 
}

% Text in right column on backside of the cover
\SpineRText{\selectlanguage{estonian}%
Oma igapäevases suhtluses puutuvad Euroopa kodanikud, äripartnerid ja poliitikud paratamatult kokku keelebarjääridega. Keeletehnoloogia suudab neid barjääre ületada ning võimaldada uudseid liideseid tehnoloogiatele ja teadmistele. Käesolev valge raamat esitab eesti keele keeletehnoloogilise toe hetkeseisu, moodustades osa seeriast, mis analüüsib 31 Euroopa keele olemasolevaid keeleressursse ja -tehnoloogiaid. Selle analüüsi viis läbi Euroopa Komisjoni rahastatud tippteadmiste võrgustik META-NET. META-NET koosneb 33 riigi 54 uurimiskeskusest, mis teevad koostööd ärimaailma, valitsus- ja teadusasutuste, valitsusväliste organisatsioonide, keelekogukondade ja Euroopa ülikoolide huvitatud osapooltega. META-NETi eesmärk on saavutada kõrgekvaliteediline keeletehnoloogia kõigile Euroopa keeltele. }

% Quotes from VIPs on backside of the cover
\quotes{
%  Kui me ei täida keeletehnoloogia arengukava ega tegutse koostöös teiste riikidega samas suunas, ei toimi eesti keel tulevikus selles keskkonnas ja marginaliseerub infoühiskonnas. \\
 % \textcolor{grey2}{--- Eesti keele arengukava 2011-2017}\\[3mm]
  "If we do not implement the development plan for language  technology or do not cooperate with other countries in the same direction, in future Estonian will [...]
%not function in this environment and will 
be marginalized in information society." \\
  \textcolor{grey2}{--- Development Plan of the Estonian Language 2011--2017}\\[3mm]
"Kas eesti keelt on võimalik kasutada kõige moodsamas tehnoloogias ja kas arvuti on võimeline suhtlema kujundlikus eesti keeles --- see määrab meie keele säilimise tuleviku maailmas." \\
 \textcolor{grey2}{--- Tõnis Lukas, haridus- ja teadusminister 2007--2011, EV90 raames korraldatud Keeletalgutel}
}

% Funding notice left column
\FundingLNotice{\selectlanguage{estonian}\vskip2mm
  Selle dokumendi autorid tänavad saksa keele valge raamatu autoreid loa eest kasutada nende väljaandes sisaldunud keelest sõltumatuid materjale. \cite{lwpgerman}.

  \bigskip
  
  Selle keeleraporti koostamist rahastas 7. raamprogramm ja Euroopa Komisjoni IKT poliitika toetusprogramm lepingute T4ME (toetusleping  249119), CESAR (toetusleping 271022), METANET4U (toetusleping 270893) ja META-NORD (toetusleping 270899) kaudu.}

% Funding notice right column
\FundingRNotice{\selectlanguage{english}\vskip2mm
  The authors of this document
  are grateful to the authors of the White Paper on German for
  permission to re-use selected language-independent materials from
  their document. \cite{lwpgerman}.
  
  \bigskip
  
  The development of this White Paper has been funded by the Seventh
  Framework Programme and the ICT Policy Support Programme of the
  European Commission under the contracts T4ME (Grant Agreement
  249119), CESAR (Grant Agreement 271022), METANET4U (Grant Agreement
  270893) and META-NORD (Grant Agreement 270899).}
