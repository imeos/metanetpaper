%                                     MMMMMMMMM                                         
%                                                                             
%  MMO    MM   MMMMMM  MMMMMMM   MM    MMMMMMMM   MMD   MM  MMMMMMM MMMMMMM   
%  MMM   MMM   MM        MM     ?MMM              MMM$  MM  MM         MM     
%  MMMM 7MMM   MM        MM     MM8M    MMMMMMM   MMMMD MM  MM         MM     
%  MM MMMMMM   MMMMMM    MM    MM  MM             MM MMDMM  MMMMMM     MM     
%  MM  MM MM   MM        MM    MMMMMM             MM  MMMM  MM         MM     
%  MM     MM   MMMMMM    MM   MM    MM            MM   MMM  MMMMMMM    MM
%
%
%          - META-NET Language Whitepaper | Portuguese Metadata -
% 
% ----------------------------------------------------------------------------

\usepackage{polyglossia}
\setotherlanguages{portuguese,english}


\title{A  língua \mbox{portuguesa} na era digital --- The \mbox{Portuguese} Language in the Digital Age}

\subtitle{White Paper Series --- Coleção Livros Brancos}

\author{
  António Branco \\
  Amália Mendes \\
  Sílvia Pereira \\
  Paulo Henriques \\
  Thomas Pellegrini \\
  Hugo Meinedo \\
  Isabel Trancoso \\
  Paulo Quaresma \\
  Vera Lúcia Strube de Lima\\
  Fernanda Bacelar
}
\authoraffiliation{
  António Branco~ {\small Universidade de Lisboa}\\
  Amália Mendes~ {\small CLUL / Universidade de Lisboa}\\
  Sílvia Pereira~ {\small Universidade de Lisboa}\\
  Paulo Henriques~ {\small CLUL / Universidade de Lisboa} \\
  Thomas Pellegrini~ {\small INESC-ID}\\
  Hugo Meinedo~ {\small INESC-ID}\\
  Isabel Trancoso~ {\small INESC-ID / IST}\\
  Paulo Quaresma~ {\small Universidade de Évora} \\
  Vera Lúcia Strube de Lima~ {\small Pontifícia Universidade Católica do Rio Grande do Sul}\\
  Fernanda Bacelar~ {\small CLUL / Universidade de Lisboa}
}
\editors{
  Georg Rehm, Hans Uszkoreit\\(organizadores, \textcolor{grey1}{editors})
}

% Text in left column on backside of the cover
\SpineLText{\selectlanguage{english}%
  In everyday communication, Europe’s citizens, business partners and politicians are inevitably confronted with language barriers.  
  Language technology has the potential to overcome these barriers and to provide innovative interfaces to technologies and knowledge. 
  This white paper presents the state of language technology support for the German language. 
  It is part of a series that analyzes the available language resources and technologies for 31~European languages. 
  The analysis was carried out by META-NET, a Network of Excellence funded by the European Commission.
  META-NET consists of 54 research centres in 33 countries, who cooperate with stakeholders from economy, government agencies, research organisations, non-governmental organisations, language communities and European universities. 
  META-NET’s vision is high-quality language technology for all European languages. 
}

% Text in right column on backside of the cover
\SpineRText{\selectlanguage{portuguese}%
  Na sua comunicação diária, os cidadãos europeus, os parceiros económicos e os responsáveis políticos são inevitavelmente confrontados com barreiras linguísticas. A Tecnologia da Linguagem tem potencial para ultrapassar essas barreiras e para fornecer interfaces inovadoras para as tecnologias e para o conhecimento. Este Livro Branco apresenta o estado da Tecnologia da Linguagem para a língua portuguesa. Faz parte de uma colecção que analisa os recursos linguísticos e as tecnologias disponíveis para 31 línguas europeias. Esta análise foi realizada no âmbito da META-NET, uma Rede de Excelência fundada pela Comissão Europeia. A META-NET é constituída por 54 centros de investigação em 33 países, que colaboram com empresas, indústria, instituições estatais, instituições não-governamentais. comunidades linguísticas e universidades europeias. A META-NET tem como visão estratégica a disponibilização de Tecnologia da Linguagem de alta qualidade para todas as línguas europeias.}

% Quotes from VIPs on backside of the cover
\quotes{%
  Excepteur sint occaecat cupidatat non proident, sunt in culpa qui officia deserunt mollit anim id est laborum. Lorem ipsum dolor sit amet, consectetur adipisicing elit, sed do eiusmod tempor labore et dolore magna aliqua. \\
  \textcolor{grey2}{--- Prof. Dr. John Doe (Member of the European Parliament and VIP)}\\[3mm]
  Excepteur sint occaecat cupidatat non proident, sunt in culpa qui officia deserunt mollit anim id est laborum. Lorem ipsum dolor sit amet, consectetur adipisicing elit, sed do eiusmod tempor labore et dolore magna aliqua. \\
  \textcolor{grey2}{--- Dr. Jane Doe (Member of the European Parliament and VIP)}
}

% Funding notice left column
\FundingLNotice{\selectlanguage{portuguese}\vskip2mm
  Os autores deste documento agradecem aos autores do Livro Branco sobre o alemão por permitirem a utilização de partes seleccionadas do seu texto original. \cite{lwpgerman}.

  \bigskip
  
  A realização deste livro branco foi financiada pelo 7º Programa-Quadro e pelo Programa de Apoio à Política das TIC (ICT PSP) da Comunidade Europeia no âmbito dos contratos T4ME (Acordo de Financiamento 249119), CESAR (Acordo de Financiamento 271022), METANET4U (Acordo de Financiamento 270893) e METANORD (Acordo de Financiamento 270899).}

% Funding notice right column
\FundingRNotice{\selectlanguage{english}\vskip2mm
  The authors of this document
  are grateful to the authors of the White Paper on German for
  permission to re-use selected language-independent materials from
  their document \cite{lwpgerman}.
  
  \bigskip
  
  The development of this white paper has been funded by the Seventh
  Framework Programme and the ICT Policy Support Programme of the
  European Commission under the contracts T4ME (Grant Agreement
  249119), CESAR (Grant Agreement 271022), METANET4U (Grant Agreement
  270893) and META-NORD (Grant Agreement 270899).}
