%                                     MMMMMMMMM                                         
%                                                                             
%  MMO    MM   MMMMMM  MMMMMMM   MM    MMMMMMMM   MMD   MM  MMMMMMM MMMMMMM   
%  MMM   MMM   MM        MM     ?MMM              MMM$  MM  MM         MM     
%  MMMM 7MMM   MM        MM     MM8M    MMMMMMM   MMMMD MM  MM         MM     
%  MM MMMMMM   MMMMMM    MM    MM  MM             MM MMDMM  MMMMMM     MM     
%  MM  MM MM   MM        MM    MMMMMM             MM  MMMM  MM         MM     
%  MM     MM   MMMMMM    MM   MM    MM            MM   MMM  MMMMMMM    MM
%
%
%          - META-NET Language Whitepaper | Portuguese Metadata -
% 
% ----------------------------------------------------------------------------

\usepackage{polyglossia}
\setotherlanguages{portuguese,english}

\hyphenation{rá-pi-da con-ti-nu-am ino-va-do-ra idio-mas exem-plo re-co-lher De-vi-do fo-no-ló-gi-ca ge-ne-ra-li-za-da coo-pe-ra-ção em-pe-nha-do Edu-ar-do bi-blio-te-cas di-fe-ren-te exem-plo do-cu-men-ta-ção ou-tros ve-ri-fi-ca-ção atu-al-men-te apre-sen-ta-ção tor-nan-do-se au-to-ma-ti-za-das rea-li-za-da es-ta-be-le-ci-dos apre-sen-ta-rem si-mi-lar di-fe-ren-tes diá-lo-go usa-do re-co-nhe-ci-men-to re-pa-re-se me-lhor qua-li-da-de es-ta-be-le-ci-men-to res-pos-ta fun-cio-na-li-da-des cons-tru-ir ge-ra-ção res-pos-ta en-con-tram--se en-con-tran-do--se re-co-nhe-ci-men-to em-pre-sa-ri-al E-xis-te Mi-nis-té-rio re-pre-sen-ta-ti-vo cons-tru-ção res-pos-ta atual-men-te li-ga-das par-ticu-lar ac-cele-rate pri-ma-ri-ly exem-plos co-nhe-ci-men-to ins-ti-tui-ção pers-pec-ti-vas es-pe-cia-li-za-do rea-li-za-do sa-tis-fa-tó-ri-os ela-bo-ra-da re-pre-sen-ta-ção fa-la-da dis-po-si-ti-vos na-tu-ral cir-cuns-cri-tas con-so-li-da-ção es-pe-cia-li-za-dos mo-de-lo me-lho-res re-gis-tam de-sem-pe-nhar exi-ge Fa-la-da des-cri-ção ou-tras le-va-do des-co-nec-ta-do cons-truí-dos res-pei-to co-nhe-ci-men-to pa-ra-le-los va-rie-da-de gra-du-al-men-te te-cno-lo-gia}


\title{A \ \ \ \ \ \ \ \ \ \ \ \ \ \ língua \mbox{portuguesa} na era \ \ \ \ \ digital --- The \mbox{Portuguese} Language in the Digital Age}

\spineTitle{The Portuguese Language in the Digital Age --- A língua portuguesa na era digital}

\subtitle{White Paper Series --- Coleção Livros Brancos}

\author{
  António Branco \\
  Amália Mendes \\
  Sílvia Pereira \\
  Paulo Henriques \\
  Thomas Pellegrini \\
  Hugo Meinedo \\
  Isabel Trancoso \\
  Paulo Quaresma \\
  Vera Lúcia Strube de Lima\\
  Fernanda Bacelar
}
\authoraffiliation{
  António Branco~ {\small Universidade de Lisboa}\newline
  Amália Mendes~ {\small CLUL, Universidade de Lisboa}\newline
  Sílvia Pereira~ {\small Universidade de Lisboa}\newline
  Paulo Henriques~ {\small CLUL, Universidade de Lisboa}\newline
  Thomas Pellegrini~ {\small INESC-ID}\newline
  Hugo Meinedo~ {\small INESC-ID}\newline
  Isabel Trancoso~ {\small INESC-ID, IST}\newline
  Paulo Quaresma~ {\small Universidade de Évora}\newline
  Vera Lúcia Strube de Lima~ {\small PUCRS}\newline
  Fernanda~Bacelar~ {\small CLUL,~Universidade~de~Lisboa}
}
\editors{
  Georg Rehm, Hans Uszkoreit\\(organizadores, \textcolor{grey1}{editors})
}

% Text in left column on backside of the cover
\SpineLText{\selectlanguage{english}%
\vspace{-30mm}
}

% Text in right column on backside of the cover
\SpineRText{\selectlanguage{portuguese}%
\vspace{-30mm}
}

% Quotes from VIPs on backside of the cover
\quotes{
  \begin{spacing}{1}
  \small
  Este livro contém uma excelente panorâmica da área das tecnologias da linguagem com ênfase no tratamento do português. Embora escrito em termos acessíveis ao grande público, os conceitos mais técnicos são descritos com o rigor adequado, como seria de esperar de um grupo de autores que inclui os investigadores desta área em Portugal com maior reconhecimento internacional. Um livro a ler por quem queira compreender a importância desta área. \\
  \textcolor{grey2}{--- Prof. Doutor Miguel Filgueiras, Professor Catedrático aposentado (Universidade do Porto)}\\[3mm]
  This book presents an overview of the language technology area with a focus on the Portuguese language. Although written for a non-technical audience, the presentation is sound, what comes as no surprise from a set of authors where the most internationally recognized researchers in this area in Portugal are to be found. This is a must-read book for anyone wishing to understand the importance of this area. \\
  \textcolor{grey2}{--- Prof. Doutor Miguel Filgueiras, Emeritus Professor (University of Oporto)}\\[5mm]% Dra. Daniela Braga
  O processamento das línguas faladas e escritas é uma área fundamental para as novas modalidades de interação natural homem-máquina. Este livro consegue,
  de uma forma acessível mas científica e rigorosa, apresentar o estado da arte do processamento do português na era digital, uma das línguas com mais rápida expansão e importância económico-tecnológica do mundo ocidental. \\
  \textcolor{grey2}{--- Dra. Daniela Braga, International Program Manager (Microsoft, Redmond WA, EUA)}\\[3mm]
  The processing of written and spoken languages is a crucial area for the new modalities of human-computer natural interaction. In an accessible yet scientific and rigorous way, this book presents the state of the art in the digital age of the computational processing of the Portuguese language, one of the languages with more rapid expansion and more economic-technological importance in the western world. \\
  \textcolor{grey2}{--- Dra. Daniela Braga, International Program Manager (Microsoft, Redmond WA, USA)}\\[5mm]% Dr. Pedro Passos Coelho
  É da maior importância a investigação realizada na área da tecnologia
  da linguagem para a consolidação do português como língua de comunicação
  global na sociedade da informação. \\
  \textcolor{grey2}{--- Dr. Pedro Passos Coelho, Primeiro-Ministro de Portugal}\\[3mm]
  The research carried out in the area of language technology is of utmost
  importance for the consolidation of Portuguese as a language of global
  communication in the information society. \\
  \textcolor{grey2}{--- Dr. Pedro Passos Coelho, Prime-Minister of Portugal}
  \end{spacing}
}

% Funding notice left column
\FundingLNotice{\selectlanguage{portuguese}\vskip2mm
  Os autores deste documento agradecem aos autores do Livro Branco sobre o alemão por permitirem a utilização de partes seleccionadas do seu texto original \cite{lwpgerman}.

  \bigskip
  
  A realização deste Livro Branco foi financiada pelo 7º Programa-Quadro e pelo Programa de Apoio à Política das TIC (ICT PSP) da Comunidade Europeia no âmbito dos contratos T4ME (Acordo de Financiamento 249119), CESAR (Acordo de Financiamento 271022), METANET4U (Acordo de Financiamento 270893) e META-NORD (Acordo de Financiamento 270899).}

% Funding notice right column
\FundingRNotice{\selectlanguage{english}\vskip2mm
  The authors of this document
  are grateful to the authors of the White Paper on German for
  permission to re-use selected language-independent materials from
  their document \cite{lwpgerman}.
  
  \bigskip
  
  The development of this White Paper has been funded by the Seventh
  Framework Programme and the ICT Policy Support Programme of the
  European Commission under the contracts T4ME (Grant Agreement
  249119), CESAR (Grant Agreement 271022), METANET4U (Grant Agreement
  270893) and META-NORD (Grant Agreement 270899).}
