%                                     MMMMMMMMM                                         
%                                                                             
%  MMO    MM   MMMMMM  MMMMMMM   MM    MMMMMMMM   MMD   MM  MMMMMMM MMMMMMM   
%  MMM   MMM   MM        MM     ?MMM              MMM$  MM  MM         MM     
%  MMMM 7MMM   MM        MM     MM8M    MMMMMMM   MMMMD MM  MM         MM     
%  MM MMMMMM   MMMMMM    MM    MM  MM             MM MMDMM  MMMMMM     MM     
%  MM  MM MM   MM        MM    MMMMMM             MM  MMMM  MM         MM     
%  MM     MM   MMMMMM    MM   MM    MM            MM   MMM  MMMMMMM    MM
%
%
%          - META-NET Language Whitepaper | Irish Metadata -
% 
% ----------------------------------------------------------------------------

\usepackage{polyglossia}
\setotherlanguages{irish,english}


\title{An Ghaeilge sa Ré Dhigiteach --- The Irish Language in the Digital Age}

\subtitle{White Paper Series --- Sraith Páipéar Bán }

\author{
  John Judge\\
  Ailbhe Ní Chasaide\\
  Rose Ní Dhubhda\\
  Kevin P. Scannell\\
  Elaine Uí Dhonnchadha
}
\authoraffiliation{
  John Judge~ {\small CNGL}\\
  Ailbhe Ní Chasaide~ {\small TCD}\\
  Rose Ní Dhubhda~ {\small OÉ Gallimh}\\
  Kevin P. Scannell~ {\small Saint Louis University}\\
  Elaine Uí Dhonnchadha~ {\small TCD}
}
\editors{
  Georg Rehm, Hans Uszkoreit\\(eagarthóirí, \textcolor{grey1}{editors})
}

% Text in left column on backside of the cover
\SpineLText{\selectlanguage{english}%
  In everyday communication, Europe’s citizens, business partners and politicians are inevitably confronted with language barriers.  
  Language technology has the potential to overcome these barriers and to provide innovative interfaces to technologies and knowledge. 
  This white paper presents the state of language technology support for the German language. 
  It is part of a series that analyzes the available language resources and technologies for 31~European languages. 
  The analysis was carried out by META-NET, a Network of Excellence funded by the European Commission.
  META-NET consists of 54 research centres in 33 countries, who cooperate with stakeholders from economy, government agencies, research organisations, non-governmental organisations, language communities and European universities. 
  META-NET’s vision is high-quality language technology for all European languages. 
}

% Text in right column on backside of the cover
\SpineRText{\selectlanguage{irish}%
I gcumarsáid laethúil, bíonn baic theanga le sárú ag saoránaigh, comhpháirtithe gnó agus polaiteoirí na hEorpa. Tá sé de chumas ag teicneolaíocht teanga na baic seo a shárú agus comhéadain nuálaíocha a chur ar fáil do theicneolaíochtaí agus faisnéis. Cuirtear i láthair sa pháipéar bán seo staid thacaíocht na teicneolaíochta teanga don Ghaeilge. Tá sé mar chuid de shraith a dhéanann anailís ar na hacmhainní agus teicneolaíochtaí teanga atá ar fáil don 31 teanga Eorpacha. Rinne META-NET, Gréasán Sármhaitheasa arna mhaoiniú ag Coimisiún na hEorpa, an anailís. Tá 54 ionad taighde i 33 tír i META-NET, a chomhoibríonn le geallsealbhóirí ón ngeilleagar, ó ghníomhaireachtaí rialtais, ó eagraíochtaí taighde, ó eagraíochtaí neamhrialtasacha, ó phobail teanga agus ó ollscoileanna Eorpacha. Is é atá mar fhís ag META-NET teicneolaíocht teanga d'ardchaighdeán a bheith ar fáil do theangacha ar fad na hEorpa.}

% Quotes from VIPs on backside of the cover
\quotes{%
"Language technology is no longer a luxury for most European languages - it is now essential to their survival as viable means of expression across the whole range of areas from business to the arts, and this is as much the case for Irish as any other European language." \\
  \textcolor{grey2}{--- Ferdie Mac an Fhailigh (CEO, Foras na Gaeilge)}\\[3mm]
  "Bheadh sé dochreidte dá bhféadfaí ardán ilteangach a fhorbairt, a éascódh úsáid gach teanga Eorpach, trí úsáid a bhaint as teicneolaíocht." \\
  \textcolor{grey2}{--- Julian De Spainn (Ardrunaí, Conradh na Gaeilge)}
}

% Funding notice left column
\FundingLNotice{\selectlanguage{irish}
  Tá údair na cáipéise seo buíoch d'údair an Pháipéir Bháin i nGearmáinis as cead a thabhairt ábhair neamhspleách ó theanga óna gcáipéis a athúsáid \cite{lwpgerman}. \\
  \bigskip
  Mhaoinigh an Seachtú Creatchlár agus Clár Tacaíochta Beartais TFC
  de Choimisiún na hEorpa faoi na conarthaí T4ME
  (Comhaontú Deontais 249119), CESAR (Comhaontú Deontais 271022),
  METANET4U (Comhaontú Deontais 270893) agus
  META-NORD (Comhaontú Deontais 270899) forbairt an pháipéir bháin seo.
}

% Funding notice right column
\FundingRNotice{\selectlanguage{english}
  The authors of this document are grateful to the authors of the white paper on German for permission to re-use selected language-independent materials from their document \cite{lwpgerman}. \\
  \bigskip
  The development of this white paper has been funded by the Seventh
  Framework Programme and the ICT Policy Support Programme of the
  European Commission under the contracts T4ME (Grant Agreement
  249119), CESAR (Grant Agreement 271022), METANET4U (Grant Agreement
  270893) and META-NORD (Grant Agreement 270899).}
