%                                     MMMMMMMMM                                         
%                                                                             
%  MMO    MM   MMMMMM  MMMMMMM   MM    MMMMMMMM   MMD   MM  MMMMMMM MMMMMMM   
%  MMM   MMM   MM        MM     ?MMM              MMM$  MM  MM         MM     
%  MMMM 7MMM   MM        MM     MM8M    MMMMMMM   MMMMD MM  MM         MM     
%  MM MMMMMM   MMMMMM    MM    MM  MM             MM MMDMM  MMMMMM     MM     
%  MM  MM MM   MM        MM    MMMMMM             MM  MMMM  MM         MM     
%  MM     MM   MMMMMM    MM   MM    MM            MM   MMM  MMMMMMM    MM
%
%
%          - META-NET Language Whitepaper | Swedish Metadata -
% 
% ----------------------------------------------------------------------------

\usepackage{polyglossia}
\setotherlanguages{swedish,english}

\hyphenation{språk-tek-no-logi språk-tek-no-login in-forma-tions-sam-hälle in-forma-tions-sam-hället maskin-över-sätt-ning maskin-över-sätt-nings-system doku-ment-pro-duk-tion dok-ment-pro-duk-tion-en vitt-för-grenat reaktor-haveriet på-verka på-verkas upp-skatt-at under-visning under-gång person-datorer fort-farande in-flyt-ande be-segrade science sciences theory processor center}


\title{Svenska språket i den digitala tidsåldern --- The Swedish Language in the Digital Age}

% Title for the spine of the cover
\spineTitle{The Swedish Language in the Digital Age --- Svenska språket i den digitala tidsåldern}

\subtitle{White Paper Series --- Vitböcker}

\author{
  Lars Borin \\
  Martha D. Brandt \\
  Jens Edlund \\
  Jonas Lindh \\
  Mikael Parkvall
}
\authoraffiliation{
  Lars Borin~ \\
  {\small Språkbanken, Göteborgs universitet} \\
  Martha D. Brandt~ \\
  {\small Språkbanken, Göteborgs universitet} \\
  Jens Edlund~ \\
  {\small Kungliga Tekniska högskolan} \\
  Jonas Lindh~ \\
  {\small Språkbanken, Göteborgs universitet} \\
  Mikael Parkvall~ \\
  {\small Stockholms universitet}  
}
\editors{
  Georg Rehm, Hans Uszkoreit\\(utgivare, \textcolor{grey1}{editors})
}

% Text in left column on backside of the cover
\SpineLText{\selectlanguage{english}%
  In everyday communication, Europe’s citizens, business partners and politicians are inevitably confronted with language barriers.  
  Language technology has the potential to overcome these barriers and to provide innovative interfaces to technologies and knowledge. 
  This white paper presents the state of language technology support for the Swedish language. 
  It is part of a series that analyzes the available language resources and technologies for 30~European languages. 
  The analysis was carried out by META-NET, a Network of Excellence funded by the European Commission.
  META-NET consists of 54 research centres in 33 countries, who cooperate with stakeholders from economy, government agencies, research organisations and others. %, non-governmental organisations, language communities and European universities. 
  META-NET’s vision is high-quality language technology for all European languages. 
}

% Text in right column on backside of the cover
\SpineRText{\selectlanguage{swedish}%
  Europas medborgare, affärsmän och politiker stöter i sin vardag ständigt och oundvikligen på språkhinder. Språkteknologi kan övervinna dessa hinder och även tillhandahålla nydanande gränsytor mot teknologi och kunskap. I denna vitbok redovisas i vilken omfattning språkteknologi och språkverktyg finns för svenska. Den ingår i en serie vitböcker med aktuella analyser av läget beträffande språkresurser och språkteknologi för 30 av Europas språk. Analyserna är utförda av META-NET, ett EU-finansierat forskningssamarbete. META-NET består av 54 forskningscentra i 33 länder, som samarbetar med företrädare för industri, offentlig sektor, forskningsorganisationer, ideella och internationella organisationer, språkgemenskaper och europeiska universitet. META-NETs vision är att åstadkomma högkvalitativ språkteknologi för alla Europas språk.
}

% Quotes from VIPs on backside of the cover
\quotes{
  \selectlanguage{swedish}%
  ``Högkvalitativ språkteknologi är kanske det mest effektiva medlet för att bevara Europas språkliga mångfald. Att alla språk ska kunna användas fullt ut i det moderna samhällslivet är en demokratisk fråga. Här fyller META-NET en viktig, för att inte säga avgörande, funktion."\\
   \textcolor{grey2}{--- Lena Ekberg (chef för Språkrådet)}\\[3mm]
%  \selectlanguage{english}%
 % High-quality language technology may be the most effective means of preserving the linguistic diversity of Europe. Being able to use all languages fully in modern society is a question of democracy. In this connection META-NET fulfils a central, even crucial, function.\\
 %  \textcolor{grey2}{--- Lena Ekberg (head of the Swedish Language Council)}\\[3mm]
 % \selectlanguage{swedish}%
 % Boken ger en tydlig bild av var vi står i Europa idag när det gäller möjligheter att tackla utmaningar kring globalisering och språkbarriärer med hjälp av dagens och framtidens språkteknologi.\\
  % \textcolor{grey2}{--- Magnus Merkel (vd, Fodina Language Technology)}\\[3mm]
  \selectlanguage{english}%
  ``This book gives a clear account of the state of language technology in Europe and how to approach challenges for globalisation using current and future language technology solutions."\\
   \textcolor{grey2}{--- Magnus Merkel (CEO, Fodina Language Technology)}
}


% Funding notice left column
\FundingLNotice{\selectlanguage{swedish}
  Författarna vill uttrycka sin tacksamhet till den tyska vitbokens författare som givit sitt tillstånd till användning av valda delar av deras text. \cite{lwpgerman}. \vfill
  \bigskip
  Arbetet med denna vitbok har utförts med finansiering från EU:s sjunde ramprogram och ICT PSP, inom projekten T4ME (avtal 249119), CESAR (avtal 271022), METANET4U (avtal 270893) och META-NORD (avtal 270899).
}

% Funding notice right column
\FundingRNotice{\selectlanguage{english}
  The authors of this document are grateful to the authors of the White Paper on German for permission to re-use selected language-independent materials from their document \cite{lwpgerman}. \vfill
  \bigskip
  The development of this white paper has been funded by the Seventh Framework Programme and the ICT Policy Support Programme of the European Commission under the contracts T4ME (Grant Agreement 249119), CESAR (Grant Agreement 271022), META\-NET4U (Grant Agreement 270893) and META-NORD (Grant Agreement 270899).
}
