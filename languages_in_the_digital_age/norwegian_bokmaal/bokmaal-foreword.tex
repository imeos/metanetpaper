Dette dokumentet er del av en serie som skal fremme kunnskap om språkteknologiens status og potensiale. Målgruppen er journalister, politikere, språkbrukere, lærere og andre interesserte. Tilgjengeligheten og bruken av språkteknologi i Europa varierer fra språk til språk. Derfor vil også nødvendige tiltak for å støtte forskning og utvikling av språkteknologi være forskjellige for hvert språk. Hvilke tiltak som er nødvendige avhenger av flere faktorer, for eksempel kompleksiteten i et gitt språk og antall språkbrukere.

Forskningsnettverket META-NET, et \emph{Network of Excellence} finansiert av Europakommisjonen, presenterer i denne serien  (jf. s.~\pageref{whitepaperseries}) sin analyse av eksisterende språkressurser og teknologier for de 23 offisielle EU-språkene og andre nasjonale og regionale språk i Europa -- deriblant norsk. Resultatene av denne analysen tyder på at det er betydelige hull i forskning og utvikling for alle språkene. Denne detaljerte ekspertanalysen av den nåværende situasjonen i denne serien vil forhåpentlig bidra til å maksimere effekten av ny forskning.

Per november 2011 består META-NET av 54 forskningsinstitusjoner i 33 land (jf. s.~\pageref{metanetmembers}) som samarbeider med kommersielle aktører (IT-bedrifter, utviklere og brukere), offentlige etater, ikke-statlige organisasjoner, representanter for språksamfunn, og universiteter. I samarbeid med disse samfunnsrepresentantene er målet å skape en felles teknologivisjon og å utvikle en strategisk forskningsagenda for flerspråklighet i Europa innen år 2020.