%                                     MMMMMMMMM                                         
%                                                                             
%  MMO    MM   MMMMMM  MMMMMMM   MM    MMMMMMMM   MMD   MM  MMMMMMM MMMMMMM   
%  MMM   MMM   MM        MM     ?MMM              MMM$  MM  MM         MM     
%  MMMM 7MMM   MM        MM     MM8M    MMMMMMM   MMMMD MM  MM         MM     
%  MM MMMMMM   MMMMMM    MM    MM  MM             MM MMDMM  MMMMMM     MM     
%  MM  MM MM   MM        MM    MMMMMM             MM  MMMM  MM         MM     
%  MM     MM   MMMMMM    MM   MM    MM            MM   MMM  MMMMMMM    MM
%
%
%     - META-NET Language White Paper | Norwegian (nynorsk) Metadata -
% 
% ----------------------------------------------------------------------------

\usepackage{polyglossia}
\setotherlanguages{norsk,english}
\newcommand{\bokmaal}[1]{} % {#1} to activate bokmaal, {} to deactivate
\newcommand{\nynorsk}[1]{#1} % {#1} to activate nynorsk, {} to deactivate


%\title{Norsk i den digitale tidsalderen --- The Norwegian Language in the Digital Age}
\title{Norsk \ \ \ \ \ \ \ \ \ \ \ \ \ \ \ \  i den \ \ \ \ \ \ \ \  digitale tidsalderen --- The Norwegian Language in the Digital Age}

% Title for the spine of the cover
\spineTitle{The Norwegian Language in the Digital Age --- Norsk i den digitale tidsalderen}

% Version of a language e.g. bokmålsversjon or nynorskversjon
\languageVersion{nynorskversjon}

\subtitle{White Paper Series --- Kvitbokserie}

\author{
  Koenraad De Smedt\\
  Gunn Inger Lyse\\
  Anje Müller Gjesdal\\
  Gyri S. Losnegaard
}
\authoraffiliation{ 
  Koenraad De Smedt~ {\small UIB}\\
  Gunn Inger Lyse~ {\small UIB}\\
  Anje Müller Gjesdal~ {\small UIB}\\
  Gyri S. Losnegaard~ {\small UIB}
}
\editors{
  Georg Rehm, Hans Uszkoreit\\(Redaktørar, \textcolor{grey1}{editors})
}

% Text in left column on backside of the cover
\SpineLText{\selectlanguage{english}%
 In everyday communication, Europe’s citizens, business partners and politicians are inevitably confronted with language barriers.  
  Language technology has the potential to overcome these barriers and to provide innovative interfaces to technologies and knowledge. 
  This white paper presents the state of language technology support for the Norwegian language. 
  It is part of a series that analyses the available language resources and technologies for 30~European languages. 
  The analysis was carried out by META-NET, a Network of Excellence funded by the European Commission.
  META-NET consists of 54 research centres in 33 countries, who cooperate with stakeholders from economy, government agencies, research organisations, non-governmental organisations, language communities and European universities. 
  META-NET’s vision is high-quality language technology for all European languages.
}

% Text in right column on backside of the cover
\SpineRText{\selectlanguage{norsk}%
Det er uunngåeleg at innbyggjarane, næringsliv og politikarar i Europa støyter på språkbarrierar. Språkteknologi er eit verkemiddel for å motverke desse barrierane, og kan gje nyskapande grensesnitt for teknologi og kunnskap. Denne kviteboka gjev eit oversyn over situasjonen for språkteknologi for norsk. Han er del av ein serie som analyserer tilgjengelege språkressursar og verktøy for 30 europeiske språk. Analysen er utført av META-NET, eit forskingsnettverk (Network of Excellence) finansiert av EU-kommisjonen. META-NET består av 54 forskingsinstitusjonar i 33 land som samarbeider med ulike aktørar frå næringslivet, forvalting, forskingsmiljø, NGOar, språkbrukarar og universitet. META-NET sin visjon er å gjere språkteknologi av høg kvalitet tilgjengeleg for alle europeiske språk.
}

% Quotes from VIPs on backside of the cover
\quotes{``Skal man lage gode språkteknologiske løsninger for norsk, må det
eksistere språklige ressurser av høy kvalitet som industrien kan
benytte. Jeg håper at denne rapporten kan bidra til at slike ressurser
etableres raskt.''\\\textcolor{grey2}{--- Torbjørn Nordgård (Utviklingsdirektør Lingit AS)}\\[3mm]
``Skal vi kommunisere med maskinene rundt oss treng vi
språkteknologi. Denne rapporten presenterer status quo og vegen
framover for språkteknologi i Noreg.''\\ 
\textcolor{grey2}{--- Trond Trosterud (Professor Universitetet i Tromsø)}}

% Funding notice left column
\FundingLNotice{\selectlanguage{norsk}\vskip2mm Forfattarane av denne
rapporten takkar forfattarane av rapporten for tysk språk for løyve
til å gjenbruke utvalt språkuavhengig material frå dokumentet deira
\cite{lwpgerman}. Forfattarane takkar òg Gisle Andersen, Torbjørg
Breivik, Helge Dyvik, Kristin Hagen, Torbjørn Nordgård, Torbjørn
Svendsen og Trond Trosterud for verdifulle bidrag og kommentarar.

 \bigskip

Arbeidet med denne utgreiinga er finansiert av det sjuande rammeprogrammet og Den europeiske kommisjonens ICT Policy Support program, gjennom kontraktane T4ME (tildelingsavtale 249 119), Cesar (tildelingsavtale 271 022), METANET4U (tildelingsavtale 270 893) og META-NORD (tildelingsavtale 270 899).

}

% Funding notice right column
\FundingRNotice{\selectlanguage{english}\vskip2mm
  The authors of this document
  are grateful to the authors of the White Paper on German for
  permission to re-use selected language-independent materials from
  their document \cite{lwpgerman}.
  They also wish to thank Gisle Andersen, Torbjørg Breivik, Helge Dyvik, Kristin Hagen, Torbjørn Nordgård, Torbjørn Svendsen and Trond Trosterud for valuable contributions and comments.
  \bigskip

  
  The development of this White Paper has been funded by the Seventh
  Framework Programme and the ICT Policy Support Programme of the
  European Commission under the contracts T4ME (Grant Agreement
  249119), CESAR (Grant Agreement 271022), METANET4U (Grant Agreement
  270893) and META-NORD (Grant Agreement 270899).}
