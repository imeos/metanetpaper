Dette dokumentet er del av ein serie som skal fremje kunnskap om språkteknologiens status og potensiale. Målgruppa er journalistar, politikarar, språkbrukarar, lærarar og andre interesserte. Tilgangen til og nytta av språkteknologi i Europa varierer frå språk til språk. Difor vil òg naudsynte tiltak for å støtte forsking og utvikling av språkteknologi vere ulike for kvart språk. Kva for tiltak som er naudsynte, avheng av fleire faktorar, til dømes kompleksiteten i eit gjeve språk og mengda språkbrukarar.

Forskingsnettverket META-NET, eit \emph{Network of Excellence} finansiert av Europakommisjonen, presenterer i denne serien  (jf. s.~\pageref{whitepaperseries}) analysen sin av eksisterande språkressursar og teknologiar for dei 23 offisielle EU-språka og andre nasjonale og regionale språk i Europa -- mellom dei norsk. Resultata av denne analysen tyder på at det er betydelege hol i forsking og utvikling for alle språka. Denne detaljerte ekspertanalysen av den noverande situasjonen i denne serien vil vonleg bidra til å maksimere effekten av ny forsking.

Per november 2011 består META-NET av 54 forskingsinstitusjonar i 33 land (jf. s.~\pageref{metanetmembers}) som samarbeider med kommersielle aktørar (IT-føretak, utviklarar og brukarar), offentlege etatar, ikkje-statlege organisasjonar, representantar for språksamfunn, og universitet. I samarbeid med desse samfunnsrepresentantane er målet å skape ein felles teknologivisjon og å utvikle ein strategisk forskingsagenda for eit fleirspråkleg Europa innan år 2020.
