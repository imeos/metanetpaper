%                                     MMMMMMMMM                                         
%                                                                             
%  MMO    MM   MMMMMM  MMMMMMM   MM    MMMMMMMM   MMD   MM  MMMMMMM MMMMMMM   
%  MMM   MMM   MM        MM     ?MMM              MMM$  MM  MM         MM     
%  MMMM 7MMM   MM        MM     MM8M    MMMMMMM   MMMMD MM  MM         MM     
%  MM MMMMMM   MMMMMM    MM    MM  MM             MM MMDMM  MMMMMM     MM     
%  MM  MM MM   MM        MM    MMMMMM             MM  MMMM  MM         MM     
%  MM     MM   MMMMMM    MM   MM    MM            MM   MMM  MMMMMMM    MM
%
%
%          - META-NET Language Whitepaper | Serbian Metadata -
% 
% ----------------------------------------------------------------------------

\usepackage{polyglossia}
\setotherlanguages{serbian,english}


\title{Српски језик у дигиталном добу --- The Serbian Language in the Digital Age}

\spineTitle{The Serbian Language in the Digital Age --- Српски језик у дигиталном добу}

\subtitle{White Paper Series --- Серија белих књига}

\author{
Duško Vitas \\
Ljubomir Popović \\
Cvetana Krstev \\
Ivan Obradović \\
Gordana Pavlović-Lažetić \\
Mladen Stanojević
}
\authoraffiliation{
 Duško Vitas, {\small University of Belgrade} \\
 Ljubomir Popović, {\small University of Belgrade} \\
 Cvetana Krstev, {\small University of Belgrade} \\
 Ivan Obradović, {\small University of Belgrade} \\
 Gordana Pavlović-Lažetić, {\small University of Belgrade} \\
 Mladen Stanojević, {\small University of Belgrade}
}
\editors{
  Georg Rehm, Hans Uszkoreit\\ (urednici, \textcolor{grey1}{editors})
}

% Text in left column on backside of the cover
\SpineLText{\selectlanguage{english}%
  In everyday communication, Europe’s citizens, business partners and politicians are inevitably confronted with language barriers.  
  Language technology has the potential to overcome these barriers and to provide innovative interfaces to technologies and knowledge. 
  This white paper presents the state of language technology support for the Serbian language. 
  It is part of a series that analyses the available language resources and technologies for 30~European languages. 
  The analysis was carried out by META-NET, a Network of Excellence funded by the European Commission.
  META-NET consists of 54 research centres in 33 countries, who cooperate with stakeholders from economy, government agencies, research organisations, non-governmental organisations, language communities and European universities. 
  META-NET’s vision is high-quality language technology for all European languages. 
}

% Text in right column on backside of the cover
\SpineRText{\selectlanguage{serbian}%
 Грађани Европе, као и пословни свет и политичари се суочавају 
у својој свакодневној комуникацији са 
језичким препрекама. Оно што доносе језичке технологије је превазилажење таквих препрека и обезбеђивање 
нове сумеђе ка технологијама и знању уопште.
Ова бела књига описује актуелни ниво подршке језичких технологија у обради српског језика.
Она је део серије која анализира расположиве језичке ресурсе и технологије за 30~европски језик.
Анализа је спроведена у оквиру META-НЕТ-а, мреже изврсности коју је основала Европска комисија.
META-НЕТ повезује 54 истраживачка центра из 33 земље, који сарађују са заинтересованим странама
из економије, владе, истраживачких организација, невладиних организација, језичких заједница 
и универзитета. 
Визија META-НЕТ-а је језичка технологија високог квалитета за све европске језике. }

% Quotes from VIPs on backside of the cover
\quotes{
%  Lorem ipsum dolor sit amet, consectetur adipisicing elit, sed do eiusmod tempor incididunt ut labore et dolore magna aliqua. Ut enim ad minim veniam, quis nostrud exercitation ullamco laboris nisi ut aliquip ex ea commodo consequat. Duis aute irure dolor in reprehenderit in voluptate velit esse cillum dolore eu fugiat nulla pariatur. Excepteur sint occaecat cupidatat non proident, sunt in culpa qui officia deserunt mollit anim id est laborum.
}

% Funding notice left column
\FundingLNotice{\selectlanguage{serbian}\vskip2mm
Захваљујемо се ауторима беле књиге о немачком језику \cite{lwpgerman} што су дозволили
да језички независне делове њиховог текста користимо у овом раду. 

\bigskip

  Израду ове беле књиге финансирали су Седми оквирни програм (FP7) и Програм подршке политици информационо-комуникационих технологија Европске комисије преко уговора T4ME (Уговор о финансирању 249119), CESAR (Уговор о финансирању 271022), METANET4U (Уговор о финансирању 270893) и META-NORD (Уговор о финансирању 270899).}

% Funding notice right column
\FundingRNotice{\selectlanguage{english}\vskip2mm
  The authors of this document
  are grateful to the authors of the White Paper on German for
  permission to re-use selected language-independent materials from
  their document \cite{lwpgerman}.
  
  \bigskip
  
  The development of this White Paper has been funded by the Seventh
  Framework Programme and the ICT Policy Support Programme of the
  European Commission under the contracts T4ME (Grant Agreement
  249119), CESAR (Grant Agreement 271022), METANET4U (Grant Agreement
  270893) and META-NORD (Grant Agreement 270899).}
