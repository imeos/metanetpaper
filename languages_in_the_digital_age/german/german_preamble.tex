%                                     MMMMMMMMM        
%                                                                             
%  MMO    MM   MMMMMM  MMMMMMM   MM    MMMMMMMM   MMD   MM  MMMMMMM MMMMMMM   
%  MMM   MMM   MM        MM     ?MMM              MMM$  MM  MM         MM     
%  MMMM 7MMM   MM        MM     MM8M    MMMMMMM   MMMMD MM  MM         MM     
%  MM MMMMMM   MMMMMM    MM    MM  MM             MM MMDMM  MMMMMM     MM     
%  MM  MM MM   MM        MM    MMMMMM             MM  MMMM  MM         MM     
%  MM     MM   MMMMMM    MM   MM    MM            MM   MMM  MMMMMMM    MM
%
%
%          - META-NET Language Whitepaper | German Metadata -
% 
% ----------------------------------------------------------------------------

\usepackage{polyglossia}
\setotherlanguages{german,english}


\title{Die Deutsche Sprache im Digitalen Zeitalter --- The German Language in the Digital Age}

\spineTitle{The German Language in the Digital Age --- Die Deutsche Sprache im Digitalen Zeitalter}

\subtitle{White Paper Series --- Weißbuch-Serie}

\author{
  Aljoscha Burchardt \\
  Markus Egg \\
  Kathrin Eichler \\
  Brigitte Krenn \\
  Jörn Kreutel \\
  Annette Leßmöllmann \\
  Georg Rehm \\
  Manfred Stede \\
  Hans Uszkoreit \\
  Martin Volk
}
\authoraffiliation{
  Aljoscha Burchardt~ {\small DFKI}\\
  Markus Egg~ {\small Humboldt-Universität zu Berlin}\\
  Kathrin Eichler~ {\small DFKI} \\
  Brigitte Krenn~ {\small ÖFAI}\\
  Jörn Kreutel~ {\small FH Brandenburg}\\
  Annette Leßmöllmann~ {\small Hochschule Darmstadt}\\
  Georg Rehm~ {\small DFKI} \\
  Manfred Stede~ {\small Universität Potsdam}\\
  Hans Uszkoreit~ {\small Universität des Saarlandes, DFKI} \\
  Martin Volk~ {\small Universität Zürich}
}
\editors{
  Georg Rehm, Hans Uszkoreit\\(Herausgeber, \textcolor{grey1}{editors})
}

% Text in left column on backside of the cover
\SpineLText{\selectlanguage{english}%
  In everyday communication, Europe’s citizens, business partners and politicians are inevitably confronted with language barriers.  
  Language technology has the potential to overcome these barriers and to provide innovative interfaces to technologies and knowledge. 
  This white paper presents the state of language technology support for the German language. 
  It is part of a series that analyses the available language resources and technologies for 30~European languages. 
  The analysis was carried out by META-NET, a Network of Excellence funded by the European Commission.
  META-NET consists of 54 research centres in 33 countries, who cooperate with stakeholders from economy, government agencies, research organisations, non-governmental organisations, language communities and European universities. 
  META-NET’s vision is high-quality language technology for all European languages. 
}

% Text in right column on backside of the cover
\SpineRText{\selectlanguage{german}%
  Im kommunikativen Miteinander stoßen Europas Bürger, die europäische
  Wirtschaft und auch die Politik schnell an sprachliche
  Grenzen. Moderne Sprachtechnologie kann Sprachgrenzen überwinden und innovative Schnittstellen
  zu Technologien und Wissen ermöglichen. Als Teil einer Serie, die die vorhandenen
  Sprachressourcen und -technologien für 30~europäische Sprachen analysiert, stellt dieses Weißbuch 
  den Stand der sprachtechnologischen Unterstützung für die deutsche
  Sprache dar.
  % 
  Die Analyse wurde von META-NET erstellt, einem von der Europäischen
  Kommission geförderten Spitzenforschungsnetzwerk. META-NET besteht
  aus 54 Forschungszentren in 33 Ländern, die mit
  Interessensvertretern aus Wirtschaft, Verwaltung, NGOs,
  Sprachgemeinschaften und europäischen Universitäten
  zusammenarbeiten. Die Vision von META-NET ist hochqualitative
  Sprachtechnologie für alle Sprachen Europas.}

% Quotes from VIPs on backside of the cover
\quotes{\selectlanguage{english}%
``Only through innovative language technologies can the multilingualism of
Europe become an advantage in realising the global export market.
META-NET is the most important initiative to turn multilingualism into
economic benefits."
  \textcolor{grey2}{--- Prof.~Dr.~Dr.~h.\,c.~mult.~Wolfgang Wahlster (CEO DFKI GmbH)}\\[3mm]
\selectlanguage{german}%
 „Bei Volkswagen sehen wir innovative Sprachtechnologien als unerlässlich an, um in einem vereinten Europa die gesamte Bandbreite von Potenzialen zum Nutzen aller voll auszuschöpfen. Dieses Buch gibt umfassende Einblicke in das Thema."
  \textcolor{grey2}{--- Jörg Porsiel (Projektmanager Maschinelle Übersetzung, Volkswagen AG)}
}

% Funding notice left column
\FundingLNotice{\selectlanguage{german}
  Die Ausarbeitung dieses Weißbuchs wurde mit Mitteln aus dem Siebten
  Rahmenprogramm und dem Programm zur Unterstützung der Politik für
  Informations- und Kommunikationstechnologien der Europäischen
  Kommission im Rahmen der Verträge T4ME (Finanzhilfevereinbarung
  249119), CESAR (Finanzhilfevereinbarung 271022), METANET4U
  (Finanzhilfevereinbarung 270893) und META-NORD
  (Finanzhilfevereinbarung 270899) finanziert.}

% Funding notice right column
\FundingRNotice{\selectlanguage{english}
  The development of this white paper has been funded by the Seventh
  Framework Programme and the ICT Policy Support Programme of the
  European Commission under the contracts T4ME (Grant Agreement
  249119), CESAR (Grant Agreement 271022), METANET4U (Grant Agreement
  270893) and META-NORD (Grant Agreement 270899).}
