%                                     MMMMMMMMM                                         
%                                                                             
%  MMO    MM   MMMMMM  MMMMMMM   MM    MMMMMMMM   MMD   MM  MMMMMMM MMMMMMM   
%  MMM   MMM   MM        MM     ?MMM              MMM$  MM  MM         MM     
%  MMMM 7MMM   MM        MM     MM8M    MMMMMMM   MMMMD MM  MM         MM     
%  MM MMMMMM   MMMMMM    MM    MM  MM             MM MMDMM  MMMMMM     MM     
%  MM  MM MM   MM        MM    MMMMMM             MM  MMMM  MM         MM     
%  MM     MM   MMMMMM    MM   MM    MM            MM   MMM  MMMMMMM    MM
%
%
%          - META-NET Language Whitepaper | Slovene Metadata -
% 
% ----------------------------------------------------------------------------

\usepackage{polyglossia}
\setotherlanguages{slovene,english}


\title{Slovenski jezik v digitalni dobi --- The Slovene Language in the Digital Age}

\subtitle{White Paper Series --- Zbirka Bela knjiga}
\spineTitle{The Slovene Language in the Digital Age --- Slovenski jezik v digitalni dobi}

\author{
  Simon Krek
}

\authoraffiliation{
  dr. Simon Krek~ {\small IJS, Amebis}\\$ $\\
  dr. Tomaž Erjavec~ {\small IJS}\\ dr. Marko Stabej~ {\small FF, Uni LJ}\\
  (redakcija, \textcolor{grey1}{contributors})
}

%\editors{
% dr. Tomaž Erjavec (IJS),\\ dr. Marko Stabej (FF, Uni LJ)\\(redakcija, \textcolor{grey1}{contributors})
%}

\editors{
  Georg Rehm, Hans Uszkoreit\\(urednika, \textcolor{grey1}{editors})
}

% Text in left column on backside of the cover
\SpineLText{\selectlanguage{english}%
  In everyday communication, Europe’s citizens, business partners and politicians are inevitably confronted with language barriers.  
  Language technology has the potential to overcome these barriers and to provide innovative interfaces to technologies and knowledge. 
  This white paper presents the state of language technology support for the German language. 
  It is part of a series that analyzes the available language resources and technologies for 31~European languages. 
  The analysis was carried out by META-NET, a Network of Excellence funded by the European Commission.
  META-NET consists of 54 research centres in 33 countries, who cooperate with stakeholders from economy, government agencies, research organisations, non-governmental organisations, language communities and European universities. 
  META-NET’s vision is high-quality language technology for all European languages. 
}

% Text in right column on backside of the cover
\SpineRText{\selectlanguage{slovene}%
  Pri vsakdanji komunikaciji se evropski državljani, poslovni partnerji in politiki neizogibno soočajo z jezikovnimi mejami. S pomočjo jezikovnih tehnologij je mogoče te meje preseči in zagotoviti inovativen dostop do tehnologij in znanja. Bela knjiga opisuje stanje glede podpore jezikovnim tehnologijam za slovenski jezik in je del zbirke, v kateri so analizirani jezikovni viri in tehnologije za 31 evropskih jezikov. Analiza je bila izdelana v okviru mreže odličnosti META-NET, ki jo financira Evropska komisija. META-NET sestavlja 54 raziskovalnih centrov v 33 državah, ki sodelujejo z deležniki iz gospodarstva, državnih agencij, raziskovalnih organizacij, nevladnih organizacij, jezikovnih skupnosti in evropskih univerz. Vizija META-NET-a je zagotavljanje jezikovnih tehnologij visoke kakovosti za vse evropske jezike.}

% Quotes from VIPs on backside of the cover
\quotes{%
 Jezikovne tehnologije za slovenski jezik je treba načrtno podpirati, da se bo slovenščina lahko uspešno razvijala tudi v prihajajočem digitalnem svetu. \\
  \textcolor{grey2}{--- Dr. Danilo Türk, predsednik Republike Slovenije}\\[3mm]
 It is imperative that language technologies for Slovene are developed systematically if we want Slovene to flourish also in the future digital world.
 \\
  \textcolor{grey2}{--- Dr Danilo Türk, President of the Republic of Slovenia}
}

% Funding notice left column
\FundingLNotice{\selectlanguage{slovenian} Avtorji bele knjige o slovenskem jeziku se zahvaljujejo avtorjem bele knjige o nemškem jeziku za dovoljenje glede uporabe jezikovno neodvisnih delov publikacije \cite{lwpgerman}.\vfill\bigskip
  Izdelava bele knjige je bila financirana s sredstvi Sedmega okvirnega programa in Programa za
  podporo razvoju politik informacijsko-komunikacijskih tehnologij Evropske komisije v okviru
  pogodb T4ME (sporazum o dodelitvi sredstev 249119), CESAR (sporazum o dodelitvi sredstev 271022), METANET4U
  (sporazum o dodelitvi sredstev 270893) in META-NORD (sporazum o dodelitvi sredstev 270899).}

% Funding notice right column
\FundingRNotice{\selectlanguage{english}\vspace{-6pt} The authors of this document
  are grateful to the authors of the white paper on German for
  permission to re-use selected language-independent materials from
  their document \cite{lwpgerman}.\vfill\bigskip
  The development of this white paper has been funded by the Seventh
  Framework Programme and the ICT Policy Support Programme of the
  European Commission under the contracts T4ME (Grant Agreement
  249119), CESAR (Grant Agreement 271022), METANET4U (Grant Agreement
  270893) and META-NORD (Grant Agreement 270899).}
