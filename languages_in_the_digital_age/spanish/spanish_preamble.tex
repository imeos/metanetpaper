%                                     MMMMMMMMM
%                                                                             
%  MMO    MM   MMMMMM  MMMMMMM   MM    MMMMMMMM   MMD   MM  MMMMMMM MMMMMMM   
%  MMM   MMM   MM        MM     ?MMM              MMM$  MM  MM         MM     
%  MMMM 7MMM   MM        MM     MM8M    MMMMMMM   MMMMD MM  MM         MM     
%  MM MMMMMM   MMMMMM    MM    MM  MM             MM MMDMM  MMMMMM     MM     
%  MM  MM MM   MM        MM    MMMMMM             MM  MMMM  MM         MM     
%  MM     MM   MMMMMM    MM   MM    MM            MM   MMM  MMMMMMM    MM
%
%
%          - META-NET Language Whitepaper | Spanish Metadata -
% 
% ----------------------------------------------------------------------------

\usepackage{polyglossia}
\setotherlanguages{english,spanish}


\title{La lengua española \ \ \ \ \ \ \ \ \ en la era \ \ \ \ \ \ \ \ \ digital --- The Spanish Language in the Digital Age}

\spineTitle{The spanish Language in the Digital Age --- La lengua española en la era digital}

\subtitle{White Paper Series --- Serie de Libros Blancos}

\author{
  Maite Melero
}

\authoraffiliation{
  Maite Melero~ {\small Barcelona Media Centre d‘Innovació}
}

\editors{
  Georg Rehm, Hans Uszkoreit\\(editores, \textcolor{grey1}{editors})
}

% Text in left column on backside of the cover
\SpineLText{\selectlanguage{english}%
  In everyday communication, Europe’s citizens, business partners and politicians are inevitably confronted with language barriers.  
  Language technology has the potential to overcome these barriers and to provide innovative interfaces to technologies and knowledge. 
  This white paper presents the state of language technology support for the spanish language. 
  It is part of a series that analys1es the available language resources and technologies for 30~European languages. 
  The analysis was carried out by META-NET, a Network of Excellence funded by the European Commission.
  META-NET consists of 54 research centres in 33 countries, who cooperate with stakeholders from economy, government agencies, research organisations, non-governmental organisations, language communities and European universities. 
  META-NET’s vision is high-quality language technology for all European languages. 
}

% Text in right column on backside of the cover
\SpineRText{\selectlanguage{spanish}%
En sus comunicaciones cotidianas, los ciudadanos europeos, los socios comerciales y los políticos se enfrentan  inevitablemente con las barreras del idioma. Las tecnologías lingüísticas tienen el potencial para superar estas barreras y proporcionar interfaces innovadoras para tecnologías y el conocimientos. Este libro blanco presenta el estado del apoyo que reciben las tecnologías lingüísticas para el español. Forma parte de una serie de estudios que analizan los recursos lingüísticos y las tecnologías disponibles para 31 idiomas europeos. El análisis fue realizado por META-NET, una red de excelencia financiada por la Comisión Europea. META-NET consta de 54 centros de investigación en 33 países, que cooperan con otros agentes interesados procedentes del sector económico, administraciones públicas, centros de investigación, organizaciones no gubernamentales, comunidades lingüísticas y universidades europeas. La visión de META-NET es la tecnología lingüística de calidad para todos los idiomas europeos.}

% Quotes from VIPs on backside of the cover
\quotes{%
Lorem ipsum dolor sit amet, consectetur adipisicing elit, sed do eiusmod tempor incididunt ut labore et dolore magna aliqua. Ut enim ad minim veniam, quis nostrud exercitation ullamco laboris nisi ut aliquip ex ea commodo consequat. Duis aute irure dolor in reprehenderit in voluptate velit esse cillum dolore eu fugiat nulla pariatur. Excepteur sint occaecat cupidatat non proident, sunt in culpa qui officia deserunt mollit anim id est laborum. \\
  \textcolor{grey2}{--- Dr. John Doe (LT Expert)}
}

% Funding notice left column
\FundingLNotice{\selectlanguage{spanish} Los autores de este documento agradecen a los autores del Libro Blanco para el alemán el permiso para reutilizar material seleccionado de su texto original \cite{lwpgerman}.

  \bigskip
  La elaboración de este Libro Blanco ha sido financiada por el Séptimo Programa Marco y el Programa de Políticas de Apoyo a las TIC de la Comisión Europea mediante los contratos T4ME (GA 249119), CESAR (GA 271022), METANET4U (GA 270893) i META-NORD (GA 270899).}

\FundingRNotice{\selectlanguage{english} The authors of this document
  are grateful to the authors of the White Paper on German for
  permission to re-use selected language-independent materials from
  their document \cite{lwpgerman}.
  
  \bigskip
  The development of this white paper has been funded by the Seventh
  Framework Programme and the ICT Policy Support Programme of the
  European Commission under the contracts T4ME (Grant Agreement
  249119), CESAR (Grant Agreement 271022), METANET4U (Grant Agreement
  270893) and META-NORD (Grant Agreement 270899).}
