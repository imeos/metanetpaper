%                                     MMMMMMMMM
%
%  MMA    MM   MMMMMM  MMMMMMM   MM    MMMMMMMM   MMA   MM  MMMMMMM MMMMMMM
%  MMMA AMMM   MM        MM     MMMM              MMMM  MM  MM        MM
%  MM MMM MM   MMMMMM    MM    IM  MI   MMMMMMM   MM MMxMM  MMMMMM    MM
%  MM  M  MM   MM        MM   .MMMMMM.            MM  MMMM  MM        MM
%  MM     MM   MMMMMM    MM   MM    MM            MM   MMM  MMMMMMM   MM
%
%
%     - META-NET Language White Paper | Norwegian (bookmal) Metadata -
% 
% ----------------------------------------------------------------------------

\usepackage{polyglossia}
\setotherlanguages{norsk,english}
\newcommand{\bokmaal}[1]{#1} % {#1} to activate bokmaal, {} to deactivate
\newcommand{\nynorsk}[1]{} % {#1} to activate nynorsk, {} to deactivate


\title{Norsk \ \ \ \ \ \ \ \ \ \ \ \ \ \ \ \  i den \ \ \ \ \ \ \ \  digitale \ \ \ \ \ \ tidsalderen --- The Norwegian Language in the Digital Age}

% Title for the spine of the cover
\spineTitle{The Norwegian Language in the Digital Age --- Norsk i den digitale tidsalderen}

% Version of a language e.g. bokmålsversjon or nynorskversjon
\languageVersion{bokmålsversjon}

\subtitle{\bokmaal{White Paper Series --- Hvitbokserie}\nynorsk{White Paper Series --- Kvitbokserie}}

\author{
  Koenraad De Smedt\\
  Gunn Inger Lyse\\
  Anje Müller Gjesdal\\
  Gyri S. Losnegaard
}

\authoraffiliation{
  Koenraad De Smedt~ {\small UIB}\\
  Gunn Inger Lyse~ {\small UIB}\\
  Anje Müller Gjesdal~ {\small UIB}\\
  Gyri S. Losnegaard~ {\small UIB}
}
\editors{
  Georg Rehm, Hans Uszkoreit\\(Redaktører, \textcolor{grey1}{editors})
}

% Text in left column on backside of the cover
\SpineLText{\selectlanguage{english}In everyday communication,
Europe’s citizens, business partners and politicians are inevitably
confronted with language barriers.  Language technology has the
potential to overcome these barriers and to provide innovative
interfaces to technologies and knowledge.  This white paper presents
the state of language technology support for the Norwegian language.
It is part of a series that analyses the available language resources
and technologies for 30~European languages.  The analysis was carried
out by META-NET, a Network of Excellence funded by the European
Commission.  META-NET consists of 54 research centres in 33 countries,
who cooperate with stakeholders from economy, government agencies,
research organisations, non-governmental organisations, language
communities and European universities.  META-NET’s vision is
high-quality language technology for all European languages.}

% Text in right column on backside of the cover
\SpineRText{\selectlanguage{norsk}Det er uunngåelig at innbyggerne,
næringsliv og politikere i Europa støter på
språkbarrierer. Språkteknologi er et virkemiddel for å motvirke disse
barrierene, og kan gi nyskapende grensesnitt for teknologi og
kunnskap. Denne hviteboken gir en oversikt over situasjonen for
språkteknologi for norsk. Den er del av en serie som analyserer
tilgjengelige språkressurser og verktøy for 30~europeiske
språk. Analysen er utført av META-NET, et forskningsnettverk (Network
of Excellence) finansiert av EU-kommisjonen. META-NET består av 54
forskningsinstitusjoner i 33 land som samarbeider med ulike aktører
fra næringslivet, forvaltning, forskningsmiljøer, NGOer, språkbrukere
og universiteter. META-NETs visjon er å gjøre språkteknologi av høy
kvalitet tilgjengelig for alle europeiske språk.}

% Quotes from VIPs on backside of the cover
\quotes{``Skal man lage gode språkteknologiske løsninger for norsk, må det
eksistere språklige ressurser av høy kvalitet som industrien kan
benytte. Jeg håper at denne rapporten kan bidra til at slike ressurser
etableres raskt.''\\\textcolor{grey2}{--- Torbjørn Nordgård (Utviklingsdirektør Lingit AS)}\\[3mm]
``Skal vi kommunisere med maskinene rundt oss treng vi
språkteknologi. Denne rapporten presenterer status quo og vegen
framover for språkteknologi i Noreg.''\\ 
\textcolor{grey2}{--- Trond Trosterud (Professor Universitetet i Tromsø)}}

% Funding notice left column
\FundingLNotice{\selectlanguage{portuguese}\vskip2mm Forfatterne av
  denne teksten takker forfatterne av hvitboken for tysk for
  tillatelsen til å gjenbruke visse språkuavhengige materialer fra
  deres tekst \cite{lwpgerman}. Forfatterne takker også Gisle
  Andersen, Torbjørg Breivik, Helge Dyvik, Kristin Hagen, Torbjørn
  Nordgård, Torbjørn Svendsen og Trond Trosterud for verdifulle bidrag
  og kommentarer.
  
  \bigskip Arbeidet med denne utredningen er finansiert av det sjuende
  rammeprogrammet og Den europeiske kommisjonens ICT Policy Support
  program, gjennom kontraktene T4ME (tildelingsavtale 249\,119), CESAR
  (tildelingsavtale 271\,022), METANET4U (tildelingsavtale 270\,893) og
  META-NORD (tildelingsavtale 270\,899).}

% Funding notice right column
\FundingRNotice{\selectlanguage{english}\vskip2mm The authors of this
 document are grateful to the authors of the White Paper on German for
 permission to re-use selected language-independent materials from
 their document \cite{lwpgerman}.  They also wish to thank Gisle
 Andersen, Torbjørg Breivik, Helge Dyvik, Kristin Hagen, Torbjørn
 Nordgård, Torbjørn Svendsen and Trond Trosterud for valuable
 contributions and comments.

  \bigskip
  The development of this White Paper has been funded by the Seventh
  Framework Programme and the ICT Policy Support Programme of the
  European Commission under the contracts T4ME (Grant Agreement
  249\,119), CESAR (Grant Agreement 271\,022), METANET4U (Grant Agreement
  270\,893) and META-NORD (Grant Agreement 270\,899).}
