\documentclass[10pt]{article}
\usepackage{color}
\usepackage{longtable}
\usepackage{tabulary}
\usepackage{rotating}
\usepackage{makecell}
\usepackage{multirow}
\usepackage{colortbl}
\usepackage{booktabs}

\usepackage{fontspec}

% The new babel, for hyphenating and localization in general...
\usepackage{polyglossia}
\setmainlanguage{basque}
%\setotherlanguage{basque}
%!TEX TS-program = xelatex
\RequireXeTeX %Force XeTeX check



% Table colors
\definecolor{orange1}{cmyk}{0, 0.56, 0.86, 0}        % #FF6600
\definecolor{orange2}{cmyk}{0, 0.34, 0.87, 0.01}     % #FF9900

\begin{document}

% Begin table
\begin{figure}
\centering

\begin{tabular}{>{\columncolor{orange1}}p{.33\linewidth}@{\hspace*{6mm}}c@{\hspace*{6mm}}c@{\hspace*{6mm}}c@{\hspace*{6mm}}c@{\hspace*{6mm}}c@{\hspace*{6mm}}c@{\hspace*{6mm}}c}
\rowcolor{orange1}
 \cellcolor{white}&
 \begin{sideways}\makecell[l]{Kantitatea}\end{sideways} &
 \begin{sideways}\makecell[l]{\makecell[l]{Eskuragarritasuna} }\end{sideways} &
 \begin{sideways}\makecell[l]{Kalitatea}\end{sideways} &
 \begin{sideways}\makecell[l]{Estaldura}\end{sideways} &
 \begin{sideways}\makecell[l]{Heldutasuna}\end{sideways} &
 \begin{sideways}\makecell[l]{Iraunkortasuna}\end{sideways} &
 \begin{sideways}\makecell[l]{Moldagarritasuna}\end{sideways} \\ \addlinespace

\multicolumn{8}{>{\columncolor{orange2}}l}{\textcolor{black}{Hizkuntza teknologiak (Tresna, Teknologiak eta Aplikazioak)}} \\ \addlinespace

Hizketa Ezagutza &2&1&1&1&4&3&2 \\ \addlinespace
Hizketa Sintesia &2&3&4&4&4&3&3 \\ \addlinespace
Analisi gramatikala &4&2.5&4&4&4&2.5&2.5 \\ \addlinespace
Analisi semantikoa &1&1.5&2&1&1&1&1\\ \addlinespace
Testu-sorkuntza &1&0&0&0&0&0&0\\ \addlinespace
Itzulpen automatikoa &3&5&2&3&3&2&2\\ \addlinespace

\multicolumn{8}{>{\columncolor{orange2}}l}{\textcolor{black}{Hizkuntza baliabideak (Baliabideak, Datuak eta Jakintza-Baseak)}} \\ \addlinespace

Testu-corpusak &2&4&3&2&3&4&2.5\\ \addlinespace
Hizketa-corpusak &3&2&3&2&3&3&2\\ \addlinespace
Corpus paraleloak &2&4&2&2&2&2&1\\ \addlinespace
Baliabide lexikalak &4&4&4&5&5&4&3\\ \addlinespace
Gramatikak &2&2&2&2&2&2&2\\
\end{tabular}
\label{tab:lrlttable}
\caption{Hizkuntza-teknologien sustapenaren egoera euskararako.}
\end{figure}

\end{document}