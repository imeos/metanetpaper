\documentclass{scrartcl}

\usepackage{fontspec}
\usepackage{polyglossia}
\setotherlanguage{english}
\setmainlanguage{basque}

\usepackage{pdfcolparallel}
\begin{document}
\begin{Parallel}[c]{70mm}{70mm}
  \ParallelLText{\selectlanguage{basque}%

Hizkuntzak multzokatzeko, bost puntuko honako eskala hau baliatu da:
\begin{itemize}
\item 1 multzoa (HTen egoera bikaina)
\item 2 multzoa (egoera ona)
\item 3 multzoa (egoera ertaina)
\item 4 multzoa (egoera osagabea)
\item 5 multzoa (egoera apala)
\end{itemize}
HTen egoera irizpide hauen bidez neurtu da:
\begin{itemize}
\item Hizketa Prozesaketa: hizketa ezagutzeko dauden teknologien kalitatea, hizketa-sintesirako dauden teknologien kalitatea, landutako eremuen kopurua, dauden hizketazko corpusen kantitatea eta tamaina, hizketan oinarritutako aplikazio eskuragarrien kantitatea eta motak.
\item Itzulpen Automatikoa: Dauden MT teknologien kalitatea, landutako hizkuntza pareen kopurua, landutako fenomeno linguistikoen eta eremuen kopurua, dauden corpus paraleloen kalitatea eta tamaina, MT aplikazio eskuragarrien kantitatea eta motak.
\item Testu Analisia: Testua analizatzeko dauden teknologien kalitatea eta motak (morfologia, sintaxia, semantika), landutako fenomeno linguistikoen eta eremuen kopurua, eskuragarri dauden aplikazioen kantitatea eta motak, dauden testu-corpusen (etiketatuen) kalitatea eta tamaina, dauden baliabide lexikalen (adibidez, WordNet) eta gramatiken kalitatea eta motak.
\item Baliabideak: Dauden testu-corpusen kalitatea eta tamaina, hizketa-corpusak eta corpus paraleloak, dauden baliabide lexikal eta gramatiken kalitatea eta motak.
\end{itemize} 
 }

  \ParallelRText{\selectlanguage{english}%

The languages were clustered using the following five-point scale: 

    \begin{itemize}
      \item Cluster 1 (excellent LT support)
      \item Cluster 2 (good support)
      \item Cluster 3 (moderate support)
      \item Cluster 4 (fragmentary support) 
      \item Cluster 5 (weak or no support)
    \end{itemize}

LT support was measured according to the following criteria:
\begin{itemize}
\item Speech Processing: Quality of existing speech recognition technologies, quality of existing speech synthesis technologies, coverage of domains, number and size of existing speech corpora, amount and variety of available speech-based applications
\item Machine Translation: Quality of existing MT technologies, number of language pairs covered, coverage of linguistic phenomena and domains, quality and size of existing parallel corpora, amount and variety of available MT applications
\item Text Analysis: Quality and coverage of existing text analysis technologies (morphology, syntax, semantics), coverage of linguistic phenomena and domains, amount and variety of available applications, quality and size of existing (annotated) text corpora, quality and coverage of existing lexical resources (e.g., WordNet) and grammars
\item Resources: Quality and size of existing text corpora, speech corpora and parallel corpora, quality and coverage of existing lexical resources and grammars
\end{itemize} 

  }
  \ParallelPar
\end{Parallel}
\end{document}