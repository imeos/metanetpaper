%                                     MMMMMMMMM        
%                                                                             
%  MMO    MM   MMMMMM  MMMMMMM   MM    MMMMMMMM   MMD   MM  MMMMMMM MMMMMMM   
%  MMM   MMM   MM        MM     ?MMM              MMM$  MM  MM         MM     
%  MMMM 7MMM   MM        MM     MM8M    MMMMMMM   MMMMD MM  MM         MM     
%  MM MMMMMM   MMMMMM    MM    MM  MM             MM MMDMM  MMMMMM     MM     
%  MM  MM MM   MM        MM    MMMMMM             MM  MMMM  MM         MM     
%  MM     MM   MMMMMM    MM   MM    MM            MM   MMM  MMMMMMM    MM
%
%
%          - META-NET Language Whitepaper | Catalan Metadata -
% 
% ----------------------------------------------------------------------------

\usepackage{polyglossia}
\setotherlanguages{basque,english}


\title{Euskara \ \ \ \ Aro \ \ \ \ Digitalean --- The Basque Language\ \ \ \ in the\ \ \ \ Digital Age}

\spineTitle{The Basque Language in the Digital Age --- Euskara Aro Digitalean}

\subtitle{White Paper Series --- Liburu Zurien Bilduma}

\author{
 Inmaculada Hernáez\\
 Eva Navas\\ 
 Igor Odriozola\\
 Kepa Sarasola\\
 Arantza Diaz de Ilarraza\\
 Igor Leturia\\
 Araceli Diaz de Lezana\\
 Beñat Oihartzabal\\
 Jasone Salaberria
}

\authoraffiliation{
 Inmaculada Hernáez~~{\small [1]}\\
 Eva Navas~~{\small [1]}\\ 
 Igor Odriozola~~{\small [1]}\\
 Kepa Sarasola~~{\small [1]}\\
 Arantza Diaz de Ilarraza~~{\small [1]}\\
 Igor Leturia~~{\small [2]}\\
 Araceli Diaz de Lezana~~{\small [3]}\\
 Beñat Oihartzabal~~{\small [4]}\\
 Jasone Salaberria~~{\small [4]}\\
 ~\\
 \footnotesize{[1]} ~ \hspace*{.3mm}\small{Univ. del País Vasco/Euskal Herriko Unibertsitatea}\\
 \footnotesize{[2]} ~ \small{Elhuyar Foundation}\\
 \footnotesize{[3]} ~ \small{Gobierno Vasco/Eusko Jaurlaritza}\\
 \footnotesize{[4]} ~ \small{UMR 5478 IKER}
}

\editors{
  Georg Rehm, Hans Uszkoreit\\(editors)
}

% Text in left column on backside of the cover
\SpineLText{\selectlanguage{english}%
  In everyday communication, Europe’s citizens, business partners and politicians are inevitably confronted with language barriers.  
  Language technology has the potential to overcome these barriers and to provide innovative interfaces to technologies and knowledge. 
  This white paper presents the state of language technology support for the Basque language. 
  It is part of a series that analyses the available language resources and technologies for 30~European languages. 
  The analysis was carried out by META-NET, a Network of Excellence funded by the European Commission.
  META-NET consists of 54 research centres in 33 countries, who cooperate with stakeholders from economy, government agencies, research organisations, non-governmental organisations, language communities and European universities. 
  META-NET’s vision is high-quality language technology for all European languages.
}

% Text in right column on backside of the cover
\SpineRText{\selectlanguage{basque}%
Europako hiritarrak, enpresak nahiz politikariak eragozpen linguistikoak gainditu beharrean izaten dira egunero-egunero. Hizkuntza-teknologiek aukera ematen dute eragozpen horiek gainditzeko eta, halaber, hainbat teknologia eta ezagupide erabiltzeko interfaze berritzaileak sortzeko. Liburu zuri honek euskararako hizkuntza-teknologien egoera aurkezten du, eta Europako 30 hizkuntzatarako eskuragarri dauden baliabide linguistikoak eta teknologiak aztertzen dituen bilduma baten lehenengo atala da. Europako batzordeak sortutako META-NET Bikaintasun Sareak bultzatu du azterketa hori, eta enpresa-munduko, administrazio publikoko, ikerketa-alorreko, alor pribatuko, komunitate linguistikoko eta unibertsitate europarretako parte hartzaileekin lanean diharduten 33 herrialdetako 54 ikerketa-zentroz osatuta dago.
}

% Quotes from VIPs on backside of the cover
\quotes{\selectlanguage{english}%
``The Language White Paper Series is an excellent initiative of
META-NET, in keeping with our motto 'Give and spread knowledge'. We
hope that it will further foster investment in Language Technology
solutions for less resourced languages like Basque." \textcolor{grey2}{--- Iñaki Goirizelaia (Rector of the Universidad del País Vasco)}\\[2mm]
\selectlanguage{basque}%
``Europa eleanitzaren testuinguruan, Informazioaren eta Komunikazioaren Teknologiak (IKT) arlo estrategikoa dira hizkuntza guztientzat baina, bereziki, hizkuntza minoritarioentzat. Egun, teknologia horien kontsumitzaileek, Interneti esker, muga geografikoak eta linguistikoak gaindituta, aukera paregabea dute IKT produktuak nahi duten hizkuntzan eskuratzeko. Baina horretarako, gure hizkuntza txikiek, ezinbestez, merkatu horretan sartu behar dute. META-NET plataforma egokia da helburu hori erdiesteko."
  \textcolor{grey2}{--- Blanca Urgell (Eusko Jaurlaritzako Kultura Sailburua)}
}

% Funding notice left column
\FundingLNotice{\selectlanguage{basque} Dokumentu honen egileek beren eskerrik beroenak adierazi nahi dizkie alemanezko liburu zuriaren \cite{lwpgerman} egileei, haien dokumentuko zenbait atal, hizkuntzaren araberakoak ez direnak, berrerabiltzeko baimena emateagatik.
  
  \bigskip
   Liburu zuri hau Europako Batzordeko Zazpigarren Esparru Programaren eta 
 IKTak Sustatzeko Programa Estrategikoaren diru-laguntzari esker garatu da, 
 T4ME (249\,119 Dirulaguntza Hitzarmena), CESAR (271\,022 Dirulaguntza Hitzarmena), 
 METANET4U (270\,893 Dirulaguntza Hitzarmena) eta META-NORD 
 (270\,899 Dirulaguntza Hitzarmena) kontratuen baitan.}

% Funding notice right column
\FundingRNotice{\selectlanguage{english} The authors of this document
  are grateful to the authors of the White Paper on German \cite{lwpgerman} for
  permission to re-use selected language-independent materials from
  their document.
  
  \bigskip
  The development of this white paper has been funded by the Seventh
  Framework Programme and the ICT Policy Support Programme of the
  European Commission under the contracts T4ME (Grant Agreement 249\,119),
  CESAR (Grant Agreement 271\,022), METANET4U (Grant Agreement 270\,893)
  and META-NORD (Grant Agreement 270\,899).}
  
  \hyphenation{ex-am-ple ex-am-ple}
