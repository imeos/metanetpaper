%                                     MMMMMMMMM
%
%  MMA    MM   MMMMMM  MMMMMMM   MM    MMMMMMMM   MMA   MM  MMMMMMM MMMMMMM
%  MMMA AMMM   MM        MM     MMMM              MMMM  MM  MM        MM
%  MM MMM MM   MMMMMM    MM    IM  MI   MMMMMMM   MM MMxMM  MMMMMM    MM
%  MM  M  MM   MM        MM   .MMMMMM.            MM  MMMM  MM        MM
%  MM     MM   MMMMMM    MM   MM    MM            MM   MMM  MMMMMMM   MM
%
%
%          - META-NET Language Whitepaper | Lithuanian Metadata -
% 
% ----------------------------------------------------------------------------

\usepackage{polyglossia}
\setotherlanguages{lithuanian,english}


\title{Lietuvių \ \ \ \ \ kalba skaitmeniniame amžiuje --- The Lithuanian Language in the Digital Age}

% Title for the spine of the cover
\spineTitle{The Lithuanian Language in the Digital Age --- Lietuvių kalba skaitmeniniame amžiuje}

\subtitle{White Paper Series --- Baltųjų knygų serija}

\author{
  Daiva Vaišnienė \\
  Jolanta Zabarskaitė
}
\authoraffiliation{
  Daiva Vaišnienė ~ {\small Lietuvių kalbos institutas} \\
  Jolanta Zabarskaitė ~ {\small Lietuvių kalbos institutas}
}
\editors{
  Georg Rehm, Hans Uszkoreit\\(redaktoriai, \textcolor{grey1}{editors})
}

% Quotes from VIPs on backside of the cover
\quotes{
\selectlanguage{english}``Having preserved a close link with the old Indo-European parent languages, 
the Lithuanian language today satisfies the needs of the modern society 
perfectly well. However, active users of the Lithuanian language only 
amount to several million. Conserving it for future generations is a 
responsibility of the whole of the European Union. How we proceed with 
developing information technology will pretty much determine the future of 
the Lithuanian language.'' \\
\textcolor{grey2}{--- Andrius Kubilius (Prime Minister of the Republic of Lithuania)}\\[3mm]
}

% Text in left column on backside of the cover
\SpineLText{\selectlanguage{english}%
 In everyday communication, Europe’s citizens, business partners and politicians are inevitably confronted with language barriers.  
  Language technology has the potential to overcome these barriers and to provide innovative interfaces to technologies and knowledge. 
  This white paper presents the state of language technology support for the Lithuanian language. 
  It is part of a series that analyses the available language resources and technologies for 30~European languages. 
  The analysis was carried out by META-NET, a Network of Excellence funded by the European Commission.
  META-NET consists of 54 research centres in 33 countries, who cooperate with stakeholders from economy, government agencies, research organisations and others. 
  META-NET’s vision is high-quality language technology for all European languages. 
  }

% Text in right column on backside of the cover
\SpineRText{\selectlanguage{lithuanian}\hspace*{-1.2mm}Kiekvieną dieną bendraudami
tarpusavyje, Europos gyventojai, verslo partneriai ir politikai
neišvengiamai susiduria su kalbos barjerais.  Kalbos technologijos
gali įveikti tokius barjerus ir pateikti novatoriškų technologinių
sąsajų ir žinių.  Šioje Baltojoje knygoje pristatoma lietuvių kalbos
technologijų būklė.  Tai - Baltųjų knygų, kuriose nagrinėjami 30
Europos kalbų ištekliai ir technologijos, serijos dalis.  Analizę
atliko META-NET, Europos Komisijos finansuojamas meistriškumo tinklas.
META-NET tinklą sudaro 54 mokslinių tyrimų centrai 33 šalyse,
bendradarbiaujantys su verslo atstovais, vyriausybinėmis
institucijomis, tyrimų organizacijomis ir kitokiomis suinteresuotomis
šalimis.  META-NET tinklo vizija – sukurti kokybiškų kalbos
technologijų, skirtų visoms Europos kalboms.}

% Funding notice left column
\FundingLNotice{\selectlanguage{lithuanian} \vskip2mm Šio dokumento
autorės nuoširdžiai dėkoja vokiečių Baltosios knygos \cite{lwpgerman}
autoriams, suteikusiems galimybę pasinaudoti medžiaga, kurioje
aptariami bendrieji kalbos technologijų dalykai.

\bigskip Šios Baltosios knygos sudarymas buvo finansuotas pagal
Europos Komisijos septintąją bendrąją programą ir IKT politikos
paramos programą: T4ME (subsidijų sutartis Nr.~249\,119), CESAR
(subsidijų sutartis Nr.~271\,022), METANET4U (subsidijų sutartis
Nr.~270\,893) ir META-NORD (subsidijų sutartis Nr.~270\,899).}

% Funding notice right column
\FundingRNotice{\selectlanguage{english} \vskip2mm The authors of this
 document are grateful to the authors of the White Paper on German
 \cite{lwpgerman} for permission to reuse selected
 language-independent materials from their document.
  
  \bigskip
  The development of this white paper has been funded by the Seventh
  Framework Programme and the ICT Policy Support Programme of the
  European Commission under the contracts T4ME (Grant Agreement
  249\,119), CESAR (Grant Agreement 271\,022), METANET4U (Grant Agreement
  270\,893) and META-NORD (Grant Agreement 270\,899).}
  
  \hyphenation{skait-me-ni-nia-me ga-li-my-bes re-mian-tis google}
