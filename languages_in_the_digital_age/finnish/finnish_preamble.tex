%                                     MMMMMMMMM
%
%  MMA    MM   MMMMMM  MMMMMMM   MM    MMMMMMMM   MMA   MM  MMMMMMM MMMMMMM
%  MMMA AMMM   MM        MM     MMMM              MMMM  MM  MM        MM
%  MM MMM MM   MMMMMM    MM    IM  MI   MMMMMMM   MM MMxMM  MMMMMM    MM
%  MM  M  MM   MM        MM   .MMMMMM.            MM  MMMM  MM        MM
%  MM     MM   MMMMMM    MM   MM    MM            MM   MMM  MMMMMMM   MM
%
%
%          - META-NET Language Whitepaper | Finnish Metadata -
% 
% ----------------------------------------------------------------------------

\usepackage[finnish, english]{babel}


\title{Suomen kieli digitaalisella aikakaudella --- The Finnish Language in the Digital Age}

\spineTitle{The Finnish Language in the Digital Age -- Suomen kieli digitaalisella aikakaudella}

\subtitle{White Paper Series --- Valkoiset kirjat}

\author{
  Kimmo Koskenniemi\\
  Krister Lindén\\
  Lauri Carlson\\
  Martti Vainio\\
  Antti Arppe\\
  Mietta Lennes\\
  Hanna Westerlund\\
  Mirka Hyvärinen\\
  Imre Bartis\\
  Pirkko Nuolijärvi\\
  Aino Piehl
}


\authoraffiliation{
  Kimmo Koskenniemi~ {\small Helsingin yliopisto}\\
  Krister Lindén~ {\small Helsingin yliopisto}\\
  Lauri Carlson~ {\small Helsingin yliopisto}\\
  Martti Vainio~ {\small Helsingin yliopisto}\\
  Antti Arppe~ {\small Helsingin yliopisto}\\
  Mietta Lennes~ {\small Helsingin yliopisto}\\
  Hanna Westerlund~ {\small Helsingin yliopisto}\\
  Mirka Hyvärinen~ {\small Helsingin yliopisto}\\
  Imre Bartis~ {\small Helsingin yliopisto}\\
  Pirkko Nuolijärvi~ {\small KOTUS}\\
  Aino Piehl~ {\small KOTUS}
}

\editors{
  Georg Rehm, Hans Uszkoreit\\(toimittajat, \textcolor{grey1}{editors})
}

% Text in left column on backside of the cover
\SpineLText{\selectlanguage{english}%
  In everyday communication, Europe’s citizens, business partners and politicians are inevitably confronted with language barriers.  
  Language technology has the potential to overcome these barriers and to provide innovative interfaces to technologies and knowledge. 
  This white paper presents the state of language technology support for the Finnish language. 
  It is part of a series that analyses the available language resources and technologies for 30~European languages. 
  The analysis was carried out by META-NET, a Network of Excellence funded by the European Commission.
  META-NET consists of 54 research centres in 33 countries, who cooperate with stakeholders from economy, government agencies, research organisations, non-governmental organisations, language communities and European universities. 
  META-NET’s vision is high-quality language technology for all European languages.}

% Text in right column on backside of the cover
\SpineRText{\selectlanguage{finnish}%
\hspace{-1.0mm}Euroopan kansalaiset, liike-elämä ja poliitikot kohtaavat väistämättä kielten erilaisuudesta johtuvia esteitä keskinäisessä viestinnässään. Kieliteknologia tarjoaa meille mahdollisuuden näiden esteiden ylittämiseen. Teknologioiden käyttäminen ja tietämyksen hyödyntäminen helpottuvat innovatiivisten käyttöliittymien avulla. Tässä valkoisessa kirjassa esitellään kieliteknologinen tuki suomen kielen osalta. Raportti on osa julkaisusarjaa, jossa analysoidaan 30 Euroopan kielen kieliresursseja ja teknologioita. Tutkimustyö on osa Euroopan komission rahoittaman huippuosaamisen verkoston META-NETin  toimintaa. META-NET koostuu 54 tutkimuslaitoksesta 33 maassa. Ne tekevät yhteistyötä talouden alan toimijoiden, julkishallinnon edustajien, tutkimuslaitosten, kansalaisjärjestöjen, kieliyhteisöjen ja eurooppalaisten yliopistojen kanssa. META-NETin visio on tuoda laadukkaita kieliteknologisia sovelluksia kaikkien Euroopan kielten puhujien saataville.}

% Quotes from VIPs on backside of the cover
\quotes{%
%"Without languages we could not communicate. The META-NET network is a valuable support for a multilingual Europe."\\
``Ilman kieliä emme voisi viestiä. META-NET-verkosto on arvokas tuki monikieliselle Euroopalle." \\
  \textcolor{grey2}{--- Alexander Stubb, Eurooppa- ja ulkomaankauppaministeri}\\[3mm]
%Kieli- ja puheteknologian sovelluksia käytetään päivittäin kääntämisessä, oikoluvussa, sanakirjojen teossa ja matkapuhelimissa. Globaalistuvassa %maailmassa, jossa englantia käytetään laajalti, on äärimmäisen tärkeää, että myös pienempien kansalliskielten kuten suomen ja ruotsin kieliteknologiaa %kehitetään täysipainoisesti.\\
 % \textcolor{grey2}{--- Fred Karlsson, yleisen kielitieteen professori, Helsingin yliopisto)}\\[3mm]
%Perustellusti on sanottu, että äidinkieli on sydämen kieli. 
%Suomalaisten yritysten on otettava tämä huomioon kansainvälistäessääntuotteitaan ja palvelujaan. 
%"Suomen kieli digitaalisella aikakaudella on erinomainen yhteenveto kieliteknologian nykytilanteesta. Toivoisinkin sen löytävän tiensä suomalaisten yritysten, %päättäjien ja toteuttajien käsiin."\\
``Our mother tongue is a matter of the heart. Finnish companies should take this into account when internationalizing their products and services. This book is an excellent round-up of the current situation of language technology.
I hope it will find its way into the hands of Finnish companies, policy makers and implementers."\\ 
\textcolor{grey2}{--- Pauli Kuosmanen, CTO (Tieto- ja viestintäteollisuuden tutkimus TIVIT Oy), Prof., Dr. Tech., eMBA
%\textcolor{grey2}{--- Pauli Kuosmanen, teknologiajohtaja (Tieto- ja viestintäteollisuuden tutkimus TIVIT Oy), Prof., Dr. Tech., eMBA
}
}

% Funding notice left column
\FundingLNotice{\selectlanguage{finnish} Tämän raportin tekijät ovat
  kiitollisia saksankielisen META-NET valkoisen kirjan tekijöille
  luvasta käyttää raporttinsa kielestä riippumattomien osioiden
  tekstejä osana tämän raportin englanninkielistä osuutta sekä
  lähteenä suomenkieliselle käännökselle \cite{lwpgerman}.

  \bigskip Tämän valkoisen kirjan tuottamiseen on myönnetty
  rahoitusta Euroopan komission seitsemännestä puiteohjelmasta ja
  tieto- ja viestintäteknologioiden tukiohjelmasta seuraavien
  sopimusten perusteella T4ME (rahoitussopimus 249119), CESAR
  (rahoitussopimus 271022), METANET4U (rahoitussopimus 270893) ja
  META-NORD (rahoitussopimus 270899).}

% Funding notice right column
\FundingRNotice{\selectlanguage{english} The authors of this document
  are grateful to the authors of the White Paper on German for
  permission to re-use selected language-independent materials from
  their document \cite{lwpgerman}.

  \bigskip The development of this white paper has been funded by the
  Seventh Framework Programme and the ICT Policy Support Programme of
  the European Commission under the contracts T4ME (Grant Agreement
  249119), CESAR (Grant Agreement 271022), METANET4U (Grant Agreement
  270893) and META-NORD (Grant Agreement 270899).}
