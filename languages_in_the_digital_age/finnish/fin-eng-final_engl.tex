

\ssection{Executive Summary}
\begin{multicols}{2}
\boxtext{Language technology builds bridges for Europe’s future}

During the last 60 years, Europe has become a distinct political and
economic structure, yet culturally and linguistically it is still very
diverse. This means that from Portuguese to Polish and Greek Italian
to GaelicIcelandic, everyday communication between Europe’s citizens
as well as communication in the spheres of business and politics is
inevitably confronted by language barriers. The EU’s institutions
spend about a billion euros a year on maintaining their policy of
multilingualism, i.e., translating texts and interpreting spoken
communication. Yet does this have to be such a burden?  Modern
language technology and linguistic research can make a significant
contribution to pulling down these linguistic borders. When combined
with intelligent devices and applications, language technology will in
the future be able to help Europeans talk easily to each other and do
business with each other even if they do not speak a common language.

The Finnish trade within the EU increases both in export and
import. Between January and July in 2011 the import from the EU
countries was over 77\% and the export approximately 73\%
\cite{SVT}. But language barriers can bring business to a halt,
especially for SMEs who do not have the financial means to reverse the
situation. The only (unthinkable) alternative to this kind of
multilingual Europe would be to allow a single language to take a
dominant position and end up replacing all other languages.

One classic way of overcoming the language barrier is to learn foreign
languages. Yet without technological support, mastering the 23
official languages of the member states of the European Union and
ca. 60 additional European languages is an insurmountable obstacle for
the citizens of our continent and its economy, political debate, and
scientific progress.

The solution is to build key enabling technologies. These will offer European
actors tremendous advantages, not only within the common European market but
also in trade relations with third countries, especially emerging economies. To
achieve this goal and preserve Europe’s cultural and linguistic diversity, it
is necessary to first carry out a systematic analysis of the linguistic
particularities of all European languages, and the current state of language
technology support for them. Language technology solutions will eventually
serve as a unique bridge between Europe’s languages.

\boxtext{Language technology as a key for the future}

The automated translation and speech processing tools currently available on
the market still fall short of this ambitious goal. The dominant actors in the
field are primarily privately-owned for-profit enterprises based in Northern
America. Already in the late 1970s, the EU realised the profound relevance of
language technology as a driver of European unity, and began funding its first
research projects, such as EUROTRA. At the same time, national projects were
set up that generated valuable results but never led to concerted European
action. In contrast to this highly selective funding effort, other multilingual
societies such as India (22 official languages) and South Africa (11 official
languages) have recently set up long-term national programmes for language
research and technology development.

The predominant actors in LT today rely on imprecise statistical approaches
that do not make use of deeper linguistic methods and knowledge. For example,
sentences are automatically translated by comparing a new sentence against
thousands of sentences previously translated by humans. The quality of the
output largely depends on the amount and quality of the available sample
corpus. While the automatic translation of simple sentences in languages with
sufficient amounts of available text material can achieve
useful results, such shallow statistical methods
are doomed to fail in the case of languages with a much smaller body of sample
material or in the case of sentences with complex
structures.

The European Union has therefore decided to fund projects such as EuroMatrix
and EuroMatrixPlus (since 2006) and iTranslate4 (since 2010) that carry out
basic and applied research and generate resources for establishing high quality
language technology solutions for all European languages. Analysing the deeper
structural properties of languages is the only way forward if we want to build
applications that perform well across the entire range of Europe’s languages.

European research in this area has already achieved a number of successes. For
example, the translation services of the European Union now use MOSES
open-source machine translation software that has been mainly developed through
European research projects. Rather than building on the outcomes of its
research projects, Europe has tended to pursue isolated research activities
with a less pervasive impact on the market. The economic value of even the
earliest efforts can be seen in the number of spin-offs. A company such as
Trados, which was founded back in 1984, was sold to the UK-based SDL in 2005.

\boxtext{Language Technology helps unify Europe}

META-NET’s long-term goal is to introduce high-quality language
technology for all languages in order to achieve political and
economic unity through cultural diversity. The technology will help
tear down existing barriers and build bridges between Europe’s
languages. This requires all stakeholders - in politics, research,
business, and society - to unite their efforts for the future.

Drawing on the insights gained so far, it appears that today’s
'hybrid' language technology mixing deep processing with statistical
methods will be able to bridge the gap between all European languages
and beyond. As this series of white papers shows, there is a dramatic
difference in the state of readiness with respect to language
solutions and the state of research between Europe’s member states.

This whitepaper series complements other strategic actions taken by
META-NET (see the appendix for an overview). Up-to-date information
such as the current version of the META-NET vision paper \cite{Vision}
or the Strategic Research Agenda (SRA) can be found on the META-NET
web site: \url{http://www.meta-net.eu}.

We are currently witnessing a digital revolution that is comparable to
Gutenberg’s invention of the printing press.


\end{multicols}
\clearpage
\ssection{Risk for Our Languages and a Challenge for Language Technology}
\begin{multicols}{2}

We are witnesses to a digital revolution that is dramatically impacting
communication and society. Recent developments in digital information and
communication technology are sometimes compared to Gutenberg’s invention of the
printing press. What can this analogy tell us about the future of the European
information society and our languages in particular?

After Gutenberg’s invention, real breakthroughs in communication and knowledge
exchange were accomplished by efforts such as Luther’s translation of the Bible
into vernacular language. In subsequent centuries, cultural techniques have
been developed to better handle language processing and knowledge exchange:
\begin{itemize}
\item the orthographic and grammatical standardisation of major languages enabled the
rapid dissemination of new scientific and intellectual ideas;

\item the development of official languages made it possible for citizens to
communicate within certain (often political) boundaries;

\item the teaching and translation of languages enabled exchanges across languages;

\item the creation of editorial and bibliographic guidelines assured the quality and
availability of printed material;

\item the creation of different media like newspapers, radio, television, books, and
other formats satisfied different communication needs.
\end{itemize}
In the past twenty years, information technology has helped to automate and
facilitate many of the processes:
\begin{itemize}
\item desktop publishing software has replaced typewriting and typesetting;

\item overhead projector transparencies have been replaced by programs such as
OpenOffice presentations or Microsoft PowerPoint;

\item e-mail send and receive documents faster than a fax machine;

\item free networking environments offer cheap Internet phone calls and hosts virtual
meetings;

\item audio and video encoding formats make it easy to exchange multimedia content;

\item search engines provide keyword-based access to web pages;

\item online services like Google Translate produce quick, approximate translations;

\item social media platforms such as Facebook, Twitter, and Google+ facilitate
communication, collaboration, and information sharing.
\end{itemize}

Although such tools and applications are helpful, they are not yet capable of
supporting a sustainable, multilingual European society for all where
information and goods can flow freely.

\subsection{Language Borders Hinder the European Information Society}

\boxtext{A global economic and information space confronts us with more languages, speakers and content.}
We cannot predict exactly what the future information society will
look like.  But there is a strong likelihood that the revolution in
communication technology is bringing people speaking different
languages together in new ways. This is putting pressure on
individuals to learn new languages and especially on developers to
create new technology applications to ensure mutual understanding and
access to shareable knowledge. In a global economic and information
space, more languages, speakers and content interact more quickly with
new types of media. The current popularity of social media (Wikipedia,
Facebook, Twitter, YouTube, and, recently, Google+) is only the tip of
the iceberg.

Today, we can transmit gigabytes of text around the world in a few
seconds before we recognise that it is in a language we do not
understand. According to a recent report from the European Commission,
57\% of Internet users in Europe purchase goods and services in
languages that are not their native language. (English is the most
common foreign language followed by French, German and Spanish.) 55\%
of users read content in a foreign language while only 35\% use
another language to write e-mails or post comments on the
Web. \cite{EC-prefer} A few years ago, English might have been the
lingua franca of the Web—the vast majority of content on the Web was
in English—but the situation has now drastically changed. The amount
of online content in other European (as well as Asian and Middle
Eastern) languages has exploded.

Surprisingly, this ubiquitous digital divide due to language borders
has not gained much public attention; yet, it raises a very pressing
question: Which European languages will thrive in the networked
information and knowledge society, and which are doomed to disappear?

\subsection{Our Languages at Risk}

While the printing press helped step up the exchange of
information in Europe, it also led to the extinction of many European
languages. Regional and minority languages were rarely printed and
languages such as Cornish and Dalmatian were limited to oral forms of
transmission, which in turn restricted their scope of use. Will the
Internet have the same impact on our languages?
% \boxtext{Europe’s approximately 80 languages are one of its richest and most important cultural assets.}
Europe’s approximately 80 languages are one of its richest and most
important cultural assets, and a vital part of its unique social
model \cite{EC-multi}. While languages such as English and Spanish are
likely to survive in the emerging digital marketplace, many European
languages could become irrelevant in a networked society. This would
weaken Europe’s global standing, and run counter to the strategic goal
of ensuring equal participation for every European citizen regardless
of language. According to a UNESCO report on multilingualism,
languages are an essential medium for the enjoyment of fundamental
rights, such as political expression, education and participation in
society \cite{UN-mid}.

\subsection{Language Technology is a Key Enabling Technology}

In the past, investment efforts in language preservation focused
on language education and translation. According to one estimate, the
European market for translation, interpretation, software localisation
and website globalisation was 8.4 billion euros in 2008 and is
expected to grow by 10\% per annum \cite{EC-size}. Yet this figure
covers just a small proportion of current and future needs in
communicating between languages. The most compelling solution for
ensuring the breadth and depth of language usage in Europe tomorrow is
to use appropriate technology, just as we use technology to solve our
transport, energy and disability needs among others.

\boxtext{Language technology helps people collaborate, conduct business, share knowledge and participate in social and political debates across different languages.}
Digital language technology (targeting all forms of written text and spoken
discourse) helps people collaborate, conduct business, share knowledge and
participate in social and political debate regardless of language barriers and
computer skills. It often operates invisibly inside complex software systems to
help us:
\begin{itemize}
\item find information with an Internet search engine;

\item check spelling and grammar in a word processor;

\item view product recommendations in an online shop;

\item hear the verbal instructions of a car navigation system;

\item translate web pages via an online service.
\end{itemize}


Language technology consists of a number of core applications that
enable processes within a larger application framework. The purpose of
the META-NET language white papers is to focus on how ready these core
technologies are for each European language.
\boxtext{Europe needs robust and affordable language technology for all Euro-pean languages.}
To maintain our position in the frontline of global innovation, Europe
will need language technology adapted to all European languages that
is robust, affordable and tightly integrated within key software
environments. Without language technology, we will not be able to
achieve a really effective interactive, multimedia and multilingual
user experience in the near future.

\subsection{Kieliteknologian mahdollisuuksia --- Opportunities for Language Technology}
In the world of print, the technology breakthrough was the rapid duplication of
an image of a text (a page) using a suitably powered printing press. Human
beings had to do the hard work of looking up, reading, translating, and
summarizing knowledge. We had to wait until Edison to record spoken language –
and again his technology simply made analogue copies.

Digital language technology can now automate the very processes of translation,
content production, and knowledge management for all European languages. It can
also empower intuitive language/speech-based interfaces for household
electronics, machinery, vehicles, computers and robots. Real-world commercial
and industrial applications are still in the early stages of development, yet R\&D
achievements are creating a genuine window of opportunity. For example,
machine translation is already reasonably accurate in specific domains, and
experimental applications provide multilingual information and knowledge
management as well as content production in many European languages.

As with most technologies, the first language applications such as voice-based
user interfaces and dialogue systems were developed for highly specialised
domains, and often exhibit limited performance. But there are huge market
opportunities in the education and entertainment industries for integrating
language technologies into games, cultural heritage sites, edutainment
packages, libraries, simulation environments and training programmes. Mobile
information services, computer-assisted language learning software, eLearning
environments, self-assessment tools and plagiarism detection software are just
some of the application areas where language technology can play an important
role. The popularity of social media applications like Twitter and Facebook
suggest a further need for sophisticated language technologies that can monitor
posts, summarise discussions, suggest opinion trends, detect emotional
responses, identify copyright infringements or track misuse.

\boxtext{Language technology helps overcome the “disability” of linguistic diversity.}

Language technology represents a tremendous opportunity for the European Union.
It can help address the complex issue of multilingualism in Europe – the fact
that different languages coexist naturally in European businesses,
organisations and schools. But citizens need to communicate across these
language borders criss-crossing the European Common Market, and language
technology can help overcome this final barrier while supporting the free and
open use of individual languages. Looking even further forward, innovative
European multilingual language technology will provide a benchmark for our
global partners when they begin to enable their own multilingual communities.
Language technology can be seen as a form of ‘assistive’ technology that helps
overcome the ‘disability’ of linguistic diversity and make language communities
more accessible to each other.

Finally, one active field of research is the use of language technology for
rescue operations in disaster areas, where performance can be a matter of life
and death: Future intelligent robots with cross-lingual language capabilities
have the potential to save lives.

\subsection{Challenges Facing Language Technology}

\boxtext{The current pace of technological progress is too slow.}
Although language technology has made considerable progress in the last few
years, the current pace of technological progress and product innovation is too
slow. Widely-used technologies such as the spelling and grammar correctors in
word processors are typically monolingual, and are only available for a handful
of languages. Online machine translation services, although useful for quickly
generating a reasonable approximation of a document’s contents, are fraught
with difficulties when highly accurate and complete translations are required.
Due to the complexity of human language, modelling our tongues in software and
testing them in the real world is a long, costly business that requires
sustained funding commitments. Europe must therefore maintain its pioneering
role in facing the technology challenges of a multiple-language community by
inventing new methods to accelerate development right across the map. These
could include both computational advances and techniques such as crowdsourcing.

\subsection{Language Acquisition in Humans and Machines}

To illustrate how computers handle language and why it is difficult to program
them to use it, let’s look briefly at the way humans acquire first and second
languages, and then see how language technology systems work.
\boxtext{Humans acquire language skills in two different ways: learning exam-ples and learning the underlying language rules.
Humans acquire language skills in two different ways: learning
examples and learning the underlying language rules. Babies acquire a
language by listening to the real interactions between its parents,
siblings and other family members. From the age of about two, children
produce their first words and short phrases. This is only possible
because humans have a genetic disposition to imitate and then
rationalise what they hear.

Learning a second language at an older age requires more effort, largely
because the child is not immersed in a language community of native speakers.
At school, foreign languages are usually acquired by learning grammatical
structure, vocabulary and spelling using drills that describe linguistic
knowledge in terms of abstract rules, tables and examples. Learning a foreign
language gets harder with age.
\boxtext{The two main types of language technology systems ‘acquire’ language capabilities in a similar manner.}
The two main types of language technology systems ‘acquire’ language
capabilities in a similar manner. Statistical (or ‘data-driven’) approaches
obtain linguistic knowledge from vast collections of concrete example texts.
While it is sufficient to use text in a single language for training, e.g., a
spell checker, parallel texts in two (or more) languages have to be available
for training a machine translation system. The machine learning algorithm then
“learns” patterns of how words, short phrases and complete sentences are
translated.

This statistical approach can require millions of sentences and performance
quality increases with the amount of text analysed. This is one reason why
search engine providers are eager to collect as much written material as
possible. Spelling correction in word processors, and services such as Google
Search and Google Translate all rely on statistical approaches. The great
advantage of statistics is that the machine learns fast in continuous series of
training cycles, even though quality can vary arbitrarily.

The second approach to language technology and machine translation in
particular is to build rule-based systems. Experts in the fields of
linguistics, computational linguistics and computer science first have to
encode grammatical analyses (translation rules) and compile vocabulary lists
(lexicons). This is very time consuming and labour intensive. Some of the
leading rule-based machine translation systems have been under constant
development for more than twenty years. The great advantage of rule-based
systems is that the experts have more detailed control over the language
processing. This makes it possible to systematically correct mistakes in the
software and give detailed feedback to the user, especially when rule-based
systems are used for language learning. But due to the high cost of this work,
rule-based language technology has so far only been developed for major
languages.

As the strengths and weaknesses of statistical and rule-based systems tend to
be complementary, current research focuses on hybrid approaches that combine
the two methodologies. However, these approaches have so far been less
successful in industrial applications than in the research lab.

As we have seen in this chapter, many applications widely used in today’s
information society rely heavily on language technology. Due to its
multilingual community, this is particularly true of Europe’s economic and
information space. Although language technology has made considerable progress
in the last few years, there is still huge potential in improving the quality
of language technology systems. In the following, we will describe the role of
Finnish in European information society and assess the current state of
language technology for the Finnish language.

\end{multicols}
\clearpage
\ssection{Finnish in the European Information Society}
\begin{multicols}{2}
\subsection{General Facts}


Finnish is the native language of approximately 4.8 million people living in
Finland and the second language of 0.5 million Finns. Finnish is also spoken in
Sweden, Estonia, Russia, the United States and Australia.
\boxtext{Finnish is one of the official languages in the European Union.}
Finnish is one of the official languages in the European Union. The Finnish
constitutional law and language law define Finnish and Swedish as the national
languages of Finland. In addition to that, Finnish is an official minority
language in Sweden. (In 2011 mainly in Northern and Central Sweden.) Besides
Finnish and Swedish, three Sámi languages (Northern Sámi, Inari Sámi and Skolt
Sámi), Romany, Karelian language and two different sign languages have long
been used in Finland. From the 19\textsuperscript{th} century onwards also Russian and Tatar
speaking people have been living in Finland. Since the end of the 1970’s
immigrants have arrived from Europe, Asia and Africa, and the amount of
immigrant languages is somewhere around 100, the major ones being Russian,
Estonian and Somali.

The Finnish literary language has a relatively short history. It has been used
in religious literature and the church since the 16\textsuperscript{th} century, and laws have
been written in Finnish since the 18\textsuperscript{th} century. Up until the 19\textsuperscript{th} century,
Swedish was used in administration, education and literature. The foundation of
contemporary Finnish was laid during the 19\textsuperscript{th} century when Finnish became a
sovereign language in all societal activity.

Finnish dialects are divided into two main categories; the Western and the
Eastern dialects. The Western dialects include the South-West dialects,
Southern-Western middle dialects, Tavastian dialects, Southern Ostrobothnian
dialect, Central and Northern Ostrobothnian dialects and the Peräpohjola
dialects. The Eastern dialects include the Savonian dialects and the
South-Eastern dialects. The difference between the Eastern and Western dialects
is mostly in the pronunciation and word forms
 (\textit{\foreignlanguage{finnish}{\textit{meijän}}},
  \textit{\foreignlanguage{finnish}{\textit{männä}}} in the East while
  \textit{\foreignlanguage{finnish}{\textit{meirän}}},
  \textit{\foreignlanguage{finnish}{\textit{mennä}}} in the West)
and partly in the vocabulary
 (\textit{\foreignlanguage{finnish}{\textit{vasta}}} in the East,
  \textit{\foreignlanguage{finnish}{\textit{vihta}}} in the West.)
The differences between dialects are clear, and speakers
from different areas can be identified by their intonation. However, the
differences are minor enough to allow speakers of different dialects to
understand each other. Urbanization and other changes in the society have
softened the dialects and smoothed out the most narrow and distinctive
features.

\subsection{Particularities of the Finnish Language}

Finnish is a part of the Finno-Ugric language group and it is one of the Baltic
Finnic languages. The other Baltic Finnic languages are Karelian, Ludic, Veps,
Ingrian, Votic, Estonian, Livonian, Võro and Seto. These languages do not
contain grammatical gender or articles.

One of the most distinctive features in Finnish is that the writing mainly
corresponds to the pronunciation. The main word stress is on the first
syllable.

\boxtext{Finnish has a rich inflectional system.}

Finnish has a rich inflectional system. Words are divided into three main
categories: 1) nouns and adjectives are inflected for case and number and
adjectives agree with their head
(\textit{\foreignlanguage{finnish}{\textit{isossa talossa}} [in a big house]},
 \textit{\foreignlanguage{finnish}{\textit{isoissa taloissa}}
         [in big houses]}),
2) verbs are inflected for person, tense and modus
(\textit{\foreignlanguage{finnish}{\textit{sanon}} [I say]},
 \textit{\foreignlanguage{finnish}{\textit{sanot}} [you say]},
 \textit{\foreignlanguage{finnish}{\textit{hän sanoo}} [he says]},
 \textit{\foreignlanguage{finnish}{\textit{sanomme}} [we say]},
 \textit{\foreignlanguage{finnish}{\textit{sanotte}} [you say]},
 \textit{\foreignlanguage{finnish}{\textit{he sanovat}} [they say]};
 \textit{\foreignlanguage{finnish}{\textit{sanon}} [I say]},
 \textit{\foreignlanguage{finnish}{\textit{sanoin}} [I said]},
 \textit{\foreignlanguage{finnish}{\textit{olen sanonut}} [I have said]},
 \textit{\foreignlanguage{finnish}{\textit{olin sanonut}} [I had said]};
 \textit{\foreignlanguage{finnish}{\textit{sanon}} [I say]},
 \textit{\foreignlanguage{finnish}{\textit{sanoisin}} [I would say]})
 and 3) adpositions and particles are mainly uninflected. There are 15
grammatical cases of which accusative only occurs in personal pronouns and the
pronoun
\textit{\foreignlanguage{finnish}{\textit{kuka}} [who]} %%% meaning who
 (\textit{\foreignlanguage{finnish}{\textit{minut}} [me]},
  \textit{\foreignlanguage{finnish}{\textit{meidät}} [us]},
  \textit{\foreignlanguage{finnish}{\textit{kenet}} [whom]}).

Each noun in Finnish is capable of having some 2,000 distinct forms
and verbs more than 12,000 forms. The number of distinct forms derives
from the agglutinative nature of Finnish, i.e. several layers of
inflectional affixes can be stacked,
e.g. \textit{halu}+\textit{tu}+\textit{imm}+\textit{i}+\textit{lla}+\textit{mme}+\textit{ko}
would express the verbal root for desiring and endings for expressing
elements “desire, something that is, most, on, our, question”.

New words in Finnish are mostly formed with derivation and composition.
Approximately 10–15 percent out of index words in dictionaries are basic words,
20–30 percent derivatives and 60–70 percent compounds.
\begin{itemize}
\item Derivatives:
   \textit{\foreignlanguage{finnish}{\textit{kirja}} [book]} $\to$
   \textit{\foreignlanguage{finnish}{\textit{kirjasto}} [library]},
   \textit{\foreignlanguage{finnish}{\textit{kirjaamo}} [registry]},
   \textit{\foreignlanguage{finnish}{\textit{kirjallisuus}} [literature]},
   \textit{\foreignlanguage{finnish}{\textit{kirjoittaa}} [to write]},
   \textit{\foreignlanguage{finnish}{\textit{kirjanen}} [booklet]},
   \textit{\foreignlanguage{finnish}{\textit{kirjallinen}} [literary]} etc.

\item Compounds:
   \textit{\foreignlanguage{finnish}{\textit{maahanmuutto}} [immigration]},
   \textit{\foreignlanguage{finnish}{\textit{kansaneläkelaitos}}
           [Social Insurance Institution]},
   \textit{\foreignlanguage{finnish}{\textit{yleisurheilumaaottelu}}
           [international event in athletics]}.
\end{itemize}
\boxtext{Certain linguistic characteristics of Finnish are challenges for computational processing.}
In addition to the stacking of the endings, Finnish is characterised by a
number of morphophonological alternations such as consonant gradation, vowel
harmony, a number of vowel mutations at the junctures between stems and
endings. Word forms are long because of inflection but also because compound
words are normally written together without hyphens or spaces. Compound words
can be further compounded resulting in even longer compounds.

The most usual order of constituents in a Finnish clause is SVX,
\textit{\foreignlanguage{finnish}{\textit{Hän osti kirjan.}}
        [He bought a book.]}
The word order, however, follows the
information structure of the clause and can therefore be employed to denote the
distinction of old and new information:
\begin{itemize}
\item \textit{\foreignlanguage{finnish}{\textit{Hän osasi läksynsä.}}
              [He mastered his homework.]}

\item \textit{\foreignlanguage{finnish}{\textit{Osasi hän läksynsä.}}
              [He did master his homework.]}
\end{itemize}

Syntactic roles are marked using inflectional marking. Therefore the word order
is relatively free, i.e. subjects and objects are identified by their case
rather than their relative position in the sentence:
\begin{itemize}
\item \textit{\foreignlanguage{finnish}{\textit{Poika osti kirjan.}}
              [The boy bought a book.]}

\item \textit{\foreignlanguage{finnish}{\textit{Kirjan poika osti.}}
              [It was a book that the boy bought.]}
\end{itemize}


\subsection{Recent Developments}

Finnish has a fairly short written history starting from religious texts
translated from German in the beginning of the New Age. The literary norm of
Finnish was, however, not established until in the 19\textsuperscript{th} century. Until the
Second World War, Finnish borrowed mostly from Swedish and German or Latin. The
vocabulary has only a small proportion of original Finno-Ugric words.

The Finnish vocabulary includes a large number of Baltic, German, Slavic and
Scandinavian loan words from different historical periods. For centuries a
strong influence came from Swedish
 ({\foreignlanguage{finnish}{\textit{pankki}}} < \textit{bank [bank]},
  {\foreignlanguage{finnish}{\textit{laki}}} < \textit{lag [law]},
  {\foreignlanguage{finnish}{\textit{treenata}}} < \textit{träna [to train]}).
Nowadays words are mostly borrowed from English
 ({\foreignlanguage{finnish}{\textit{liisaus}}} < \textit{leasing},
  {\foreignlanguage{finnish}{\textit{meili}}} < \textit{mail}),
although special languages also lend from
other languages
 (\textit{\foreignlanguage{finnish}{\textit{pitsa}} [pizza]},
  \textit{\foreignlanguage{finnish}{\textit{karate}}}).
It is typical for the loan words to
assimilate quickly to the Finnish structure and inflection conventions. Loan
words often live side by side with the Finnish variants:
\textit{\foreignlanguage{finnish}{\textit{tulostin}} $\sim$
        \foreignlanguage{finnish}{\textit{printteri}} [printer]}.

A new kind of influence from English has lately been recognised. The usage of
Finnish in some areas of life has been narrowed down, leaving Finnish less
often used. This tendency can be more clearly seen in natural sciences and
technology, but it is there also in other scientific forums. The scientific
community is also more aware of the fact that Finnish requires more attention
than during the past centuries.

The relationship between spoken and written language is also in a state of
change. It is usual to publish texts in the Internet that actually are speech.
Conventions of speech are therefore moving towards becoming part of the written
language much stronger than before.

\subsection{Language Cultivation in Finland}

\boxtext{The acts and degrees state that the language of Finnish is the task the Research Institute for the Languages in Finland.}

The acts and degrees state that the language planning of Finnish is the task of
the Research Institute for the Languages in Finland. The institute gives out
recommendations, offers counselling, educates, and collects and administers
up-to-date databases of Finnish. Counselling has long traditions and is widely
known amongst Finns. Language planning in Finland is all the more moving
towards counselling on the textual level, although details on spelling and
inflections are issues where the Finns still ask for advice.

The Finnish Terminology Centre TSK is one of the central developers of
terminology in Finnish, and work on terminology is also carried out in several
scientific societies. At the onset of 2011, the University of Helsinki launched
the project The Bank of Finnish terminology in Arts and Sciences, with the
objective of enhancing the creation and wide use of Finnish scientific terms.

Interest in the quality and intelligibility of the language used by the
authorities has grown during the 21st century. Cooperating closely with the
legislators, the Research Centre for Languages in Finland has made several
initiatives in suggesting improvements in the discourse of the authorities.

\subsection{Language in Education}


\boxtext{Language skills are a key qualification for education.}
Approximately 56 000 children start in the Finnish comprehensive schools each
year in an integrated nine-year school system. The Finnish language plays an
important part of the studies in all grades. The total amount of lesson hours
is defined in the national decree, but how the lesson hours are divided between
different grades is decided locally. During the nine years in the comprehensive
school the Finnish students attend 1554 hours teaching of their mother tongue
and literature.
\boxtext{Finland has taken part in all PISA cycles, in 2000, 2003, 2006 and 2009.}
Finland has taken part in all PISA cycles, in 2000, 2003, 2006 and
2009. The results of the tests show that the basic education has been
a Finnish success story, even if the difference in the level of
performance of girls and boys is the largest of all PISA countries
\cite{Literacy}. In 2009 with reading literacy as the main focus area,
the mean performance of Finnish students was ranked third, following
the trend of all previous PISA cycles \cite{Pisa2006}. The Finnish
language has been offered as one plausible explanation for the
excellent results, because it is easy to read, and children learn to
read subtitles on TV very early since there is no dubbing in the
TV. Reading is also supported by other means, such as creating a dense network of
libraries and a wide variety of newspapers made available for all age
groups.

During the three to four upper secondary education years, the
students (aged 16-19) attend six mandatory courses and they can also
choose three advanced courses in mother tongue and literature. The
mother tongue is a mandatory subject in the matriculation examination
after which the students are eligible for higher education studies
where they have two options to choose from, the more professionally
oriented polytechnic education or the university education where the
focus is mostly on scientific research. Approximately 36 000 students
enrol in the polytechnics and 20 000 in the universities each year
\cite{Education}. The curriculum of all 26 polytechnics and 16
universities include mandatory courses in mother tongue and in
communication.

The students in Finland study Finnish at the upper secondary
school level less than students in other OECD countries, and
taking extra classes in Finnish language studies or literature is not very
popular, even if the subject is regarded as important. The work group behind the report
\textit{\foreignlanguage{finnish}{Suomen kielen tulevaisuus}} (The
future of the Finnish language) \cite{Tulevaisuus2009} recommends that
the course tray should also include studies improving other that text
production skills or literary studies, such as more formal and linguistic
approaches to languages.

Finnish can be studied as the major in 8 out of 15 Finnish
universities: the universities of Helsinki, Jyväskylä, Oulu, Tampere,
Turku, Vaasa, Eastern Finland, and Åbo Akademi, and Finnish literature
in the first six of these \cite{hum-ulko}. In several other
universities it is possible to study individual courses of
Finnish. The role of English in the universities overall has grown
with the increasing number of international students but the language
of instruction in the degree programmes in Finland studies is mostly
Finnish \cite{Board}.

\subsection{International Aspects}

\boxtext{Until the late 20th century Finnish was a receiving language for influences from other languages in international settings. World literature, as well as scientific achievements has been available for the Finns through translations.}	
Until the late 20th century Finnish was a receiving language for
influences from other languages in international settings. World
literature, as well as scientific achievements has been available for
the Finns through translations. Also translations of popular
culture such as lyrics had a strong status in Finland until the
1990's. A strong tradition of translating with a habit of reading and
listening to translated language has thus been rooted into Finland.
The last few decades have, however, witnessed a change in this respect
with the growing importance of the internet multiplying the usage of
texts and other cultural works in other languages than Finnish, most
often English.

Translating from Finnish into other languages has also been important
for the Finns. Finnish has rarely been an option in international
business contacts, nor scientific interaction, and translating Finnish
source texts has always been necessary. Although Finnish is offered in
several universities around the world, it is more often studied rather
for personal than professional reasons.  With the increasing number of
international contacts, the situation of translating from Finnish has
changed, since Finns nowadays use more often foreign languages, mostly
English, in producing texts. Some large Finnish enterprises have
chosen English as the concern/consolidated company language.

The status of Finnish faced a significant shift when Finland joined
the European Union in 1995. For the first time in its history, Finnish
became one of the official languages of an international
organisation. While Finnish is not one of the working languages in the
EU, Finland participates in the activities in the EU as well as in
other international settings through translation and
interpretation \cite{Tulevaisuus2009}. The number of texts and genres
translated are very different from the translation activities in the
past, as texts in the EU are translated into Finnish from the working
languages, most often English.  Among the genres translated, the EU
legislation enjoys a special status. When the Finns wish to contact
the institutions of the EU, the texts are translated from Finnish into
the working languages, but the number of texts translated from Finnish
into other languages is quite small.

Speeches of the Finnish representatives and officials are interpreted
from Finnish or into Finnish. Interpretation services have, however,
not been used as often as would have been possible, especially in the
meetings the Finnish officials participate in. In 2003 the EU changed
the way the costs for interpretations are covered by the member
states, and it has since been possible to finance other costs by
saving in the interpretation costs, an option that Finland chose to
take.

The fact that Finns use less interpretation services than before might
have an impact on how they tend to react to the EU translations in
general. Finns tend to read the texts prepared for the meetings in
English, and they often choose to speak English in them. Half of the
officials that answered a poll on interpretation gave a negative
answer to a question whether they get interpretation services as often
as they would wish. The same officials consider the Finnish versions
of the EU texts as harder to understand than the versions of the same
texts in other languages, or similar texts written in Finnish
\cite{Piehl2008}. Linguistic problems occur in the national
implementation process of the EU acts \cite{OECD2010}. A network
for translation of the EU acts has been established to enhance
cooperation between the EU translators EU and the officials.

An issue in the question of whether to request interpretation in the
EU or not can possibly be the fact that knowledge of foreign languages
is very highly appreciated in Finland. The media pay attention to the
language skills of the politicians, such as ministers of parliament
and how they cope with speaking English. Using Finnish is easily
regarded as not being competent in the particular foreign language
instead of a sign of appreciation towards Finnish and its status as
one of the official languages in the EU. The bond between the usage of
Finnish and its development does apparently not appear as relevant to
those choosing English for pragmatic reasons: the more specialists use
Finnish, the better and more idiomatic expressions are available for
its users - and vice versa.

Language technology could be more widely employed than it currently
is. Large and up-to-date databases of terms and phrases in
administration with solid quality assurance are an example of a
welcomed resource to both translators and interpreters. Machine
translation into or from Finnish would require more effort to reach a
level of quality that would benefit translation activities in
practice.
\subsection{Finnish on the Internet}

\boxtext{There were almost 1.5 million broadband subscriptions and more than a million wireless subscriptions in Finland in 2009.}	
Between 2000 and 2009, the number of households using computers at
home in Finland has risen steadily from 47 percent in 2000 to 81
percent in 2009 \cite{OECD-ICT}. For the wired broadband
subscriptions, Finland ranked 15 out of 31 countries in 2009, with the
total of 1,407,500 subscriptions \cite{OECD-wired} and for the
wireless subscriptions, Finland ranked 20 out of 29 countries, a total
of 1,182,300 subscriptions \cite{OECD-mobile}.

The Finns are active users of the Internet. According to Statistics
Finland 86 per cent of the population use the Internet, and the elders
seem to pace up in this development surprisingly fast, the growth in
the statistics for the age groups 64 to 74 was 10 percent in one
year. Most Finns (72 percent) use the Internet on a daily basis for
banking (76 percent), for maintaining social contacts via email (77
percent), and for looking up information on products and goods (74
percent). It is also usual to search for information on the
authorities and the services provided, and more and more people send
forms filled with information required for the authorities via the
Internet. 74 per cent of the population watch news or TV programmes in
the Internet \cite{SVT}.

The National Library of Finland documents the contents of the Finnish web
sites. This task is statutory. The library has also as one of its tasks to
digitise printed matter and it reports that the number of digitised pages in
2010 was 1,064,000. The FinElib library containing electronic articles and
other licensed materials was during one year visited 68,900,000 times with
19,6000,000 items downloaded by the users \cite{natlibstat}.

Social media is rapidly gaining ground in Finland. In 2010, 42 percent
of Finns aged 16 to 74 have registered as a user in at least one of the
community based applications (Facebook, Twitter etc.) Two thirds of
them visit the groups daily. According to Google Analytics, the most
popular search all in all since 2004 in Finland is Facebook, with
YouTube on the second place followed by two local tabloid papers
Iltalehti and Iltasanomat. Discussion groups like irc and suomi24 are
also popular with frequent searches at all times. Alexa reports Google
as the top site in Finland, which means that the other search engines
have not gained much ground in Finland \cite{topsites}.

The Finnish Communication Regulatory Authority (Ficora) keeps the
records of the registration of .fi -domains in Finland, and it is
possible to follow the development of the registrations within a
certain period of time. For example in January 2000, about 10 years
ago, a total of 357 new .fi -domains were registered, whereas in 2011,
a total of 164 new .fi -domains were registered on April 5th alone,
i.e. during one day only. There are now more than 270,000 registered
.fi -domains. Based on the Google-results (April 5, 2011) the number
of other web sites besides the .fi-domains is approximately
110,000. That would add up to an estimate of almost 300, 000 Finnish
web sites altogether.

For Language Technology, the growing importance of the Internet is
important in two ways. On the one hand, the large amount of digitally
available language data represents a rich source for analysing the
usage of natural language, in particular by collecting statistical
information. On the other hand, the Internet offers a wide range of
application areas for Language Technology.

The most commonly used web application is certainly Web Search, which
involves the automatic processing of language on multiple levels, as
we will see in more detail the second part of this paper. It involves
sophisticated Language Technology, differing for each language. For
Finnish, this includes coping with polysemy
i.e. words denoting the same thing, e.g.
\textit{\foreignlanguage{finnish}{\textit{kuusi}}} (six) or
\textit{\foreignlanguage{finnish}{\textit{kuusi}}} (spruce tree).

It is an expressed political aim in Finland as well as other European
countries to ensure equal opportunities for everyone. As early as 1998
the Sitra, the Finnish Innovation Fund, published a report “Kohti
esteetöntä tietoyhteiskuntaa” (Towards a barrier-free information
society free from barriers), stating that the information society
shall be open for all citizens who wish to access services, information
and entertainment, act interactively in the internet,
participate in the decision making and the society, communicate and
participate also while mobile, develop oneself, and work at any time
and in any place. The report highlights the possibilities of technology in
providing support for the disabled in coping with everyday tasks but
it also states that in Finland the know-how in 1997 was still
scattered and not enough practical solutions and products emerge to
answer the demand both in the national and the international
markets today. Language technology has provided valuable aids such as speech
synthesizer and Braille screen, an optical reader with a
text-to-speech application will make it possible for a visually
impaired person to read or listen to newspapers. Making the
barrier-free society happen requires political commitment, cooperation
and interaction between the relevant players \cite{Sitra1998}.

The public agencies need to make sure that their web sites and internet
services can be used by the disabled without restrictions. User-friendly
language technology tools offer the principal solution to satisfy this
regulation, for example by offering speech synthesis for the blind.

Internet users and providers of web content can also profit from
Language Technology in less obvious ways, e.g., if it is used to
automatically translate web contents from one language into
another. Considering the high costs associated with manually
translating these contents, comparatively little usable Language
Technology is developed and applied, compared to the anticipated
need. This may be due to the complexity of the Finnish language and
the number of technologies involved in typical Language Technology
applications. In the next chapter, we will present an introduction to
Language Technology and its core application areas as well as an
evaluation of the current situation of Language Technology support for
Finnish.

\end{multicols}
\clearpage
\ssection{Language Technology Support for Finnish}
\begin{multicols}{2}
Language technologies are software systems designed to handle
human language and are therefore often called “human language
technology”. Human language comes in spoken and written forms. While
speech is the oldest and in terms of human evolution the most natural
form of language communication, complex information and most human
knowledge is stored and transmitted in written texts. Speech and text
technologies process or produce these different forms of language,
though they both use dictionaries and rules of grammar and semantics.
This means that language technology (LT) links language to various
forms of knowledge, independently of the media (speech or text) it is
expressed in. Figure \ref{X} illustrates the LT
landscape. When we communicate, we combine language with other modes
of communication and information media – for example speaking can
involve gestures and facial expressions. Digital texts link to
pictures and sounds. Movies may contain language in spoken and written
form. In other words, speech and text technologies overlap and
interact with other technologies that facilitate processing of
multimodal communication and multimedia documents.

In the following, we will discuss the main application areas of
language technology, i.e., language checking, web search, speech
technology, and machine translation. This includes
applications and basic technologies such as
\begin{itemize}
\item spelling correction

\item authoring support

\item computer-assisted language learning

\item information retrieval

\item information extraction

\item text summarization

\item question answering

\item speech recognition

\item speech synthesis
\end{itemize}

Before discussing the above application areas, we will shortly
describe the architecture of a typical LT system.

\subsection{Application Architectures}

Software applications for language processing typically consist of
several components that mirror different aspects of language.
Figure \ref{X} shows a highly simplified architecture that can be found in a
typical text processing system. The first three modules handle the
structure and meaning of the text input:

\begin{enumerate}
\item Pre-processing: cleans the data, analyses or removes formatting,
detects the input language, and so on.

\item Grammatical analysis: finds the verb, its objects, modifiers and
other parts of speech as well as detects the sentence structure.

\item Semantic analysis: performs disambiguation (i.e., computes the
appropriate meaning of words in a given context); resolves anaphora
(i.e., which pronouns refer to which nouns in the sentence) and
substitute expressions; and represents the meaning of the sentence in
a machine-readable way.
\end{enumerate}
After analysing the text, task-specific modules can perform other
operations, such as automatic summarization and database
look-ups. This is a simplified and idealised description of the
application architecture and illustrates the complexity of LT
applications.

After introducing the core application areas for language technology,
we shall provide a brief overview of the state of LT research and
education today, and end with an overview of past and present research
programmes. We shall then present an expert estimate of core LT tools
and resources in terms of various dimensions such as availability,
maturity and quality. The general situation of LT for the Finnish
language is then summarised in tabular form.

\subsection{Keskeiset sovellusalat --- Core Application Areas}

In this section, we focus on the most important LT tools and
resources, and give an overview of LT activities in Finland. Tools and
resources that are \underline{underlined} in the text can also be
found in the summary table.


\subsubsection{Language Checking}


Anyone who has used a word processor knows that a spelling checker highlights 
spelling mistakes and proposes corrections.  The first spelling correction programs compared
a list of extracted words against a dictionary of correctly spelled
words. Today these programs are far more sophisticated. Using
language-dependent algorithms for \underline{grammatical analysis},
they detect errors
related to morphology (e.g., plural formation) as well as
syntax–related errors, such as a missing verb or a conflict of
verb-subject agreement (e.g.,
\textit{\foreignlanguage{finnish}{\textit{me *kirjoittaa kirjeen}}}
 [a similar concept in English would be
\textit{she *write a letter}]).
But most spell checkers will not find any errors in the
following text:
\begin{itemize} % abuse of itemize
\item[] \textit{I have a spelling checker,}

\item[] \textit{It came with my PC.}

\item[] \textit{It plane lee marks four my revue}

\item[] \textit{Miss steaks aye can knot sea.} \cite{Surprise}
\end{itemize}
Handling these kinds of errors usually requires an analysis of the
context. For example: if a word needs to be written in upper case in
Finnish or not:
\begin{itemize}
\item[] {\foreignlanguage{finnish}{\textit{Muista ottaa kaneli mukaan.}}} \\
        \textit{[Remember to take the cinnamon with you.]}
\item[] {\foreignlanguage{finnish}{\textit{Muista ottaa Kaneli mukaan.}}} \\
        \textit{[Remember to take Kaneli with you.]}
\end{itemize}
This type of analysis either needs to draw on language-specific
\underline{grammars} laboriously coded into the software by experts, or on a
statistical language model. In this case, a model calculates the
probability of a particular word as it occurs in a specific position
(e.g., between the words that precede and follow it). For example,
 {\foreignlanguage{finnish}{\textit{kaneli}}} is a much more probable
 as a noun than a proper noun
 {\foreignlanguage{finnish}{\textit{Kaneli}}}.
A statistical language model can be automatically created by using a
large amount of (correct) language data (called a \underline{text corpus}).
Most
of these two approaches have been developed around data from
English. Neither approach can transfer easily to Finnish because the
language has a flexible word order, unlimited compound building and a
richer inflection system.

\boxtext{The use of language checking is not limited to word processors; it also applies to authoring support systems.}

Language checking is not limited to word processors; it is also used
in “authoring support systems”, i.e., software environments in which
manuals and other documentation are written to special standards for
complex IT, healthcare, engineering and other products. Fearing
customer complaints about incorrect use and damage claims resulting
from poorly understood instructions, companies are increasingly
focusing on the quality of technical documentation while targeting the
international market (via translation or localization) at the same
time. Advances in natural language processing have led to the
development of authoring support software, which helps the writer of
technical documentation use vocabulary and sentence structures that
are consistent with industry rules and (corporate) terminology
restrictions.

% \textbf{[[[Figure 4. Web search architecture, somehow here. A mess
% that did not fit in the margin even in the monolingual document.]]]}

%                                     Search
% 
%                                     Results
% 
%                               Semantic Processing
% 
%                                      Query
% 
%                                    Analysis
% 
%                                       Web
% 
%                                      pages
% 
%                                 Pre-processing
% 
%                                   User query
% 
%                                 Pre-processing
% 
%                                    Indexing
% 
%                              Matching \& Relevance

Finnish has a history of several small Finnish companies and Language
Service Providers developing products based on various language
models. Finnish is a challenging language to model, or as Antti Arppe
put it in 2002: "Whereas in English one can in principle create a
prototypical language engineering tool such as a simple spell-checker
by merely listing and compressing the most common 100,000 words or so,
in Finnish one would need to list tens if not hundreds of millions of
word forms to create a speller with comparable coverage using the same
technique." \cite{NoPath} Since the late 1980's, there has been a
series of language proofing tools from available from Kielikone,
nowadays specializing in dictionaries, Connexor specializing in
language analysis tools, Gurusoft specializing in SOM-applications,
and Lingsoft offering a wide selection of tools,
including hyphenation and proofreading for Finnish.

Besides spell checkers and authoring support, language checking is
also important in the field of computer-assisted language
learning. And language checking applications also automatically
correct search engine queries, as found in Google's \textit{Did you
mean\dots} suggestions.

\subsubsection{Web Search}


Searching the Web, intranets or digital libraries is probably the
most widely used yet largely underdeveloped language technology
application today. The Google search engine, which started in 1998,
now handles about 80\% of all search queries \cite{Spiegel}. The verb
\textit{\foreignlanguage{finnish}{\textit{guuglata}}} is used in everyday
speech in Finnish although there is no conventional way to spell it
yet. The Google search interface and results page display has not
significantly changed since the first version. Yet in the current
version, Google offers spelling correction for misspelled words and
has now incorporated basic semantic search capabilities that can
improve search accuracy by analysing the meaning of terms in a search
query context \cite{Google-rolls}. The Google success story shows that
a large volume of available data and efficient indexing techniques can
deliver satisfactory results for a statistically-based approach.

For more sophisticated information requests, it is essential to
integrate deeper linguistic knowledge to \underline{semantic
analysis}. Experiments using \underline{lexical resources} such
as machine-readable thesauri or ontological language resources (e.g.,
WordNet for English or the equivalent Finnish FinnWordNet) have
demonstrated improvements in finding pages using synonyms of the
original search terms, such as
{\foreignlanguage{finnish}{\textit{atomienergia}}} \textit{[atomic energy]},
{\foreignlanguage{finnish}{\textit{ydinvoima}}} \textit{[atomic power]} and
{\foreignlanguage{finnish}{\textit{ydinenergia}}} \textit{[nuclear energy]},
or even more loosely related terms.

\boxtext{The next generation of search engines will have to include much more sophisticated language technology.}

The next generation of search engines will have to include much more
sophisticated language technology, in particular in order to deal with
search queries consisting of a question or other sentence type rather
than a list of keywords. For the query, “Give me a list of all
companies that were taken over by other companies in the last five
years,” the LT system needs to analyse the sentence syntactically and
semantically as well as provide an index to quickly retrieve relevant
documents. A satisfactory answer will require syntactic parsing to
analyse the grammatical structure of the sentence and determine that
the user wants companies that have been acquired, not companies that
acquired other companies. For the expression \textit{last five years},
the system needs to determine the relevant years. And, the query needs
to be matched against a huge amount of unstructured data to find the
piece or pieces of relevant information the user wants. This is called
“information retrieval”, and involves searching and ranking relevant
documents. To generate a list of companies, the system also needs to
recognise a particular string of words in a document as a company
name, a process called “named entity recognition”.

A more demanding challenge is matching a query in one language with
documents in another language. Cross-lingual information retrieval
involves automatically translating the query into all possible source
languages and then translating the results back into the target
language.

Now that data is increasingly found in non-textual formats, there is a
need for services that deliver multimedia information retrieval by
searching images, audio files and video data. In the case of audio and
video files, a speech recognition module must convert the speech
content into text (or into a phonetic representation) that can then be
matched against a user query.

In Finland, there are few small and medium size enterprises to actively develop and apply search
technologies at the moment, although Gurusoft specializes in applying
language independent Self-organizing maps (SOM methods) to information
retrieval tasks, but the product Docunaut is designed to apply the method in searches within the intranets of their customers instead of the world wide web. At present
there are no ongoing large-scale Finnish language search engine
projects.

\subsubsection{Speech Technology}

\boxtext{Speech technology is the basis for creating interfaces that allow a user to interact with spoken language instead of a graphical display, keyboard and mouse.}

Speech technology is used to create interfaces that enable users to
interact in spoken language instead of a graphical display, keyboard and mouse.
Today, voice user interfaces (VUI) are used for partially or
fully automated telephone services provided by companies to customers,
employees or partners. Business domains that rely heavily on VUIs include
banking, supply chain, public transportation, and telecommunications. Other
uses of speech technology include interfaces to car navigation systems and the
use of spoken language as an alternative to the graphical or touch-screen
interfaces in smartphones.

Speech technology comprises four technologies:
\begin{enumerate}
\item Automatic \underline{speech recognition} (ASR) determines which words are actually
    spoken in a given sequence of sounds uttered by a user.

\item Natural language understanding analyses the syntactic structure of a user’s
    utterance and interprets it according to the system in question.

\item Dialogue management determines which action to take given the user input
    and system functionality.

\item \underline{Speech synthesis} (text-to-speech or TTS) transforms the system’s reply into
    sounds for the user.
\end{enumerate}

One of the major challenges of ASR systems is to accurately recognise the words
a user utters. This means restricting the range of possible user utterances to
a limited set of keywords, or manually creating language models that cover a
large range of natural language utterances. Using machine learning techniques,
language models can also be generated automatically from \underline{speech corpora}, i.e.,
large collections of speech audio files and text transcriptions. Restricting
utterances usually forces people to use the voice user interface in a rigid way
and can damage user acceptance; but the creation, tuning and maintenance of
rich language models will significantly increase costs. VUIs that employ
language models and initially allow a user to express their intent more
flexibly — prompted by a \textit{How may I help you?} greeting — tend to be automated
and are better accepted by users.

Companies tend to use pre-recorded utterances by professional speakers for
generating the output of the voice user interface. For static utterances where
the wording does not depend on particular contexts of use or personal user
data, this can deliver a rich user experience. But more dynamic content in an
utterance may suffer from unnatural intonation because bits of audio files have
simply been strung together. Today’s TTS systems are getting better (though
they can still be optimised) at producing natural-sounding dynamic utterances.

Interfaces in the market for speech technology have been
considerably standardised during the last decade in terms of their various
technology components. There has also been strong market consolidation in
speech recognition and speech synthesis. The national markets in the G20
countries (economically resilient countries with high populations) have been
dominated by just five global players, with Nuance (USA) and Loquendo (Italy)
being the most prominent players in Europe. In 2011, Nuance announced the
acquisition of Loquendo, which represents a further step in market
consolidation.

Research in speech technology has been undertaken in Finland as early as the
1960s, with some results having an international renown or impact such as the
portable Synte 2 speech synthesizer, developed by the Acoustics Laboratory in
the 1970s and the phonetic typewriter in the 1980s, both developed at the
University of Technology (currently Aalto University). There have also been
some individual speech products on the market since the early 1990s; however,
their clientele have been limited mainly to special groups such as the visually
impaired. After the turn of the millennium a clear change has been witnessed.
Both the public and the private sectors have embarked on major research and
development projects in speech technology, which are starting to bear fruit -
there now exist several basic technological solutions for both speech
recognition and synthesis of Finnish that are on par with any language. Most
speech technology companies working on a global level offer both TTS and ASR
for Finnish. Two Finnish companies (Bitlips Oy and Timehouse Oy) offer Finnish
TTS. Bitlips also has English, Finland Swedish and Welsh synthesis. Lingsoft Oy
and Suomen Puheentunnistus Oy both have Finnish ASR systems and provide VUI
services for several Finnish corporations.

There are currently several major research projects on speech on both TTS and
ASR in Finland. The bulk of the research is done at Aalto University,
University of Helsinki, Tampere University of Technology. The main industrial
contributor to speech research in Finland has traditionally been Nokia.

Regarding dialogue management technology and know-how, there exist no SMEs
offering products in these areas. Finally, within the domain of Speech
Interaction, a genuine market for the linguistic core technologies for
syntactic and semantic analysis does not exist yet.

Looking forward, there will be significant changes due to the spread of
smartphones as a new platform for managing customer relationships in addition
to fixed telephones, the Internet and e-mail. This will also affect how speech
technology is used. In the long run, there will be fewer
telephone-based VUIs and spoken language will play a far more central role as a
user-friendly input for smartphones. This will be largely driven by stepped
improvements in the accuracy of speaker-independent speech recognition via
speech dictation services already offered as centralised services to smartphone
users.

\subsubsection{Machine Translation}

\boxtext{At its basic level, Machine Translation simply substitutes words in one natural language with words in another language.}

The idea of using digital computers to translate natural languages goes back to
1946 and was followed by substantial funding for research during the 1950s and
again in the 1980s. Yet machine translation (MT) still cannot meet its initial
promise of across-the-board automated translation.

The most basic approach to machine translation is to automatically replace the
words in a text in one natural language by words in another language. This can
be useful in subject domains that have a very restricted, formulaic language
such as weather reports. But to produce a good translation of less standardised
texts, larger text units (phrases, sentences, or even whole passages) need to
be matched to their closest counterparts in the target language. The major
difficulty is that human language is ambiguous. Ambiguity creates challenges on
multiple levels, such as word sense disambiguation on the lexical level (a
\textit{jaguar} is a brand of car or an animal) or the assignment of case on the
syntactic level, for example:
\begin{itemize}
\item[] {\foreignlanguage{finnish}
         {\textit{Poliisi tarkkaili miestä mäellä.}}} \\
        \textit{[The policeman observed the man on the hill.]}

\item[] {\foreignlanguage{finnish}{\textit{Poliisi tarkkaili miestä
                kiikarilla.}}} \\
        \textit{[The policeman observed the man with binoculars.]}
\end{itemize}
One way to build an MT system is to use linguistic rules. For translations
between closely related languages, a direct substitution translation may be
feasible in cases like the above example. But, rule-based (or linguistic
knowledge-driven) systems often analyse the input text and create an
intermediary symbolic representation from which the text can be generated into
the target language. The success of these methods is highly dependent on the
availability of extensive lexicons with morphological, syntactic, and semantic
information, and large sets of grammar rules carefully designed by skilled
linguists. This is a very long and therefore costly process.

In the late 1980s when computational power increased and became cheaper, there
was more interest in statistical models for machine translation. Statistical
models are derived from analysing bilingual text corpora, such as the Europarl
\underline{parallel corpus}, which contains the proceedings of the European Parliament in
11 European languages. Given enough data, statistical MT works well enough to
derive an approximate meaning of a foreign language text by processing parallel
versions and finding plausible patterns of words. But unlike knowledge-driven
systems, statistical (or data-driven) MT often generates ungrammatical output.
Data-driven MT is advantageous because less human effort is required, and it
can also cover special particularities of the language (e.g., idiomatic
expressions) that can get ignored in knowledge-driven systems.

\boxtext{Machine Translation is particularly challenging for the Finnish language.}

The strengths and weaknesses of knowledge-driven and data-driven machine
translation tend to be complementary, so that nowadays researchers focus on
hybrid approaches that combine both methodologies. One approach uses both
knowledge-driven and data-driven systems together with a selection module that
decides on the best output for each sentence. However, results for sentences
longer than say 12 words will often be far from perfect. A better solution is
to combine the best parts of each sentence from multiple outputs; this can be
fairly complex, as corresponding parts of multiple alternatives are not always
obvious and need to be aligned.

Finland missed out on first generation machine translation, but caught the
second wave of rule-based machine translation in the 80’s. A long-term
nationally funded R\&D project Kielikone first developed the necessary Finnish
analysis tools and used them to build a rule-based Finnish-to-English MT system
in the 90’s that subsequently became a commercial product. IBM Finland
researched English-to-Finnish transfer based on the IBM English parser at the
turn of the 90’s but did not reach product stage. Sunda, a newer rule based
system developed from the Kielikone technology base, now sells relatively good
quality English-to-Finnish MT. Google and Microsoft provide statistical MT for
Finnish, but the quality remains poor, due to the complexity of Finnish
morphology and the free word order which current statistical MT is poorly
equipped for. The technical university has a group working on statistical
language modelling of Finnish, including Finnish morphology and SMT.

There is still a huge potential for improving the quality of MT
systems. The challenges involve adapting language resources to a given
subject domain or user area, and integrating the technology into
workflows that already have term bases and translation
memories. Another problem is that most of the current systems are
English-centred and only support a few languages from and into
Finnish. This leads to friction in the translation workflow and forces
MT users to learn different lexicon coding tools for different
systems.

Evaluation campaigns help compare the quality of MT systems, the
different approaches and the status of the systems for different
language pairs. Table \ref{X}, which was prepared during the EC
Euromatrix+ project, shows the pair-wise performances obtained for 22
of the 23 official EU languages. (Irish was not compared.). The
results are ranked according to a BLEU score, which indicates higher
scores for better translations\cite{BLEU}. A human translator would
achieve a score of around 80 points.

The best results (in green and blue) were achieved by languages that
benefit from a considerable research effort in coordinated programs
and from the existence of many parallel corpora (e.g., English,
French, Dutch, Spanish and German). The languages with poorer results
are shown in red. These languages either lack such development efforts
or are structurally very different from other languages (e.g.,
Hungarian, Maltese and Finnish).

% \textbf{[[[HUGE matrix missing here. Only two words there: Target
% Language. (Clearly "Source Language" should too be.)]]]}



% \textbf{Kuva Konekäännösjärjestelmien suorituskyvyn arviointi
%   kielipareittain Euromatrix+ -projektissa -- Performance of Machine
%   Translation for Language Pairs in the Euromatrix+
% Project}


\subsection{Other Application Areas}

\boxtext{Language technology applications often provide significant service functionalities “under the hood” of larger software systems.}

Building language technology applications involves a range of subtasks that do
not always surface at the level of interaction with the user, but they provide
significant service functionalities “under the hood” of the system in question.
They all form important research issues that have now evolved into individual
sub-disciplines of computational linguistics.

Question answering, for example, is an active area of research for which
annotated corpora have been built and scientific competitions have been
initiated. The concept of question answering goes beyond keyword-based searches
(in which the search engine responds by delivering a collection of potentially
relevant documents) and enables users to ask a concrete question to which the
system provides a single answer. For example:
\begin{itemize}
\item[] \textit{Question: How old was Neil Armstrong when he stepped on the
              moon?}

\item[] \textit{Answer: 38.}
\end{itemize}
While question answering is obviously related to the core area of web search,
it is nowadays an umbrella term for such research issues as what different
types of questions there are, and how they should be handled; how a set of
documents that potentially contain the answer can be analysed and compared (do
they provide conflicting answers?); and how specific information (the answer)
can be reliably extracted from a document without ignoring the context.

This is in turn related to information extraction (IE), an area that was
extremely popular and influential when computational linguistics took a
statistical turn in the early 1990s. IE aims to identify specific pieces of
information in specific classes of documents, such as detecting the key players
in company takeovers as reported in newspaper stories. Another common scenario
that has been studied is reports on terrorist incidents. The problem here is to
map the text to a template that specifies the perpetrator, target, time,
location and results of the incident. Domain-specific template-filling is the
central characteristic of IE, which makes it another example of a “behind the
scenes” technology that forms a well-demarcated research area that in practice
needs to be embedded into a suitable application environment.

\boxtext{For the Finnish language, research in most text technologies is much less developed than for the English language.}

Text summarization and \underline{text generation} are two borderline areas that can act
either as standalone applications or play a supporting role “under the hood”.
Summarization attempts to give the essentials of a long text in a short form,
and is one of the features available in Microsoft Word. It mostly uses a
statistical approach to identify the “important” words in a text (i.e., words
that occur very frequently in the text in question but less frequently in
general language use) and determine which sentences contain the most of these
“important” words. These sentences are then extracted and put together to
create the summary. In this very common commercial scenario, summarization is
simply a form of sentence extraction, and the text is reduced to a subset of
its sentences. An alternative approach, for which some research has been
carried out, is to generate brand new sentences that do not exist in the source
text. This requires a deeper understanding of the text, which means that so far
this approach is far less robust. On the whole, a text generator is rarely used
as a stand-alone application but is embedded into a larger software
environment, such as a clinical information system that collects, stores and
processes patient data. Creating reports is just one of many applications for
text summarization.

For the Finnish language, research in these text technologies is much
less developed than for the English language. Question answering,
information extraction, and summarization have been the focus of
numerous open competitions in the USA since the 1990s, primarily
organised by the government-sponsored organisations DARPA and
NIST. These competitions have significantly improved the
start-of-the-art, but their focus has mostly been on the English
language.  As a result, there are hardly any annotated corpora or
other special resources needed to perform these tasks in Finnish. When
summarization systems use purely statistical methods, they are largely
language-independent and a number of research prototypes are
available. For text generation, reusable components have traditionally
been limited to surface realization modules (generation grammars) and
most of the available software is for the English language.  }


\subsection{Educational Programmes}

Language Technology is a highly interdisciplinary field, involving the
expertise of linguists, computer scientists, mathematicians, philosophers,
psycholinguists, and neuroscientists, among others Language Technology has been
taught as a major subject at the University of Helsinki since 1994, and it has
been active in cooperation with other universities offering courses in the
neighbouring fields on both national and international level. The national
level includes the establishment of KIT Network for Language Technology Studies
in 2001 with 10 universities all over Finland participating in course exchange
and a common syllabus. The formal agreement between the universities ended in
2007 but the students enrolled in Finnish universities can apply for a grant
from their faculties to take Language Technology courses within the network.
The KIT Network universities include Aalto University, University of Eastern
Finland, University of Helsinki, University of Tampere, Technical University of
Tampere, University of Turku, University of Vaasa, University of Oulu, and Åbo
Akademi in Turku.

During 2006 - 2009 the students with a sufficient knowledge in Language
Technology could, after completing their BA, apply for a special master's
degree in Language Technology at the University of Helsinki. The master's
degree programme offered an option to focus on language technology, speech
technology or translation studies as a major. In 2009 the formal Master's
degree programme came to end with the new organisation structures taking place,
and it is now possible to apply to study advanced studies offered by the
language technology subject towards an MA in language technology.

The Graduate School of Language Technology in Finland (the KIT Graduate School)
was a multidisciplinary national graduate school, functioning during t2004 -
2009 as part of the emerging network of graduate schools of language technology
in the Nordic countries, Nordic Graduate School of Language Technology, NGSLT.
The KIT Graduate School was granted five PhD student positions for two
four-year periods 2002 - 2005 and 2006 - 2009. From the beginning of 2010 the
graduate school merged with LANGNET, the Finnish doctoral programme in language
studies, and became one of its programmes.

The education of language technology researchers in sufficient numbers
is nevertheless a prerequisite for the diverse research and thus the
development of successful commercial activity.\cite{FinExp}

\subsection{National Projects and Efforts}

The most important agencies for research funding in Finland are
the Academy of Finland financed by the Ministry of Education and
Culture and the Finnish Funding Agency for Technology and Innovation
(Tekes) financed by the Ministry of Trade and Industry \cite{Leading}.
Sitra, The Finnish National Fund for Research and Development had
provided funding for the MT project Kielikone in the 1980's Public
support from TEKES has been an important source of funding for basic
research especially through two large technology programs, USIX
(User-Oriented Information Technology) 1999 – 2002 and FENIX
(Interactive Computing) 2003 – 2007.

The USIX technology program aimed at raising the needs of the users and the
consumers of products and technologies by providing Finnish enterprises and
research institutions with funding for improving the quality of the products
and technologies. Some of the core technologies identified in the program were
Finnish speech recognition, large data management and search interfaces. The
program financed 181 projects with the total volume of 84 MEUR (44 MEUR
provided by Tekes) of which 29\% were research projects. Examples of NLP USIX
projects are WEBSOM developing Self-Organizing Map (SOM) technologies and GILTA
on Managing Large Text Masses, INTERACT, STT Speech-to-Text (research and
development of the phonemic speech recognition for Finnish), the joint project
for Finnish speech technology SuoPuhe, Noise Robust Multilingual Speech
Recognition, Dictionaries and language checking tools, and Multilingual
adaptative translation knowledge base, led jointly by most Finnish universities
and several enterprises. Several commercial products developed within the USIX
framework are available in the market today \cite{LoppuUSIX}.

The NLP projects carried out within the FENIX technology program include FENIX
4M (Mobile and Multilingual Maintenance Man) and FinnONTO (Semantic Web
Ontologies) at the University of Helsinki, New methods and applications in
speech processing and Search-in-a-Box (University of Turku), Rich semantic
media for personal and professional users (VTT Technical Research Centre of
Finland) and Intelligent Web Services (Helsinki School of Science and
Technology), StatHouse Semantics and Automatic content classification and
ontologies (Seerco Ltd) \cite{FinalFENIX}.

Recently, A joint project on speech synthesis between the University of
Helsinki and Aalto University has been very successful in the new field of
statistical parametric synthesis based on Hidden Markov Models and a new,
physiologically grounded vocoding technology. Developing speech synthesis is
very data oriented.

EU funded projects in Finland since the 1980’s include LR SIMPLE, LR
PAROLE and EU MLIS 5008 LINGMACHINE.  The Common Language Resources
and Technologies Infrastructure (CLARIN) was funded by the Commission
during 2008 – 2010, and the work within the initiative continues. The
national part FIN-CLARIN is funded by the Ministry of Education and
Culture. The FIN-CLARIN consortium comprises the following partners:
IT Center for Science CSC, The Research Institute for the Languages of
Finland KOTUS, the universities of Helsinki, Eastern Finland,
Jyväskylä, Oulu, Tampere, Turku, Vaasa, Aalto University and Åbo
Akademi. HFST (Helsinki Finite State Transducer Technology), OMor
(Open Source Morphologies), FinnWordNet, and FinnTreeBank are examples
of currently ongoing projects.

Language Technology at the University of Helsinki also cooperated in 2000 –
2004 on an international level in several projects within the
\textit{Språgteknologiprogram} (Nordic Language Technology Research Program)
funded by
the Nordic Council of Ministers. The Finnish Language Technology documentation
centre FiLT was established to promote availability of language technology
resources, both for commercial and academic players.

As we have seen, previous programmes have led to the development of a number of
LT tools and resources for the Finnish language. In the following section, the
current state of LT support for Finnish is summarised.

\subsection{Availability of Tools and Resources}



Table \ref{X} summarises the current state of language technology support
for the Finnish language. The rating for existing tools and resources was
generated by leading experts in the field who provided
estimates based on a scale from 0 (very low) to 6 (very high) according to
seven criteria.

 \begin{table}
 \centering
 \begin{tabular}{>{\columncolor[RGB]{255,190,000}}p{.33\linewidth}ccccccc}
 \toprule
 \rowcolor[RGB]{255,190,000}
  \cellcolor{white}&\begin{sideways}\makecell[l]{Määrä \\
Quantity}\end{sideways}
 &\begin{sideways}\makecell[l]{\makecell[l]{Saatavuus \\ Availability}
}\end{sideways} &\begin{sideways}\makecell[l]{Laatu \\ Quality}\end{sideways}
 &\begin{sideways}\makecell[l]{Kattavuus \\ Coverage}\end{sideways}
&\begin{sideways}\makecell[l]{Valmiusaste \\ Maturity}\end{sideways}
&\begin{sideways}\makecell[l]{Vakaus \\ Sustainability}\end{sideways}
&\begin{sideways}\makecell[l]{Soveltuvuus \\ Adaptability}\end{sideways} \\
 \midrule
 \multicolumn{8}{>{\columncolor[RGB]{255,155,000}}l}{Kieliteknologia:
työkalut, teknologiat ja sovellukset} \\\addlinespace[{-.009cm}]
 \multicolumn{8}{>{\columncolor[RGB]{255,155,000}}l}{Language Technology: Tools,
Technologies and Applications} \\
 \midrule
 Puheentunnistus \newline Speech Recognition & 3 & 2 & 4 & 3 & 3 & 3 & 4 \\
 Puhesynteesi \newline Speech Synthesis & 3 & 3 & 5 & 4 & 4 & 4 & 4 \\
 Kieliopillinen analyysi \newline Grammatical analysis
                                     & 3,5 & 3,5 & 3,5 & 4 & 4 & 3,5 & 3,5\\
 Semanttinen analyysi \newline Semantic analysis
                                     & 0,4 & 0,4 & 1 & 1 & 1 & 1,4 & 0,7 \\
 Tekstin tuottaminen \newline Text generation & 3 & 3 & 4 & 2 & 3 & 3 & 4 \\
 Konekäännös \newline Machine translation & 3 & 1 & 4 & 2 & 3 & 1 & 2 \\
 \midrule
 \multicolumn{8}{>{\columncolor[RGB]{255,155,000}}l}{Kieliaineistot:
aineistot, tietokannat ja tietämyskannat} \\\addlinespace[{-.009cm}]
 \multicolumn{8}{>{\columncolor[RGB]{255,155,000}}l}{Language Resources:
Resources, Data and Knowledge Bases} \\
 \midrule
 Tekstikorpukset \newline Text corpora & 3 & 4 & 4 & 3,5 & 3,5 & 3,5 & 4 \\
 Puhekorpukset \newline Speech corpora & 2 & 3 & 3 & 2 & 2 & 2 & 2 \\
 Rinnakkaiskorpukset \newline Parallel corpora
                                       & 1 & 2 & 3 & 2 & 2 & 3 & 3 \\
 Leksikaaliset resurssit \newline Lexical resources
                                & 3 & 4 & 3,5 & 4 & 3,5 & 3,5 & 3,5 \\
 Kieliopit \newline Grammars & 2 & 5 & 4 & 4 & 4 & 3 & 3 \\
 \bottomrule
 \end{tabular}
 \label{tab:finnish_table}
 \caption{Suomen kielen kieliteknologian tuki -- State of
language technology support for Finnish}
 \end{table}


The key results for the Finnish language can be summed up as follows:
\begin{itemize}
\item While some specific corpora of high quality exist, sufficiently large
    syntactically annotated corpora are not available yet and many of the
    resources lack standardization. The commercial sector in Finland needs
    large, up-to date resources for the product development targeted to the big
    public.

\item There are several tools for syntactical analysis available based on various
    linguistic models. In general, they work well given the particularities of
    the Finnish language. Work on semantics has not led to applications yet.

\item In Speech technology, the biggest leap forward in Finland has been taken in
    the area of speech recognition. Due to the particularities of Finnish, the
    word lists or lexicons required for speech recognition have been
    impractically large. A speech technology research group at the Helsinki
    University of Technology presented already in 2002 a method for automated
    word segmentation that enabled reducing the size of the lexicon
    dramatically. This breakthrough has not yet been implemented in the
    commercial sector. Speech synthesis research has moved forward considerably
    during the last few years. However, the work is still in the laboratory
    phase, and considerable resources are needed to bring the system to the
    market. Speech corpora are hard to collect and require a lot of work.

\item There are only very few projects working on information retrieval for
    Finnish. It is more usual to take an existing tool and implement a Finnish
    stemmer as part of it leading to licensing and limited rights to use the
    tool in other environments.

\item There are few multimodal resources and virtually no advanced discourse
    processing tools available for Finnish.

\item An unclear legal situation restricts making use of digital texts, such as
    those published online by newspapers, for empirical linguistic and language
    technology research, for example, to train statistical language models.
    Together with politicians and policy makers, researchers should try to
    establish laws or regulations that enable researchers to use publicly
    available texts for language-related R\&D activities.
\end{itemize}

To conclude, in a number of specific areas of Finnish language research, we
have software with limited functionality available today. Obviously, further
research efforts are required to meet the current deficit in processing texts
on a deeper semantic level and to address the lack of resources such as
parallel corpora for machine translation.

\subsection{Cross-language comparison}

The current state of LT support varies considerably from one language
community to another. In order to compare the situation between
languages, this section will present an evaluation based on two sample
application areas (machine translation and speech processing) and one
underlying technology (text analysis), as well as basic resources
needed for building LT applications.

The languages were positioned into clusters based on the following
five-point scale:
\begin{itemize}
\item[] Cluster 1: excellent support
\item[] Cluster 2: good support
\item[] Cluster 3: moderate support
\item[] Cluster 4: fragmentary support
\item[] Cluster 5: weak or no support
\end{itemize}

LT support was measured according to the following criteria:
\begin{itemize}
\item Speech Processing: Quality of existing speech recognition
       technologies, quality of existing speech synthesis
       technologies, coverage of domains, number and size of existing
       speech corpora, amount and variety of available speech-based
       applications
\item Machine Translation: Quality of existing MT technologies, number
       of language pairs covered, coverage of linguistic phenomena and
       domains, quality and size of existing parallel corpora, amount
       and variety of available MT applications

\item Text Analysis: Quality and coverage of existing text analysis
       technologies (morphology, syntax, semantics), coverage of
       linguistic phenomena and domains, amount and variety of
       available applications, quality and size of existing
       (annotated) text corpora, quality and coverage of existing
       lexical resources (e.g., WordNet) and grammars
\item Resources: Quality and size of existing text corpora, speech
       corpora and parallel corpora, quality and coverage of existing
       lexical resources and grammars
\end{itemize}

 \begin{figure}
 \small
 \centering
 \begin{tabular}{>{\columncolor[RGB]{255,155,000}}
p{.15\linewidth}@{\hspace{.05\linewidth}}>{\columncolor[RGB]{255,155,000}}p{.15\linewidth}@{\hspace{.05\linewidth}}>{\columncolor[RGB]{255,155,000}}p{.15\linewidth}@{\hspace{.05\linewidth}}>{\columncolor[RGB]{255,155,000}}p{.15\linewidth}@{\hspace{.05\linewidth}}>{\columncolor[RGB]{255,155,000}}p{.15\linewidth}
}
 \begin{center}\vspace*{-2mm}\textbf{Klusteri Cluster 1}\end{center} &
\begin{center}\vspace*{-2mm}\textbf{Klusteri Cluster 2}\end{center} &
\begin{center}\vspace*{-2mm}\textbf{Klusteri Cluster 3}\end{center} &
\begin{center}\vspace*{-2mm}\textbf{Klusteri Cluster 4}\end{center} &
\begin{center}\vspace*{-2mm}\textbf{Klusteri Cluster 5}\end{center}
 \\ \addlinespace
\addlinespace
 \rowcolor[RGB]{255,190,000}
 & englanti -- English
 & tšekki -- Czech, hollanti -- Dutch, suomi -- Finnish,
ranska -- French, saksa -- German, italia -- Italian, portugali
-- Portuguese, espanja -- Spanish
 & baski -- Basque, bulgaria -- Bulgarian, katalaani -- Catalan, tanska
-- Danish, viro -- Estonian, galicia -- Galician, kreikka -- Greek,
unkari -- Hungarian, iiri -- Irish, norja -- Norwegian, puola --
Polish, serbia -- Serbian, slovakki -- Slovak, sloveeni -- Slovene,
ruotsi -- Swedish
 & kroatia -- Croatian, islanti -- Icelandic, latvia -- Latvian,
liettua -- Lithuanian, malta -- Maltese, romania -- Romanian\\
 \end{tabular}
 \label{fig:speech_cluster}
 \caption{Puheenkäsittelyn kieliklusterit -- Language
clusters for Speech Processing}
 \end{figure}

 \begin{figure}
 \small
 \centering

\begin{tabular}{>{\columncolor[RGB]{255,155,000}}p{.15\linewidth}@{\hspace{.05\linewidth}}
>{\columncolor[RGB]{255,155,000}}p{.15\linewidth}@{\hspace{.05\linewidth}}>{\columncolor[RGB]{255,155,000}}p{.15\linewidth}@{\hspace{.05\linewidth}}>{\columncolor[RGB]{255,155,000}}p{.15\linewidth}@{\hspace{.05\linewidth}}>{\columncolor[RGB]{255,155,000}}p{.15\linewidth}
}
  \begin{center}\vspace*{-2mm}\textbf{Klusteri Cluster 1}\end{center} &
\begin{center}\vspace*{-2mm}\textbf{Klusteri Cluster 2}\end{center} &
\begin{center}\vspace*{-2mm}\textbf{Klusteri Cluster 3}\end{center} &
\begin{center}\vspace*{-2mm}\textbf{Klusteri Cluster 4}\end{center} &
\begin{center}\vspace*{-2mm}\textbf{Klusteri Cluster 5}\end{center} \\ \addlinespace
\addlinespace
 \rowcolor[RGB]{255,190,000}
 &englanti -- English
 &ranska -- French, espanja -- Spanish
 &katalaani -- Catalan, hollanti -- Dutch, saksa -- German,
  unkari --Hungarian, italia -- Italian, puola -- Polish, romania -- Romanian
 &baski -- Basque, bulgaria -- Bulgarian, kroatia -- Croatian,
 tšekki -- Czech, tanska -- Danish, viro -- Estonian, suomi -- Finnish,
 galicia -- Galician,
 kreikka -- Greek, islanti -- Icelandic, iiri -- Irish, latvia -- Latvian,
 liettua -- Lithuanian, malta -- Maltese, norja -- Norwegian,
 portugali -- Portuguese,
 serbia -- Serbian, slovakki -- Slovak, sloveeni -- Slovene,
 ruotsi -- Swedish\\
 \end{tabular}
 \label{fig:mt_cluster}
 \caption{Konekäännöksen kieliklusterit -- Language clusters for
Machine Translation}
 \end{figure}

 \begin{figure}
  \small
  \centering
 
\begin{tabular}{>{\columncolor[RGB]{255,155,000}}p{.15\linewidth}@{\hspace{.05\linewidth}}
>{\columncolor[RGB]{255,155,000}}p{.15\linewidth}@{\hspace{.05\linewidth}}>{\columncolor[RGB]{255,155,000}}p{.15\linewidth}@{\hspace{.05\linewidth}}>{\columncolor[RGB]{255,155,000}}p{.15\linewidth}@{\hspace{.05\linewidth}}>{\columncolor[RGB]{255,155,000}}p{.15\linewidth}
}
 \begin{center}\vspace*{-2mm}\textbf{Klusteri Cluster 1}\end{center} &
\begin{center}\vspace*{-2mm}\textbf{Klusteri Cluster 2}\end{center} &
\begin{center}\vspace*{-2mm}\textbf{Klusteri Cluster 3}\end{center} &
\begin{center}\vspace*{-2mm}\textbf{Klusteri Cluster 4}\end{center} &
\begin{center}\vspace*{-2mm}\textbf{Klusteri Cluster 5}\end{center}
 \\ \addlinespace
\addlinespace
 \rowcolor[RGB]{255,190,000}
 &englanti -- English
 &hollanti -- Dutch, ranska -- French, saksa -- German, italia -- Italian,
 espanja -- Spanish
 &baski -- Basque, bulgaria -- Bulgarian, katalaani -- Catalan,
 tšekki -- Czech, tanska -- Danish, suomi -- Finnish, galicia -- Galician,
 kreikka -- Greek,
 unkari -- Hungarian, norja -- Norwegian, puola -- Polish,
 portugali -- Portuguese, romania -- Romanian, slovakki -- Slovak,
 sloveeni -- Slovene, ruotsi -- Swedish
 &kroatia -- Croatian, viro -- Estonian, islanti -- Icelandic, iiri -- Irish,
 latvia -- Latvian, liettua -- Lithuanian, malta -- Maltese, serbia -- Serbian
 \\
 \end{tabular}
 \label{fig:text_cluster}
 \caption{Tekstinanalyysin kieliklusterit --
 Language clusters for Text Analysis}
 \end{figure}

 \begin{figure}
  \small
  \centering

\begin{tabular}{>{\columncolor[RGB]{255,155,000}}p{.15\linewidth}@{\hspace{.05\linewidth}}
>{\columncolor[RGB]{255,155,000}}p{.15\linewidth}@{\hspace{.05\linewidth}}>{\columncolor[RGB]{255,155,000}}p{.15\linewidth}@{\hspace{.05\linewidth}}>{\columncolor[RGB]{255,155,000}}p{.15\linewidth}@{\hspace{.05\linewidth}}>{\columncolor[RGB]{255,155,000}}p{.15\linewidth}}
 \begin{center}\vspace*{-2mm}\textbf{Klusteri Cluster 1}\end{center} &
\begin{center}\vspace*{-2mm}\textbf{Klusteri Cluster 2}\end{center} &
\begin{center}\vspace*{-2mm}\textbf{Klusteri Cluster 3}\end{center} &
\begin{center}\vspace*{-2mm}\textbf{Klusteri Cluster 4}\end{center} &
\begin{center}\vspace*{-2mm}\textbf{Klusteri Cluster 5}\end{center}
 \\ \addlinespace
\addlinespace
  \rowcolor[RGB]{255,190,000}
  & englanti -- English
  & saksa -- German, unkari -- Hungarian, ruotsi -- Swedish,
      ranska -- French, hollanti -- Dutch, tšekki -- Czech
  & baski -- Basque, bulgaria -- Bulgarian, katalaani -- Catalan,
      kroatia -- Croatian, tanska -- Danish, viro -- Estonian,
     suomi -- Finnish,
     galicia -- Galician, kreikka -- Greek, norja -- Norwegian,
    portugali -- Portuguese, romania -- Romanian, serbia -- Serbian,
    slovakki -- Slovak, sloveeni -- Slovene
  & islanti -- Icelandic, iiri -- Irish, latvia -- Latvian,
      liettua -- Lithuanian, malta -- Maltese \\
  \end{tabular}
  \label{fig:resources_cluster}
  \caption{Kieliaineistojen kieliklusterit -- Language clusters for Resources}
 \end{figure}


The tables
show that the LT funding and thus the resources available for
developing resources for the Finnish language in the recent decades has been
smaller than for the major European languages in general, and particularly
English. Based on the evaluation, machine translation technologies for Finnish
have been classified to the cluster of low support. For speech processing,
current technologies perform well enough to be successfully integrated into a
number of industrial applications such as spoken dialogue and dictation
systems, especially for special languages. The need for language resources both
for text and speech technologies is evident. Text analysis components already
cover the linguistic phenomena of Finnish to a certain extent and form part of
many applications, e.g. spelling correction and function on a satisfactory
level.

For building more sophisticated applications, such as machine translation,
there is a clear need for resources and technologies that cover a wider range
of linguistic aspects and allow a deep semantic analysis of the input text. By
improving the quality and coverage of these basic resources and technologies,
we shall be able to open up new opportunities for tackling a vast range of
advanced application areas, including high-quality machine translation.

\subsection{Conclusions}


\textbf{In this series of white papers, we have made an important
initial effort to assess language technology support for 30 European
languages, and provide a high-level comparison across these
languages. By identifying the gaps, needs and deficits, the European
language technology community and related stakeholders are now in a
position to design a large scale research and development programme
aimed at building a truly multilingual, technology-enabled Europe.}

We have seen that there are huge differences between Europe’s
languages. While there are good quality software and resources
available for some languages and application areas, others (usually
‘smaller’ languages) have substantial gaps.  Many languages lack basic
technologies for text analysis and the essential resources for
developing these technologies. Others have basic tools and resources
but are as yet unable to invest in semantic processing. We therefore
still need to make a large-scale effort to attain the ambitious goal
of providing high-quality machine translation between all European
languages.

Basic research in language technology was well funded in the 1980's
and 1990's but since then the funding has been less satisfying. Even
if some language technology development projects received funding in
the 2000's from the leading Finnish funding agencies, the Finnish
Funding Agency for Technology and Innovation (Tekes) and the Academy
of Finland, the results and material developed in these projects have
not been widely and openly distributed. As the present report shows,
the situation in language technology is acceptable only for the most
basic tools and resources. Finland is lagging behind in the
development of essential digital resources necessary for the survival
of a language as defined in the BLARK (Basic Language Resource Kit)
for speech, text and lexicons. The BLARK is essential in developing
the language technology modules for creating language technology
tools. There is a growing demand for large-scale up-to-date resources
for the language technology research and product development for the
benefit of the Finnish society.

Current efforts within the large-scale European research
infrastructure project Common Language Re-sources and Technologies
Infrastructure (CLARIN) and in the Multilingual Europe Technology
Alliance (META) aim at supporting language resource and technology
distribution and access on a European level. However, the national
needs in Finland have not yet been adequately addressed.

Our findings show that the only alternative is to make a substantial
effort to create LT resources for Finnish, and use them to drive
forward research, innovation and development. The need for large
amounts of data and the extreme complexity of language technology
systems makes it vital to develop a new infrastructure and a more
coherent research organization to spur greater sharing and
cooperation.

There is also a lack of continuity in research and development
funding.  Short-term coordinated programmes tend to alternate with
periods of sparse or zero funding. We can therefore conclude that
there is a desperate need for a large, coordinated initiative focused
on overcoming the differences in language technology readiness for
European languages as a whole.

META-NET’s long-term goal is to introduce high-quality language
technology for all languages in order to achieve political and
economic unity through cultural diversity. The technology will help
tear down existing barriers and build bridges between Europe’s
languages. This requires all stakeholders - in politics, research,
business, and society - to unite their efforts for the future.

\end{multicols}
\clearpage
\ssection{About META-NET}
\begin{multicols}{2}
\boxtext{The Multilingual Europe Technology Alliance (META)}

META-NET is a Network of Excellence funded by the European
Commission. The network currently consists of 47 members from 31
European countries. META-NET fosters the Multilingual Europe
Technology Alliance (META), a growing community of language technology
professionals and organisations in Europe.

% Figger: Countries Represented in META-NET

META-NET cooperates with other initiatives like the Common Language
Resources and Technology Infrastructure (CLARIN), which is helping
establish digital humanities research in Europe. META-NET fosters the
technological foundations for a truly multilingual European
information society that:
\begin{itemize}
\item makes communication and cooperation possible across languages;

\item provides equal access to information and knowledge in any language;

\item offers advanced and affordable networked information technology to European
    citizens.
\end{itemize}
META-NET stimulates and promotes multilingual technologies for all
European languages. The technologies enable automatic translation,
content production, information processing and knowledge management
for a wide variety of applications and subject domains. The network
wants to improve current approaches, so better communication and
cooperation across languages can take place. Europeans have an equal
right to information and knowledge regardless of language.

\subsection{Lines of Action}

META-NET launched on 1 February 2010 with the goal of advancing
research in language technology (LT). The network supports a Europe
that unites as a single digital market and information space. META-NET
has conducted several activities that further its goals. META-VISION,
META-SHARE and META-RESEARCH are the network’s three lines of action.

% Fig: Three Lines of Action in META-NET

\textbf{META-VISION} fosters a dynamic and influential stakeholder
community that unites around a shared vision and a common strategic
research agenda (SRA). The main focus of this activity is to build a
coherent and cohesive LT community in Europe by bringing together
representatives from highly fragmented and diverse groups of
stakeholders. In the first year of META-NET, presentations at the
FLaReNet Forum (Spain), Language Technology Days (Luxembourg),
JIAMCATT 2010 (Luxembourg), LREC 2010 (Malta), EAMT 2010 (France) and
ICT 2010 (Belgium) centred on public outreach. According to initial
estimates, META-NET has already contacted more than 2,500 LT
professionals to develop its goals and visions with them. At the
META-FORUM 2010 event in Brussels, META-NET communicated the initial
results of its vision building process to more than 250
participants. In a series of interactive sessions, the participants
provided feedback on the visions presented by the network.

\textbf{META-SHARE} creates an open, distributed facility for
exchanging and sharing resources. The peer-to-peer network of
repositories will contain language data, tools and web services that
are documented with high-quality metadata and organised in
standardised categories. The resources can be readily accessed and
uniformly searched. The available resources include free, open source
materials as well as restricted, commercially available, fee-based
items. META-SHARE targets existing language data, tools and systems as
well as new and emerging products that are required for building and
evaluating new technologies, products and services. The reuse,
combination, repurposing and re-engineering of language data and tools
plays a crucial role. META-SHARE will eventually become a critical
part of the LT marketplace for developers, localisation experts,
researchers, translators and language professionals from small,
mid-sized and large enterprises. META-SHARE addresses the full
development cycle of LT—from research to innovative products and
services. A key aspect of this activity is establishing META-SHARE as
an important and valuable part of a European and global infrastructure
for the LT community.

\textbf{META-RESEARCH} builds bridges to related technology
fields. This activity seeks to leverage advances in other fields and
to capitalise on innovative research that can benefit language
technology. In particular, this activity wants to bring more semantics
into machine translation (MT), optimise the division of labour in
hybrid MT, exploit context when computing automatic translations and
prepare an empirical base for MT. META-RESEARCH is working with other
fields and disciplines, such as machine learning and the Semantic Web
community.  META-RESEARCH focuses on collecting data, preparing data
sets and organising language resources for evaluation purposes;
compiling inventories of tools and methods; and organising workshops
and training events for members of the community. This activity has
already clearly identified aspects of MT where semantics can impact
current best practices. In addition, the activity has created
recommendations on how to approach the problem of integrating semantic
information in MT. META-RESEARCH is also finalising a new language
resource for MT, the Annotated Hybrid Sample MT Corpus, which provides
data for English-German, English-Spanish and English-Czech language
pairs. META-RESEARCH has also developed software that collects
multilingual corpora that are hidden on the Web.
\end{multicols}

