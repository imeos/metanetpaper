%                                     MMMMMMMMM
%
%  MMA    MM   MMMMMM  MMMMMMM   MM    MMMMMMMM   MMA   MM  MMMMMMM MMMMMMM
%  MMMA AMMM   MM        MM     MMMM              MMMM  MM  MM        MM
%  MM MMM MM   MMMMMM    MM    IM  MI   MMMMMMM   MM MMxMM  MMMMMM    MM
%  MM  M  MM   MM        MM   .MMMMMM.            MM  MMMM  MM        MM
%  MM     MM   MMMMMM    MM   MM    MM            MM   MMM  MMMMMMM   MM
%
%
%          - META-NET Language Whitepaper | Croatian Metadata -
% 
% ----------------------------------------------------------------------------


\usepackage{polyglossia}
\usepackage{nth}
\setotherlanguages{croatian, english}

\title{Hrvatski \ \ \ \ \ \ \ \ \ Jezik u \ \ \ \ \ \ \ \ \ Digitalnom \ \ \ \ \ \ \ \ \ Dobu --- The Croatian Language in the Digital Age}

% Title for the spine of the cover
\spineTitle{The Croatian Language in the Digital Age --- Hrvatski Jezik u Digitalnom Dobu}

\subtitle{White Paper Series --- Niz Bijele Knjige}

\author{
  Marko Tadić\\
  Dunja Brozović-Rončević\\
  Amir Kapetanović\\
}

\authoraffiliation{
  Marko Tadić~~{\small [1]}\\
  Dunja Brozović-Rončević~~ {\small [2]}\\
  Amir Kapetanović~~ {\small [2]} \\
  ~\\
  \footnotesize{[1]} ~ \hspace*{.3mm}\small{Filozofski Fakultet, Zagreb}\\
  \footnotesize{[1]} ~ \hspace*{.3mm}\small{Institut za hrvatski jezik i jezikoslovlje}
}

\editors{
  Georg Rehm, Hans Uszkoreit\\ (urednici, \textcolor{grey1}{editors})
}

% Text in left column on backside of the cover
\SpineLText{\selectlanguage{english}%
  In everyday communication, Europe’s citizens, business partners and politicians are inevitably confronted with language barriers.  
  Language technology has the potential to overcome these barriers and to provide innovative interfaces to technologies and knowledge. 
  This white paper presents the state of language technology support for the Icelandic language. 
  It is part of a series that analyses the available language resources and technologies for 30~European languages. 
  The analysis was carried out by META-NET, a Network of Excellence funded by the European Commission.
  META-NET consists of 54 research centres in 33 countries, who cooperate with stakeholders from economy, government agencies, research organisations and others. %, non-governmental organisations, language communities and European universities. 
  META-NET’s vision is high-quality language technology for all European languages. 
}

% Text in right column on backside of the cover
\SpineRText{\selectlanguage{croatian}%
  U svakodnevnoj komunikaciji građani Europe, poslovni partneri i političari neizbježno su suočeni s jezičnim barijerama. Potencijal koji imaju jezične tehnologije mogao bi savladati te prepreke i osigurati inovativna sučelja za tehnologije i znanja.
 Ovaj dokument prikazuje stanje jezičnih tehnologija za hrvatski jezik. Jedan je od dokumenata u nizu bijele knjige koji analizira dostupne jezične resurse i tehnologije za 30 europski jezik.
 Analizu je proveo Meta-NET - mreža izvrsnosti koju financira Europska komisija. META-NET se sastoji od 54 istraživačka centra u 33 zemalje, koji surađuju s partnerima iz gospodarstva, državnih agencija, istraživačkih organizacija i drugih nevladinih organizacija, jezičnih zajednica i europskih sveučilišta. Vizija je Meta-NET-a povećanje kvalitete jezičnih tehnologija za sve europske jezike.
}
\vspace{-5mm}

% Quotes from VIPs on backside of the cover
\quotes{%
  \selectlanguage{croatian}%
  „Niz "Jezičnih bijelih knjiga" otvara nove uvide u europsku jezičnu raznolikost dok istodobno relativizira pojam tzv. "malih" jezika, poput hrvatskoga. Stoga jezične tehnologije imaju ne samo ključnu ulogu u iskazivanju jezičnoga bogatstva u današnjoj Europi, već predstavljaju metodološko ishodište za daljnji razvitak digitalnih humanističkih znanosti, osobito ako ih se promatra kao temelj za daljnja istraživanja u raznim humanističkih disciplinama.“ \\
 \textcolor{grey2}{--- Prof. dr. Milena Žic Fuchs (redoviti član Hrvatske akademije znanosti i umjetnosti, predsjednica Stalnoga odbora za hmanističke znanosti Europske znanstvene zaklade)}\\[3mm]
  \selectlanguage{english}%
  „The Language White Paper Series opens up new insights into language diversity in Europe as well as relativizing the concept of so-called "small" languages, such as Croatian. Thus, language technologies play not only a crucial role in showcasing the linguistic  richness of Europe, but at the same time providing the methodological backbone for further development of Digital Humanities, especially seen as the basis for further research in a variety of Humanities disciplines.“ \\
\textcolor{grey2}{--- Professor Milena Žic Fuchs (Fellow of the Croatian Academy of Sciences and Arts, Chair of the Standing Committee for the Humanities of the European Science Foundation)}\\[3mm]
}

% Funding notice left column
\FundingLNotice{\selectlanguage{croatian}
 Autori ovoga dokumenta zahvalni su autorima Bijele knjige o njemačkome jeziku za dopuštenje uporabe odabrane jezičnoneovisne građe iz njihovoga teksta \cite{boo1}.\vfill
  \bigskip
  \begin{spacing}{1.2}
  Izradba ove bijele knjige poduprta je od strane Sedmoga okvirnoga programa i ICT programa za podršku politici Europske komisije u skladu s ugovorima T4ME (opći ugovor 249\,119), CESAR (opći ugovor 271\,022), METANET4U (opći ugovor 270\,893) i META-NORD (opći ugovor 270\,899).\end{spacing}
}

% Funding notice right column
\FundingRNotice{\selectlanguage{english}
  The authors of this document are grateful to the authors of the White Paper on German for permission to re-use selected language-independent materials from their document \cite{boo1}.\vfill
  \bigskip
  \begin{spacing}{1.2}
  The development of this white paper has been funded by the Seventh Framework Programme and the ICT Policy Support Programme of the European Commission under the contracts T4ME (Grant Agreement 249\,119), CESAR (Grant Agreement 271\,022), META\-NET4U (Grant Agreement 270\,893) and META-NORD (Grant Agreement 270\,899).\end{spacing}
}
