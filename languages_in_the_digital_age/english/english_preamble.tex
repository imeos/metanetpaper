%                                     MMMMMMMMM
%
%  MMA    MM   MMMMMM  MMMMMMM   MM    MMMMMMMM   MMA   MM  MMMMMMM MMMMMMM
%  MMMA AMMM   MM        MM     MMMM              MMMM  MM  MM        MM
%  MM MMM MM   MMMMMM    MM    IM  MI   MMMMMMM   MM MMxMM  MMMMMM    MM
%  MM  M  MM   MM        MM   .MMMMMM.            MM  MMMM  MM        MM
%  MM     MM   MMMMMM    MM   MM    MM            MM   MMM  MMMMMMM   MM
%
%
%          - META-NET Language Whitepaper | English Metadata -
% 
% ----------------------------------------------------------------------------

\usepackage{polyglossia}
\setotherlanguages{english}


\title{The English Language in the Digital Age --- ~}

\spineTitle{The English Language in the Digital Age}

\subtitle{~ --- White Paper Series}

\author{
  Sophia Ananiadou \\
  John McNaught \\
  Paul Thompson
}

\authoraffiliation{
  Sophia Ananiadou~ {\small University of Manchester}\\
  John McNaught~ {\small University of Manchester}\\
  Paul Thompson~ {\small University of Manchester}
}

\editors{
  Georg Rehm, Hans Uszkoreit\\(editors)
}

% Text in left column on backside of the cover
\SpineLText{%
}

% Text in right column on backside of the cover
\SpineRText{%
}

% Quotes from VIPs on backside of the cover
\quotes{%
  In everyday communication, Europe’s citizens, business partners and politicians are inevitably confronted with language barriers. Language technology has the potential to overcome these barriers and to provide innovative interfaces to technologies and knowledge. This white paper presents the state of language technology support for the English language. It is part of a series that analyses the available language resources and technologies for 30 European languages. The analysis was carried out by META-NET, a Network of Excellence funded by the European Commission. META-NET consists of 54 research centres in 33 countries, who cooperate with stakeholders from economy, government agencies, research organisations, non-governmental organisations, language communities and European universities. META-NET’s vision is high-quality language technology for all European languages.\\\\
``As an information solution provider and academic publisher, we at Elsevier appreciate the great benefits that integrating language technology solutions into our platforms such as SciVerse can bring to researchers in allowing them to improve their research outcome and to find the information they are looking for quickly and easily. We hope that the META-NET initiative, and in particular this white paper, will allow people working in different areas to gain an understanding of the significant potential of language technology solutions, and help to drive further research into this area." \\
  \textcolor{grey2}{--- Rafael Sidi (Vice President, Product Management for ScienceDirect, Elsevier)}\\[3mm]
 ``Language technology has the potential to add enormous value to the UK economy. Without language technology, and in particular text mining,
there is a real risk that we will miss discoveries that could have significant social and economic impact." \\
  \textcolor{grey2}{--- Douglas B. Kell (Research Chair in Bioanalytical Science, University of Manchester)}
}

% Funding notice left column
\FundingLNotice{
\ 
}

% Funding notice right column
\FundingRNotice{
\textcolor{black}{ The development of this white paper has been funded by the Seventh
  Framework Programme and the ICT Policy Support Programme of the
  European Commission under the contracts T4ME (Grant Agreement
  249119), CESAR (Grant Agreement 271022), METANET4U (Grant Agreement
  270893) and META-NORD (Grant Agreement 270899).\\\\
  The authors of this document are grateful to the authors of the White Paper on German for permission to re-use selected language-independent materials from their document \cite{lwpgerman}. Furthermore, the authors would like to thank Kevin B. Cohen (University of Colorado, USA), Yoshinobu Kano (National Institute of Informatics, Japan), Ioannis Korkotzelos (University of Manchester, UK), BalaKrishna Kolluru (University of Manchester, UK), Tayuka Matzuzaki (University of Tokyo, Japan), Chikashi Nobata (Yahoo Labs, USA), Naoaki Okazaki (Tohoku University, Japan), Martha Palmer (University of Colorado, USA), Sampo Pyysalo (University of Manchester, UK), Rafal Rak (University of Manchester, UK) and Yoshimasa Tsuruoka (University of Tokyo), for their contributions to this white paper.}}
