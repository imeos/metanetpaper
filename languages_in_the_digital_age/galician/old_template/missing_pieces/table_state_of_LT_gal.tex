\documentclass[10pt]{article}
\usepackage{color}
\usepackage{longtable}
\usepackage{tabulary}
\usepackage{rotating}
\usepackage{makecell}
\usepackage{multirow}
\usepackage{colortbl}
\usepackage{booktabs}

\usepackage{fontspec}

% The new babel, for hyphenating and localization in general...
\usepackage{polyglossia}
\setmainlanguage{galician}
%!TEX TS-program = xelatex
\RequireXeTeX %Force XeTeX check



% Table colors
\definecolor{orange1}{cmyk}{0, 0.56, 0.86, 0}        % #FF6600
\definecolor{orange2}{cmyk}{0, 0.34, 0.87, 0.01}     % #FF9900

\begin{document}

% Begin table
\begin{figure}
\centering

\begin{tabular}{>{\columncolor{orange1}}p{.33\linewidth}@{\hspace*{6mm}}c@{\hspace*{6mm}}c@{\hspace*{6mm}}c@{\hspace*{6mm}}c@{\hspace*{6mm}}c@{\hspace*{6mm}}c@{\hspace*{6mm}}c}
\rowcolor{orange1}
 \cellcolor{white}&
 \begin{sideways}\makecell[l]{Cantidade}\end{sideways} &
 \begin{sideways}\makecell[l]{\makecell[l]{Dispoñibilidade} }\end{sideways} &
 \begin{sideways}\makecell[l]{Calidade}\end{sideways} &
 \begin{sideways}\makecell[l]{Cobertura}\end{sideways} &
 \begin{sideways}\makecell[l]{Madurez}\end{sideways} &
 \begin{sideways}\makecell[l]{Sustentabilidade}\end{sideways} &
 \begin{sideways}\makecell[l]{Adaptabilidade}\end{sideways} \\ \addlinespace

\multicolumn{8}{>{\columncolor{orange2}}l}{\textcolor{black}{Tecnoloxía da linguaxe: ferramentas, tecnoloxías e aplicacións}} \\ \addlinespace

Recoñecemento da fala &3&2&3&3&3&3&3 \\ \addlinespace
Síntese da fala &4&3&4&5&4&3&3\\ \addlinespace
Análise gramatical &3&5&4&4&3&2&3\\ \addlinespace
Análise semántica &1&1&2&1&1&1&1\\ \addlinespace
Xeración de texto &0&0&0&0&0&0&0\\ \addlinespace
Tradución automática &3&5&2&3&4&1&2\\ \addlinespace


\multicolumn{8}{>{\columncolor{orange2}}l}{\textcolor{black}{Recursos de lingua: recursos, datos e bases de coñecemento}} \\ \addlinespace

Corpus textual &3&2&3&3&3&2&2\\ \addlinespace
Corpus da fala &3&4&4&2&3&3&2\\ \addlinespace
Corpus paralelos &2&5&3&2&2&1&1\\ \addlinespace
Recursos léxicos &3&2&3&2&3&3&2\\ \addlinespace
Gramáticas &2&2&2&2&2&2&2\\
\end{tabular}
\label{tab:lrlttable}
\caption{Estado de apoio da tecnoloxía da linguaxe para o idioma galego}
\end{figure}

\end{document}