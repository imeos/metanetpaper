%                                     MMMMMMMMM        
%                                                                             
%  MMO    MM   MMMMMM  MMMMMMM   MM    MMMMMMMM   MMD   MM  MMMMMMM MMMMMMM   
%  MMM   MMM   MM        MM     ?MMM              MMM$  MM  MM         MM     
%  MMMM 7MMM   MM        MM     MM8M    MMMMMMM   MMMMD MM  MM         MM     
%  MM MMMMMM   MMMMMM    MM    MM  MM             MM MMDMM  MMMMMM     MM     
%  MM  MM MM   MM        MM    MMMMMM             MM  MMMM  MM         MM     
%  MM     MM   MMMMMM    MM   MM    MM            MM   MMM  MMMMMMM    MM
%
%
%          - META-NET Language Whitepaper | Galician Metadata -
% 
% ----------------------------------------------------------------------------

\usepackage{polyglossia}
\setotherlanguages{galician,english}


\title{O idioma galego na era dixital --- The Galician Language\ \ \ \ in the\ \ \ \ Digital Age}
\spineTitle{The Galician Language in the Digital Age --- O idioma galego na era dixital}
\subtitle{White Paper Series --- Serie de Libros Brancos}

\author{
 Carmen García Mateo\\
 Monsterrat Arza Rodríguez
}

%%% CGM
\authoraffiliation{
 Carmen García Mateo~ {\small UVIGO}\\
 Monsterrat Arza Rodríguez~ {\small UVIGO}\\
{\small UVIGO --- Universidade de Vigo}
}



\editors{
  Georg Rehm, Hans Uszkoreit\\(editores)
}

% Text in left column on backside of the cover
\SpineLText{\selectlanguage{english}%
  In everyday communication, Europe’s citizens, business partners and politicians are inevitably confronted with language barriers.  
  Language technology has the potential to overcome these barriers and to provide innovative interfaces to technologies and knowledge. 
  This white paper presents the state of language technology support for the Galician language. 
  It is part of a series that analyses the available language resources and technologies for 30~European languages. 
  The analysis was carried out by META-NET, a Network of Excellence funded by the European Commission.
  META-NET consists of 54 research centres in 33 countries, who cooperate with stakeholders from economy, government agencies, research organisations, non-governmental organisations, language communities and European universities. 
  META-NET’s vision is high-quality language technology for all European languages.
}

% Text in right column on backside of the cover
\SpineRText{\selectlanguage{galician}%
Na comunicación cotiá, os cidadáns europeos, os homes e mulleres de negocios e os políticos enfróntanse inevitablemente coas barreiras do idioma.
A tecnoloxía da linguaxe ten potencial para superar estas barreiras e proporcionar interfaces innovadoras ás tecnoloxías e ao coñecemento.
Este libro branco presenta a situación actual da tecnoloxía da linguaxe para o galego.
É parte dunha serie de libros que analiza os recursos e tecnoloxías dispoñibles para 30~linguas europeas.
A análise foi realizada por META-NET, unha rede de excelencia financiada pola Comisión Europea.
META-NET componse de 54 centros de investigación de 33 países, que colaboran con partes interesadas de economía, axencias gobernamentais, organizacións de investigación, organizacións non gobernamentais, comunidades lingüísticas e universidades europeas.
A visión de META-NET é tecnoloxía da linguaxe de alta calidade para todas as linguas europeas. 
}

%%% CGM : No cites  are included
%FIXME Quotes from VIPs on backside of the cover
%\quotes{%
 %``Lorem Ipsum Dolor ....'' \\
  %\textcolor{grey2}{--- Dr. Danilo Türk, VIP-affiliation}\\[3mm]
 %``Lorem Ipsum Dolor ....''
% \\
 % \textcolor{grey2}{--- Dr Danilo Türk, VIP-affiliation}
%}

% Funding notice left column
\FundingLNotice{\selectlanguage{galician} As autoras deste documento agradecen 
aos autores do “Libro Branco sobre o alemán” \cite{lwpgerman} o seu consentimento
 para reutilizar material seleccionado do seu documento orixinal. Asemesmo, as autoras agradecen a colaboración dos seguintes expertos no idioma galego: 
Dr. Xavier G. Guinovart (Universidade de Vigo),
Dr. Eduardo R. Banga (Universidade de Vigo),
Dr. Xosé Luis Regueira (Universidade de Santiago de Compostela), e
Don José Ramom Pichel  (Imaxin Software).
Iinformación das páxinas web do 
Consello da Cultura Galega ( Proxecto LOIA) e da  
Secretaría Xeral de Política Lingüística – Xunta de Galicia foi empregada na elaboración deste texto.

 
  \bigskip
O desenvolvemento deste libro branco foi financiado polo Sétimo 
Programa Marco e o Programa de apoio ás TIC (ICT Policy support programme) da 
Comisión Europea en virtude dos contratos T4ME (acordo de subvención 249119), 
CESAR (acordo de subvención 271022), METANET4U (acordo de subvención 270893) 
e META-NORD (acordo de subvención 270899).}

% Funding notice right column
\FundingRNotice{\selectlanguage{english} The authors of this document
  are grateful to the authors of the White Paper on German \cite{lwpgerman} for
  permission to re-use selected language-independent materials from
  their document.  Furthermore, the authors would like to thank Dr. Xavier G. Guinovart (University of Vigo), Dr. Eduardo R. Banga (University of Vigo), Dr. Xosé Luis Regueira (University of Santiago de Compostela) and Mr. José Ramom Pichel (Imaxin Software) for their contributions to this white paper. Material availabe at the web pages of "Consello da Cultura Galega – Proxecto LOIA"  and "Secretaría Xeral de Política Lingüística – Xunta de Galicia" has been used.

  
  \bigskip
  The development of this white paper has been funded by the Seventh
  Framework Programme and the ICT Policy Support Programme of the
  European Commission under the contracts T4ME (Grant Agreement
  249119), CESAR (Grant Agreement 271022), METANET4U (Grant Agreement
  270893) and META-NORD (Grant Agreement 270899).}
