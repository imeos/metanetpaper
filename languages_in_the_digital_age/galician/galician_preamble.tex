%                                     MMMMMMMMM                                         
%                                                                             
%  MMO    MM   MMMMMM  MMMMMMM   MM    MMMMMMMM   MMD   MM  MMMMMMM MMMMMMM   
%  MMM   MMM   MM        MM     ?MMM              MMM$  MM  MM         MM     
%  MMMM 7MMM   MM        MM     MM8M    MMMMMMM   MMMMD MM  MM         MM     
%  MM MMMMMM   MMMMMM    MM    MM  MM             MM MMDMM  MMMMMM     MM     
%  MM  MM MM   MM        MM    MMMMMM             MM  MMMM  MM         MM     
%  MM     MM   MMMMMM    MM   MM    MM            MM   MMM  MMMMMMM    MM
%
%
%          - META-NET Language Whitepaper | Galician Metadata -
% 
% ----------------------------------------------------------------------------

\usepackage{polyglossia}
\setotherlanguages{galician,english}


\title{O idioma galego na era dixital --- The Galician Language\ \ \ \ in the\ \ \ \ Digital Age}

\subtitle{White Paper Series --- Serie de Libros Brancos}

\author{
 Dr. Xavier G. Guinovart\\
 Dr. Eduardo R. Banga\\ 
 Dr. Xosé Luis Regueira\\
 Mr. José Ramom Piche\\
}

\authoraffiliation{
 Dr. Xavier G. Guinovart~ {\small [UVIGO]}\\
 Dr. Eduardo R. Banga~{\small [UVIGO]}\\ 
 Dr. Xosé Luis Regueira~{\small [USTC]}\\
 Mr. José Ramom Piche~{\small [Imaxin Software]}\\
 \footnotesize{[UVIGO]} ~ \small{University of Vigo}\\
 \footnotesize{[USTC]} ~ \small{University of Santiago de Compostela}
}


\editors{
  Georg Rehm, Hans Uszkoreit\\(editores)
}

% Text in left column on backside of the cover
\SpineLText{\selectlanguage{english}%
  In everyday communication, Europe’s citizens, business partners and politicians are inevitably confronted with language barriers.  
  Language technology has the potential to overcome these barriers and to provide innovative interfaces to technologies and knowledge. 
  This white paper presents the state of language technology support for the Galician language. 
  It is part of a series that analyzes the available language resources and technologies for 31~European languages. 
  The analysis was carried out by META-NET, a Network of Excellence funded by the European Commission.
  META-NET consists of 54 research centres in 33 countries, who cooperate with stakeholders from economy, government agencies, research organisations, non-governmental organisations, language communities and European universities. 
  META-NET’s vision is high-quality language technology for all European languages.
}

% Text in right column on backside of the cover
\SpineRText{\selectlanguage{galician}%
***HELP En la comunicació quotidiana, els ciutadans europeus, socis de negocis i polítics s'enfronten inevitablement amb les barreres de l'idioma.
La tecnologia del llenguatge té el potencial per superar aquestes barreres i proporcionar interfícies innovadores a les tecnologies i coneixements.
Aquest llibre blanc presenta l'estat del suport de la tecnologica del llenguatge per al català.
És part d'una sèrie de llibres que analitza els recursos i tecnologies disponibles per 31~llengües europees.
L'anàlisi va ser realitzat per META-NET, una xarxa d'excel·lència finançada per la Comissió Europea.
META-NET es compon de 54 centres de recerca a 33 països, que col·laboren amb les parts interessades de l'economia, les agències governamentals, organitzacions de recerca, organitzacions no governamentals, comunitats lingüístiques i universitats europees.
La visió de META-NET és tecnologia del llenguatge d'alta qualitat per totes les llengües europees.
***FHELP
}


% Funding notice left column
\FundingLNotice{\selectlanguage{galician} Os autores deste documento agradecen 
aos autores do “Libro Branco sobre o alemán” \cite{lwpgerman} o seu consentimento
 para reutilizar material seleccionado do seu documento orixinal.
 
  \bigskip
O desenvolvemento deste libro branco foi financiado polo Sétimo 
Programa Marco e o Programa de apoio ás TIC (ICT Policy support programme) da 
Comisión Europea en virtude dos contratos T4ME (acordo de subvención 249119), 
CESAR (acordo de subvención 271022), METANET4U (acordo de subvención 270893) 
e META-NORD (acordo de subvención 270899).}

% Funding notice right column
\FundingRNotice{\selectlanguage{english} The authors of this document
  are grateful to the authors of the White Paper on German \cite{lwpgerman} for
  permission to re-use selected language-independent materials from
  their document.
  
  \bigskip
  The development of this white paper has been funded by the Seventh
  Framework Programme and the ICT Policy Support Programme of the
  European Commission under the contracts T4ME (Grant Agreement
  249119), CESAR (Grant Agreement 271022), METANET4U (Grant Agreement
  270893) and META-NORD (Grant Agreement 270899).}
