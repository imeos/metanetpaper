The automated translation and speech processing tools currently available on the market still fall short of this ambitious goal. The dominant actors in the field are primarily privately-owned for-profit enterprises based in Northern America. Already in the late 1970s, the EU realized the profound relevance of language technology as a driver of European unity, and began funding its first research projects, such as EUROTRA. At the same time, national projects were set up that generated valuable results but never led to concerted European action. In contrast to this highly selective funding effort, other multilingual societies such as India (22 official languages) and South Africa (11 official languages) have recently set up long-term national programmes for language research and technology development. 

The predominant actors in LT today rely on imprecise statistical approaches that do not make use of deeper linguistic methods and knowledge. For example, sentences are automatically translated by comparing a new sentence against thousands of sentences previously translated by humans. The quality of the output largely depends on the amount and quality of the available sample corpus. While the automatic translation of simple sentences in languages with sufficient amounts  of available text material can achieve useful results, such shallow statistical methods doomed to fail in the case of languages with a much smaller body of sample material or in the case of sentences with the complex structures.

The European Union has therefore decided to fund projects such as EuroMatrix and EuroMatrixPlus (since 2006) and iTranslate4 (since 2010) that carry out basic and applied research and generate resources for establishing high quality language technology solutions for all European languages. Analysing the deeper structural properties of languages is the only way forward if we want to build applications that perform well across the entire range of Europe’s languages.

European research in this area has already achieved a number of successes. For example, the translation services of the European Union now use MOSES open-source machine translation software that has been mainly developed through European research projects. The Verbmobil project, funded by the German Ministry of Education and Research, pushed Germany into the lead in the world of speech translation research for a time. Many of the research and development labs have since been closed down or moved elsewhere. Europe has tended to pursue isolated research activities with a less pervasive impact on the market.  
