To illustrate how computers handle language and why it is difficult to program them to use it, let’s look briefly at the way humans acquire first and second languages, and then see how language technology systems work. 

Humans acquire language skills in two different ways. Babies acquire a language by listening to the real interactions between its parents, siblings and other family members. From the age of about two, children produce their first words and short phrases. This is only possible because humans have a genetic disposition to imitate and then rationalise what they hear. 

Learning a second language at an older age requires more effort, largely because the child is not immersed in a language community of native speakers. At school, foreign languages are usually acquired by learning grammatical structure, vocabulary and spelling using drills that describe linguistic knowledge in terms of abstract rules, tables and examples. Learning a foreign language gets harder with age.

\boxtext{Humans acquire language skills in two different ways: learning examples and learning the underlying language rules}

The two main types of language technology systems ‘acquire’ language capabilities in a similar manner. Statistical (or ‘data-driven’) approaches obtain linguistic knowledge from vast collections of concrete example texts. While it is sufficient to use text in a single language for training, e.g., a spell checker, parallel texts in two (or more) languages have to be available for training a machine translation system. The machine learning algorithm then “learns” patterns of how words, short phrases and complete sentences are translated. 

This statistical approach can require millions of sentences and performance quality increases with the amount of text analysed. This is one reason why search engine providers are eager to collect as much written material as possible. Spelling correction in word processors, and services such as Google Search and Google Translate all rely on statistical approaches. The great advantage of statistics is that the machine learns fast in continuous series of training cycles, even though quality can vary arbitrarily.

The second approach to language technology and machine translation in particular is to build rule-based systems. Experts in the fields of linguistics, computational linguistics and computer science first have to encode grammatical analyses (translation rules) and compile vocabulary lists (lexicons). This is very time consuming and labour intensive. Some of the leading rule-based machine translation systems have been under constant development for more than twenty years. The great advantage of rule-based systems is that the experts have more detailed control over the language processing. This makes it possible to systematically correct mistakes in the software and give detailed feedback to the user, especially when rule-based systems are used for language learning. But due to the high cost of this work, rule-based language technology has so far only been developed for major languages. 

As the strengths and weaknesses of statistical and rule-based systems tend to be complementary, current research focuses on hybrid approaches that combine the two methodologies. However, these approaches have so far been less successful in industrial applications than in the research lab. 

As we have seen in this chapter, many applications widely used in today’s information society rely heavily on language technology. Due to its multilingual community, this is particularly true of Europe’s economic and information space. Although language technology has made considerable progress in the last few years, there is still huge potential in improving the quality of language technology systems. In the following, we will describe the role of Slovak in European information society and assess the current state of language technology for the Slovak language. 
