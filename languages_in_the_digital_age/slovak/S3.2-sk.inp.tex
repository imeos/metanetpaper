\noindent Slovenčina sa začala samostatne vyvíjať priamo z~praslovančiny od 10. storočia. Hlavné zmeny v~nej prebehli a~ustálili sa do 15. storočia, niektoré rovnomerne (zánik nosoviek), iné diferencovane (vokalizácia tvrdých jerov vo východnej a~západnej časti dnešného Slovenska bola západoslovanského typu, v~centrálnej časti nezápadoslovanského typu). Súčasťou týchto zmien bol aj rozpad praslovanskej stavby slabiky, čo ovplyvnilo zmeny v~skloňovaní a~časovaní. Aj keď sa slovenčina a~čeština dlhý čas vyvíjali za rozdielnych podmienok (Slovensko sa v~11. storočí stalo súčasťou Uhorského kráľovstva), ostali si navzájom blízke. Niektoré špecifické črty slovenského jazyka (formy \emph{l\textbf{a}keť}, \emph{Če\textbf{s}i}, prípona \emph{-m} pri slovesách v~prvej osobe jednotného čísla atď.) však súčasne existujú aj v~jazykoch južných Slovanov. Niektorými menej dôležitými charakteristikami slovenčina pripomína poľštinu (predpona \emph{pre-} na rozdiel od českého \emph{pro-}, zachovanie spoluhlásky \emph{dz} a~niekoľko výrazov, napríklad \emph{teraz}, \emph{pivnica}). Inými charakteristikami sa blíži k~východoslovanským jazykom. Hovorí sa preto o~centrálnej pozícii slovenčiny medzi slovanskými jazykmi a~o~dobrej zrozumiteľnosti slovenčiny pre príslušníkov ostatných slovanských národov.

\boxtext{Niektoré špecifické črty slovenského jazyka existujú aj v jazykoch južných Slovanov}

Slovenčina používa upravené latinské písmo. Keďže pre niektoré slovenské hlásky v~latinskej abecede chýbajú osobitné písmená, slovenská abeceda si vypomáha diakritickými znamienkami. Mäkkosť konsonantov sa zaznačuje mäkčeňom (\emph{ď}, \emph{ť}, \emph{ň}, \emph{ľ}, používa sa aj pri grafémach \emph{ž}, \emph{š}, \emph{č}, \emph{dž}), dĺžka vokálov, resp. konsonantov dĺžňom (\emph{á}, \emph{é}, \emph{í}, \emph{ý}, \emph{ó}, \emph{ú}, \emph{ŕ}, \emph{ĺ}). Vokály nepodliehajú redukcii, v~každej pozícii sa vyslovujú v~plnej forme. Okrem samohlások a~spoluhlások existujú v~slovenčine takzvané \emph{i}‑dvojhlásky (\emph{ia}, \emph{ie}, \emph{iu}) a~jedna \emph{u}-dvojhláska (/\textsubarch{u}o/, píše sa ô; /o\textsubarch{u}/ sa nepovažuje za dvojhlásku), pričom prvý úsek sa vyslovuje ako polosamohláska.

\boxtext{Slovenčina používa upravené latinské písmo}

Fonetickým špecifikom štandardnej slovenčiny (a~stredoslovenských dialektov) je takzvaný rytmický zákon, podľa ktorého by vedľa seba nemali byť dve dlhé slabiky (\emph{pekný} – \emph{krásny}, \emph{prosím} – \emph{smútim}). Slovenčina má prízvuk viazaný na prvú slabiku slova, ktorý nie je príliš silný (slabší ako v~ruštine alebo v~poľštine). V~predložkových frázach s~jednoslabičnou predložkou sa prízvuk zvyčajne kladie na predložku: {\em \underbar{pri} škole}.

Slovenčina má v~porovnaní s~ruštinou, ale napríklad aj
s~češtinou, jednoduchšiu štruktúru skloňovacích a~časovacích
paradigiem. Tvarový systém substantívnych a~slovesných foriem je
však napriek unifikačným tendenciám dostatočne jasne členený.
Slovenský jazyk má šesť gramatických pádov (nominatív, genitív,
datív, akuzatív, lokál a~inštrumentál). Vokatív sa v~slovenčine
na rozdiel od češtiny aktívne nevyužíva, zvyčajne je identický
s~nominatívom. Slovenčina rozoznáva 4~rody: mužský životný
a~mužský neživotný, ženský a~stredný rod podstatných mien
a~súvisiacich prídavných mien, zámen a~čísloviek. Mužský
a~ženský rod sa pri životných konkrétach určuje podľa
prirodzeného rodu, v~ostatných prípadoch je to vec konvencie, ktorá
nie je signalizovaná nijakým členom, iba niekedy zakončením (napr.
\emph{strom} – masculínum inanimatum, \emph{jabloň} – feminínum,
\emph{jablko} – neutrum). Pre každý rod sú v~školských
učebniciach uvedené viaceré vzory, ktorých paradigmy sa odlišujú
najmä v~G/A~sg. a~N/G pl. (napr. mužský životný
\emph{chlap}  / \emph{chlap\textbf{a}} / \emph{chlap\textbf{i}}   / \emph{chlap\textbf{ov}},
\emph{hrdina} / \emph{hrdin\textbf{u}} / \emph{hrdin\textbf{ovia}} / \emph{hrdin\textbf{ov}};
\emph{žena}   / \emph{žen\textbf{y}}   / \emph{žen\textbf{u}}     / \emph{žen\textbf{y}}    / \emph{žien},
\emph{dlaň}   / \emph{dlan\textbf{e}}  / \emph{dlaň}              / \emph{dlan\textbf{e}}   / \emph{dlan\textbf{í}}).
Súčasne je v~niektorých vzoroch a~pádoch značná pádová
homonymia: G a~A~sg. životných maskulín, N a~A~sg. neživotných
maskulín, v~ženskom rode G sg. a~N pl. a~pod. Medzi vzormi sú možné
prechody, napr. ženský vzor \emph{kosť} je v~súčasnosti
produktívnejší ako vzor \emph{dlaň}. Slová zaradené k~istému
vzoru sa od neho často odlišujú, čo sa rieši vymenovaním
výnimiek; vo vedeckých a~počítačovo-lingvistických prácach sa
však uvádza oveľa väčší počet vzorov
\cite{pales1994,sokolova1999,sokolova2007a}.

Pri slovesách sa rozlišujú tri časy: minulý, prítomný a~budúci.
Okrem troch slovesných spôsobov – indikatívu, imperatívu
a~kondicionálu – má väčšina slovies jeden z~nasledujúcich vidov
– nedokonavý (\emph{volať}) a~dokonavý (\emph{zavolať}).
Slovenčina je silne flektívny jazyk s~prvkami analytických
konštrukcií (hlavne v~slovesných formách ako \emph{budem písať},
\emph{bol by som prišiel}). Gramatickú funkciu slova jasne určuje
skloňovanie, slovosled vety je teda pomerne voľný. V~syntaktickej
typológii slovenčinu charakterizuje základná konštrukcia S(ubjekt)
– V(erbum) – O(bjekt), ide však skôr o~teoretickú schému, ktorá
v~praxi nadobúda rozličné formy v~dôsledku voľného slovosledu.
Jednoznačnému určeniu S~a~O~napomáhajú pády (S~je v~N, O~je
zvyčajne v~A~alebo G, D, zriedkavejšie v~ostatných pádoch),
homonymia tvarov však môže spôsobiť neistotu v~obsadení funkcie
subjektu a~objektu (najmä pri cudzích vlastných menách, ale
v~školskej praxi a~v~počítačovej analýze vo viacerých ďalších
prípadoch). 

Osobitné problémy cudzincom a~počítačovému
spracovaniu slovenčiny robia slovesné morfémy
\emph{sa}, \emph{si}, ktoré môžu stáť pred slovesom alebo za ním,
a~to aj vo vzdialenosti viacerých slov či dokonca v~inej časti
rozdelenej vety v~súvetnej štruktúre (\emph{Netrvalo dlho, keď
\textbf{sa} im ich hviezda, ktorú predtým videli v~diaľke, zrazu 
\textbf{priblížila}}). V~slovenčine sú najčastejšie dvojčlenné
frázy so subjektom (agensom), ale často sa používajú aj
jednočlenné frázy bez agensa (\emph{Prší. – Prišlo mu zle. –
Na stavbe sa tvrdo pracuje.}). Subjekt známy z~kontextu a~tvaru
prísudkového slovesa sa formálne nevyjadruje (\emph{Našiel som
ho.}), jeho prítomnosť vo vete v~podobe osobného zámena je pre
slovenčinu príznaková (\emph{\underbar{Ja} som ho našiel!}).
