\noindent V tejto sérii bielych kníh sme podnikli úvodné kroky na stanovenie
stupňa podpory jazykových technológií mnohých európskych jazykoch. Umožňuje to
na vysokej úrovni porovnať situáciu medzi jazykmi a následne určiť nedostatky a
potreby.

Táto biela kniha dokazuje, že na Slovensku existuje kvalitné prostredie pre
lingvistický výskum aj napriek tomu, že daný technologický priemysel sa tu
dostatočne nerozvinul. Slovenský výskum sa realizuje iba s malým počtom
dostupných technológií a zdrojov. Tento počet je nižší ako v prípade iných
jazykov ako sú čeština alebo poľština a podstatne nižší ako je to v prípade
hlavných jazykov EÚ (angličtiny, nemčiny alebo francúzštiny). Slovenské jazykové
technológie a zdroje majú navyše zjavne horšiu kvalitu.

Náš pohľad na technologickú podporu slovenského jazyka naozaj nemôže byť
optimistický. Na Slovensku máme rodiaci sa výskum v oblasti jazykových
technológií pre slovenčinu, a to najmä na univerzitách, vo vedeckých
pracoviskách, ako aj v malých a stredných podnikoch, ktoré sa zameriavajú na
základný výskum a riešenia špecifických problémov v oblasti jazykových
technológií. Rôzne inštitúcie zasvätili svoj výskum rozvoju jazykových
technológií, ako sú tvorba veľkých korpusov slovenčiny (písaných textov ale aj
hovoreného jazyka), morfologická analýza, strojový preklad, komplexné
rečové interaktívne systémy, rozpoznávanie reči a podobné.  Ich rozvoj je však
nutné ďalej rozvíjať a podporovať.

Ako uvádza táto správa, skôr ako bude možné urobiť nejaký posun v spracovávaní
slovenčiny, musia sa podniknúť okamžité kroky. Je jasné, že sa musí vynaložiť
väčšia snaha vytvoriť zdroje jazykových technológií pre slovenčinu a viesť
výskum, inováciu a rozvoj ako taký. Potreba veľkoobjemových dát a extrémna
komplikovanosť systémov jazykových technológií robí rozvoj novej infraštruktúry
veľmi dôležitým. Podporilo by to spoluprácu všetkých zainteresovaných strán.

Vo financovaní výskumu a rozvoja chýba kontinuita. Krátkodobo koordinované
programy sa striedajú s obdobiami nízkeho až zriedkavého financovania, pričom
je tu badateľný celkový nedostatok koordinácie programov v ďalších krajinách EÚ
a v samotnej Európskej komisii.

Slovenčine by spolu s ďalšími jazykmi pomohol väčší záujem o jazykové technológie a vytvorenie viacjazyčného programu pre európske krajiny a celý svet.\footnote{V. Reding, J. Figeľ, Predslov, in \emph{Human Language
Technologies for Europe}, TC-Star projekt,
\url{http://www.tcstar.org/pubblicazioni/D17_HLT_ENG.pdf}}
