V tejto sérii bielych kníh sa podnikli úvodné kroky na stanovenie podpory jazykových technológií mnohých európskych jazykoch. Umožňuje to porovnávať danú situáciu medzi jazykmi na vysokej úrovni a určenie nedostatkov a potrieb.

Táto biela kniha dokazuje, že na Slovensku existuje kvalitné prostredia pre lingvistický výskum aj napriek tomu, že daný technologický priemysel sa tu dostatočne nerozvinul. Slovenský výskum pracuje iba s malým počtom dostupných technológií a zdrojov. Tento počet je nižší ako v prípade iných jazykov ako sú čeština alebo poština a podstatne nižší ako je to v prípade hlavných jazkov EÚ (angličtiny, nemčiny alebo francúzštiny). Slovenské jazykové technológie a zdroje majú navyše zjavne horšiu kvalitu.

Náš pohľad na technologickú podporu slovenského jazyk naozaj nemôže byť optimistický. Na Slovensku máje rodiacu sa scénu výskumu v oblasti jazykových technológií slovenčiny a to najmä na univerzitách, vo vedeckých inštitúciách, ako aj v malých a stredných podnikoch (MSP), ktoré sa zameriavajú na základný výskum a riešenia špecifických problémov v oblasti jazykových technológií. Rôzne inštitúcie sa zasvätili výskumu a rozvoju jazykových technológií, ako sú akustické modely telefonických systémov, vytváranie rečových signálov v hlučných podmienkach (detekcia reči/ticha, výpočet vektora príznakov, atď), komplexný rečový interaktívny systém, informačno-dialógový systém hlasového vyhľadávania, rozpoznávací systém nepretržitej reči, diktovací systém nezávislí na rečníkovi, atď. Toto je však nutné ďalej rozvíjať a podpodrovať.

Podľa hodnotenia, ktoré je detailne rozpracované v tejto správe, skôr ako bude možné urobiť nejaký posun vo veci  slovenčiny, musia sa podniknúť okamžité kroky. Je jasné, že sa musí vynaložiť väčšia snaha na vytváranie zdrojov pre jazykové technológie v slovenčine a viesť výskum, inovácia a rozvoj ako taký. Potreba veľkoobjemových dát a extrémna komplikovanosť systémov jazykových technológií robí rozvoj novej infraštruktúry životne dôležitým, aby sa tak urýchlila širšia spolupráca a zdieľanie.

Chýba tiež kontinuita vo financovaní výskumu a rozvoja. Krátkodobo koordinované programy sa zvyknú striedať s obdobiami nízkeho alebo zriedkavého financovania a pociťujeme celkový nedostatok spolupráce medzi programami v iných krajinách EÚ a v samotnej Európskej komisii.

Slovenskému jazyku by pomohlo úsilie o väčšiu koordináciu
zameranú na jazykové technológie v súčinnosti s inými jazykmi a na
vytvorenie skutočnej viacjazyčnej agendy pre Európu a svet ako
celok.\endnote{V. Reding, J. Figeľ, Predhovor, in \emph{Human Language
Technologies for Europe}, TC-Star projekt,
\url{http://www.tcstar.org/pubblicazioni/D17_HLT_ENG.pdf}}
