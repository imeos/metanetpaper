Jazykové technológie a ich vývoj sa na Slovensku stále považujú za súčasť vedy a výskumu. Zaraďujú sa najmä do oblasti aplikovaného výskumu, a to v rámci lingvistiky (predovšetkým lexikografie) alebo informatiky. Kontakt s komerčnou sférou je nedostatočný až sporadický. V súčasnosti sa však začínajú jazykové technológie v značnej miere využívať v rôznych softvérových aplikáciách.

Prvé veľké projekty zamerané na jazykové technológie a zdroje na Slovensku boli osobitne schválené a financované vládou. Išlo o projekty {\em Vybudovanie Národného korpusu slovenského jazyka a elektronizácia jazykovedného výskumu v rokoch 2002 -- 2006} a {\em Komplexné spracovanie slovenského jazyka a jeho elektronizácia na účely jazykovedného výskumu}. Oba projekty sa realizovali v Jazykovednom ústave Ľudovíta Štúra Slovenskej akadémie vied.

Projekt {\em Vybudovanie Národného korpusu slovenského jazyka a elektronizácia jazykovedného výskumu v rokoch 2002 – 2006} bol schválený uznesením vlády č. 137/2002. Jeho cieľom bolo vybudovať reprezentatívny korpus slovenského jazyka, ktorý je nevyhnutným základom a materiálovým zdrojom pre všetky lingvistické výskumy a výskumy počítačového spracovania prirodzeného jazyka. Jazykový materiál korpusu je základnou bázou pri tvorbe veľkého lexikografického diela -- Slovníka súčasného slovenského jazyka. 

V rámci projektu sa vytvorilo oddelenie Slovenského národného korpusu, ktoré sa následne stalo vedúcim pracoviskom v oblasti spracovania prirodzeného jazyka na Slovensku. V rokoch 2007 -- 2011 (druhá fáza) projekt pokračoval pod názvom {\em Budovanie Slovenského národného korpusu a elektronizácia jazykovedného výskumu na Slovensku} na základe zmluvy a jeho spolufinancovaní medzi Ministerstvom školstva SR, Ministerstvom kultúry SR a SAV.

V rokoch 2003 -- 2006 sa v rámci štátneho programu výskumu a vývoja Aktuálne otázky rozvoja spoločnosti zároveň realizovala úloha č. 2003SP200280307 {\em Komplexné spracovanie slovenského jazyka a jeho elektronizácia na účely jazykovedného výskumu}. Vďaka riešeniu tejto úlohy sa mohli vyvíjať potrebné nástroje na počítačové spracovanie slovenského jazyka a rozširovať ďalšie zdroje: morfologická a syntaktická anotácia, elektronické lingvistické zdroje, terminologická databáza a pod. Výsledky tohto projektu sa využívajú a ďalej rozvíjajú v pokračujúcom projekte a našli si cestu aj do komerčnej sféry.

Ďalším významným projektom v spracovaní slovenského jazyka bol projekt {\em APD
--  Automatický prepis diktátu pre Ministerstvo spravodlivosti Slovenskej
republiky} koordinovaný Oddelením analýzy a syntézy reči Ústavu informatiky
Slovenskej akadémie vied v spolupráci s Katedrou elektroniky a multimediálnych
komunikácií Technickej univerzity v Košiciach. Projekt sa realizoval v rokoch
2009 -- 2011. Cieľom bolo vytvoriť kompletný systém na prepis hovoreného
slovenského jazyka, špeciálne v oblasti súdnictva. Projekt bol financovaný
Ministerstvom spravodlivosti Slovenskej republiky. V súčasnosti sa systém začína
využívať v pilotnej prevádzke na súdoch Slovenskej republiky.

Tieto projekty boli na Slovensku doteraz jedinou významnou iniciatívou v oblasti počítačového spracovania slovenčiny. Pripravili východisko pre hlbší výskum, ako aj rozmach komerčných projektov v tejto oblasti. Financovanie ďalšieho výskumu je však jednoznačne nevyhnuté. 


