Jazykové technológie predstavujú vysoko interdisciplinárnu oblasť, ktorá si okrem iného vyžaduje expertízy lingvistov, vedcov výpočtovej techniky, matematikov, filozofov, psycholingvistov a~neurológov. Jazykové technológie si na slovenských fakultách stále hľadajú pevné miesto.

Od roku 2007 viedli výskumníci z~Ústavu informatiky Slovenskej akadémie vied (Michal Laclavík a~Martin Šeleng) na Fakulte informačných technológií STU kurz získavania informácií, ktorý sa sústredil na problematiku získavania a~extrahovania informácií\footnote{\url{http://vi.ikt.ui.sav.sk/}}, grafových algoritmov na ich podporu a~spracovanie veľkého množstva dát. Študenti riešia v~tejto doméne rozličné projekty, pričom viacerí používajú slovenské zdroje a~niektorí riešia priamo problémy spracovania slovenského jazyka. Ako príklad uvádzame viaceré projekty zamerané na vytvorenie štatistického, slovníkovo orientovaného alebo algoritmického stemera založeného na projektoch Snowbal alebo Egothor, ako aj projekty zamerané na určovanie účinnosti a~štatistiky pri jednoduchých stemeroch, ktoré fungujú na princípe vynechania samohlások, diakritických znamienok alebo nakoniec celých slovných zakončení atď. Takisto sem patria aj súbežne prebiehajúce projekty štatistických prekladov alebo tvorba automatického slovníka, ktorý prekladá medzi slovenčinou a~inými jazykmi (angličtinou, češtinou). Napokon sú to projekty využívajúce slovníky alebo frekvenčné jazykové slovníky pre aplikácie ako T9, extrahovanie pomenovaných entít s~použitím metód strojového učenia, knižnice ako OpenNLP, tvorba morfologického analyzátora, ako aj extrahovanie udalostí z~e-mailov alebo zo slovenských webových stránok a~pod.

Dodnes neexistuje žiadny pravidelný študijný program počítačovej lingvistiky.
