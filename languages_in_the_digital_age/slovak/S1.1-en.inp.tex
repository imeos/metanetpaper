During the last 60 years, Europe has become a distinct political and economic structure, yet culturally and linguistically it is still very diverse. This means that from Portuguese to Polish and Italian to Icelandic, everyday communication between Europe’s citizens as well as communication in the spheres of business and politics is inevitably confronted with
 language barriers. The EU’s institutions spend about a billion euros a year on maintaining their policy of multilingualism, i.e., translating texts and interpreting spoken communication. Yet does this have to be such a burden? Modern language technology and linguistic research can make a significant contribution to pulling down these linguistic borders. When combined with intelligent devices and applications, language technology will in the future be able to help Europeans talk easily to each other and do business with each other even if they do not speak a common language. 

\boxtext{Language technology builds bridges}

One classic way of overcoming the language barrier is to learn foreign languages. Yet without technological support, mastering the 23 official languages of the member states of the European Union and some 60 other European languages is an insurmountable obstacle for the citizens of Europe and its economy, political debate, and scientific progress.   

The solution is to build key enabling technologies. These will offer European actors tremendous advantages, not only within the common European market but also in trade relations with third countries, especially emerging economies. To achieve this goal and preserve Europe’s cultural and linguistic diversity, it is necessary to first carry out a systematic analysis of the linguistic particularities of all European languages, and the current state of language technology support for them. Language technology solutions will eventually serve as a unique bridge between Europe’s languages. 
