{\bf META-NET} is a Network of Excellence partially funded by the European Commission. The network currently consists of 54 research centres members from 33 European countries. META-NET forges {\bf META}, the Multilingual Europe Technology Alliance, a growing community of language technology professionals and organisations in Europe. 

%obrazok

META-NET fosters the technological foundations for a truly multilingual European information society that:

\begin{itemize}
\item makes communication and cooperation possible across languages;
\item grants all Europeans equal access to information and knowledge in any language;
\item builds upon and advances functionalities of networked information technology.
\end{itemize}

The network supports a Europe that unites as a single digital market and information space. It stimulates and promotes multilingual technologies for all European languages. These technologies support automatic translation, content production, information processing and knowledge management for a wide variety of subject domains and applications. They also enable intuitive language-based interfaces to technology ranging from household electronics, machinery and vehicles to computers and robots. 
