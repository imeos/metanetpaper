META-NET is a Network of Excellence funded by the European Commission. The network currently consists of 47 members from 31 European countries. META-NET fosters the Multilingual Europe Technology Alliance (META), a growing community of language technology professionals and organisations in Europe. 

%obrazok

META-NET cooperates with other initiatives like the Common Language Resources and Technology Infrastructure (CLARIN), which is helping establish digital humanities research in Europe. META-NET fosters the technological foundations for a truly multilingual European information society that:

\begin{itemize}
\item makes communication and cooperation possible across languages;
\item provides equal access to information and knowledge in any language;
\item offers advanced and affordable networked information technology to European citizens.
\end{itemize}

META-NET stimulates and promotes multilingual technologies for all European languages. The technologies enable automatic translation, content production, information processing and knowledge management for a wide variety of applications and subject domains. The network wants to improve current approaches, so better communication and cooperation across languages can take place. Europeans have an equal right to information and knowledge regardless of language. 
