Slovak Online\footnote{\url{http://www.slovake.eu}} is a project providing a web portal enabling free-of-charge Slovak language studies by means of e-learning. Provided language courses in different levels (mini course for tourists, courses A1 and A2 according to the Common European Framework of Reference for Languages) are divided into topical chapters and they are supplemented by audio and video recordings and exercises. The site includes an outline of Slovak grammar and orthography, a multilingual dictionary and language games. It also provides some basic information and trivia about Slovakia and the Slovak language, a library with  extracts from Slovak literary works and the possibility of instant messaging communication between registered users.

The target group of the site is foreigners living in Slovakia, partners in mixed marriages, inhabitants of border area, Slovak living abroad, slovakists and slavists, immigrants, students and tourists. Currently, the site has a German, English, Esperanto, French, Lithuanian, Polish and Slovak version.

The project, the first of its kind, came into existence on the basis of experience gained by the operation of the \emph{lernu!}\footnote{\url{http://www.lernu.net}} site – the biggest portal for Esperanto language studies. The Slovak Online project was supported by European Committee in the frame of the KA2 programme – languages – lifelong learning. The project is coordinated by a civic association Edukácia@Internet (Slovakia), with the partnership of Ľudovít Štúr Institute of Linguistics (Slovakia), Studio GAUS (Germany), Vilniaus universitetas (Lithuania), Wyższa Szkoła Informatyki, Zarządzania i Administracji w Warszawie (Poland) and Slovak Centre London (UK).
