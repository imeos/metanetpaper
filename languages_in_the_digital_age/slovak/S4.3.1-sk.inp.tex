Každý, kto používa kancelársky balík, ako napríklad LibreOffice, už pravdepodobne narazil na funkciu Kontrola pravopisu a~gramatiky, ktorá poukazuje na pravopisné chyby a~navrhuje ich opravu. 40 rokov po tom, čo Ralph Gorin uviedol prvý program na kontrolu pravopisu, sa tieto programy jazyka stali oveľa sofistikovanejšími a~už nepracujú len na princípe porovnávania zoznamu vybraných slov s~pravopisným slovníkom. Oproti jazykovo závislým algoritmom na spracovanie \underbar{morfológie} (napr. tvorenie plurálu) existujú aj algoritmy schopné rozpoznať syntaktické chyby ako chýbajúce sloveso alebo sloveso nezhodné s~podmetom v~osobe a~čísle, ako to môžeme pozorovať napríklad aj vo vete ‘She *write a~letter.’ („Ona písať list.“). Najdostupnejšie funkcie kontroly pravopisu (vrátane uplatnených v~balíku LibreOffice) však v~nasledujúcej prvej strofe básne Jerrolda H. Zara založenej na homofónii nenájdu žiadnu chybu (1992)\footnote{\url{http://www.bios.niu.edu/zar/zar.shtml}}:

\begin{verse}
\emph{%
Eye have a~spelling chequer\\
It came with my Pea Sea.\\
It plane lee marks four my revue\\
Miss Steaks I~can knot sea.
}
\end{verse}

Na spracovanie tohto typu chýb je v~mnohých prípadoch potrebná analýza daného \underbar{kontextu}, ktorá je napríklad potrebná aj na rozhodnutie, či sa má isté slovo písať s~„y“ alebo s~„i“, ako napríklad v~prísloví:

\begin{verse}
\emph{%
Kto chce psa biť, palicu si nájde.\\
\smallskip
Kto chce psom byť, pána si nájde.
}
\end{verse}

Takýto postup si vyžaduje buď formuláciu \underbar{gramatických} pravidiel špecifických pre daný jazyk, čo zároveň predpokladá vysoký stupeň expertízy a~manuálnej práce, alebo využitie takzvaného štatistického \underbar{jazykového modelu}. Takéto modely kalkulujú s~možnosťou výskytu istého slova v~danom kontexte (tzn. s~predchádzajúcimi a~nasledujúcimi slovami). Napríklad, \emph{chce psom byť} je oveľa pravdepodobnejší sled slov ako \emph{chce psom biť} a~naopak, \emph{chce psa biť} je oveľa pravdepodobnejšia vetná konštrukcia než \emph{chce psa byť} (napriek tomu by sme nepochybne dokázali vymyslieť kontexty, v~ktorých sú gramaticky správne všetky štyri uvedené príklady). Štatistický jazykový model môže byť automaticky derivovaný využívaním veľkého množstva (korektných) jazykových dát (t.~j. \underbar{korpusu}). Tieto prístupy však boli vyvinuté a~hodnotené len na anglických jazykových dátach a~nedajú sa automaticky priamo aplikovať na slovenčinu s~jej nestálym slovosledom a~bohatou flexiou.

Používanie funkcie Kontrola pravopisu a~gramatiky nie je obmedzené len na nástroje kancelárskych balíkov, ale využíva sa aj v~redakčných (sub)systémoch. Spolu s~rastúcim počtom technických produktov sa za posledné obdobie rapídne zvýšil aj počet technickej dokumentácie. Strach spoločností zo sťažností zákazníkov a~z~nárokov na náhradu škody, ktorá bola zapríčinená nesprávnymi alebo nesprávne pochopenými inštrukciami, spôsobil, že sa spoločnosti začali viac sústreďovať na kvalitu technickej dokumentácie a~zároveň na medzinárodný trh. Pokroky v~spracovávaní prirodzeného jazyka viedli k~rozvoju autorizovaného podporného softvéru, ktorý slúži zostavovateľovi technickej dokumentácie na využívanie slovnej zásoby a~vetných štruktúr v~súlade s~istými pravidlami a~(spoločnými) terminologickými obmedzeniami.

Existujúce zariadenia kontroly pravopisu a~gramatiky pre slovenský jazyk sú väčšinou založené na slovníku základných slovných tvarov (lem) skombinovanom so súborom morfologických pravidiel, ktoré umožňujú analýzu alebo generovanie všetkých (správnych) slovných tvarov. Hoci sa tento jednoduchý postup javí ako uspokojivý, má dve zásadné nevýhody. Prvou nevýhodou je nesprávne určenie zdanlivo správnych slovných tvarov v~dôsledku nevhodného kontextu. Druhou nevýhodou je neschopnosť rozlišovať skutočné pravopisné chyby od správnych slovných tvarov, ktoré však nie sú obsiahnuté v~slovníku. Takéto slová však budú vzhľadom na prirodzené pribúdanie nových slov, vedeckých a~technických termínov v~lexikóne existovať stále.

Okrem kontroly pravopisu a~autorizovanej podpory je funkcia Kontrola pravopisu a~gramatiky takisto dôležitá v~oblasti výučby jazyka prostredníctvom počítača a~využíva sa aj v~automaticky korigovaných zadaniach odoslaných do webových vyhľadávačov, napríklad do Google, ktorý v~niektorých jazykových mutáciách (ale nie v~slovenčine) uvádza opravené návrhy frázou „Mali ste na mysli\dots“
