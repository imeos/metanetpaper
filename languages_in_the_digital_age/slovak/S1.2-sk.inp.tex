Automatické preklady a~nástroje na spracovanie reči v~súčasnosti nedosahujú svoje ciele v~dostatočnej miere. V~tomto odvetví sú dominantné najmä súkromné podniky zo Severnej Ameriky. Na konci 70. rokov si Európa začala uvedomovať hlboký význam jazykových technológií pre integráciu Európy a~začala podporovať prvé výskumné projekty, napr. projekt EUROTRA. Zároveň sa realizovali viaceré národné projekty s~hodnotnými výsledkami, ktoré sa však výraznejšie do európskeho povedomia nedostali. V~iných jazykových spoločenstvách, napr. v~Indii s~22 úradnými jazykmi a~v~Južnej Afrike s~11 úradnými jazykmi, sa predsa len podarilo vybudovať dlhodobé národné programy na výskum a~rozvoj jazykových technológií.

Riadiaci činitelia v oblasti jazykových technológií v~súčasnosti nevyužívajú presné štatistické prístupy s~dokonalejšími jazykovými metódami a~znalosťami, napr. vety sa automaticky prekladajú porovnávaním novej vety s~tisíckami iných viet, ktoré predtým už niekto preložil. Preto kvalita závisí do značnej miery od~množstva a~kvality príkladov v~korpuse. Automatický preklad jednoduchých viet akéhokoľvek jazyka, ktorý obsahuje dostatočné množstvo textového materiálu, dosahuje dobré výsledky. Takéto štatistické metódy však zlyhávajú, pretože čím je prekladaný text dlhší a~rozmanitejší, tým vyššia je pravdepodobnosť nepresného prekladu textu. 

Európska únia sa preto od roku 2006 rozhodla financovať projekty, ako je EuroMatrix alebo EuroMatrixPlus a~od roku 2010 aj iTranslate4, ktoré vykonávajú základný aj~aplikovaný výskum a~zhromažďujú zdroje, ktoré by pre všetky európske jazyky stanovili sofistikované riešenia jazykových technológií. Aby sme vybudovali kvalitné aplikácie, je nevyhnutné detailnejšie preskúmať štruktúru jazyka.

Európsky výskum v~tejto oblasti už dosiahol svoje úspechy. Napríklad EÚ používa na prekladateľské služby prekladový softvér MOSES, ktorý bol vyvinutý vďaka viacerým európskym výskumným projektom. Jeden z~nich, projekt Verbmobil, ktorý financovalo ministerstvo školstva a~výskumu v~Nemecku, mal vo sfére prekladu reči na istý čas vedúcu pozíciu. Mnohé z~výskumných laboratórií boli, bohužiaľ, zrušené, príp. presunuté, čoho dôsledkom je tendencia v~Európe rozvíjať výskumnú činnosť v~izolácii, a~teda so~slabším prienikom na~svetový trh.
