V~oblasti tlače bolo technologickým zlomom vynájdenie tlačiarne. Ľudia sa namáhali pri prácnom vyhľadávaní, čítaní, prekladaní a~sumarizácií poznatkov. Čakali sme až na Edisona, ktorý zachytil hovorenú reč, a~teda jeho technológia vytvárala analógové kópie. 

Digitálne jazykové technológie dokážu vytvoriť automatický preklad, vygenerovať obsah, spracúvať informácie a~riadiť vedomostný manažment, ktorý je aplikovateľný na všetky európske jazyky. Jazykové technológie môžu tiež podporovať rozvoj používateľských rozhraní pre domácu elektroniku, zariadenia, dopravné prostriedky, počítače či~roboty. Hoci existuje mnoho takýchto prototypov, komerčné a~priemyselné aplikácie sú stále iba v~prvotných štádiách rozvoja. Nedávne úspechy vo výskume a~rozvoji vytvorili skutočný priestor na nové možnosti. Povedzme strojový preklad je už primerane presný v~špecifických oblastiach; experimentálne aplikácie poskytujú mnohojazyčnú informáciu a~vedomostný manažment, ako aj generovanie obsahu v~mnohých európskych jazykoch.

Ako pri väčšine technológií, aj prvé jazykové aplikácie, ako napríklad hlasové používateľské rozhrania a~dialógové systémy, boli vyvinuté pre vysoko špecializované domény a~často vykazujú obmedzenú použiteľnosť. Ale v~oblasti vzdelávania a~zábavného priemyslu sú obrovské príležitosti na integráciu jazykových technológií do hier, edukačných pomôcok, simulačných prostredí, prípadne vzdelávacích programov. Mobilné informačné služby, softvéry na počítačovú podporu učenia sa jazyka, e‑learningové prostredia, nástroje na sebahodnotenie a~softvéry na detekciu plagiátorstva sú len zlomkom možností, v~ktorých zohrávajú jazykové technológie dôležitú úlohu. Popularita sociálnych aplikácií ako Twitter, Pokec alebo~Facebook naznačuje potrebu sofistikovanejších jazykových technológií, ktoré dokážu monitorovať príspevky, sumarizovať diskusie, navrhnúť názorové trendy, detegovať emocionálne reakcie, identifikovať porušenie autorských práv alebo sledovať zneužitie diela.

Jazykové technológie predstavujú pre Eu\-róp\-sku úniu obrovskú
príležitosť. Môžu pomôcť pri problematike viacjazyčnosti
v~Eu\-ró\-pe – keďže  obchodná sféra, rôzne organizácie či školy
sú charakteristické svojou národnostnou rozmanitosťou. Ľudia chcú
komunikovať napriek jazykovým hraniciam. Jazykové technológie môžu
pomôcť prekonať tieto bariéry vďaka slobodnému a~otvorenému
používaniu rozličných jazykov. Pri pohľade na budúcnosť nám
zavedenie inovatívnych a~multilingválnych jazykových technológií
pre Európu takisto môže pomôcť v~komunikácii s~celosvetovými
partnermi a~s~ich   viacjazyčnými spoločenstvami. Jazykové
technológie možno vnímať aj ako „podporné“ prostriedky, ktoré
prekonávajú jazykovú rozmanitosť a~zbližujú jazykové
spoločenstvá.

Napokon, jedno odvetvie výskumu predstavuje aj používanie jazykových technológií pri záchranných akciách v~oblastiach postihnutých katastrofami, kde ich použitie môže byť otázkou života a~smrti, napríklad budúce inteligentné roboty s~mnohorakými jazykovými schopnosťami majú potenciál zachraňovať ľudské životy. 
