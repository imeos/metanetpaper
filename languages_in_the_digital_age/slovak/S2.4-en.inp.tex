In the world of print, the technology breakthrough was the rapid
duplication of an image of a text (a page) using a suitably powered
printing press. Human beings had to do the hard work of looking up,
reading, translating, and summarizing knowledge. We had to wait until
Edison to record spoken language – and again his technology simply
made analogue copies.

Digital language technology can now automate the very processes of translation, content production, and knowledge management for all European languages. It can also empower intuitive language/speech-based interfaces for household electronics, machinery, vehicles, computers and robots. Real-world commercial and industrial applications are still in the early stages of development, yet R\&D achievements are creating a genuine window of opportunity. For example, machine translation is already reasonably accurate in specific domains, and experimental applications provide multilingual information and knowledge management as well as content production in many European languages. 

As with most technologies, the first language applications such as voice-based user interfaces and dialogue systems were developed for highly specialised domains, and often exhibit limited performance. But there are huge market opportunities in the education and entertainment industries for integrating language technologies into games, cultural heritage sites, edutainment packages, libraries, simulation environments and training programmes. Mobile information services, computer-assisted language learning software, e-learning environments, self-assessment tools and plagiarism detection software are just some of the application areas where language technology can play an important role. The popularity of social media applications like Twitter, Pokec or Facebook suggest a further need for sophisticated language technologies that can monitor posts, summarise discussions, suggest opinion trends, detect emotional responses, identify copyright infringements or track misuse.

\boxtext{Language technology helps overcome the ``disability'' of linguistic diversity}

Language technology represents a tremendous opportunity for the European Union. It can help address the complex issue of multilingualism in Europe – the fact that different languages coexist naturally in European businesses, organisations and schools. But citizens need to communicate across these language borders criss-crossing the European Common Market, and language technology can help overcome this final barrier while supporting the free and open use of individual languages. Looking even further forward, innovative European multilingual language technology will provide a benchmark for our global partners when they begin to enable their own multilingual communities. Language technology can be seen as a form of ‘assistive’ technology that helps overcome the ‘disability’ of linguistic diversity and make language communities more accessible to each other.

Finally, one active field of research is the use of language technology for rescue operations in disaster areas, where performance can be a matter of life and death: Future intelligent robots with cross-lingual language capabilities have the potential to save lives.
