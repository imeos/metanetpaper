V~súčasnosti nemôžeme presne odhadnúť, aká bude informačná
spoločnosť o~niekoľko rokov. Je však veľmi pravdepodobné, že
revolúcia v~komunikačných technológiách spojí ľudí, ktorí
hovoria rozličnými jazykmi, napriek jazykovým bariéram. Momentálne
môžeme cítiť istý tlak na ľudí, aby sa učili cudzie jazyky,
a~najmä na ľudí, ktorí by mali vytvárať nové technologické
aplikácie na zabezpečenie vzájomného dorozumenia.
V~aktuálnej globálnej ekonomike a~informačnom priestore sa denne
konfrontujeme s~narastajúcim počtom jazykov, hovoriacimi a~novými
témami. Súčasná popularita sociálnych médií (Wikipedia, Facebook,
Twitter, Pokec, YouTube a~pod.) je len špičkou tohto pokrokového
ľadovca.

Dnes dokážeme prenášať gigabajty textu po celom svete za pár
sekúnd, hoci sú v~jazyku, ktorému nerozumieme. Podľa nedávnej
správy, ktorú vydala Európska komisia, 57~\% používateľov
internetu platí za tovar a~služby v~cudzom jazyku (angličtina je
najbežnejšia, hneď za ňou nasleduje francúzština, nemčina
a~španielčina). 55~\% používateľov číta obsah v~cudzom jazyku,
pričom iba 35~\% používa iný jazyk na písanie e-mailov alebo
posielanie komentárov na webe.\footnote{European Commission
Directorate-General Information Society and Media, \emph{User language
preferences online}, Flash Eurobarometer \#313, 2011 \newline
(\url{http://ec.europa.eu/public_opinion/flash/fl_313_en.pdf}).} Pred
niekoľkými rokmi mohla byť angličtina internetová lingua franca,
pretože prevažná väčšina materiálov na webe bola v~angličtine.
Situácia sa však medzičasom modifikovala – rozrástlo sa množstvo
inojazyčného on-line obsahu (najmä~ázijského a~arabského).

Táto digitálna priepasť, ktorá je zapríčinená jazykovými
bariérami, prekvapivo nezískala dostatok pozornosti na verejnosti.
Digitálny svet si kladie naliehavú otázku: „Ktorým európskym
jazykom sa bude dariť v~zosieťovanej informačnej a~znalostnej
spoločnosti a~ktoré zaniknú?“
