We cannot predict exactly what the future information society will look like. But there is a strong likelihood that the revolution in communication technology is bringing people speaking different languages together in new ways. This is putting pressure on individuals to learn new languages and especially on developers to create new technology applications to ensure mutual understanding and access to shareable knowledge. In a global economic and information space, more languages, speakers and content interact more quickly with new types of media. The current popularity of social media (Wikipedia, Facebook, Twitter, YouTube, Pokec, Google+) is only the tip of the iceberg.

\boxtext{A global economy and information space confronts us with different languages, speakers and content}

Today, we can transmit gigabytes of text around the world in a few seconds before we recognise that it is in a language we do not understand. According to a recent report from the European Commission, 57\% of Internet users in Europe purchase goods and services in languages that are not their native language. (English is the most common foreign language followed by French, German and Spanish.) 55\% of users read content in a foreign language while only 35\% use another language to write e-mails or post comments on the Web.\footnote{European Commission Directorate-General Information Society and Media, \emph{User language preferences online}, Flash Eurobarometer \#313, 2011 (\url{http://ec.europa.eu/public_opinion/flash/fl_313_en.pdf}).} A few years ago, English might have been the lingua franca of the Web—the vast majority of content on the Web was in English—but the situation has now drastically changed. The amount of online content in other European (as well as Asian and Middle Eastern) languages has exploded.

Surprisingly, this ubiquitous digital divide due to language borders has not gained much public attention; yet, it raises a very pressing question: Which European languages will thrive in the networked information and knowledge society, and which are doomed to disappear?
