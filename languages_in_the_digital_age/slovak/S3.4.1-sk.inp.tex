\noindent Slovak Online\footnote{\url{http://www.slovake.eu}} je projekt umožňujúci bezplatné štúdium
slovenského jazyka prostredníctvom e-learningu na webovom portáli. Ponúkané jazykové
kurzy rôznych úrovní (minikurz pre turistov, kurzy A1 a~A2 podľa
Spoločného európskeho referenčného rámca) sú rozdelené do
tematických kapitol a~doplnené audio a~videonahrávkami a~cvičeniami.
Stránka obsahuje aj prehľad slovenskej gramatiky a~pravopisu,
prekladový slovník a~jazykové hry. Sprístupňujú sa tu takisto základné
informácie a~zaujímavosti o~Slovensku a~slovenčine, knižnica
s~ukážkami diel slovenských autorov a~možnosť komunikácie medzi
zaregistrovanými používateľmi formou textových správ.

\boxtext{Cieľovou skupinou sú cudzinci, partneri v~zmiešaných manželstvách, obyvatelia pohraničných oblastí, Slováci žijúci v~zahraničí, slovakisti, slavisti a ďalší záujemcovia}

Cieľovou skupinou projektu sú cudzinci žijúci na území Slovenska,
partneri v~zmiešaných manželstvách, obyvatelia pohraničných
oblastí, Slováci žijúci v~zahraničí, slovakisti a~slavisti,
imigranti, študenti a~turisti. V~súčasnosti stránka existuje
v~nemeckej, anglickej, esperantskej, francúzskej, litovskej, poľskej
a~slovenskej verzii.

Projekt, ktorý je prvým svojho druhu, vznikol na základe skúseností
získaných prevádzkou stránky
lernu!\footnote{\url{http://www.lernu.net}}, najväčšieho portálu na
učenie sa jazyka esperanto. Projekt Slovak Online podporila Európska
komisia v~rámci programu KA2 – languages – program celoživotného
vzdelávania. Realizátorom projektu je občianske združenie
Edukácia@Internet (Slovensko), partnermi sú Jazykovedný ústav
Ľudovíta Štúra SAV (Slovensko), Studio GAUS (Nemecko), Vilniaus
universitas (Litva), Wyższa Szkoła Informatyki, Zarządzania
i~Administracji w Warszawie (Poľsko) a~Slovak Centre London (Spojené
kráľovstvo Veľkej Británie a~Severného Írska).
