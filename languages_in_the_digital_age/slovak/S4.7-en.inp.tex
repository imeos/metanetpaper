The following table summarises the current state of language technology support for the Slovak language. The rating for existing tools and resources was generated by leading experts in the field who provided estimates based on a scale from 0 (very low) to 6 (very high) according to seven criteria.

%tabulka

The key results for the Slovak language can be summed up as follows:

\begin{itemize}
\item While some specific corpora of high quality exist, a very large syntactically annotated corpus is not available.
\item For Slovak, the Slovak National Corpus is the reference language corpus, but only the query interface is generally available, due to licensing restrictions.
\item On the other hand, the Corpus of Spoken Slovak is not encumbered by copyright law and is therefore publicly available, but its size is minuscule compared to the corpus of written language.
\item Many of the resources lack standardisation, i.e., even if they exist, sustainability is not given; concerted programs and initiatives are needed to standardise data and interchange formats.
\item Semantics is more difficult to process than syntax; text semantics is more difficult to process than word and sentence semantics.
\item There is an ontological resource for Slovak (even mapped to English ontological resources) but its coverage is limited.
\item Standards do exist for semantics in the sense of world knowledge (RDF, OWL, etc.); they are, however, not easily applicable to NLP tasks.
\item Written text processing is more mature than speech processing (especially speech recognition)
\item Many of the resources taken as standard in other languages are missing for Slovak; NLP language research in Slovakia is severely underfunded.
\item Some of the research and development activities for the Slovak language is carried out in the Czech Republic by Czech universities and Czech SMEs.
\item Speech Recognition of the Slovak language is studied at several universities and workplaces but the amount of free tools and data is limited.
\item In contrast with speech recognition, speech synthesis is less covered by universities and other workplaces.
\item In the field of speech synthesis, there are open source packages available together with several other simple synthesizers but the speech synthesis with more natural voices is not available.
\item Slovak dialogue systems are not extended due to the poor accessibility of high quality speech recognition modules of the Slovak language.
\end{itemize}


Cross-language comparison

