V~nasledujúcej tabuľke ponúkame sumarizáciu súčasného stavu jazykových technológií pre slovenský jazyk. Kritériá existujúcich nástrojov a~zdrojov v~rozmedzí 0 (veľmi nízky) až 6 (veľmi vysoký) navrhli poprední odborníci.

%tabulka

\emph{Tabuľka sa dá zhrnúť do niekoľkých kľúčových bodov:}

\begin{itemize}
\item Na Slovensku existuje niekoľko špecializovaných kvalitných korpusov, ale dosiaľ tu nie je dostupný žiaden veľký, syntakticky anotovaný korpus.
\item Referenčným korpusom pre slovenčinu je Slovenský národný korpus. Kvôli licenčným obmedzeniam je však prístupné len jeho vyhľadávacie rozhranie.
\item Na druhej strane, korpus hovorených textov nepodlieha zákonu o~ochrane autorských práv a~je verejne dostupný. Jeho rozsah je však oproti rozsahu korpusu písaných textov nepatrný.
\item Mnohé zdroje sú neštandardizované, t.~j. aj keď existujú, nie sú udržiavané. Na štandardizáciu dát a~výmenu formátov je nevyhnutné spoločné úsilie a~iniciatíva.
\item Spracovať sémantiku je ťažšie ako spracovať syntax; spracovať textovú sémantiku je ťažšie než spracovať lexikálnu a~vetnú sémantiku.
\item Slovenčina má ontologický zdroj (zmapovaný na anglické ontologické zdroje), no jeho pokrytie je obmedzené.
\item V~zmysle reprezentácie vedomostí o~svete existujú štandardy pre sémantiku (RDF, OWL, atď.), ktoré sa však ťažko aplikujú na úlohy NLP.
\item Spracovanie písaného textu je rozvinutejšie ako spracovanie hovoreného textu (najmä rozpoznávania reči).
\item V~slovenčine chýbajú mnohé zdroje, ktoré sa v~iných jazykoch považujú za štandard; jazykový výskum NLP je na Slovensku veľmi slabo financovaný.
\item Niektoré výskumné a~vývojové aktivity pre slovenčinu sa realizujú v~Českej republike – ~na českých univerzitách a~v~súkromnom sektore.
\item Výskum rozpoznávania reči pre slovenčinu prebieha na niekoľkých univerzitách a~výskumných pracoviskách, no množstvo voľne dostupných nástrojov a~dát je obmedzené.
\item Naopak, syntézu reči spracúvajú univerzity a~iné vedecké pracoviská v~oveľa menšom rozsahu.
\item V~oblasti syntézy reči sú dostupné OpenSource balíky a~niekoľko jednoduchých syntetizátorov reči, no syntéza reči s~prirodzenejšími hlasmi nie je dostupná.
\item Slovenské dialógové systémy sú veľmi málo rozšírené v~dôsledku nízkej dostupnosti kvalitných modulov rozpoznávania reči pre slovenčinu.
\end{itemize}
