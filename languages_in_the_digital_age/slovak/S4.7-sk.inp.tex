\noindent Súčasný stav jazykových technológií je rozdielny v každej krajine. Na porovnanie situácie medzi jednotlivými jazykmi slúži nasledujúce ohodnotenie vzorových aplikácií v oblasti strojového prekladu a spracovania jazyka, textovej analýzy a zdrojov príslušného jazyka, ktoré sú nevyhnutné na tvorbu jazykových technológií. Tieto jazyky sa zoskupili na základe nasledujúcej päťbodovej škály:


\begin{figure*}
\small
\centering
\begin{tabular}
{ % defines color for each column.
>{\columncolor{corange5}} p{.17\linewidth}@{\hspace{.027\linewidth}}
>{\columncolor{corange4}}p{.17\linewidth}@{\hspace{.027\linewidth}}
>{\columncolor{corange3}}p{.17\linewidth}@{\hspace{.027\linewidth}}
>{\columncolor{corange2}}p{.17\linewidth}@{\hspace{.027\linewidth}}
>{\columncolor{corange1}}p{.17\linewidth} 
}
\rowcolor{orange1} % redefines color for all columns in row 1
\begin{center}\vspace*{-2mm}\textbf{Klaster 1}\end{center} & 
\begin{center}\vspace*{-2mm}\textbf{Klaster 2}\end{center} & 
\begin{center}\vspace*{-2mm}\textbf{Klaster 3}\end{center} & 
\begin{center}\vspace*{-2mm}\textbf{Klaster 4}\end{center} & 
\begin{center}\vspace*{-2mm}\textbf{Klaster 5}\end{center} \\ \addlinespace

& \vspace*{0.5mm}angličtina
& \vspace*{0.5mm}nemčina \newline   
taliančina \newline  
fínčina \newline 
francúzština \newline 
holandčina \newline 
portugalčina \newline 
španielčina \newline
čeština \newline 
& \vspace*{0.5mm}baskičtina \newline 
bulharčina \newline 
dánčina \newline 
estónčina \newline 
galícijčina \newline 
gréčtina \newline  
írčina \newline  
katalánčina \newline 
nórčina \newline 
poľština \newline 
švédčina \newline
srbčina \newline 
slovenčina \newline 
slovinčina \newline 
maďarčina  \newline
& \vspace*{0.5mm}islandčina \newline  
chorvátčina \newline 
lotyština \newline 
litovčina \newline 
maltčina \newline 
rumunčina\\
\end{tabular}
\label{fig:speech_cluster_sk}
\caption{Klastre jazykov pre spracovanie reči}
\end{figure*}

\begin{figure*}
\small
\centering
\begin{tabular}
{ % defines color for each column.
>{\columncolor{corange5}} p{.17\linewidth}@{\hspace{.027\linewidth}}
>{\columncolor{corange4}}p{.17\linewidth}@{\hspace{.027\linewidth}}
>{\columncolor{corange3}}p{.17\linewidth}@{\hspace{.027\linewidth}}
>{\columncolor{corange2}}p{.17\linewidth}@{\hspace{.027\linewidth}}
>{\columncolor{corange1}}p{.17\linewidth} 
}
\rowcolor{orange1} % redefines color for all columns in row 1
\begin{center}\vspace*{-2mm}\textbf{Klaster 1}\end{center} & 
\begin{center}\vspace*{-2mm}\textbf{Klaster 2}\end{center} & 
\begin{center}\vspace*{-2mm}\textbf{Klaster 3}\end{center} & 
\begin{center}\vspace*{-2mm}\textbf{Klaster 4}\end{center} & 
\begin{center}\vspace*{-2mm}\textbf{Klaster 5}\end{center} \\ \addlinespace

& \vspace*{0.5mm} angličtina 
& \vspace*{0.5mm} francúzština \newline 
španielčina
& \vspace*{0.5mm}nemčina \newline 
taliančina \newline 
katalánčina \newline
holandčina \newline 
poľština \newline 
rumunčina \newline 
maďarčina 
& \vspace*{0.5mm}baskičtina \newline 
bulharčina \newline 
dánčina \newline 
estónčina \newline 
fínčina \newline 
galícijčina \newline 
gréčtina \newline 
írčina \newline 
islandčina \newline 
chorvátčina \newline 
lotyština \newline 
litovčina \newline 
maltčina \newline 
nórčina \newline 
portugalčina \newline 
švédčina \newline 
srbčina \newline 
slovenčina \newline 
slovinčina \newline 
čeština \newline
\end{tabular}
\label{fig:mt_cluster_sk}
\caption{Klastre jazykov pre strojový preklad}
\end{figure*}

\begin{figure*}
  \small
  \centering
  \begin{tabular}
{ % defines color for each column.
>{\columncolor{corange5}} p{.17\linewidth}@{\hspace{.027\linewidth}}
>{\columncolor{corange3}}p{.17\linewidth}@{\hspace{.027\linewidth}}
>{\columncolor{corange3}}p{.17\linewidth}@{\hspace{.027\linewidth}}
>{\columncolor{corange2}}p{.17\linewidth}@{\hspace{.027\linewidth}}
>{\columncolor{corange1}}p{.17\linewidth} 
}
\rowcolor{orange1} % redefines color for all columns in row 1
\begin{center}\vspace*{-2mm}\textbf{Klaster 1}\end{center} & 
\begin{center}\vspace*{-2mm}\textbf{Klaster 2}\end{center} & 
\begin{center}\vspace*{-2mm}\textbf{Klaster 3}\end{center} & 
\begin{center}\vspace*{-2mm}\textbf{Klaster 4}\end{center} & 
\begin{center}\vspace*{-2mm}\textbf{Klaster 5}\end{center} \\ \addlinespace

& \vspace*{0.5mm}angličtina
& \vspace*{0.5mm}nemčina \newline 
  francúzština \newline 
  taliančina \newline 
  holandčina \newline 
  španielčina
& \vspace*{0.5mm}baskičtina \newline 
  bulharčina \newline 
  dánčina \newline 
  fínčina \newline 
  galícijčina \newline 
  gréčtina \newline 
  katalánčina \newline 
  nórčina \newline 
  poľština \newline 
  portugalčina \newline 
  rumunčina \newline 
  švédčina \newline 
  slovenčina \newline 
  slovinčina \newline 
  čeština \newline 
  maďarčina \newline 
& \vspace*{0.5mm}estónčina \newline 
  írčina \newline 
  islandčina \newline 
  chorvátčina \newline 
  lotyština \newline 
  litovčina \newline 
  maltčina \newline 
  srbčina \\
  \end{tabular}
\label{fig:text_cluster_sk}
\caption{Klastre jazykov pre textovú analýzu}
\end{figure*}

\begin{figure*}
  \small
  \centering
\begin{tabular}
{ % defines color for each column.
>{\columncolor{corange5}} p{.17\linewidth}@{\hspace{.027\linewidth}}
>{\columncolor{corange4}}p{.17\linewidth}@{\hspace{.027\linewidth}}
>{\columncolor{corange3}}p{.17\linewidth}@{\hspace{.027\linewidth}}
>{\columncolor{corange2}}p{.17\linewidth}@{\hspace{.027\linewidth}}
>{\columncolor{corange1}}p{.17\linewidth} 
}
\rowcolor{orange1} % redefines color for all columns in row 1
\begin{center}\vspace*{-2mm}\textbf{Klaster 1}\end{center} & 
\begin{center}\vspace*{-2mm}\textbf{Klaster 2}\end{center} & 
\begin{center}\vspace*{-2mm}\textbf{Klaster 3}\end{center} & 
\begin{center}\vspace*{-2mm}\textbf{Klaster 4}\end{center} & 
\begin{center}\vspace*{-2mm}\textbf{Klaster 5}\end{center} \\ \addlinespace
    
& \vspace*{0.5mm}angličtina
& \vspace*{0.5mm}nemčina \newline 
    francúzština \newline 
    holandčina \newline 
    švédčina \newline 
    čeština \newline 
    poľština \newline 
    maďarčina 
& \vspace*{0.5mm} baskičtina \newline 
    bulharčina \newline 
    dánčina \newline 
    estónčina \newline 
    fínčina \newline 
    galícijčina \newline 
    gréčtina \newline 
    katalánčina \newline 
    chorvátčina \newline 
    nórčina \newline 
    portugalčina \newline 
    rumunčina \newline 
    srbčina \newline 
    slovenčina \newline 
    slovinčina \newline
&  \vspace*{0.5mm} írčina \newline 
    islandčina \newline 
    lotyština \newline 
    litovčina \newline 
    maltčina  \\
  \end{tabular}
  \caption{Klastre jazykov pre zdroje}
  \label{fig:resources_cluster_sk}
\end{figure*}



 

\begin{itemize}
\item Klaster 1: vynikajúca podpora jazykových technológií
\item Klaster 2: dobrá podpora
\item Klaster 3: mierna podpora
\item Klaster 4: čiastočná podpora
\item Klaster 5: slabá alebo žiadna podpora
\end{itemize}

Podpora jazykových technológií sa merala podľa nasledovných kritérií:

\begin{itemize}
\item Spracovanie reči: Kvalita existujúcich technológií na rozpoznávanie reči, kvalita existujúcich technológií rečovej syntézy, záber domén, počet a~veľkosť existujúcich hovorených korpusov, množstvo a~pestrosť dostupných na reči založených aplikácií
\item Strojový preklad: Kvalita existujúcich technológií strojového prekladu, počet pokrytých jazykových párov, pokrytie lingvistických fenoménov a~domén, kvalita a~veľkosť existujúcich paralelných korpusov, množstvo a~pestrosť dostupných aplikácií strojového prekladu
\item Textová analýza: Kvalita a~pokrytie existujúcich technológií textovej analýzy (morfológie, syntaxe, sémantiky), pokrytie lingvistických fenoménov a~domén, množstvo a~pestrosť dostupných aplikácií, kvalita a~veľkosť existujúcich (anotovaných) textových korpusov, kvalita a~pokrytie existujúcich lexikálnych zdrojov (napr. WordNet) a~gramatík
\item Zdroje: Kvalita a~veľkosť existujúcich textových korpusov, hovorených korpusov a~paralelných korpusov, kvalita a~pokrytie existujúcich lexikálnych zdrojov a~gramatík
\end{itemize} 

