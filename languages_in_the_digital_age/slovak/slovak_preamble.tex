%                                     MMMMMMMMM                                         
%                                                                             
%  MMO    MM   MMMMMM  MMMMMMM   MM    MMMMMMMM   MMD   MM  MMMMMMM MMMMMMM   
%  MMM   MMM   MM        MM     ?MMM              MMM$  MM  MM         MM     
%  MMMM 7MMM   MM        MM     MM8M    MMMMMMM   MMMMD MM  MM         MM     
%  MM MMMMMM   MMMMMM    MM    MM  MM             MM MMDMM  MMMMMM     MM     
%  MM  MM MM   MM        MM    MMMMMM             MM  MMMM  MM         MM     
%  MM     MM   MMMMMM    MM   MM    MM            MM   MMM  MMMMMMM    MM
%
%
%          - META-NET Language Whitepaper | Slovak Metadata -
% 
% ----------------------------------------------------------------------------

\usepackage{polyglossia}
%\usepackage{tipa}
\usepackage{nth}
\setotherlanguages{slovak,english}


\title{Slovenský jazyk v~di\-gitál\-nom veku --- The \mbox{Slovak} Language in the Digital Age}

% Title for the spine of the cover
\spineTitle{The Slovak Language in the Digital Age --- Slovenský jazyk v digitálnom veku}

\subtitle{White Paper Series --- Séria bielych kníh}

\author{
  Mária Šimková\\
  Radovan Garabík \\
  Katarína Gajdošová \\
  Michal Laclavík \\
  Slavomír Ondrejovič \\
  Jozef Juhár \\
  Ján Genči \\
  Karol Furdík \\
  Helena Ivoríková \\
  Jozef Ivanecký 
}
\authoraffiliation{
  Mária Šimková {\footnotesize  Jazykovedný ústav Ľ. Štúra SAV}\\
  Radovan Garabík {\footnotesize Jazykovedný ústav Ľ. Štúra SAV}\\
  Katarína Gajdošová {\footnotesize Jazykovedný ústav Ľ. Štúra SAV}\\
  Michal Laclavík {\footnotesize Ústav informatiky SAV}\\
  Slavomír Ondrejovič {\footnotesize Jazykovedný ústav Ľ. Štúra SAV}\\
  Jozef Juhár {\footnotesize Technická univerzita v~Košiciach}\\
  Ján Genči {\footnotesize Technická univerzita v~Košiciach}\\
  Karol Furdík {\footnotesize Technická univerzita v~Košiciach}\\
  Helena Ivoríková {\footnotesize Studia Academica Slovaca UK}\\
  Jozef Ivanecký {\footnotesize European Media Laboratory}
}

%\authoraffiliation{
%  Mária Šimková~ {\small JÚĽŠ SAV}\\
%  Radovan Garabík~ {\small JÚĽŠ SAV}\\
%  Katarína Gajdošová~ {\small JÚĽŠ SAV}\\
%  Michal Laclavík~ {\small ÚI SAV}\\
%  Slavomír Ondrejovič~ {\small JÚĽŠ SAV}\\
%  Jozef Juhár~ {\small TUKE}\\
%  Ján Genči~ {\small TUKE}\\
%  Karol Furdík~ {\small TUKE}\\
%  Helena Ivoríková~ {\small SAS UK}\\
%  Jozef Ivanecký~ {\small EML}\\
%}

\editors{
  Georg Rehm, Hans Uszkoreit\\(redakcia, \textcolor{grey1}{editors})
}

% Text in left column on backside of the cover
\SpineLText{\selectlanguage{english}%
  In everyday communication, Europe’s citizens, business partners and politicians are inevitably confronted with language barriers.  
  Language technology has the potential to overcome these barriers and to provide innovative interfaces to technologies and knowledge. 
  This white paper presents the state of language technology support for the Slovak language. 
  It is part of a series that analyzes the available language resources and technologies for 31~European languages. 
  The analysis was carried out by META-NET, a Network of Excellence funded by the European Commission.
  META-NET consists of 54 research centres in 33 countries, who cooperate with stakeholders from economy, government agencies, research organisations, NGOs, language communities and European universities. 
  META-NET’s vision is high-quality language technology for all European languages.
}

% Text in right column on backside of the cover
\SpineRText{\selectlanguage{slovak}%
  V bežnej komunikácii sú občania Európy, obchodní partneri či politici neustále konfrontovaní s jazykovými bariérami. Jazykové technológie by mohli časom prekonať tieto bariéry a poskytnúť inovatívne technologické a znalostné prístupy. Táto biela kniha odráža súčasný stav jazykových technológií pre slovenčinu. Je súčasťou série, ktorá analyzuje dostupné jazykové zdroje a technológie pre 31 jazykov Európy. Analýza sa realizuje pod záštitou META-NET-u, siete excelentnosti, ktorá je financovaná Európskou komisiou. META-NET pozostáva z 54 výskumných centier v 33 krajinách, ktoré spolupracujú so zainteresovanými stranami z oblasti ekonómie, vládnych agentúr, výskumných organizácií, nevládnych organizácií, jazykových komunít a európskych univerzít. Víziou META-NET-u je tvorba vysokokvalitných jazykových technológií pre všetky európske jazyky.
}

% Quotes from VIPs on backside of the cover
%FIXME
\quotes{
 % \begin{spacing}{1}
   % \small
    ``This book is proof of deeper European
    integration increasing with the need for qualitative progress in LT for
    Slovak."\\
      \textcolor{grey2}{--- Jozef Ivanecký (European Media Laboratory)}\\[3mm]
    ``Aktuálne globalizačné procesy, mnohojazyčná Európa
    a existencia jazykových technológií vytvárajú podmienky pre vývoj
    nástrojov uľahčujúcich komunikáciu v rôznych oblastiach.
    Interdisciplinárny projekt META-NET predstavuje cestu, ktorou sa bude
    uberať ďalší rozvoj jazykových technológií v podmienkach
    jazykovej plurality zohľadňujúc potreby nositeľov väčších aj
    menších jazykov."\\
      \textcolor{grey2}{--- Viera Rosová (podpredsedníčka Slovenskej akadémie vied pre ekonomiku)}
  %\end{spacing}
}

% Funding notice left column
\FundingLNotice{\selectlanguage{slovak}
  Autori tohto dokumentu ďakujú autorom Bielej knihy pre nemčinu za povolenie používať vybrané jazykovo nezávislé materiály z ich dokumentu. \cite{lwpgerman}.

  \bigskip
  
  Táto biela kniha bola financovaná prostredníctvom Siedmeho rámcového programu a~Programu podpory politiky v~oblasti informačných a~komunikačných technológií Európskej komisie na základe dohôd T4ME (Grantová dohoda 249119), CESAR (Grantová dohoda 271022), METANET4U (Grantová dohoda 270893) a META-NORD (Grantová dohoda 270899).}

% Funding notice right column
\FundingRNotice{\selectlanguage{english}
  The authors of this document
  are grateful to the authors of the White Paper on German for
  permission to re-use selected language-independent materials from
  their document. \cite{lwpgerman}.
  
  \bigskip
  
  The development of this White Paper has been funded by the Seventh
  Framework Programme and the ICT Policy Support Programme of the
  European Commission under the contracts T4ME (Grant Agreement
  249119), CESAR (Grant Agreement 271022), METANET4U (Grant Agreement
  270893) and META-NORD (Grant Agreement 270899).}
