The primary, general corpus {\bf\emph{prim}} covers Slovak texts
which arose after the year 1955. Three major styles are represented in the
corpus: journalistic, fiction, professional (including popular science) as well as
various other genres and areas. The corpus database comprises texts from
throughout Slovakia as well as texts by Slovaks living abroad, texts originally
in Slovak and translated from other languages. For specialized research, the
general corpus \emph{prim-*-all} can be divided into independent subcorpora:

\begin{itemize}
\item \emph{sane} – does not contain linguistic texts, texts without diacritics, texts from Slovaks living abroad etc.
\item \emph{vyv} – journalistic, fiction, and professional texts are represented by a third share each
\item \emph{inf} – journalistic texts only
\item \emph{prf} – professional texts only
\item \emph{img} – fiction texts only
\item \emph{skimg} – original Slovak fiction texts only
\end{itemize}

The use of the texts the in Slovak National Corpus is governed by the provisions of the Copyright Act.

The corpus texts and text units are accompanied by: external,
bibliographical, style, and genre
annotation\footnote{\url{http://korpus.sk/bibstyle.html}} and internal,
morphological or morphosyntactic
annotation\footnote{\url{http://korpus.sk/morpho.html}}. All the words
are lemmatised.


 
