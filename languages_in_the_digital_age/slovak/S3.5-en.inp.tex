The creation of the Slovak National Corpus Department of Ľ. Štúr
Institute of Linguistics has been stimulated by the worldwide trend
involving the language and information technology development, the need
to create the source data for dictionaries. The department was founded
in 2002 with the support of the Ministry of Culture of the Slovak
Republic (program for maintaining the national language), the Ministry
of Education (informatisation and use of innovative  methods in
teaching) and the Slovak Academy of Sciences. An eight-member team of
predominantly young scientists is involved in the project: Construction
of the Slovak National Corpus and the electronization of linguistic
research in Slovakia \cite{simkova2006b}.

In the initial stages of forming the department, its corpus database,
and the specific tools for its construction and use, the Slovak National
Corpus department regularly held scientific seminars presented by
eminent foreign specialists. Selected contributions were compiled in
publication \cite{simkova2006a}. Since 2005 the Slovak National Corpus
team has organized the biennial international conference Slovko\footnote{\url{http://korpus.juls.savba.sk/~slovko}} on
natural language processing and corpus linguistic research, with
participation by Slovak as well as foreign researchers (from Austria,
Bulgaria, Croatia, the Czech Republic, France, Germany, Hungary, Poland,
Russia, Slovenia, Spain, Ukraine, etc.). The published conference
proceedings contain contributions on the preparation, research, and
results of diverse national and international projects in the field of
construction and use of general and specific corpora and databases, in
the field of language analysis and synthesis, automatic translations,
computer lexicography and terminography, e-learning etc. 

The department
members have been involved in 7 Slovak projects and 6 international
projects and cooperation. In 2005 they were awarded the Slovak Academy
of Sciences Prize for construction of scientific infrastructure. 
