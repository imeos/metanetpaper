\noindent Slovenské nárečia predstavujú dorozumievací prostriedok autochtónneho obyvateľstva príslušných nárečových oblastí v~každodennom spoločenskom a~pracovnom styku s~najbližším okolím. Slovenské nárečia sa doteraz dedia z~generácie na generáciu v~ústnej podobe, hoci aj tu dochádza v~porovnaní s~minulosťou k~procesu nivelizácie.

Slovnú zásobu jednotlivých nárečí na území Slovenska opisuje Slovník slovenských nárečí, podrobnejšie a~v~rozšírení na ďalšie jazykové roviny sú viaceré nárečia opísané v~samostatných monografiách.

Slovenské nárečia sa členia na tri základné skupiny (pozri obrázok~\ref{fig:dialects_sk}):

\begin{itemize}
\item[a)] Západoslovenské nárečia sú rozšírené v~trenčianskej, nitrianskej, trnavskej, myjavskej oblasti a~v~ďalších regiónoch.

\begin{enumerate}
\setcounter{enumi}{19}
\item Hornotrenčianske nárečia
\item Dolnotrenčianske nárečie
\item Považské nárečie 
\item Stredonitrianske nárečia 
\item Dolnonitrianske nárečia 
\item Nárečia trnavského okolia
\item Záhorské nárečia 
\end{enumerate}

\item[b)] Stredoslovenskými nárečiami sa hovorí v~regiónoch Liptov, Orava, Turiec, Tekov, Hont, Novohrad, Gemer a~vo~zvolenskej oblasti.

\begin{enumerate}
\setcounter{enumi}{9}
\item Liptovské nárečia
\item Oravské nárečia
\item Turčianske nárečie
\item Hornonitrianske nárečia
\item Zvolenské nárečia
\item Tekovské nárečia
\item Hontianske nárečie
\item Novohradské nárečia
\item Gemerské nárečia
\end{enumerate}

\item[c)] Východoslovenské nárečia možno nájsť v~regiónoch Spiš, Šariš, Zemplín a~Abov.

\begin{enumerate}
\setcounter{enumi}{29}
\item Spišské nárečia
\item Abovské nárečia
\item Šarišské nárečia
\item Zemplínske nárečia
\item Sotácke nárečia
\item Užské nárečia
\setcounter{enumi}{39}
\item Oblasť goralských nárečí
\item Oblasť ukrajinských nárečí
\item Nárečovo rôznorodé oblasti 
\item Oblasť maďarských nárečí
\end{enumerate}
\end{itemize}

\medskip

Tieto skupiny sa ďalej bohato a~pestro členia („Čo dedina, to reč iná“), pričom členitosťou sa nárečia vyznačujú predovšetkým v~hornatých oblastiach. Práve hornatosť krajiny spôsobovala v~minulosti istú (rečovú) izolovanosť obyvateľstva v~rámci jednotlivých žúp. Pod tieto špecifiká sa podpísalo ďalej aj prevrstvovanie a~migrácia obyvateľstva, kolonizácie, miešanie odlišných nárečových typov, pôsobenie susedných slovanských i~neslovanských jazykov, zmeny v~zamestnaní obyvateľstva a~pod. Podľa povahy nárečí a~výskytu jednotlivých charakteristických javov možno zaradiť do uvedených skupín aj slovenské nárečia v~Maďarsku, Srbsku, Chorvátsku, Rumunsku, Bulharsku a~v~iných krajinách, kam sa v~minulosti presídlili veľké kompaktné skupiny. Pri menšom počte starých písomných pamiatok sú slovenské nárečia základným prameňom slovenskej historickej gramatiky.

