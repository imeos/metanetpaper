\noindent Táto biela kniha je súčasťou série, ktorá propaguje najnovšie poznatky
a~potenciál jazykových technológií. Je určená učiteľom, novinárom, politikom
i~rôznym jazykovým spoločnostiam.

V~európskych krajinách majú jazykové technológie rozličnú úroveň aj využitie.
Z~toho dôvodu sú aj opatrenia potrebné na ďalšiu podporu výskumu a~vývoja
jazykových technológií pre každý jazyk odlišné. Požadované opatrenia závisia od
faktorov, akými sú napríklad zložitosť daného jazyka či veľkosť jazykovej
komunity.

META-NET, sieť excelentnosti, financovaná z~fondov Európskej komisie,
vypracovala analýzu súčasných jazykových zdrojov a~technológií. Analýza
zahŕňala okrem 23 oficiálnych európskych jazykov aj iné dôležité národné
i~regionálne jazyky. V~každom jazyku výsledky analýzy poukázali na značné
medzery vo výskume. Podrobnejšia expertná analýza a~zhodnotenie
momentálnej situácie pomôže maximalizovať efektivitu ďalších výskumov
a~minimalizovať prípadné riziká.

META-NET pozostáva z~54 výskumných centier v~33 krajinách, ktoré spolupracujú
so zainteresovanými stranami z~oblasti komerčných firiem, vládnych agentúr,
priemyslu, výskumných organizácií, softvérových spoločností, poskytovateľov
technológií a~európskych univerzít. Spoločne majú vytvoriť jednotnú víziu
strategického plánu výskumu, ktorá ukazuje, ako môžu jazykové technológie
riešiť rozdiely vo výskume do roku 2020.
