Biela kniha je súčasťou série, ktorá propaguje najnovšie poznatky a~potenciál jazykových technológií. Môže slúžiť učiteľom, novinárom, politikom i~rôznym jazykovým spoločnostiam.

V~európskych krajinách majú jazykové technológie rozličné využitie aj dostupnosť. Z~toho dôvodu sú aj opatrenia potrebné na ďalšiu podporu výskumu a~vývoja jazykových technológií pre každý jazyk odlišné. Požadované opatrenia závisia od faktorov, akými sú napríklad zložitosť daného jazyka či veľkosť jazykovej komunity.

META-NET, sieť excelentnosti, financovaná z~fondov Európskej komisie, vypracovala analýzu súčasných jazykových zdrojov a~technológií. Analýza zahŕňala okrem 23 oficiálnych európskych jazykov aj iné dôležité národné i~regionálne jazyky. V~každom jazyku výsledky analýzy po\-uká\-zali na množstvo nedostatkov vo výs\-ku\-me. Podrobnejšia expertná analýza a~zhodnotenie momentálnej situácie po\-mô\-že maximalizovať efektivitu ďalších výskumov a~minimalizovať prípadné riziká.

META-NET sa realizuje v~47 výskumných centrách v~31 krajinách, ktoré spolupracujú so zainteresovanými stranami z~oblasti komerčných firiem, vládnych agentúr, priemyslu, výskumných organizácií, softvérových spoločností, poskytovateľov technológií a~európskych univerzít. Spoločne majú vytvoriť jednotnú víziu strategického plánu výskumu, ktorý ukazuje, ako môžu jazykové technológie riešiť rozdiely vo výskume do roku 2020.
