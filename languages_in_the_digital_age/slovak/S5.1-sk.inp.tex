\noindent  Od 1. februára 2010 v sebe META-NET zahŕňa tri nasledujúce okruhy činností: META-VISION, META-SHARE a~META-RESEARCH. 

{\bf META-VISION} podporuje dynamickú a~vplyvnú komunitu zainteresovaných strán, ktorú zjednocuje strategický výskumný program (SRA; Strategic Research Agenda) pre oblasť európskych jazykových technológií. Hlavným cieľom META-VISION je vytvoriť v~Európe ucelenú a~súdržnú komunitu jazykových technológií cez zoskupenie rôznych zainteresovaných strán. Súčasná séria Bielych kníh bola pripravovaná v 29 jazykoch. Spoločná technologická vízia sa vytvárala v troch vizionárskych skupinách. Technologická rada META vznikla s cieľom prediskutovať a pripraviť strategický výskumný program založený na vízii vzájomnej spoluprácie celej komunity jazykových technológií.

{\bf META-SHARE} vytvára možnosti na výmenu a~sprístupnenie zdrojov. Sieť dátových úložísk bude obsahovať publikácie, súbory dát, multimediálne súbory, výpočtové nástroje, služby a~aplikácie usporiadané do štandardizovaných kategórií. Zdroje sa dajú jednoducho vyhľadať. Sú to jednak bezplatné a~voľne prístupné materiály, ale aj zdroje s~obmedzeným a~spoplatneným použitím. 

{\bf META-RESEARCH} spája príbuzné technologické oblasti. Táto oblasť sa snaží využiť poznatky iných oblastí a~zužitkovať ich na výskum jazykových technológií. Tento okruh sa obzvlášť zameriava na špičkový výskum v oblasti strojového prekladu, zbierania dát, prípravy dátových súborov a organizovania jazykových zdrojov na účel hodnotenia; na zostavovanie inventára nástrojov a metód a organizovanie seminárov či školení pre členov komunity. 




