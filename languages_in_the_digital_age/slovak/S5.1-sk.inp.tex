META-NET sa začal realizovať 1. februára 2010 s~cieľom vylepšiť výskum jazykových technológií. Sieť podporuje Európu tým, že ju spája ako jediný digitálny trh a~informačný priestor. Pod META-NET spadajú nasledujúce okruhy činností v~oblastiach: META-VISION, META-SHARE a~META-RESEARCH, ktoré ďalej prehlbujú ciele META-NET-u.

%obrazok

META-VISION podporuje dynamickú a~vplyvnú komunitu zainteresovaných strán, ktorú zjednocuje \emph{Strategický výskumný program} (SRA; Strategic Research Agenda) pre oblasť európskych jazykových technológií. Hlavným cieľom META-VISION je vytvoriť v~Európe ucelenú a~súdržnú komunitu jazykových technológií cez zoskupenie rôznych zainteresovaných strán. Cieľom prezentácií META-NET-u v~prvom roku bolo dostať sa do povedomia verejnosti (FLaReNet Forum v~Španielsku, Dni jazykových technológií v~Luxembursku, JIAMCATT 2010 v~Luxembursku, LREC 2010 na Malte, EAMT 2010 vo Francúzsku a~ICT 2010 v~Belgicku). Podľa prvých odhadov už META-NET kontaktoval viac než 2500 odborníkov na jazykové technológie. Na stretnutí META-FORUM 2010 v~Bruseli predložil META-NET prvé vízie spoločného budovania viac ako 250 účastníkom, pričom zúčastnení aktéri vyjadrili svoje názory na tieto vízie na niekoľkých interaktívnych stretnutiach.

META-SHARE vytvára možnosti na výmenu a~sprístupnenie zdrojov. Sieť dátových úložísk bude obsahovať publikácie, súbory dát, multimediálne súbory, výpočtové nástroje, služby a~aplikácie usporiadané do štandardizovaných kategórií. Zdroje sa dajú jednoducho vyhľadať. Sú to jednak bezplatné a~voľne prístupné materiály, ale aj zdroje s~obmedzeným a~spoplatneným použitím. META-SHARE sleduje aj súčasné jazykové dáta, nástroje a~systémy, ale aj produkty potrebné na budovanie a~hodnotenie nových technológií, produktov a~služieb. Slúži poskytovateľom a~používateľom jazykových prostriedkov a~technológií, expertom na lokalizáciu, výskumníkom, prekladateľom a~jazykovým odborníkom malých či veľkých spoločností. META-SHARE rieši celý cyklus vývoja jazykových technológií od výskumu až k~tvorbe inovatívnych produktov a~služieb. META-SHARE si kladie za cieľ poskytnúť takúto infraštruktúru ako otvorenú, distribuovanú, zabezpečenú a~schopnú spolupracovať v~oblasti jazykových technológií. 

META-RESEARCH spája príbuzné technologické oblasti. Táto oblasť sa snaží využiť poznatky iných oblastí a~zužitkovať ich na výskum jazykových technológií, ktoré sa zaoberajú sémantikou strojového prekladu, optimalizáciou rozdelenia práce v~hybridnom strojovom preklade a~využitím kontextu pre preklad a~pre prípravu bázy strojového prekladu. META-RESEARCH spolupracuje s~inými odvetviami a~disciplínami, ako je napr. strojové učenie alebo sémantická webová komunita (Semantic Web community). META-RESEARCH zbiera dáta, spracúva ich a~štrukturalizuje jazykové zdroje na účely hodnotenia, zostavuje inventár nástrojov a~metód, organizuje workshopy a~vzdelávacie akcie pre členov. Táto oblasť META-NET-u jasne identifikovala aspekty strojového prekladu, v~ktorom zohráva sémantika dôležitú úlohu. Okrem toho odporučila riešenie problému integrácie sémantickej informácie do strojového prekladu. META-RESEARCH takisto dokončuje nový jazykový zdroj pre strojový preklad (Annotated Hybrid Sample MT Corpus), ktorý obsahuje dáta z~anglicko-nemeckého, anglicko-španielskeho a~anglicko-českého jazyka, podobne ako softvér pre viacjazyčné korpusy na webe.
