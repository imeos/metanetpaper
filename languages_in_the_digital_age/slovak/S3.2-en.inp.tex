The Slovak language started to develop directly from Old Church Slavonic in the
\nth{10} century. Main changes took place and were stabilised before the
\nth{15} century; some of them equally (reduction of the nasal
vowels) and the others differentially (vocalisation of hard jers in eastern and
the western parts of contemporary Slovakia was of western Slavic type and in
the central part it was of non-western Slavic type). A part of these changes
was also the decomposition of the Old Church Slavonic syllable structure, which
influenced the changes in declension and conjugation. Although the Slovak and
Czech languages developed under different conditions for a long period
(Slovakia became a part of the Kingdom of Hungary in the \nth{11}
century), they have remained close to each other. However, some specific
features of the Slovak language (the forms \emph{l\textbf{a}keť}/elbow,
\emph{Če\textbf{s}i}/the Czechs, the suffix \emph{-m} in the first person
singular, etc.) are parallel in South Slavic languages. With some less
significant characteristics, Slovak resembles Polish (prefix \emph{pre-} unlike
the Czech \emph{pro-}, preservation of the consonant \emph{dz}, and several
expressions such as \emph{teraz/now, pivnica/cellar}). By other characteristics
it approaches East Slavic languages. Therefore we talk about the central
position of Slovak among the Slavic languages and about the good
understandability of Slovak for the members of other Slavic nations. 

\boxtext{Some specific features of the Slovak languages are parallel in South Slavic languages}

Modified Latin with diacritical marks is used in Slovakia. The palatalisation of consonants is marked with a caron (\emph{ď, ť, ň, ľ}; also used for graphemes \emph{ž, š, č, dž}) and the length of vowels and consonants by an acute accent (\emph{á, é, í, ý, ó, ú, ŕ, ĺ}). Vowels are not subject to reduction, they are pronounced in full form in each position. In Slovak, besides vowels and consonants, several diphthongs (ia, ie, iu) and one u-diftong (ô) occur. 

\boxtext{Modified Latin with diacritical marks is used in Slovakia}

A phonetic speciality of the Slovak standard language (and of Central Slovak dialects) is the so-called rhythmic rule, which is a tendency not to have two long syllables adjacent (\emph{pekný}/nice – \emph{krásny}/beautiful, \emph{prosím}/please – \emph{smútim}/I am sad). Slovak has dynamic stress on the first syllable of the word that is not very strong (it is weaker than in Russian or Polish). In prepositional phrases with one-syllable prepositions, the stress is usually put on the prepositions: \emph{v škole}/near the school.

Unlike Russian or Czech, Slovak has a simpler structure of declension and conjugation paradigms. However, the system of substantive and verbal forms is clearly structured, in spite of unification tendencies. The Slovak language has six grammatical cases (nominative, genitive, dative, accusative, locative and instrumental). Unlike Czech, the vocative is not frequently used in Slovak anymore; it is usually identical with the nominative. Slovak recognises 4 genders: masculine animate and masculine inanimate, feminine, and neuter for nouns and related adjectives, pronouns and numerals. Masculine and feminine genders with animate concreta are determined according to the natural gender and in other cases it is a matter of convention, which is not signalised by any article, and only sometimes by the ending (e.g.: \emph{strom/tree} – masc. inanimate, \emph{jabloň/apple tree} – fem., \emph{jablko/apple} – neuter.). For each gender there are given several patterns in student grammar books and their paradigms differ especially in –  G/A sing. and N/G plur. (e.g.: masculine animate \emph{chlap} / \emph{chlap\textbf{a}} / \emph{chlap\textbf{i}} / \emph{chlap\textbf{ov}}, \emph{hrdina} / \emph{hrdin\textbf{u}} / \emph{hrdin\textbf{ovia}} / \emph{hrdin\textbf{ov}}; \emph{žena} / \emph{žen\textbf{y}} / \emph{žen\textbf{u}} / \emph{žen\textbf{y}} / \emph{žien}, \emph{dlaň} / \emph{dlan\textbf{e}} / \emph{dlaň} / \emph{dlan\textbf{e}} / \emph{dlan\textbf{í}}). In some patterns and cases there is some significant homonymy: G and A sing. of animate masculine, N and A sing. of inanimate masculine, in feminine gender of G sing. and N plur., etc. There are possible transitions among the paradigms, e.g. the feminine paradigm \emph{kosť} is nowadays more productive than the paradigm \emph{dlaň}. Words formally assigned to a certain paradigm quite often do not follow the pattern, which is the reason for many exceptions. In NLP literature a much larger number of paradigms is mentioned \cite{pales1994,sokolova1999,sokolova2007a}.

In the conjugation of verbs, three tenses are distinguished: past, present, and future. In addition to the three forms – indicative, imperative, and conditional, most of the verbs exist in two aspects – perfective (\emph{zavolať}) and imperfective (\emph{volať}). Slovak is a highly inflectional language with elements of analytical constructions (especially in verb forms such as \emph{budem písať}, \emph{bol by som prišiel}). The grammar function of words is clearly designated by inflection, therefore the word order in a sentence is relatively free. From the syntactic point of view, Slovak is characterised by a basic construction scheme S(ubject) – V(erb) – O(bject), however, it is a rather theoretical scheme, whose realisation varies as a consequence of the free word order. Cases are helpful for the unambiguous determination of S and O (S is in N case, O is usually in A or G, D cases, rarely in other cases), homonymy of the forms, however, can be a cause of an uncertainty in subject and object functions (especially in foreign proper names but also in several other cases). 

\boxtext{Highly unbound verbal morphemes cause problems for foreigners and computer processing}

Special problems for foreigners and computer processing of the Slovak language are caused by highly movable verbal morphemes \emph{sa}, \emph{si}, by which the verb can be preceded or followed even in distance of several words, or even in a different part of the sentence structure (\emph{Netrvalo dlho, keď \textbf{sa} im ich hviezda, ktorú predtým videli v diaľke, zrazu \textbf{priblížila}}). In Slovak, two-unit sentences with a subject (agents) are the most frequent but one-unit constructions without agents are also frequently used (\emph{Prší., Prišlo mu zle., Na stavbe sa tvrdo pracuje.}). The subject is known from the context and the form of the predicative verb is not expressed formally (\emph{Našiel som ho.}); its presence in the sentence in the form of a personal pronoun marks an emphasis (\emph{\underbar{Ja} som ho našiel!}).

