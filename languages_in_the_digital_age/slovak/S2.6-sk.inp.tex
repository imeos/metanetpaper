Aby sme si vedeli lepšie predstaviť prácu počítača s~osvojovaním si jazyka, stručne zhrnieme spôsoby, akými si ľudia osvojujú prvý a~druhý jazyk. Potom si načrtneme, ako si jazyk osvojujú jazykové technológie. 

Ľudia si jazyk osvojujú dvoma rozličnými spôsobmi. V~prvom prípade sa dieťa učí jazyk tak, že počúva rozhovory medzi hovoriacimi v~danom jazyku. Presnejšie, jazykovými vzormi sú preňho používatelia jazyka, ako napríklad rodičia, súrodenci alebo ~iní rodinní príslušníci. Dieťa začína produkovať prvé slová a~krátke frázy vo veku približne dvoch rokov. Deje sa to vďaka špeciálnej genetickej dispozícii imitovať zvuky a~následne si odôvodniť to, čo počuje.

Učenie sa druhého jazyka zvyčajne vyžaduje oveľa viac úsilia, lebo dieťa už nie je súčasťou jazykového spoločenstva rodených hovoriacich. V~školskom veku sa cudzie jazyky väčšinou osvojujú učením gramatických štruktúr, slovnej zásoby a~pravopisu z~kníh a~vzdelávacích materiálov, ktoré opisujú jazykové systémy pomocou abstraktných pravidiel, tabuliek a~textových ukážok. Učenie sa cudzieho jazyka si vyžaduje veľa času i~úsilia a~s~pribúdajúcim vekom to už nie je také jednoduché.
Jazykové technológie nadobúdajú jazykové schopnosti podobným spôsobom ako ľudia. Štatistické prístupy získavajú jazykové schopnosti z~rozmanitého výberu konkrétnych príkladov textov. Tieto algoritmy strojového učenia modelujú istý druh jazykovej schopnosti, ktorá dokáže odvodzovať vzory ako slová, krátke frázy a~celé vety používané v~jednom jazyku alebo prekladané z~jedného jazyka do druhého.  

Tento štatistický prístup vyžaduje obsah miliónov viet a~svoj kvalitatívny výkon zvyšuje s~narastajúcim množstvom analyzovaných textov. To je jeden z~dôvodov, prečo sa prevádzkovatelia vyhľadávačov snažia získať čo najviac písomných materiálov. Korekcia pravopisu v~textových procesoroch a~služby ako Google Hľadať na webu\footnote{oficiálny názov služby} a~Google Translate sú závislé od štatistických prístupov. Veľkou výhodou štatistiky je, že stroj sa učí veľmi rýchlo, hoci kvantita nie vždy korešponduje s~kvalitou.

Systémy založené na pravidlách sú druhým najväčším typom jazykových technológií. Vysoko špecializovaní odborníci z~oblasti lingvistiky, počítačovej lingvistiky a~počítačovej vedy kódujú gramatické analýzy (pravidlá prekladu) a~zostavujú zoznam slovnej zásoby (lexikóny). Vytvorenie týchto systémov je časovo náročné a~prácne. Niektoré z~týchto hlavných systémov strojového prekladu založených na pravidlách sa  rozvíjajú už viac než 20 rokov. Ich výhodou je, že odborní pracovníci môžu systematickejšie kontrolovať spracúvanie jazyka, čo prispieva k~oprave prípadných chýb v~softvéri. Vďaka týmto systémom sa používateľovi poskytne detailnejšia spätná väzba, osobitne vtedy, keď sa tieto systémy používajú na výučbu jazykov. Z~finančných dôvodov sú systémy založené na pravidlách prístupné iba pre rozšírenejšie jazyky.

Silné a~slabé stránky štatistických systémov a~systémov založených na pravidlách sa navzájom dopĺňajú. Aktuálny výskum sa sústreďuje na hybridné prístupy, ktoré tieto dva systémy kombinujú. Doteraz sa však viac uplatnili viac v~priemyselných aplikáciách než v~oblasti výskumu.

Ako sme si v~tejto kapitole mohli prečítať, v~dnešnej informačnej spoločnosti sa využíva množstvo jazykových technológií. Kvôli viacjazyčnosti to platí najmä pre európsky ekonomický a~informačný priestor. Jazykové technológie zaznamenali v~posledných rokoch značný rozmach. Ich permanentné zdokonaľovanie však je nevyhnutnosťou. V~nasledujúcich kapitolách opíšeme úlohu slovenského jazyka v~európskej informačnej spoločnosti a~zhodnotíme súčasný stav jazykových technológií pre slovenský jazyk.
