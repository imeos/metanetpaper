META-NET was launched on 1 February 2010 with the goal of advancing research in language technology (LT). The network supports a Europe that unites as a single digital market and information space. META-NET has conducted several activities that further its goals. META-VISION, META-SHARE and META-RESEARCH are the network’s three lines of action.

%obrazok

META-VISION fosters a dynamic and influential stakeholder community that unites around a shared vision and a common strategic research agenda (SRA). The main focus of this activity is to build a coherent and cohesive LT community in Europe by bringing together representatives from highly fragmented and diverse groups of stakeholders. In the first year of META-NET, presentations at the FLaReNet Forum (Spain), Language Technology Days (Luxembourg), JIAMCATT 2010 (Luxembourg), LREC 2010 (Malta), EAMT 2010 (France) and ICT 2010 (Belgium) centred on public outreach. According to initial estimates, META-NET has already contacted more than 2,500 LT professionals to develop its goals and visions with them. At the META-FORUM 2010 event in Brussels, META-NET communicated the initial results of its vision building process to more than 250 participants. In a series of interactive sessions, the participants provided feedback on the visions presented by the network. 

META-SHARE creates an open, distributed facility for exchanging and sharing resources. The peer-to-peer network of repositories will contain language data, tools and web services that are documented with high-quality metadata and organised in standardised categories. The resources can be readily accessed and uniformly searched. The available resources include free, open source materials as well as restricted, commercially available, fee-based items. META-SHARE targets existing language data, tools and systems as well as new and emerging products that are required for building and evaluating new technologies, products and services. The reuse, combination, repurposing and re-engineering of language data and tools plays a crucial role. META-SHARE will eventually become a critical part of the LT marketplace for developers, localisation experts, researchers, translators and language professionals from small, mid-sized and large enterprises. META-SHARE addresses the full development cycle of LT—from research to innovative products and services. A key aspect of this activity is establishing META-SHARE as an important and valuable part of a European and global infrastructure for the LT community. 

META-RESEARCH builds bridges to related technology fields. This activity seeks to leverage advances in other fields and to capitalise on innovative research that can benefit language technology. In particular, this activity wants to bring more semantics into machine translation (MT), optimise the division of labour in hybrid MT, exploit context when computing automatic translations and prepare an empirical base for MT. META-RESEARCH is working with other fields and disciplines, such as machine learning and the Semantic Web community. META-RESEARCH focuses on collecting data, preparing data sets and organising language resources for evaluation purposes; compiling inventories of tools and methods; and organising workshops and training events for members of the community. This activity has already clearly identified aspects of MT where semantics can impact current best practices. In addition, the activity has created recommendations on how to approach the problem of integrating semantic information in MT. META-RESEARCH is also finalising a new language resource for MT, the Annotated Hybrid Sample MT Corpus, which provides data for English-German, English-Spanish and English-Czech language pairs. META-RESEARCH has also developed software that collects multilingual corpora that are hidden on the Web.
