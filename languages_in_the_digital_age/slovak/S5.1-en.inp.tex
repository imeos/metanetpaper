Launched on 1 February 2010, META-NET has already conducted various activities in its three lines of action META-VISION, META-SHARE and META-RESEARCH.  

%obrazok

{\bf META-VISION} fosters a dynamic and influential stakeholder community that
unites around a shared vision and a common strategic research agenda
(SRA). The main focus of this activity is to build a coherent and
cohesive LT community in Europe by bringing together representatives
from highly fragmented and diverse groups of stakeholders. The present White Paper was prepared together with volumes for 29 other languages. The shared technology vision was developed in three sectorial Vision Groups. The META Technology Council was established in order to discuss and to prepare the SRA based on the vision in close interaction with the entire LT community.

{\bf META-SHARE} creates an open, distributed facility for exchanging and
sharing resources. The peer-to-peer network of repositories will contain
language data, tools and web services that are documented with
high-quality metadata and organised in standardised categories. The
resources can be readily accessed and uniformly searched. The available
resources include free, open source materials as well as restricted,
commercially available, fee-based items. 

{\bf META-RESEARCH} builds bridges to related technology fields. This activity
seeks to leverage advances in other fields and to capitalise on
innovative research that can benefit language technology. In particular, the action line focuses on conducting leading-edge research in machine translation, collecting data, preparing data sets and organising language resources for evaluation purposes; compiling inventories of tools and methods; and organising workshops and training events for members of the community. 
