\noindent Celosvetový trend rozvoja jazykových a informačných
technológií a potreba zodpovedajúcej materiálovej bázy pre
koncipovanie slovníkov a opis slovenského jazyka podnietil vznik
korpusov a korpusovej lingvistiky aj na Slovensku. V r. 2002 vzniklo s
podporou Ministerstva kultúry SR (program starostlivosti o štátny
jazyk), Ministerstva školstva SR (informatizácia a využívanie
inovatívnych metód vo výučbe) a Slovenskej akadémie vied oddelenie
Slovenského národného korpusu Jazykovedného ústavu Ľ. Štúra SAV
(SNK JÚĽŠ SAV). Kolektív ôsmich, prevažne mladých vedeckých,
odborných a technických pracovníkov bol poverený riešením úlohy
Budovanie Slovenského národného korpusu a elektronizácia
jazykovedného výskumu na Slovensku \cite{simkova2006b}.

V začiatkoch budovania pracoviska, korpusovej databázy a
špecifických nástrojov na jej tvorbu a využívanie sa na pôde
oddelenia SNK konali pravidelné vedecké semináre, na ktorých
prednášali významní zahraniční odborníci. Vybrané príspevky
boli zhrnuté do publikácie \emph{Insight into the Slovak and Czech Corpus Linguistics} \cite{simkova2006a}. Od r. 2005 organizuje kolektív SNK bienálnu medzinárodnú konferenciu
Slovko\footnote{\url{http://korpus.juls.savba.sk/~slovko/}} o
počítačovom spracovaní prirodzených jazykov a
korpusovolingvistických výskumoch. Na podujatí sa pravidelne
zúčastňujú domáci aj zahraniční bádatelia (z Bulharska, Česka,
Francúzska, Chorvátska, Maďarska, Nemecka, Poľska, Rakúska, Ruska,
Slovinska, Španielska, Ukrajiny a i.). V zborníkoch z týchto
konferencií je publikovaných vyše sto príspevkov o príprave,
riešení a výsledkoch národných a medzinárodných projektov v
oblasti budovania a využívania všeobecných i špecifických korpusov
a databáz, v oblasti analýzy a syntézy reči, automatizovaného
prekladu, počítačovej lexikografie a termínografie, e-learningu a
pod. 

Pracovníci oddelenia SNK JÚĽŠ SAV sa doteraz zapojili do 7
projektov v rámci Slovenska a do 6 medzinárodných projektov a
spoluprác\footnote{\url{http://korpus.sk/projects.html}}. V r. 2005
získali Cenu SAV za budovanie infraštruktúry pre vedu. 

