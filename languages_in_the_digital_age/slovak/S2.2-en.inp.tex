While the printing press helped step up the exchange of information in Europe, it also led to the extinction of many European languages. Regional and minority languages were rarely printed and languages such as Romani and Rusyn were mostly limited to oral forms of transmission, which in turn restricted their scope of use. Will the Internet have the same impact on our languages?

\boxtext{The wide variety of languages in Europe is one of its richest and most important cultural assets}
 
Europe’s approximately 80 languages are one of its richest and most important cultural assets, and a vital part of its unique social model.\footnote{European Commission, \emph{Multilingualism: an asset for Europe and a shared commitment}, Brussels, 2008 (\url{http://ec.europa.eu/education/languages/pdf/com/2008_0566_en.pdf}).} While languages such as English and Spanish are likely to survive in the emerging digital marketplace, many European languages could become irrelevant in a networked society. This would weaken Europe’s global standing, and run counter to the strategic goal of ensuring equal participation for every European citizen regardless of language. According to a UNESCO report on multilingualism, languages are an essential medium for the enjoyment of fundamental rights, such as political expression, education and participation in society.\footnote{UNESCO Director-General, \emph{Intersectoral mid-term strategy on languages and multilingualism}, Paris, 2007 (\url{http://unesdoc.unesco.org/images/0015/001503/150335e.pdf}).}
