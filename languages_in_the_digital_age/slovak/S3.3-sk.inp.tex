Na konci roka 2010 bola veľkosť slovenskej internetovej populácie približne 2 394 000, čo je viac ako 44~\% všetkých Slovákov. V~prípade mladšej generácie je toto percento omnoho vyššie, keďže mladí ľudia trávia počas dňa mnoho času na internete. Do konca roka 2010 prekročil počet slovenských domén hranicu 231-tisíc\footnote{\url{https://www.sk-nic.sk/documents/pdf/2010-12-31_SK-NIC_PS.pdf}}. Podiel .sk domén na svetovom internete bol na konci roka 2010 približne \emph{1~\textperthousand}\footnote{Počet všetkých domén podľa \url{http://www.verisigninc.com} dosahoval na konci roka 2010 približne 200 miliónov.}. Na internete sa slovenčina s~diakritikou objavila v~polovici 90. rokov 20. storočia. Sféra internetovej komunikácie a~texty, ktoré sa na internete nachádzajú, sú zaujímavé z~hľadiska výskumu prirodzeného jazyka, ale aj z~hľadiska možnosti zberu štatistických materiálov. Internet je aj miestom využívania rôznych aplikačných oblastí, ktoré ako zdroj využívajú jazykové dáta. 

Rovnako ako pri mnohých iných európskych jazykoch, aj pre začiatky používania slovenčiny na internete\footnote{V čomkoľvek, čo sa týka počítačov.} bolo typické vynechávanie diakritiky. Kvôli zmätkom s~kódovaním znakov na konci 80. a~začiatkom 90. rokov 20. storočia a~nedostatočnej softvérovej podpore rozličných znakových kódovaní začal „správny“ pravopis na internete prevládať až koncom 90. rokov. V~súčasnosti, pri takmer univerzálnom používaní kódovania Unicode a~UTF-8, neexistujú žiadne nevyriešené problémy a~diakritika sa používa univerzálne (v~neformálnych kontextoch, napríklad v~e-mailoch a~na diskusných fórach a~hlavne v~SMS správach sa bežne používa slovenčina bez diakritiky).

Osobitnou kategóriou sú bilingválne slovníky, ktoré sú voľne prístupné slovenským používateľom internetu na troch veľkých slovenských portáloch (\emph{azet.sk}, \emph{centrum.sk}, \emph{zoznam.sk}).

Spoločnosť Google vyvíja voľne dostupný automatický prekladač textov z~rôznych jazykov do slovenčiny a~naopak. Miera správnosti je však v~prípade väčšiny jazykov nízka. Zaujímavý je vzájomný preklad medzi blízkopríbuznými jazykmi slovenčina – čeština/čeština – slovenčina, kde je úspešnosť a~správnosť prekladu dobrá. Samozrejme, aj tento preklad je miestami nesprávny, ale je omnoho úspešnejší ako preklad medzi slovenčinou a~angličtinou, nemčinou, francúzštinou a~inými rozšírenými jazykmi.

O~využívaní internetových zdrojov používateľmi slovenského internetu svedčí aj vyše 60-tisíc slovenských registrovaných používateľov internetovej encyklopédie Wikipédia v~slovenskom jazyku. Slovenská Wikipédia \emph{obsahuje vyše 285-tisíc článkov}.
