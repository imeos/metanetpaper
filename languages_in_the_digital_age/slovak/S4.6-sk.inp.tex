Nasledujúca tabuľka poskytuje prehľad súčasnej situácie jazykových technológií pre slovenčinu. Klasifikácia existujúcich nástrojov je založená na odborných odhadoch popredných expertov s~použitím nasledujúcich kritérií (v~rozmedzí od 0 do 6).

\begin{enumerate}
\item Kvantita: Existuje pre daný jazyk nejaký nástroj/zdroj? Čím viac nástrojov/zdrojov existuje, tým je hodnotenie vyššie.
\begin{itemize}
\item 0: neexistujú žiadne nástroje/zdroje
\item 6: mnoho nástrojov/zdrojov, veľká rôznorodosť
\end{itemize}
\item Dostupnosť: Sú nástroje/zdroje dostupné? - t.~j. sú Open Source voľne použiteľné na akejkoľvek platforme alebo sú dostupné len za vysokú cenu, resp. za obmedzených podmienok?
\begin{itemize}
\item 0: takmer všetky nástroje/zdroje sú dostupné len za vysokú cenu
\item 6: veľké množstvo nástrojov/zdro\-jov je voľne dostupných vďaka licenciám OpenSource, ako napr. Creative Commons, ktoré umožňujú opätovné použitie a~prispôsobenie potrebám používateľa\footnote{V prípade, že tam sú napr. dva rôzne zdroje, jeden z~nich úplne otvorený a~druhý úplne uzavretý, do tabuľky zadáme priemer (t.~j. 3)}
\end{itemize}
\item Kvalita: Do akej miery sa jednotlivé kritériá správania nástrojov a~ukazovatele kvality zdrojov približujú ku kvalite najlepších dostupných nástrojov, aplikácií či zdrojov? Sú tieto nástroje/zdroje aktuálne a~udržiavané?
\begin{itemize}
\item 0: amatérsky nástroj/zdroj
\item 6: kvalitný nástroj/zdroj, anotácie v~zdroji sa kvalitou rovnajú ručným anotáciám
\end{itemize}
\item Pokrytie: Do akej miery spĺňajú najlepšie dostupné nástroje príslušné kritériá pokrytia (štýlov, žánrov, druhov textov, jazykových javov, typov vstupov/výstupov, počtu jazykov podporovaných MT systémami atď.)? Do akej miery sú zdroje reprezentantmi daných jazykov, resp. subjazykov?
\begin{itemize}
\item 0: zdroj/nástroj určený na špecifické účely, osobité prípady, malé pokrytie, používa sa len vo veľmi špecifických, neobvyklých prípadoch
\item 6: zdroj so širokým pokrytím, robustný nástroj, široko uplatniteľný, veľké množstvo podporovaných jazykov
\end{itemize}
\item Vyspelosť: Môže sa nástroj/zdroj považovať za vyspelý, stabilný a~pripravený na trh? Dajú sa najlepšie dostupné nástroje/zdroje priamo použiť alebo sa musia upraviť? Je výkon takýchto technológií dostatočný a~použiteľný? Alebo sú to len prototypy, ktoré sú nevhodné pre produktívne systémy? Ukazovateľom vyspelosti môže byť prijatie nástrojov/zdrojov do komunity a~ich úspešné používanie v~systémoch jazykových technológií.
\begin{itemize}
\item 0: predbežný prototyp, amatérsky systém, overenie koncepcie
\item 6: okamžite integrovateľný/po\-u\-ži\-teľný prvok systému
\end{itemize}
\item Udržateľnosť: Ako sa dá nástroj/zdroj udržiavať, resp. integrovať do súčasných informačných systémov? Spĺňa nástroj/zdroj určitú úroveň udržateľnosti vzhľadom na dokumentáciu/manuály, vysvetlenie prípadov použitia, front‑endy, GUI atď.? Využíva daný nástroj štandardné/najspoľahlivejšie programovacie jazyky (napr. Java EE)? Existujú technické/výskumné normy, resp. kvázinormy? Ak áno, vyhovuje nástroj/zdroj týmto normám (dátové formáty a~pod.)?
\begin{itemize}
\item 0: súkromné zdroje, dátové formáty a~API ad hoc
\item 6: zdroje úplne vyhovujúce normám, kompletná dokumentácia
\end{itemize}
\item Adaptabilnosť: Do akej miery sa dajú najlepšie nástroje/zdroje adaptovať, resp. rozšíriť na nové úlohy/domény/žánre/typy textov/prí\-pa\-dy použitia atď.?
\begin{itemize}
\item 0: je prakticky nemožné adaptovať nástroj/zdroj na nové úlohy, dokonca ani s~použitím veľkého množstva zdrojov či človekohodín
\item 6: vysoká úroveň adaptabilnosti; nástroje/zdroje sa dajú veľmi jednoducho a~efektívne adaptovať
\end{itemize}
\end{enumerate}
