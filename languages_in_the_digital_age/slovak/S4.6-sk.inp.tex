\noindent V tabuľke \ref{tab:lrlttable-sk} ponúkame sumarizáciu súčasného stavu jazykových technológií pre slovenčinu. Kritériá existujúcich nástrojov a~zdrojov v~rozmedzí 0 (veľmi nízky) až 6 (veľmi vysoký) navrhli poprední odborníci.

% Begin table
\begin{table*}[htb]
\centering

\begin{tabular}{>{\columncolor{orange1}}p{.33\linewidth}@{\hspace*{6mm}}c@{\hspace*{6mm}}c@{\hspace*{6mm}}c@{\hspace*{6mm}}c@{\hspace*{6mm}}c@{\hspace*{6mm}}c@{\hspace*{6mm}}c}
\rowcolor{orange1}
 \cellcolor{white}&
 \begin{sideways}\makecell[l]{Kvantita}\end{sideways} &
 \begin{sideways}\makecell[l]{\makecell[l]{Dostupnosť} }\end{sideways} &
 \begin{sideways}\makecell[l]{Kvalita}\end{sideways} &
 \begin{sideways}\makecell[l]{Pokrytie}\end{sideways} &
 \begin{sideways}\makecell[l]{Zrelosť}\end{sideways} &
 \begin{sideways}\makecell[l]{Udržateľnosť~~~}\end{sideways} &
 \begin{sideways}\makecell[l]{Adaptabilita}\end{sideways} \\ \addlinespace

\multicolumn{8}{>{\columncolor{orange2}}l}{\textcolor{black}{Jazyková technológia: Nástroje, technológie a aplikácie}} \\ \addlinespace

Rozpoznávanie reči	&3	&1	&2	&2	&3	&3	&2 \\ \addlinespace
Syntéza reči 		&3	&3	&3	&3	&3	&3	&3 \\ \addlinespace
Gramatická analýza 	&2	&2	&3	&2	&2	&3	&3 \\ \addlinespace
Sémantická analýza 	&1	&2	&1	&1	&1	&3	&3 \\ \addlinespace
Generovanie textu 		&1	&1	&1	&1	&0	&1	&1 \\ \addlinespace
Strojový preklad 	&2	&2	&2	&2	&2	&1	&2 \\ \addlinespace

\multicolumn{8}{>{\columncolor{orange2}}l}{\textcolor{black}{Jazykové zdroje: Zdroje, dáta a znalostné databázy}} \\ \addlinespace

Textové korpusy 		&2	&4	&4	&5	&4	&4	&4  \\ \addlinespace
Hovorené korpusy 		&3	&4	&2	&2	&3	&3	&3 \\ \addlinespace
Paralelné korpusy 	&2	&3	&2	&2	&2	&2	&3 \\ \addlinespace
Lexikálne zdroje 	&3	&2	&3	&4	&3	&4	&3 \\ \addlinespace
Gramatiky 		&2	&3	&3	&2	&1	&2	&1 \\
\end{tabular}
\caption{Stav podpory jazykových technológií v slovenčine}
\label{tab:lrlttable-sk}
\end{table*}



\begin{enumerate}
\item Kvantita: Existuje pre daný jazyk nejaký nástroj/zdroj? Čím viac nástrojov/zdrojov existuje, tým je hodnotenie vyššie.
\begin{itemize}
\item 0: neexistujú žiadne nástroje/zdroje
\item 6: mnoho nástrojov/zdrojov, veľká rôznorodosť
\end{itemize}
\item Dostupnosť: Sú nástroje/zdroje dostupné? - t.~j. sú Open Source voľne použiteľné na akejkoľvek platforme alebo sú dostupné len za vysokú cenu, resp. za obmedzených podmienok?
\begin{itemize}
\item 0: takmer všetky nástroje/zdroje sú dostupné len za vysokú cenu
\item 6: veľké množstvo nástrojov/zdro\-jov je voľne dostupných vďaka licenciám OpenSource, ako napr. Creative Commons, ktoré umožňujú opätovné použitie a~prispôsobenie potrebám používateľa\footnote{V prípade, že tam sú napr. dva rôzne zdroje, jeden z~nich úplne otvorený a~druhý úplne uzavretý, do tabuľky zadáme priemer (t.~j. 3).}
\end{itemize}
\item Kvalita: Do akej miery sa jednotlivé kritériá správania nástrojov a~ukazovatele kvality zdrojov približujú ku kvalite najlepších dostupných nástrojov, aplikácií či zdrojov? Sú tieto nástroje/zdroje aktuálne a~udržiavané?
\begin{itemize}
\item 0: amatérsky nástroj/zdroj
\item 6: kvalitný nástroj/zdroj, anotácie v~zdroji sa kvalitou rovnajú ručným anotáciám
\end{itemize}
\item Pokrytie: Do akej miery spĺňajú najlepšie dostupné nástroje príslušné kritériá pokrytia (štýlov, žánrov, druhov textov, jazykových javov, typov vstupov/výstupov, počtu jazykov podporovaných MT systémami atď.)? Do akej miery sú zdroje reprezentantmi daných jazykov, resp. subjazykov?
\begin{itemize}
\item 0: zdroj/nástroj určený na špecifické účely, osobité prípady, malé pokrytie, používa sa len vo veľmi špecifických, neobvyklých prípadoch
\item 6: zdroj so širokým pokrytím, robustný nástroj, široko uplatniteľný, veľké množstvo podporovaných jazykov
\end{itemize}
\item Vyspelosť: Môže sa nástroj/zdroj považovať za vyspelý, stabilný a~pripravený na trh? Dajú sa najlepšie dostupné nástroje/zdroje priamo použiť alebo sa musia upraviť? Je výkon takýchto technológií dostatočný a~použiteľný? Alebo sú to len prototypy, ktoré sú nevhodné pre produktívne systémy? Ukazovateľom vyspelosti môže byť prijatie nástrojov/zdrojov do komunity a~ich úspešné používanie v~systémoch jazykových technológií.
\begin{itemize}
\item 0: predbežný prototyp, amatérsky systém, overenie koncepcie
\item 6: okamžite integrovateľný/po\-u\-ži\-teľný prvok systému
\end{itemize}
\item Udržateľnosť: Ako sa dá nástroj/zdroj udržiavať, resp. integrovať do súčasných informačných systémov? Spĺňa nástroj/zdroj určitú úroveň udržateľnosti vzhľadom na dokumentáciu/manuály, vysvetlenie prípadov použitia, front‑endy, GUI atď.? Využíva daný nástroj štandardné/najspoľahlivejšie programovacie jazyky (napr. Java EE)? Existujú technické/výskumné normy, resp. kvázinormy? Ak áno, vyhovuje nástroj/zdroj týmto normám (dátové formáty a~pod.)?
\begin{itemize}
\item 0: súkromné zdroje, dátové formáty a~API ad hoc
\item 6: zdroje úplne vyhovujúce normám, kompletná dokumentácia
\end{itemize}
\item Adaptabilnosť: Do akej miery sa dajú najlepšie nástroje/zdroje adaptovať, resp. rozšíriť na nové úlohy/domény/žánre/typy textov/prí\-pa\-dy použitia atď.?
\begin{itemize}
\item 0: je prakticky nemožné adaptovať nástroj/zdroj na nové úlohy, dokonca ani s~použitím veľkého množstva zdrojov či človekohodín
\item 6: vysoká úroveň adaptabilnosti; nástroje/zdroje sa dajú veľmi jednoducho a~efektívne adaptovať
\end{itemize}
\end{enumerate}

\emph{Tabuľka sa dá zhrnúť do niekoľkých kľúčových bodov:}

\begin{itemize}
\item Na Slovensku existuje niekoľko špecializovaných kvalitných korpusov, ale dosiaľ tu nie je dostupný žiaden veľký, syntakticky anotovaný korpus.
\item Referenčným korpusom pre slovenčinu je Slovenský národný korpus. Kvôli licenčným obmedzeniam je však prístupné len jeho vyhľadávacie rozhranie.
\item Na druhej strane, korpus hovorených textov nepodlieha zákonu o~ochrane autorských práv a~je verejne dostupný. Jeho rozsah je však oproti rozsahu korpusu písaných textov nepatrný.
\item Mnohé zdroje sú neštandardizované, t.~j. aj keď existujú, nie sú udržiavané. Na štandardizáciu dát a~výmenu formátov je nevyhnutné spoločné úsilie a~iniciatíva.
\item Spracovať sémantiku je ťažšie ako spracovať syntax; spracovať textovú sémantiku je ťažšie než spracovať lexikálnu a~vetnú sémantiku.
\item Slovenčina má ontologický zdroj (zmapovaný na anglické ontologické zdroje), no jeho pokrytie je obmedzené.
\item V~zmysle reprezentácie vedomostí o~svete existujú štandardy pre sémantiku (RDF, OWL, atď.), ktoré sa však ťažko aplikujú na úlohy NLP.
\item Spracovanie písaného textu je rozvinutejšie ako spracovanie hovoreného textu (najmä rozpoznávania reči).
\item V~slovenčine chýbajú mnohé zdroje, ktoré sa v~iných jazykoch považujú za štandard; jazykový výskum NLP je na Slovensku veľmi slabo financovaný.
\item Niektoré výskumné a~vývojové aktivity pre slovenčinu sa realizujú v~Českej republike – ~na českých univerzitách alebo v~súkromnom sektore.
\item Výskum rozpoznávania reči pre slovenčinu prebieha na niekoľkých univerzitách a~výskumných pracoviskách, no množstvo voľne dostupných nástrojov a~dát je obmedzené.
\item Naopak, syntézu reči spracúvajú univerzity a~iné vedecké pracoviská v~oveľa menšom rozsahu.
\item V~oblasti syntézy reči sú dostupné OpenSource balíky a~niekoľko jednoduchých syntetizátorov reči, no syntéza reči s~prirodzenejšími hlasmi nie je dostupná.
\item Slovenské dialógové systémy sú veľmi málo rozšírené v~dôsledku nízkej dostupnosti kvalitných modulov rozpoznávania reči pre slovenčinu.
\end{itemize}

