\noindent Primárny, všeobecný korpus {\bf\emph{prim}} obsahuje texty
v slovenskom jazyku, ktoré vznikli po r. 1955.  Zastúpené sú tri
hlavné štýly: publicistický, umelecký, odborný
(populárno-náučný), ako aj rôzne žánre a vecné oblasti. Texty
sú z celého Slovenska i od Slovákov žijúcich v zahraničí,
originálne slovenské aj preložené z iných jazykov. Na špecifické
výskumy sa z hlavného korpusu \emph{prim-*-all} tvoria samostatné
podkorpusy:

\begin{itemize}
\item \emph{sane} – neobsahuje lingvistické texty, texty bez diakritiky, texty od zahraničných Slovákov a pod.;
\item \emph{vyv} – publicistické, umelecké a odborné texty sú zastúpené tretinovým podielom;
\item \emph{inf} – iba publicistické texty;
\item \emph{prf} – iba odborné texty;
\item \emph{img} – iba umelecké texty;
\item \emph{skimg} – iba pôvodné slovenské umelecké texty.
\end{itemize}

Použitie textov v Slovenskom národnom korpuse sa riadi ustanoveniami
slovenského autorského zákona. 

Textom a textovým jednotkám v korpuse sa štandardne priraďuje
vonkajšia: bibliografická a štýlovo-žánrová
anotácia\footnote{\url{http://korpus.sk/bibstyle.html}} a vnútorná,
morfologická alebo morfosyntaktická
anotácia\footnote{\url{http://korpus.sk/morpho.html}}. Všetky slová
sú lematizované. 
