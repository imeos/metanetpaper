At the end of 2010, the size of the Slovak Internet population reached approximately 2 394 000 which is more than 44\% of all Slovak inhabitants. In the case of the younger generation, this percentage has been much higher as young people spend a lot of time on the Internet. By the end of 2010 the number of Slovak domains exceeded the level of 231 thousand\footnote{\url{https://www.sk-nic.sk/documents/pdf/2010-12-31_SK-NIC_PS.pdf}}. The amount of \emph{.sk} domains on the worldwide web was about 1\textperthousand\footnote{Number of all domains according to \url{http://www.verisigninc.com} was reaching approximately 200 million at the end of year 2010.} by the end of 2010. The style of Internet communication and the texts to be found on the Internet are interesting for natural language research but also for text collecting purposes. The  Internet is also a place for the usage of various applications which use language data as a source.  

Shared with many other European languages, a specific feature of early Slovak language presence on the internet\footnote{And generally, in anything computer related} was the habit of using the language without diacritics. Owing to the “character encoding mess” in the late 80's and 90's and the lack of software support for different character encodings, the “proper” language on the Internet started to dominate only in the late 1990's. Nowadays, with the almost universal Unicode and UTF-8 encoding, there are no more outstanding problems and the diacritics are used universally (however, in informal contexts such as in e-mails and discussion forums, and especially in SMS, Slovak without diacritics is common). 

A special category consists of bilingual dictionaries, which are freely accessible to Slovak users through three major Slovak portals (\emph{azet.sk}, \emph{centrum.sk}, \emph{zoznam.sk}). 

Google is developing a freely accessible automatic text translator from various languages into Slovak and vice versa. The degree of correctness is, however, low in the case of the majority of languages. There is an interesting result regarding the mutual translation between the closely related languages Slovak$\leftrightarrow$Czech, where the percentage of correctness of the translation is good. Of course, even these translations are sometimes incorrect, however, they are much more successful than translations between Slovak and English, German, French, and other major languages.  

The use of the Internet by Slovak Internet users is reflected by more than 60~000 registered Slovak users of the Internet encyclopedia Wikipedia in the Slovak language. Slovak Wikipedia includes more than 285~000 articles. 
