\noindent Letná škola slovenského jazyka a~kultúry Studia Academica Slovaca (SAS) je určená zahraničným slovakistom a~slavistom, študentom na zahraničných univerzitách, kultúrnym pracovníkom, manažérom, lektorom, prekladateľom a~všetkým záujemcom o~štúdium slovenského jazyka a~kultúry. Cieľom kurzu je získanie a prehĺbenie komunikačnej kompetencie v~slovenskom jazyku na rôznych stupňoch a~rozšírenie poznatkov zo slovenskej lingvistiky, literatúry, histórie a~kultúry.

\boxtext{Cieľom je získanie a prehĺbenie komunikačnej kompetencie v slovenskom jazyku}

Letná škola SAS je najstaršou letnou univerzitou na Slovensku - existuje od roku 1965 a~od roku 1966 pod názvom Studia Academica Slovaca. SAS si od svojho vzniku kontinuálne zachováva profil slovakistických akademických štúdií. Letnú školu SAS každoročne absolvuje približne 150 frekventantov z~viac ako 30 krajín sveta. Na príprave a~realizácii vzdelávacieho programu sa podieľajú vysokoškolskí pedagógovia a~lektori odborne vyškolení v~oblasti slovenčiny ako cudzieho jazyka, z~ktorých mnohí majú skúsenosti aj z~pôsobenia na lektorátoch zahraničných univerzít.
