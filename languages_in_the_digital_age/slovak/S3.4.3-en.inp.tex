The Summer School of Slovak Language and Culture Studia Academica Slovaca is aimed at Slovakists and Slavists abroad, cultural workers, managers, lecturers, language teachers, translators and all those interested in studying Slovak language and culture. The aim of the course is to enable students to acquire and improve their Slovak language competence on various levels, as well as to extend their knowledge in Slovak linguistics, literature, history and culture.
Established in 1965, Summer School SAS is the oldest summer university in Slovakia and has been under the name Studia Academica Slovaca since 1966. Since its establishment, SAS has continually maintained its profile of Slovakist academic studies. The Summer School SAS is usually attended by approximately 150 participants from more than 25 countries all over the world. Those creating and holding the seminars are professional teachers and lecturers, experts in teaching Slovak as a foreign language, often experienced in teaching in Slovakia as well as abroad.
