\noindent Kníhtlač značne prispela k~výmene informácií v~Európe, ale napomohla tiež zániku mnohých európskych jazykov. V~regionálnych a~menšinových jazykoch sa dokumenty rozmnožovali zriedkakedy. Výsledkom bolo, že mnohé jazyky, ako napríklad rómsky alebo rusínsky, sa zredukovali viacmenej len na ústne podanie, čo obmedzovalo ich kontinuálne osvojenie a~rozšírenie. Bude mať internet podobný vplyv aj na naše jazyky?

\boxtext{Rôznorodosť jazykov v Európe je súčasťou najvzácnejšieho a kultúrneho bohatstva Európy}

Približne 80 jazykov je časťou najvzácnejšieho a~najdôležitejšieho kultúrneho bohatstva Európy. Množstvo európskych jazykov je takisto nevyhnutnou súčasťou jej sociálneho úspechu.\footnote{European Commission, \emph{Multilingualism: an asset for Europe and a~shared commitment}, Brussels, 2008 \newline (\url{http://ec.europa.eu/education/languages/pdf/com/2008_0566_en.pdf}).} Zatiaľ čo sa budú populárne jazyky ako angličtina a~španielčina v~rozvíjajúcej sa digitálnej spoločnosti a~na~trhu určite udržiavať, mnohé európske jazyky sa vynechajú z~digitálnych komunikácií a~pre internetovú spoločnosť sa stanú irelevantné. Takýto vývoj by oslabil európsku stabilitu, pretože by bol v~rozpore  s~cieľom zabezpečiť rovnaké postavenie každého európskeho občana bez ohľadu na jazykovú príslušnosť. V~správe Unesca o~multilingvizme sa uvádza, že jazyky sú médiom uplatňovania základných ľudských práv, ako je právo na vyjadrenie politického názoru, vzdelanie a~účasť na~spoločenskom živote.\footnote{UNESCO Director-General, \emph{Intersectoral mid-term strategy on languages and multilingualism}, Paris, 2007 \newline (\url{http://unesdoc.unesco.org/images/0015/001503/150335e.pdf}).}
