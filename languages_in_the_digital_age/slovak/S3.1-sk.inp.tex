
Slovenský jazyk patrí – v~rámci indoeurópskej rodiny jazykov –
spolu s~poľštinou, češtinou a lužickou srbčinou k~západnej vetve
slovanských jazykov. Jazykové, historické a~archeologické fakty
ukazujú, že slovenčina sa vyvíjala priamo z~praslovančiny (nie cez
štádium pračeskoslovenčiny). Praslovanský základ slovenčiny sa
sformoval v~priestore medzi Karpatmi, Dunajom a~dolnou Moravou, a~to
v~dotyku so západoslovanským areálom na západ od tohto priestoru
a~s~východoslovanským areálom na sever a~severovýchod. Do tohto
priestoru prišli Slovania, predchodcovia Slovákov, v~6. storočí
z~juhovýchodu. Za základ slovenčiny možno pokladať rekonštruovaný
jazyk veľkomoravského etnika členený na nárečia, ale reprezentovaný aj istou kultúrnou podobou. Najbúrlivejší vývin
slovenčina prekonala v~10. – 12. storočí, v~13. – 15. storočí
sa predovšetkým stabilizovala. V~16. – 18. storočí sa na území
Slovenska používala ako kultúrny jazyk čeština, ale aj niekoľko
typov kultúrnej slovenčiny: kultúrna západoslovenčina, kultúrna
stredoslovenčina a~kultúrna východoslovenčina. Od konca 18.
storočia sa začínajú pokusy o~formovanie spisovnej slovenčiny.
Anton Bernolák koncom 18. storočia založil svoju kodifikáciu na
západnej kultúrnej slovenčine, ale v~dôsledku zmenených
spoločenských a~hospodárskych podmienok nemal želaný úspech.
Ľudovít Štúr vychádzal zo stredoslovenského základu, ním kodifikovaná spisovná slovenčina sa ujala a~po istých úpravách (Martina
Hattalu, Michala Miloslava Hodžu) používa až dodnes.

Slovenský jazyk je štátnym jazykom Slovenskej republiky a od mája 2004 je slovenčina jedným z úradných jazykov EÚ. Po slovensky hovorí okolo štyri a pol milióna obyvateľov Slovenska, viac ako milión vysťahovalcov v~USA a~okolo 300-tisíc v~Českej republike. Menšie rečové skupiny sa nachádzajú aj v~Maďarsku, Rumunsku, Srbsku, Chorvátsku, Bulharsku, Poľsku, vo Francúzsku, v~Nemecku, Belgicku, Rakúsku, Nórsku, Dánsku, vo Fínsku, Švédsku, v~Taliansku, vo Švajčiarsku, v~Holandsku, na Cypre, v~Rusku a~na Ukrajine, v~Kirgizsku, Izraeli, Kanade, Juhoafrickej republike, Argentíne, Brazílii, Uruguaji, Austrálii, na Novom Zélande, vo Veľkej Británii a~v~niektorých ďalších krajinách. Slovenčina je známa ako „esperanto“ slovanských jazykov, vníma sa ako najzrozumiteľnejšia aj pre používateľov iných slovanských jazykov.

\boxtext{Slovenčina je známa ako „esperanto“ slovanských jazykov}

Slováci v~zahraničí predstavujú rôzne skupiny: sú to potomkovia pôvodných obyvateľov Slovenska, ktorí odchádzali do iných oblastí bývalého Rakúsko-Uhorska, potomkovia novších vysťahovalcov zo Slovenska v~zámorí (emigrantské vlny od konca 19. do polovice 20. storočia), politicko-ekonomickí emigranti po r. 1945, resp. 1948 a~po r. 1968 a~ich potomkovia, napokon prevažne mladí ľudia usídlení v~zahraničí po r. 1990. Odhaduje sa, že pri poslednej emigrantskej vlne v~r. 2007 – 2008 odišlo do zahraničia asi 270-tisíc Slovákov. Osobitnú skupinu predstavujú potomkovia Slovákov, ktorí ostali za hranicami Slovenska po politicko-geografických zmenách po r. 1918, resp. 1945. Na Slovensku zároveň žijú národnostné menšiny (Maďari, Rómovia, Česi, Rusíni, Ukrajinci, Nemci, Poliaci, Moravania, Chorváti, Bulhari, Židia), ktoré spolu tvoria 14,2~\% obyvateľov Slovenska. Používanie štátneho jazyka a~jazykov menšín na území Slovenska upravuje Zákon o~štátnom jazyku a~Zákon o~používaní jazykov národnostných menšín.

Slovenský jazyk má viacero foriem: spisovná slovenčina je predovšetkým jazykom písanej podoby a~úradnej, oficiálnej komunikácie, hovorová slovenčina je štandardnou podobou predovšetkým hovorenej komunikácie. V~každej forme sú osobitné podskupiny, ktoré tvoria stratifikačný systém slovenčiny: spisovná forma/celoslovenská štandardná forma/celoslovenská subštandardná forma/regionálne varianty/lokálne varianty, teritoriálna forma (nárečia), sociálne formy (slangy, žargóny, argoty, profesionálne jazyky). Za reguláciu jazyka a~jazykovú politiku bolo v dobe písania tohto dokumentu zodpovedné Ministerstvo kultúry SR (Zákon o~štátnom jazyku SR, Ústredná jazyková rada). Vo svojich rozhodnutiach by sa malo opierať o~poznatky a~názory vedeckej a~odbornej obce, na ktorej čele stojí Jazykovedný ústav Ľudovíta Štúra Slovenskej akadémie vied (ďalej JÚĽŠ SAV).
%\footnote{S každými voľbami sa situácia mení, pretože pre niektoré politické strany je Zákon o~štátnom jazyku vhodnou príležitosťou ľahko získať volebné hlasy.}
JÚĽŠ SAV je zriaďovateľom a~koordinátorom činnosti viacerých komisií s~celoslovenskou pôsobnosťou: pravopisná komisia, ortoepická komisia, onomastická komisia a~kodifikačná komisia. Jednotlivé komisie pripravujú a~odporúčajú kodifikáciu ortoepickej, pravopisnej, gramatickej a~lexikálnej normy. Pravopisné pravidlá prechádzajú osobitnou diskusiou aj so zapojením širšej verejnosti, ale vzhľadom na vzájomnú prepojenosť mnohých faktorov a~celospoločenský dosah každej zmeny sa nemenia príliš často. Posledné zmeny najmä v~oblasti pravidla o~rytmickom krátení a~v~písaní veľkých písmen sa udiali v~r. 1991. V lexikografických príručkách, ktoré vznikajú v JÚĽŠ SAV a z rôznych hľadísk opisujú slovnú zásobu slovenčiny (Krátky slovník slovenského jazyka, Slovník súčasného slovenského jazyka A – G, H – L, Synonymický slovník, Slovník cudzích slov – akademický \cite{kssj2003,sssj2006,sssj2011,sss2004,scs2005}), sa okrem pravopisnej normy zachytáva aj lexikálna a sčasti aj gramatická a ortoepická norma. Stav slovenčiny v rôznych jej podobách mapujú aj osobitné monografie a štúdie v časopisoch vydávaných JÚĽŠ SAV.

Územné usporiadanie Slovenska (územie s~rozlohou necelých 50-tisíc km² je situované najmä na dĺžku, ktorá dosahuje od východnej po západnú hranicu takmer 430 km) a~špecifiká jednotlivých nárečí ovplyvňujú aj podobu slovenčiny v~jednotlivých regiónoch a~lokalitách, s~čím sa musia vyrovnávať predovšetkým cudzinci učiaci sa slovenčinu a~pohybujúci sa na území SR.
