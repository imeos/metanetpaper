We are witnesses to a digital revolution that is dramatically impacting communication and society. Recent developments in digital information and communication technology are sometimes compared to Gutenberg’s invention of the printing press. What can this analogy tell us about the future of the European information society and our languages in particular?

\boxtext{We are witnessing a digital revolution that is comparable to Gutenberg's invention of the printing press}

After Gutenberg’s invention, real breakthroughs in communication and knowledge exchange were accomplished by efforts such as Luther’s translation of the Bible into vernacular language. In subsequent centuries, cultural techniques have been developed to better handle language processing and knowledge exchange:

\begin{itemize}
\item the orthographic and grammatical standardisation of major languages enabled the rapid dissemination of new scientific and intellectual ideas;
\item the development of official languages made it possible for citizens to communicate within certain (often political) boundaries;
\item the teaching and translation of languages enabled exchanges across languages;
\item the creation of editorial and bibliographic guidelines assured the quality and availability of printed material;
\item the creation of different media like newspapers, radio, television, books, and other formats satisfied different communication needs. 
\end{itemize}

In the past twenty years, information technology has helped to automate and facilitate many of the processes:

\begin{itemize}
\item desktop publishing software has replaced typewriting and typesetting;
\item presentation software, such as OpenOffice/LibreOffice Impress or Microsoft PowerPoint has replaced overhead projector transparencies;
\item e-mail send and receive documents faster than a fax machine;
\item SIP telephony and Skype offers cheap Internet phone calls and hosts virtual meetings;
\item audio and video encoding formats make it easy to exchange multimedia content;
\item search engines provide keyword-based access to web pages;
\item online services like Google Translate produce quick, approximate translations;
\item social media platforms such as Facebook, Twitter, and Google+ facilitate communication, collaboration, and information sharing.
\end{itemize}

Although such tools and applications are helpful, they are not yet capable of supporting a sustainable, multilingual European society for all where information and goods can flow freely.
