In the past, investment efforts in language preservation focused on language education and translation. According to one estimate, the European market for translation, interpretation, software localisation and website globalisation was \euro8.4 billion\footnote{Short scale, i.e. $8.4\cdot 10^9$.} in 2008 and is expected to grow by 10\% per annum.\footnote{European Commission Directorate-General for Translation, \emph{Size of the language industry in the EU}, Kingston Upon Thames, 2009 (\url{http://ec.europa.eu/dgs/translation/publications/studies}).} Yet this figure covers just a small proportion of current and future needs in communicating between languages. The most compelling solution for ensuring the breadth and depth of language usage in Europe tomorrow is to use appropriate technology, just as we use technology to solve our transport, energy and disability needs among others.

Digital language technology (targeting all forms of written text and spoken discourse) helps people collaborate, conduct business, share knowledge and participate in social and political debate regardless of language barriers and computer skills. It often operates invisibly inside complex software systems to help us:

\begin{itemize}
\item find information with an Internet search engine;
\item check spelling and grammar in a word processor;
\item view product recommendations in an online shop;
\item hear the verbal instructions of a car navigation system;
\item translate web pages via an online service.
\end{itemize}

Language technology consists of a number of core applications that enable processes within a larger application framework. The purpose of the META-NET language white papers is to focus on how ready these core technologies are for each European language. 

\boxtext{Europe needs robust and affordable language technology for all European languages}

To maintain our position in the frontline of global innovation, Europe will need language technology adapted to all European languages that is robust, affordable and tightly integrated within key software environments. Without language technology, we will not be able to achieve a really effective interactive, multimedia and multilingual user experience in the immediate future. 
