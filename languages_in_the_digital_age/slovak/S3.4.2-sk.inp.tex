\noindent Studia Academica Slovaca -- centrum pre slovenčinu ako cudzí jazyk (SAS) je špecializovaným pracoviskom Filozofickej fakulty Univerzity Komenského (FF UK) v~Bratislave. Ťažiskom pedagogickej a~vedeckovýskumnej činnosti je vzdelávanie zahraničných záujemcov o~slovenský jazyk a~kultúru, propagácia slovenskej vedy, kultúry a~umenia v~zahraničí, realizácia a~koordinácia výskumu slovenčiny ako cudzieho jazyka, riešenie medzinárodných a~domácich vedeckovýskumných projektov a~edičná činnosť zameraná na tvorbu a~vydávanie vedeckých slovakistických publikácií a~učebníc slovenčiny ako cudzieho jazyka. Okrem toho SAS ako odborné centrum pre slovenčinu ako cudzí jazyk už tradične participuje na odborno-metodickej príprave lektorov slovenčiny ako cudzieho jazyka pôsobiacich na zahraničných univerzitách. Výsledkom spolupráce s~lektorátmi a~zahraničnými slovakistami je databáza slovakistiky v~zahraničí.

\boxtext{Ťažiskom činnosti je vzdelávanie zahraničných záujemcov, propagácia slovenskej vedy, kultúry a umenia}

Osobitnou zložkou činnosti centra je každoročná organizácia a~realizácia letnej školy slovenského jazyka a~kultúry Studia Academica Slovaca, ktorú FF UK ponúka zahraničným záujemcom už od roku 1965. Na jej úspešnú históriu nadviazalo Metodické centrum SAS (1992), ktoré sa v~roku 2006 pretransformovalo na SAS – centrum pre slovenčinu ako cudzí jazyk. Za takmer polstoročie existencie SAS využilo služby tejto inštitúcie takmer šesťtisíc zahraničných záujemcov o~slovenský jazyk, kultúru a~slovenské reálie z~viac ako 50 štátov sveta. Na pôde Studia Academica Slovaca boli~položené základy vedeckého opisu a~didaktiky slovenčiny ako cudzieho jazyka a~vznikli tu prvé učebnice a~príručky slovenčiny pre cudzincov. Vzhľadom na svoju bohatú tradíciu a~skúsenosti v~súčasnosti pôsobí SAS ako koordinačné a~informačné centrum s~celoslovenskou a~exteritoriálnou pôsobnosťou.
V~roku 2006 centrum SAS získalo akreditáciu Ministerstva školstva Slovenskej republiky na poskytovanie vzdelávacích aktivít Slovenčina ako cudzí jazyk – jazykový kurz v~kontaktnej a~dištančnej forme pre všetky stupne jazykového vzdelávania, a~to pre začiatočníkov A1, A2, mierne a~stredne pokročilých B1, B2 a~pokročilých C1, C2, ktorých obsah je publikovaný v~tlačenej verzii \cite{pekarovicova2007} a~takisto na webovej stránke.\footnote{\url{http://www.fphil.uniba.sk/fileadmin/user_upload/editors/sas/slavic/Vzdelavaci_program.pdf}}

Na základe grantu Ministerstva školstva Slovenskej republiky sa v rámci projektu Vzdelávací program Slovenčina ako cudzí jazyk ponúka záujemcom o slovenský jazyk e-learningový kurz slovenčiny\footnote{\url{http://www.e-slovak.sk}} pre 1. stupeň A1 (úplný začiatočník) a 2. stupeň A2 (začiatočník). Cieľom projektu je tvorba obsahu a~foriem jazykového vzdelávania cudzincov pre jednotlivé stupne podľa Spoločného európskeho referenčného rámca pre jazyky, ako aj špecifikácia jednotných kritérií hodnotenia a~certifikácie jazykovej kompetencie. Hlavnou náplňou je príprava štandardných a~špecializovaných učebných materiálov pre študentov a~metodických príručiek pre učiteľov. Každoročne sa koná odborno-metodický seminár pre učiteľov základných a~stredných škôl v~zahraničí, ako aj pre lektorov pôsobiacich na zahraničných univerzitách s~cieľom informovať o~novinkách v~oblasti lingvistiky, literatúry, kultúry a~didaktiky slovenčiny ako cudzieho jazyka.

Za realizáciu projektu, ktorý rieši kolektív Studia Academica Slovaca, \emph{Vzdelávací program Slovenčina ako cudzí jazyk}, získala Filozofická fakulta UK ocenenie iniciatívy Európskej komisie v~oblasti jazykového vzdelávania \emph{Európska značka 2007}.
