Studia Academica Slovaca – centrum pre slovenčinu ako cudzí jazyk (SAS) je špecializovaným pracoviskom Filozofickej fakulty Univerzity Komenského (FF UK) v~Bratislave. Ťažiskom pedagogickej a~vedeckovýskumnej činnosti je vzdelávanie zahraničných záujemcov o~slovenský jazyk a~kultúru, propagácia slovenskej vedy, kultúry a~umenia v~zahraničí, realizácia a~koordinácia výskumu slovenčiny ako cudzieho jazyka, riešenie medzinárodných a~domácich vedeckovýskumných projektov a~edičná činnosť zameraná na tvorbu a~vydávanie vedeckých slovakistických publikácií a~učebníc slovenčiny ako cudzieho jazyka. Okrem toho SAS ako odborné centrum pre slovenčinu ako cudzí jazyk už tradične participuje na odborno-metodickej príprave lektorov slovenčiny ako cudzieho jazyka pôsobiacich na zahraničných univerzitách. Výsledkom spolupráce s~lektorátmi a~zahraničnými slovakistami je databáza slovakistiky v~zahraničí.

Osobitnou zložkou činnosti centra je každoročná organizácia a~realizácia letnej školy slovenského jazyka a~kultúry Studia Academica Slovaca, ktorú FF UK ponúka zahraničným záujemcom už od roku 1965. Na jej úspešnú históriu nadviazalo Metodické centrum SAS (1992), ktoré sa v~roku 2006 pretransformovalo na SAS – centrum pre slovenčinu ako cudzí jazyk. Za takmer polstoročie svojej existencie sa SAS obsahovo i~metodicky vyprofiloval a~stal sa uznávanou a~vyhľadávanou inštitúciou, ktorú absolvovalo takmer šesťtisíc zahraničných záujemcov o~slovenský jazyk, kultúru a~slovenské reálie z~viac ako 50 štátov sveta. Práve na pôde Studia Academica Slovaca boli~položené základy vedeckého opisu a~didaktiky slovenčiny ako cudzieho jazyka a~zásluhou jeho spolupracovníkov vznikli prvé učebnice a~príručky slovenčiny pre cudzincov. Vzhľadom na svoju bohatú tradíciu a~skúsenosti v~súčasnosti pôsobí SAS ako koordinačné a~informačné centrum s~celoslovenskou a~exteritoriálnou pôsobnosťou.

Za realizáciu projektu, ktorý rieši kolektív Studia Academica Slovaca, \emph{Vzdelávací program Slovenčina ako cudzí jazyk}, získala Filozofická fakulta UK ocenenie iniciatívy Európskej komisie v~oblasti jazykového vzdelávania \emph{Európska značka 2007}.
