Typické softvérové aplikácie na spracovanie jazyka sa skladajú z~niekoľkých zložiek, ktoré odrážajú rôzne aspekty jazyka a~úlohu, ktorú plnia. Obrázok zobrazuje veľmi zjednodušenú architektúru, ktorú možno nájsť v~systéme na spracovanie textu. Prvé tri moduly sa zaoberajú štruktúrou a~významom textového vstupu:

\begin{itemize}
\item Predbežné spracovanie: vyčistenie dát, odstránenie formátovania, detekcia vstupného jazyka, detekcia chýbajúcej diakritiky atď.
\item Gramatická analýza: hľadanie slovesa a~jeho objektov, modifikátorov, atď.; zistenie vetnej štruktúry.
\item Sémantická analýza: odstránenie viacznačnosti (Ktorý význam slova \emph{mier} je správny v~danom kontexte?), vyriešenie anafory a~odkazujúcich výrazov ako \emph{on}, \emph{to auto} atď.; prezentácia významu vety v~strojovo čitateľnej forme.
\end{itemize}

Účelové moduly potom vykonávajú mnoho rôznych operácií, ako je automatická sumarizácia vstupného textu, databázové hľadania a~mnoho ďalších. V~ďalšom texte ukážeme základné aplikačné oblasti a~zdôrazníme ich základné moduly. Opäť je potrebné zdôrazniť, že architektúry aplikácií sú veľmi zjednodušené a~idealizované pre vyjadrenie komplexnosti aplikácií jazykových technológií všeobecne zrozumiteľným spôsobom. Najdôležitejšie využívané nástroje a~zdroje sú v~texte podčiarknuté a~možno ich nájsť aj v~tabuľke na konci kapitoly.

Po predstavení základných aplikačných oblastí poskytneme stručný prehľad situácie v~oblasti výskumu a~vzdelávania jazykových technológií, pričom na záver uvedieme prehľad minulých a~prebiehajúcich výskumných programov. Na konci tejto časti budeme prezentovať odborný odhad situácie oblasti základných nástrojov a~zdrojov jazykových technológií z~viacerých hľadísk, napríklad dostupnosti, platnosti alebo kvality. Táto tabuľka poskytuje dobrý prehľad situácie jazykových technológií pre slovenčinu.
