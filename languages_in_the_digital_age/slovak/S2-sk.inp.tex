\noindent V~poslednej dekáde sme svedkami digitálnej revolúcie, ktorá má značný vplyv na komunikáciu a~spoločnosť. Nedávne pokroky v~digitálnych a~sieťových komunikačných technológiách sa niekedy prirovnávajú ku Gutenbergovmu vynájdeniu kníhtlače. Ako nám môže táto analógia konkrétne priblížiť budúcnosť európskej informačnej spoločnosti a~našich jazykov?

\boxtext{Sme svedkami digitálnej revolúcie, ktorú môžeme prirovnať
ku Gutenbergovmu vynálezu kníhtlače}

Po Gutenbergovom vynáleze nastal skutočný prelom v~komunikácii
a~výmene poznatkov vďaka takým snahám, ako bol napr. Lutherov
preklad Biblie do zrozumiteľného jazyka. V~ďalších storočiach
nastal rozvoj kultúrnych postupov, ktoré rozšírili
výmenu poznatkov a~zefektívnili spracovávanie jazyka. Zmeny,
ktoré nastali:

\begin{itemize}
\item ortografické a~gramatické ustálenie významnejších jazykov umožnilo rýchle rozšírenie nových vedeckých a~intelektuálnych ideí;
\item rozvoj oficiálnych jazykov pomohol obyvateľom komunikovať v~rámci určitých (často politických) hraníc;
\item vyučovanie a~preklad jazykov umožnil výmenu poznatkov medzi jazykmi;
\item vytvorenie žurnalistických a~bibliografických príručiek prinieslo zlepšenie kvality a~dostupnosti tlačeného materiálu;
\item vytvorenie rôznych médií, akými sú knihy, noviny, rozhlas, televízia a~i. uspokojilo rozmanité komunikačné potreby.
\end{itemize}

Za posledných dvadsať rokov pomohli informačné technológie automatizovať a~uľahčiť celý rad procesov:

\begin{itemize}
\item DTP softvér nahradil strojopis a~sadzbu;
\item prezentačný softvér, ako napríklad OpenOffice/LibreOffice Impress alebo Microsoft PowerPoint nahradili spätný projektor;
\item zasielanie a~prijímanie dokumentov e-mailom je rýchlejšie ako prostredníctvom faxu;
\item SIP telefónia alebo Skype umožňujú internetové volania a~virtuálne stretnutia;
\item efektívne kódovanie zvukových a~obrazových súborov uľahčuje výmenu multimediálneho obsahu;
\item nástroje na vyhľadávanie umožňujú na báze kľúčových slov efektívny prístup na webové stránky;
\item on–line služby, ako napríklad Google Translate, ponúkajú rýchle, aj keď približné preklady;
\item platformy sociálnych médií (Pokec, Facebook, Twitter, Google a~i.) uľahčujú spoluprácu a~sprístupnenie informácií.
\end{itemize}

Spomenuté nástroje a~aplikácie ľuďom pomáhajú, no v~súčasnosti nedokážu dostatočne pokryť potreby multilingválnej modernej európskej informačnej spoločnosti, v~ktorej je neustály tok informácií a~tovaru.
