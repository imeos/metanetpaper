Language Technology is a highly interdisciplinary field involving the expertise of linguists, computer scientists, mathematicians, philosophers, psycholinguists, and neuroscientists among others. As such, it has not yet acquired a fixed place in the Slovak faculty system.

Since 2007 the researchers from the Institute of Informatics of the Slovak Academy of Sciences (Michal Laclavík and Martin Šeleng) have been teaching the Information retrieval course\footnote{\url{http://vi.ikt.ui.sav.sk/}} at the Faculty of Information Technologies of the Slovak Technical University. This course focuses on such themes as information retrieval, information extraction, graph algorithms for their support as well as processing large amounts of data. The students solve various practical projects in this domain, while many of them use Slovak text sources, and some of them directly solve the NLP problems of Slovak language processing. As an example, let us mention several projects aimed at the creation of a statistical, dictionary-oriented or algorithmic stemmer based on the  “snowball” or “Egothor” projects, and at the determination of the efficiency and statistics for the simple stemmers which function on the principle of omitting the vowels, diacritic marks or, eventually, word endings etc. At the same time, there are also statistical translation projects or the automatic dictionary creation between the Slovak or other languages (English, Czech). Finally, let us mention the projects utilising dictionaries or frequency language dictionaries for applications such as T9, named entities extraction using computer learning methods and libraries such as OpenNLP, the creation of POS tagging algorithms as well as the extraction of events from e-mails or from Slovak webpages and the like.

There is no regular Computational Linguistics study programme otherwise.
