In Slovakia, the language technologies and their development are still considered mostly a scientific
area and are included predominantly in applied research, either linguistic
(particularly lexicography) or computer science. The connection with the
business sector has been rather weak and sporadic. However, recently the
language technologies have been making strong and resolute entrance to many
software applications.

The first two big government funded research projects with a focus on language
technologies and resources in Slovakia were {\em National Corpus of the Slovak Language and
Electronisation of Linguistic Research in the years 2002 -- 2006} and {\em Integrated Computational
Processing of the Slovak Language for Linguistic Research Purposes}, both
carried out at Ľ. Štúr Institute of Linguistics, Slovak Academy of Sciences.

{\em National Corpus of the Slovak Language and Electronisation of Linguistic
Research in years 2002 -- 2006}, approved by a government resolution n.
137/2002, was aimed at building a representative corpus of Slovak language, as a necessary foundation
and data source for any linguistic and natural language processing research.
The corpus data form the base in compiling the comprehensive Dictionary of
Contemporary Slovak.


In this project, the Slovak National Corpus Department was created and
subsequently became the leading institution in NLP research in Slovakia. The
project continued in its \nth{2} period as {\em Construction of Slovak National Corpus and Electronisation of Linguistic Research in Slovakia} (in the years 2007--2011) as agreed by the Ministry of Education of the Slovak Republic, Ministry of Culture of the Slovak Republic and the Slovak Academy of Sciences.


The project {\em Integrated Computational Processing of the Slovak Language for Linguistic
Research Purposes}, n. 2003SP200280307 was carried out in years 2003 -- 2006 in the frame of the State research and development programme {\em Current Issues in Society Development}. The project supplemented the Slovak language resources with necessary tools and additional data (morphological a stylistic annotation, electronic linguistic resources, terminology database etc.). The results of the project are further used in subsequent projects and also in commercial environment.


Another major project concerning the Slovak language processing was the project
{\em Automatic Transcription of Dictate for the Ministry of Justice of
the Slovak Republic},  coordinated by the Department of Speech Analysis
and Synthesis of the Institute of Informatics of the Slovak Academy of
Sciences, with participation of the Department of Electronics and
Multimedia Communications of the Technical University of Košice,
carried out in the years 2009--2011. The goal of the project was to
create a complete system for transcribing spoken Slovak language,
specialised for judicial domain. The project has been funded by the
Ministry of Justice of the Slovak Republic, and is currently being
deployed commercially in the courts of law throughout the Slovak
Republic.

These three projects were so far the only major initiatives concerning natural
language processing of the Slovak language. They paved the way for further
research and commercial projects, but the need for additional research and its funding is clearly necessary.



