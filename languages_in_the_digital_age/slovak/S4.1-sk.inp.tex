Jazykové technológie sú informačné technológie, ktoré sa zameriavajú na prácu s~ľudským jazykom, preto sa tieto technológie často zaraďujú pod pojem ľudské jazykové technológie. Ľudský jazyk existuje v~hovorenej a~písomnej forme. Kým reč je najstarší a~najprirodzenejší spôsob jazykovej komunikácie, komplexné informácie a~súhrn ľudského poznania sa zaznamenávajú a~prenášajú vo forme písomných textov. Rečové a~textové technológie spracúvajú alebo produkujú jazyk v~uvedených dvoch formách. Ale jazyk má tiež aspekty spoločné pre obe formy, ako sú slovníky, väčšina z~gramatiky a~zmysel viet. Veľkú časť jazykových technológií teda nemožno zaradiť výlučne pod rečovú alebo textovú technológiu. Znalostné technológie sú technológie, ktoré spájajú jazyk s~vedomosťami. Príslušná schéma znázorňuje krajinu jazykových technológií. V~našej komunikácii miešame jazyk s~inými druhmi komunikácie a~ďalšími informačnými médiami. Reč kombinujeme s~gestami a~výrazmi tváre. Texty je možné kombinovať s~obrázkami a~zvukmi. Filmy môžu obsahovať jazyk v~hovorenej aj písomnej forme. Rečové a~textové technológie sa teda prekrývajú a~pôsobia v~interakcii s~mnohými ďalšími technológiami, ktoré uľahčujú spracovanie multimodálnej komunikácie a~multimediálnych dokumentov.

%obrazok
