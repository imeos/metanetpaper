\boxtext{Jazykové technológie pomáhajú zjednotiť Európu}

\noindent Na základe súčasných poznatkov sa zdá, že dnešné „hybridné“ jazykové technológie, ktoré využívajú štatistické prístupy, by mohli prekonať jazykovú priepasť v~rámci Európy aj mimo nej. Séria bielych kníh poukazuje na značné rozdiely medzi jazykmi v~stave jazykových technológií. Aj slovenský jazyk potrebuje mimoriadne efektívne riešenia na ďalší výskum a~rozvoj jazykových technológií.

Dlhodobým cieľom META-NET-u je pos\-kyt\-núť kvalitné jazykové technológie všetkým jazykom, aby sa zároveň dosiahla politická a~ekonomická jednota napriek kultúrnym rozdielom. Technologické nástroje pomôžu prekonať existujúce bariéry. Všetky zainteresované strany (z~oblasti politiky, vedy, obchodu a~pod.) by sa mali snažiť o~zjednotenie.

Táto séria bielych kníh dopĺňa aj ďalšie aktivity META-NET-u (pozri prílohu). Aktuálne informácie, napríklad najnovšie vízie alebo strategický výskumný program META-NET-u sú dostupné na oficiálnej webovej stránke META-NET-u: \url{http://www.meta-net.eu}.

