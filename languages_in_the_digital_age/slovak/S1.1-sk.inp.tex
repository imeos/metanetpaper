\noindent Európa sa počas posledných 60 rokov stala výraznou politickou
a~ekonomickou štruktúrou, ale kultúrne a~jazykovo je stále
veľmi rôznorodá. To znamená, že od Portugalska po Poľsko
a~od Talianska po Island je bežná komunikácia medzi občanmi Európy podobne ako komunikácia v~oblasti podnikania a~politiky neustále konfrontovaná s~jazykovými bariérami. Európske inštitúcie minú ročne približne miliardu eur na preklady inojazyčných textov a~na tlmočenie. Nemuselo by to tak byť, ak by moderné jazykové technológie a~lingvistický výskum pomohli prekonať jazykové hranice. Ak vhodne využijeme inteligentné zariadenia a~aplikácie, budeme môcť vzájomne diskutovať alebo obchodovať a~rozličné jazyky nebudú pre nás prekážkou.

\boxtext{Jazykové technológie predstavujú mosty}

Jedným zo spôsobov, ako prekonať jazykové bariéry, je naučiť sa
niekoľko cudzích jazykov. Zvládnuť 23 oficiálnych jazykov
členských štátov EÚ a~približne 60 ďalších európskych jazykov
je málo pravdepodobné. Vďaka technologickej podpore však dokážeme
viesť politické aj ekonomické rokovania, ako aj napredovať vo
výskume.

Riešením mnohojazyčnosti je vybudovanie kľúčových technológií,
ktoré európskym činiteľom ponúknu obrovské výhody, a~to nielen
v~rámci spoločného európskeho trhu, ale aj pri obchodných vzťahoch
s~krajinami  tretieho sveta, najmä s~krajinami rozvíjajúcej sa
ekonomiky. Aby sme dosiahli tento cieľ a~zároveň zachovali kultúrnu
a~jazykovú rozmanitosť, musíme systematicky analyzovať špecifiká
všetkých európskych jazykov, ako aj stav súčasných jazykových
technológií. Navrhnuté riešenia budú mostom medzi jazykmi. 
