Slovak dialects are a means of communication of the autochthonous population of the respective dialect areas in everyday social and working relations with the nearest environment. Slovak dialects are inherited from one generation to the next in verbal form, although the process of levelling can be observed in this area. 

Vocabularies of individual dialects in Slovakia are described in more detail in the Dictionary of Slovak Dialects and several dialects are described in separate studies with an extension to other linguistic levels.

Slovak dialects are divided into three basic groups (see figure~\ref{fig:dialects_en}):

\begin{itemize}

\item[a)] The Western Slovak dialects are spread throughout the Trenčín, Nitra, Trnava, Myjava areas and other regions.

\begin{enumerate}
\setcounter{enumi}{19}
\item Upper Trenčín dialects
\item Lower Trenčín dialect
\item Váh river dialect
\item Central Nitra dialects
\item Lower Nitra dialects
\item Trnava area dialects
\item Záhorie dialect
\end{enumerate}

\item[b)] The central Slovak dialects are spoken in the regions of Liptov, Orava, Turiec, Tekov, Hont, Novohrad, Gemer and in the Zvolen area.

\begin{enumerate}
\setcounter{enumi}{9}
\item Liptov dialects
\item Orava dialects
\item Turiec dialect
\item Upper Nitra dialects
\item Zvolen dialects
\item Tekov dialects
\item Hont dialect
\item Novohrad dialects
\item Gemer dialects
\end{enumerate}

\item[c)] The eastern Slovak dialects can be found in the regions of Spiš, Šariš, Zemplín and Abov.

\begin{enumerate}
\setcounter{enumi}{29}
\item Spiš dialects
\item Abov dialects
\item Šariš dialects
\item Zemplín dialect
\item Soták dialects
\item Už dialects
\setcounter{enumi}{39}
\item Goral dialects
\item Ukrainian dialects
\item Various dialects
\item Hungarian dialects
\end{enumerate}
\end{itemize}

\medskip

These groups are further divided into a variety of subdialects (each village has its own dialect); especially mountainous regions have highly varied dialects. In the past, the mountainous character of the country caused certain (language) isolation of the population in individual provinces. These specific characteristics were also caused by the reorganisation and migration of the population, colonization, mixing of different dialect types, influence from neighbouring Slavic and non Slavic languages, changes in the employment of the population, etc. According to the nature of dialects and the occurrence of the individual characteristics, Slovak dialects in Hungary, Serbia, Croatia, Romania, Bulgaria and other countries, where large compact groups moved to in the past, can be included in these groups. In view of the limited number of old written monuments, Slovak dialects are the basic source of historical Slovak grammar.
