\noindent V~minulosti sa najviac investovalo do jazykového vzdelávania a~prekladu. Podľa
niektorých odhadov sa napríklad v~roku 2008 v~Európe minulo na preklad,
interpretáciu, softvérovú lokalizáciu a~internetovú globalizáciu približne 8,4
miliardy eur, pričom sa rátalo s~10-percentným nárastom ročne.\footnote{European
Commission Directorate-General for Translation, \emph{Size of the language
industry in the EU}, Kingston Upon Thames, 2009 \newline
(\url{http://ec.europa.eu/dgs/translation/publications/studies}).} Faktom je,
že tieto finančné prostriedky napriek tomu nestačia na uspokojenie súčasných
ani budúcich potrieb. Najlepšie riešenie pre dostatočný výskum používania
jazyka je výber technológie, ktorú používame aj na riešenie problémov
v~doprave, energetike, sociálnej oblasti a~pod.

Digitálne jazykové technológie (v~písanom aj hovorenom diskurze) pomáhajú ľuďom spolupracovať, podnikať, sprístupňovať vedomosti a~zúčastňovať sa na sociálnych a~politických diskusiách bez ohľadu na jazykové bariéry alebo počítačové zručnosti. Sú užitočné v~prípade:

\begin{itemize}
\item vyhľadávania informácií pomocou internetového vyhľadávača,
\item kontroly pravopisu a~gramatiky v~textových procesoroch,
\item odporúčania produktu v~internetovom obchode,
\item počúvania inštrukcií automobilového navigačného systému,
\item prekladu webových stránok prostredníctvom on-line služieb.
\end{itemize}

Jazykové technológie sa skladajú z~niekoľkých základných
aplikácií, ktoré sú bázou väčšieho aplikačného rámca.
Účelom bielej knihy META-NET-u je preskúmať stav základných
technológií všetkých európskych jazykov.

\boxtext{Európa potrebuje vhodné a cenovo dostupné jazykové technológie pre všetky európske jazyky}

Aby si Európa udržala svoju pozíciu na čele inovatívneho pokroku,
mali by sa jazykové technológie adaptovať dôkladne a~cenovo dostupne
na všetky európske jazyky a~zároveň sa pevne integrovať do
kľúčových softvérových prostredí. Bez jazykových technológií
Európa nedosiahne efektívne, interaktívne, multimediálne
a~viacjazyčné používateľské prostredie. 
