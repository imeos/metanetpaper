In this series of white papers, an initial effort has been made to assess support for language technology for many European languages. This makes it possible to compare the situation across languages at a high-level, and to identify gaps and needs.

This white paper demonstrates the existence of the quality environment for
linguistic research in Slovakia, despite the technology industry here is not
sufficiently developed. The Slovak research exists only in a small number of
available technologies and resources. This number is lower than for languages,
such as Czech and Polish, and substantially lower than for the main EU
languages (English, German or French). In addition, Slovak language
technologies and resources are of noticeably poorer quality.

We cannot really be optimistic about technology support for the Slovak
language. There is a nascent research scene in Slovakia concerning Slovak
Language LT, mostly in universities, scientific institutions, much like at the
small and medium enterprises that focus on basic research and solutions of
specific LT problems. Various institutions have devoted their efforts to
research and development of the LT products such as production of huge corpora of Slovak (of both written and spoken language), the morphology analysis, machine translation, complex speech interactive system, speech recognition system, etc. But those must be further developed and supported. 

According to the assessment detailed in this report, immediate action must
occur before any breakthroughs for the Slovak language can be achieved. It is
clear that there must be a greater effort to create LT resources for Slovak,
and drive research, innovation and development in general. The need for large
amounts data and the extreme complexity of language technology systems makes it
vital to develop a new infrastructure to spur greater sharing and cooperation.

There is also a lack of continuity in research and development funding. Short-term coordinated programmes tend to alternate with periods of low or sparse funding, and there is an overall lack of coordination among programmes in other EU countries and at the European Commission.

A large coordinated effort focused on language technologies would help
save the Slovak language, together with other languages, and establish a
genuine multilingual agenda for Europe and the world as a
whole.\footnote{V. Reding, J. Figeľ, Preface, in \emph{Human Language
Technologies for Europe}, TC-Star project,
\url{http://www.tcstar.org/pubblicazioni/D17_HLT_ENG.pdf}}
