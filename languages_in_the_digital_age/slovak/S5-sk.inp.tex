\noindent {\bf META-NET} je sieť excelentnosti čiastočne financovaná z~fondov Európskej komisie. Sieť tvorí v~súčasnosti 54 výskumných centier z~33 krajín. META-NET buduje Multilingválnu európsku technologickú alianciu {\bf META}, ktorá predstavuje narastajúcu komunitu profesionálov jazykových technológií a~organizácií v~Európe.

%obrazok

META-NET podporuje technologické základy multilingválnej európskej informačnej spoločnosti tým, že:

\begin{itemize}
\item umožní komunikáciu a~spoluprácu v~rôznych jazykoch;
\item garantuje všetkým Európanom rovnaký prístup k~informáciám a~vedomostiam v~ľubovoľnom jazyku;
\item buduje a vylepšuje funkcie zosieťovaných informačných technológií. 
\end{itemize}

Sieť podporuje Európu tým, že ju spája ako jediný digitálny trh a informačný priestor. META-NET stimuluje a~podporuje rozvoj viacjazyčných technológií všetkých európskych jazykov. Tieto technológie využívajú automatický preklad, generujú obsah, spracúvajú informácie, riadia vedomostný manažment a~i. Využívajú tiež intuitívne jazykové rozhrania aplikovateľné na rozmanité technologické výdobytky ako je domáca elektronika, stroje, autá alebo počítače či roboty.

