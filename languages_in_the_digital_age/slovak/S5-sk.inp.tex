\noindent META-NET je sieť excelentnosti financovaná z~fondov Európskej komisie. Sieť tvorí v~súčasnosti 47 členov z~31 krajín. META-NET podporuje Multilingválnu európsku technologickú alianciu (META) a~narastajúci počet komunít profesionálov jazykových technológií a~organizácií v~Európe.

%obrazok

META-NET spolupracuje s~ďalšími subjektmi, ako sú Common Language Sources alebo Technologická infraštruktúra (Clarino), ktorý pomáha digitalizovať humanitné vedy v~Európe. META-NET podporuje technologické základy multilingválnej európskej informačnej spoločnosti tým, že:

\begin{itemize}
\item umožní komunikáciu a~spoluprácu v~rôznych jazykoch;
\item zabezpečí rovnaký prístup k~informáciám a~vedomostiam v~ľubovoľnom jazyku;
\item ponúkne moderné a~cenovo dostupné zosieťované informačné technológie európskym občanom. 
\end{itemize}

META-NET stimuluje a~podporuje rozvoj viacjazyčných technológií všetkých európskych jazykov. Tieto technológie využívajú mnohoraké aplikácie, ktoré tvoria automatický preklad, generujú obsah, spracúvajú informácie, riadia vedomostný manažment a~i. Sieť chce zlepšiť súčasné prístupy, aby bola v~rôznych jazykoch lepšia spolupráca a~komunikácia. Všetci Európania majú rovnaké právo na informácie a~znalosti bez ohľadu na jazyk.
