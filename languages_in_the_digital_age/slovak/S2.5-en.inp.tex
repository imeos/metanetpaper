Although language technology has made considerable progress in the last few years, the current pace of technological progress and product innovation is too slow. Widely-used technologies such as the spelling and grammar correctors in word processors are typically monolingual, and are only available for a handful of languages. Online machine translation services, although useful for quickly generating a reasonable approximation of a document’s contents, are fraught with difficulties when highly accurate and complete translations are required. Due to the complexity of human language, modelling our tongues in software and testing them in the real world is a long, costly business that requires sustained funding commitments. Europe must therefore maintain its pioneering role in facing the technology challenges of a multiple-language community by inventing new methods to accelerate development right across the map. These could include both computational advances and techniques such as crowdsourcing.
