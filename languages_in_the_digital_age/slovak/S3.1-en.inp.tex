The Slovak Republic is a country in Central Europe neighbouring both Slavic (Czech Republic, Poland, Ukraine) and non-Slavic countries (Hungary, Austria). Its geographic location, mostly mountainous landscape and historical development created the considerably multi-ethnic and multicultural character of the country. It also caused a variety of Slovak dialects and the subsequent codification of (modern) standard Slovak as an over-regional communication mean by as late as 1843. Although part of the territory of Slovakia belonged to the historic Great Moravia, where Constantine and Methodius, invited from the Byzantine Empire in the 9\textsuperscript{th} century were spreading the Christian religion and education through Old Church Slavonic and the Glagolitic alphabet. Later development of Slovakia and the Slovak language was influenced by the Latin alphabet and Roman culture. Several influences subsequently occurred that left traces on the Slovak language as well.

The Slovak language– in the Indo-European family of languages, together with Polish, Czech, Lower and Upper Sorbian – belongs to the West branch of Slavic languages. Linguistic, historic, and archaeological sources prove that Slovak developed directly from Proto-Slavic. The Proto-Slavic basis of Slovak was formed in the area between the Carpathians, the Danube, and the Upper Moravia. The Slavonians, predecessors of the Slovaks, came to this area in the 6\textsuperscript{th} century from the south-east. The reconstructed language of the Great Moravian ethnic group, which was divided into dialects but formed a certain cultural form can be regarded as the basis of Slovak. The Slovak language went through fast development in the 10\textsuperscript{th} to 12\textsuperscript{th} centuries (jer vocalisation, disappearance of nasal vowels), and stabilised in the 13\textsuperscript{th} to 15\textsuperscript{th} centuries. In the 16\textsuperscript{th} to 18\textsuperscript{th} centuries, Czech was used as the cultural language in Slovakia, together with several types of cultural Slovak, such as cultural West Slovak, cultural Central Slovak and cultural East Slovak. By the end of the 18\textsuperscript{th} century, attempts at the formation of literary Slovak had started. At the end of the 18\textsuperscript{th} century, Anton Bernolák based his codification on cultural West Slovak, but failed to get wide recognition due to changed social and economic conditions. Ľudovít Štúr used Central Slovak as the basis and his idea took hold very soon, and with certain modifications (Martin Hattala, Michal Miloslav Hodža) lasts up to these days.

Slovak is the official language in the Slovak Republic. Since May 2004 it has also been one of the administrative languages of the European Union. Slovak is spoken by 4.5 million inhabitants of Slovakia, more than 1 million emigrants in the United States, and approx. 300 thousand people in the Czech Republic. Smaller language groups of Slovaks are situated in Hungary, Romania, Serbia, Croatia, Bulgaria, Poland, the United Kingdom, France, Germany, Belgium, Austria, Norway, Denmark, Finland, Sweden, Italy, Switzerland, the Netherlands, Cyprus, Russia, Ukraine, Kyrgyzstan, Israel, Canada, South Africa, Argentina, Brazil, Uruguay, Australia, New Zealand,  and other countries. Slovak language is ``esperanto'' of all the Slavic languages due to its most comprehensible character for other users of Slavic langugages.

\boxtext{Slovak language is ``esperanto'' of all the Slavic languages}

 Slovaks abroad pertain to different groups: they are descendants of indigenous
inhabitants of Slovakia, who moved to other areas of former Austro-Hungary;
descendants of later migrants from Slovakia, living overseas (emigration wave
from the late 19\textsuperscript{th} to the mid 20\textsuperscript{th}
century); political and economic migrants after 1945, 1948, and 1968 and their
descendants; and finally, mostly young people settled abroad after the year
1990. It is estimated that some 270\,000 Slovaks went abroad in the last wave
of emigration in the years 2007 – 2008. A special group consists of descendants
of  Slovaks, who remained abroad due to political and geographical changes
after the year 1918 or the year 1945. At the same time, there are ethnic
minorities living in Slovakia (Hungarians, Gypsies, Czechs, Ruthenians,
Ukrainians, Germans, Poles, Moravians, Croatians, Bulgarians, Jews), which
together account for 14.2\% of population of Slovakia.

The Slovak language has several forms: standard Slovak is mainly used in
written form and in official communication and colloquial Slovak represents a
standard mainly used in verbal communication. Each form has specific subgroups,
which form the Slovak language stratification: literary language / nationwide
standard language / nationwide substandard language / regional variant / local
variant, territorial variant (dialects), social variant (slang, jargon, argot,
professional languages). At the time of compiling this dcument, responsibility
for control over language and language policy was borne by the Ministry of
Culture (Act on State Language, Central Language Board). Its decisions should
be based on the knowledge and opinions of the scientific and professional
community led by the Ľudovít Štúr Institute of Linguistics of the Slovak
Academy of Sciences. The Institute is a founder and coordinator of several
commissions with nationwide coverage: spelling committee, orthoepic committee,
onomastic committee, and the committee for codification. The committees prepare
and recommend codification of orthoepic, spelling, grammatical and lexical
rules. Spelling rules are subject to a broader discussion with the involvement
of the general public, but due to the interconnection of many factors and
social impact of any changes they are not amended too often. The last
amendments, especially in the rules of rhythmic alternation and capitalisation,
were made in 1991. The lexicographic works (Krátky slovník slovenského jazyka,
Slovník súčasného slovenského jazyka A -- G, H -- L, Synonymický slovník,
Slovník cudzích slov – akademický
\cite{kssj2003,sssj2006,sssj2011,sss2004,scs2005})\footnote {Short Dictionary
of Slovak, Dictionary of Contemporary Slovak, Slovak Synonym Dictionary,
Dictionary of Foreign Words -- academic} compiled at the Ľ. Štúr Institute of
Linguistics of the Slovak Academy of Sciences cover not only the orthography
but also lexical, grammar and orthoepic rules. Monographs and scientific
articles published by the Institute capture the Slovak language situations in
all its areas. 

The territorial arrangement of Slovakia (a territory with an area of
almost 50\,000 km\textsuperscript{2} is mainly situated lengthwise; the
length between eastern and western borderlines is almost 430 km) and
specifics of individual dialects also affect forms of Slovak language in
specific regions and locations, which represents a problem to be coped
with mainly by foreigners learning Slovak and moving throughout the
territory of the Slovak Republic.

%obrazok
