Studia Academica Slovaca – The Centre for Slovak as a Foreign Language (SAS) is a specialised centre at the Faculty of Arts\footnote{officially also called Faculty of Philosophy}, Comenius University (FF UK) in Bratislava. The pedagogical and research activities focus on the education of foreigners interested in Slovak language and culture, propagation of Slovak science, culture and art abroad, implementation and coordination of the research of Slovak as a foreign language, realisation of international and domestic research projects and activities aimed at creating and publishing academic Slovakist material and textbooks of Slovak as a foreign language. Besides the SAS being an expert centre for Slovak as a foreign language, it also traditionally participates in scientific methodical preparation for lecturers of Slovak as a foreign language at universities abroad. The result of the cooperation with the lectorates and foreign Slavists builds a database of Slavonic studies abroad.

\boxtext{The activities focus on the education of foreigners interested
in Slovak language and culture}

Another part of the Centre's activities is the annual organisation and realisation of a Summer School of Slovak Language and Culture \emph{Studia Academica Slovaca}, which has been offered to foreign applicants since 1965. The Methodical Centre SAS reassumed its successful history in 1992, and in 2006 it was transformed into SAS – The Centre for Slovak as a Foreign Language. In its almost half-century of existence of SAS, almost 6 000 foreign alumni interested in Slovak language, culture and realia from more than 50 countries all over the world have  utilized its services. On the grounds of Studia Academica Slovaca the basis of scientific description and didactics of Slovak as a foreign language was laid, and the first textbooks and didactics of Slovak as a foreign language were written. In relation to its wide tradition and experience, SAS currently works as a coordination and information centre with slovakiawide as well as an exterior sphere of activity.
In 2006 the SAS Centre acquired accreditation from the Ministry of Education of the Slovak Republic for providing educational activities concerning Slovak as a Foreign Language – language courses in contact and distance form for all levels of language development including beginners (A1, A2), intermediate and upper-intermediate (B1, B2) and advanced (C1, C2). Their contents are published in printed version \cite{pekarovicova2007} and published on the web\footnote{\url{http://www.fphil.uniba.sk/fileadmin/user_upload/editors/sas/slavic/Vzdelavaci_program.pdf}}.
Based on a grant from the Ministry of Education of the Slovak Republic under the project Educational Programme Slovak as a Foreign Language, SAS offers those who are interested in Slovak language a Slovak e-learning course for level A1 (Basic User – Breakthrough) and level A2 (Basic User – Waystage). The objective of the project is to create both content and forms of language development for foreigners on individual levels corresponding to The Common European Framework of Reference for Languages, as well as to specify individual criteria of the evaluation and certification of language competence. The main scope is the preparation of standard and specialised learning materials for students and methodical materials for teachers. Every year a Methodical seminary on Slovak as a foreign language for teachers of grammar and secondary schools abroad and for university lectors takes place to inform about new approaching linguistics, literature, culture and didactics of Slovak as a foreign language.

A product of the implementation of the project by the Studia Academica Slovaca group “Educational programme Slovak as a Foreign Language”, the Faculty of Arts of Comenius University has been awarded the \emph{European label 2007} by the European Commission in the field of language education. 
