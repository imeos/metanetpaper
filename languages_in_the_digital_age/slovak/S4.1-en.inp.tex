%obrazok

Language technologies are information technologies specialised in human language processing. Therefore these technologies are also often subsumed under the term human language technology. Human language occurs in spoken and written form. While speech is the oldest and most natural mode of language communication, complex information and the bulk of human knowledge is recorded and transmitted in written texts. Speech and text technologies process or produce language in these two forms. However, language also has aspects common to both forms such as dictionaries, most of the grammar, and the meaning of sentences. Thus, large parts of language technology cannot be subsumed under either speech or text technologies. Knowledge technologies include technologies that link language to knowledge. Figure 3 illustrates the language technology landscape. In our communication, we mix language with other modes of communication and other information media. We combine speech with gestures and facial expressions. Texts can be combined with pictures and sounds. Films may contain language in spoken and written form. Thus, speech and text technologies overlap and interact with many other technologies that facilitate the processing of multi-modal communication and multimedia documents.
