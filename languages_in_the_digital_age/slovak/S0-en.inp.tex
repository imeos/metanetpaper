This white paper is part of a series that promotes knowledge about language technology and its potential. It addresses educators, journalists, politicians, language communities and others.

The availability and use of language technology in Europe varies between languages. Consequently, the actions that are required to further support research and development of language technologies also differ for each language. The required actions depend on many factors, such as the complexity of a given language and the size of its community.

META-NET, a Network of Excellence funded by the European Commission, has conducted an analysis of current language resources and technologies. This analysis focused on the 23 official European languages as well as other important national and regional languages in Europe. The results of this analysis suggest that there are many significant research gaps for each language. A more detailed expert analysis and assessment of the current situation will help maximise the impact of additional research and minimize any risks.

META-NET consists of 47 research centres from 31 countries that are working with stakeholders from commercial businesses, government agencies, industry, research organisations, software companies, technology providers and European universities. Together, they are creating a common technology vision while developing a strategic research agenda that shows how language technology applications can address any research gaps by 2020.
