
\noindent Hoci jazykové technológie za posledné roky napredujú, súčasné tempo technologického vývoja a~inovácie produktov je pomalé. Jazykové technológie so širokým využitím (napríklad kontrola pravopisu a~gramatiky v~textových editoroch) jestvujú v~monolingválnej forme, a~preto sú dostupné len pre hŕstku jazykov. On-line služby, ako sú profesionálne aplikácie strojových prekladov, prinášajú so sebou mnohé ťažkosti v~situáciách, v~ktorých sú potrebné veľmi presné a~úplné preklady. Vzhľadom na zložitosť ľudského jazyka a~modelovanie nášho jazyka do softvéru je následné testovanie pridlhé a~nákladné a~vyžaduje si neustálu finančnú podporu. Ak si chce Európa zachovať svoje postavenie priekopníka v~prijímaní technologických výziev viacjazyčnej jazykovej komunity, musí neustále predkladať nové metódy na urýchlenie technologického rozvoja, napríklad progres v~oblasti počítačovej technológie a~techník ako crowdsourcing.

\boxtext{Súčasné tempo technologického vývoja je príliš pomalé}

