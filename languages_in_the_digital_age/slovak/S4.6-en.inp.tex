
The current state of LT support varies considerably from one language community to another. In order to compare the situation between languages, this section will present an evaluation based on two sample application areas (machine translation and speech processing) and one underlying technology (text analysis), as well as basis resources needed for building LT applications. The languages were categorised using the following five-point scale:


\begin{figure*}
\small
\centering
\begin{tabular}
{ % defines color for each column.
>{\columncolor{corange5}} p{.17\linewidth}@{\hspace{.027\linewidth}}
>{\columncolor{corange4}}p{.17\linewidth}@{\hspace{.027\linewidth}}
>{\columncolor{corange3}}p{.17\linewidth}@{\hspace{.027\linewidth}}
>{\columncolor{corange2}}p{.17\linewidth}@{\hspace{.027\linewidth}}
>{\columncolor{corange1}}p{.17\linewidth} 
}
\rowcolor{orange1} % redefines color for all columns in row 1
\begin{center}\vspace*{-2mm}\textbf{Cluster 1}\end{center} & 
\begin{center}\vspace*{-2mm}\textbf{Cluster 2}\end{center} & 
\begin{center}\vspace*{-2mm}\textbf{Cluster 3}\end{center} & 
\begin{center}\vspace*{-2mm}\textbf{Cluster 4}\end{center} & 
\begin{center}\vspace*{-2mm}\textbf{Cluster 5}\end{center} \\ \addlinespace

& \vspace*{0.5mm}English
& \vspace*{0.5mm}German \newline   
Italian \newline  
Finnish \newline 
French \newline 
Dutch \newline 
Portuguese \newline 
Spanish \newline
Czech \newline 
& \vspace*{0.5mm}Basque \newline 
Bulgarian \newline 
Danish \newline 
Estonian \newline 
Galician\newline 
Greek \newline  
Irish \newline  
Catalan \newline 
Norwegian \newline 
Polish \newline 
Swedish \newline
Serbian \newline 
Slovak \newline 
Slovene \newline 
Hungarian  \newline
& \vspace*{0.5mm}Icelandic \newline  
Croatian \newline 
Latvian \newline 
Lithuanian \newline 
Maltese \newline 
Romanian\\
\end{tabular}
\label{fig:speech_cluster_en}
\caption{Language clusters for Speech Processing}
\end{figure*}

\begin{figure*}
\small
\centering
\begin{tabular}
{ % defines color for each column.
>{\columncolor{corange5}} p{.17\linewidth}@{\hspace{.027\linewidth}}
>{\columncolor{corange4}}p{.17\linewidth}@{\hspace{.027\linewidth}}
>{\columncolor{corange3}}p{.17\linewidth}@{\hspace{.027\linewidth}}
>{\columncolor{corange2}}p{.17\linewidth}@{\hspace{.027\linewidth}}
>{\columncolor{corange1}}p{.17\linewidth} 
}
\rowcolor{orange1} % redefines color for all columns in row 1
\begin{center}\vspace*{-2mm}\textbf{Cluster 1}\end{center} & 
\begin{center}\vspace*{-2mm}\textbf{Cluster 2}\end{center} & 
\begin{center}\vspace*{-2mm}\textbf{Cluster 3}\end{center} & 
\begin{center}\vspace*{-2mm}\textbf{Cluster 4}\end{center} & 
\begin{center}\vspace*{-2mm}\textbf{Cluster 5}\end{center} \\ \addlinespace

& \vspace*{0.5mm} English 
& \vspace*{0.5mm} French \newline 
Spanish
& \vspace*{0.5mm}German \newline 
Italian \newline 
Catalan \newline 
Dutch \newline 
Polish \newline 
Romanian \newline 
Hungarian 
& \vspace*{0.5mm}Basque \newline 
Bulgarian \newline 
Danish \newline 
Estonian \newline 
Finnish \newline 
Galician \newline 
Greek \newline 
Irish \newline 
Icelandic \newline 
Croatian \newline 
Latvian \newline 
Lithuanian \newline 
Maltese \newline 
Norwegian \newline 
Portuguese \newline 
Swedish \newline 
Serbian \newline 
Slovak \newline 
Slovene \newline 
Czech \newline
\end{tabular}
\label{fig:mt_cluster_en}
\caption{Language clusters for Machine Translation}
\end{figure*}

\begin{figure*}
  \small
  \centering
  \begin{tabular}
{ % defines color for each column.
>{\columncolor{corange5}} p{.17\linewidth}@{\hspace{.027\linewidth}}
>{\columncolor{corange3}}p{.17\linewidth}@{\hspace{.027\linewidth}}
>{\columncolor{corange3}}p{.17\linewidth}@{\hspace{.027\linewidth}}
>{\columncolor{corange2}}p{.17\linewidth}@{\hspace{.027\linewidth}}
>{\columncolor{corange1}}p{.17\linewidth} 
}
\rowcolor{orange1} % redefines color for all columns in row 1
\begin{center}\vspace*{-2mm}\textbf{Cluster 1}\end{center} & 
\begin{center}\vspace*{-2mm}\textbf{Cluster 2}\end{center} & 
\begin{center}\vspace*{-2mm}\textbf{Cluster 3}\end{center} & 
\begin{center}\vspace*{-2mm}\textbf{Cluster 4}\end{center} & 
\begin{center}\vspace*{-2mm}\textbf{Cluster 5}\end{center} \\ \addlinespace

& \vspace*{0.5mm}English
& \vspace*{0.5mm}German \newline 
  French \newline 
  Italian \newline 
  Dutch \newline 
  Spanish
& \vspace*{0.5mm}Basque \newline 
  Bulgarian \newline 
  Danish \newline 
  Finnish \newline 
  Galician \newline 
  Greek \newline 
  Catalan \newline 
  Norwegian \newline 
  Polish \newline 
  Portuguese \newline 
  Romanian \newline 
  Swedish \newline 
  Slovak \newline 
  Slovene \newline 
  Czech \newline 
  Hungarian \newline 
& \vspace*{0.5mm}Estonian \newline 
  Irish \newline 
  Icelandic \newline 
  Croatian \newline 
  Latvian \newline 
  Lithuanian \newline 
  Maltese \newline 
  Serbian \\
  \end{tabular}
\label{fig:text_cluster_en}
\caption{Language clusters for Text Analysis}
\end{figure*}

\begin{figure*}
  \small
  \centering
\begin{tabular}
{ % defines color for each column.
>{\columncolor{corange5}} p{.17\linewidth}@{\hspace{.027\linewidth}}
>{\columncolor{corange4}}p{.17\linewidth}@{\hspace{.027\linewidth}}
>{\columncolor{corange3}}p{.17\linewidth}@{\hspace{.027\linewidth}}
>{\columncolor{corange2}}p{.17\linewidth}@{\hspace{.027\linewidth}}
>{\columncolor{corange1}}p{.17\linewidth} 
}
\rowcolor{orange1} % redefines color for all columns in row 1
\begin{center}\vspace*{-2mm}\textbf{Cluster 1}\end{center} & 
\begin{center}\vspace*{-2mm}\textbf{Cluster 2}\end{center} & 
\begin{center}\vspace*{-2mm}\textbf{Cluster 3}\end{center} & 
\begin{center}\vspace*{-2mm}\textbf{Cluster 4}\end{center} & 
\begin{center}\vspace*{-2mm}\textbf{Cluster 5}\end{center} \\ \addlinespace
    
& \vspace*{0.5mm}English
& \vspace*{0.5mm}German \newline 
    French \newline 
    Dutch \newline 
    Swedish \newline 
    Czech \newline 
    Polish \newline
    Hungarian 
& \vspace*{0.5mm} Basque\newline 
    Bulgarian\newline 
    Danish \newline 
    Estonian \newline 
    Finnish \newline 
    Galician \newline 
    Greek \newline 
    Catalan \newline 
    Croatian \newline 
    Norwegian \newline 
    Portuguese \newline 
    Romanian \newline 
    Serbian \newline 
    Slovak \newline 
    Slovene \newline
&  \vspace*{0.5mm} Irish \newline 
    Icelandic \newline 
    Latvian \newline 
    Lithuanian \newline 
    Maltese  \\
  \end{tabular}
  \caption{Language clusters for Resources}
  \label{fig:resources_cluster_en}
\end{figure*}




\begin{itemize}
  \item Cluster 1 (excellent LT support)
  \item Cluster 2 (good support)
  \item Cluster 3 (moderate support)
  \item Cluster 4 (fragmentary support) 
  \item Cluster 5 (weak or no support)
\end{itemize}

LT support was measured according to the following criteria:
\begin{itemize}
\item Speech Processing: Quality of existing speech recognition technologies, quality of existing speech synthesis technologies, coverage of domains, number and size of existing speech corpora, amount and variety of available speech-based applications
\item Machine Translation: Quality of existing MT technologies, number of language pairs covered, coverage of linguistic phenomena and domains, quality and size of existing parallel corpora, amount and variety of available MT applications
\item Text Analysis: Quality and coverage of existing text analysis technologies (morphology, syntax, semantics), coverage of linguistic phenomena and domains, amount and variety of available applications, quality and size of existing (annotated) text corpora, quality and coverage of existing lexical resources (e.g., WordNet) and grammars
\item Resources: Quality and size of existing text corpora, speech corpora and parallel corpora, quality and coverage of existing lexical resources and grammars
\end{itemize} 

