The following table provides an overview of the current situation of Language Technology support for Slovak. The rating of existing tools and resources was generated by leading experts in the field who provided estimates based on a scale from 0 (very low) to 6 (very high) according to seven criteria.

\begin{enumerate}
\item Quantity: Does a tool/resource exist for the language at hand? The more tools/resources exist, the higher the rating.
\begin{itemize}
\item 0: no tools/resources whatsoever
\item 6: many tools/resources, large variety
\end{itemize}
\item Availability: Are tools/resources accessible, i.e.,are they Open Source, freely usable on any platform or only available for a high price or under very restricted conditions?
\begin{itemize}
\item 0: practically all tools/resources are only available for a high price
\item 6: a large amount of tools/resources is freely, openly available under sensible Open Source or Creative Commons licenses that allow re-use and re-purposing\footnote{If there are e.g. two resources, one of them completely open and the other completely closed, we put the average (i.e. 3)}
\end{itemize}
\item Quality: How well are the respective performance criteria of tools and quality indicators of resources met by the best available tools, applications or resources? Are these tools/resources current and also actively maintained?
\begin{itemize}
\item 0: toy resource/tool
\item 6: high-quality tool, human-quality annotations in a resource
\end{itemize}
\item Coverage: To what degree do the best tools meet the respective coverage criteria (styles, genres, text sorts, linguistic phenomena, types of input/output, number of languages supported by an MT system etc.)? To what degree are resources representative of the targeted language or sublanguages?
\begin{itemize}
\item 0: special-purpose resource or tool, specific case, very small coverage, only to be used for very specific, non-general use cases
\item 6: very broad coverage resource, very robust tool, widely applicable, many languages supported
\end{itemize}
\item Maturity: Can the tool/resource be considered mature, stable, ready for the market? Can the best available tools/resources be used out-of-the-box or do they have to be adapted? Is the performance of such a technology adequate and ready for production use or is it only a prototype that cannot be used for production systems? An indicator may be whether resources/tools are accepted by the community and successfully used in LT systems. 
\begin{itemize}
\item 0: preliminary prototype, toy system, proof-of-concept, example resource exercise
\item 6: immediately integratable/applicable component
\end{itemize}
\item Sustainability: How well can the tool/resource be maintained/integrated into current IT systems? Does the tool/resource fulfill a certain level of sustainability concerning documentation/manuals, explanation of use cases, front-ends, GUIs etc.? Does it use/employ standard/best-practice programming environments (such as Java EE)? Do industry/research standards/quasi-standards exist and if so, is the tool/resource compliant (data formats etc.)?
\begin{itemize}
\item 0: completely proprietary, ad hoc data formats and APIs
\item 6: full standard-compliance, fully documented
\end{itemize}
\item Adaptability: How well can the best tools or resources be adapted/extended to new tasks/domains/genres/text types/use cases etc.?
\begin{itemize}
\item 0: practically impossible to adapt a tool/resource to another task, impossible even with large amounts of resources or person months at hand
\item 6: very high level of adaptability; adaptation also very easy and efficiently possible
\end{itemize}
\end{enumerate}
