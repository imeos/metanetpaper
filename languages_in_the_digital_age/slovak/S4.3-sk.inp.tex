\noindent Tvorba aplikácií jazykových technológií v~sebe zahŕňa množstvo čiastkových úloh, ktoré síce v~interakcii s~používateľom nevyjdú vždy na povrch, ale poskytujú rozličné funkcie skrytého systému. Koncipujú preto v~danej oblasti výskumu dôležité otázky, ktoré sa stali samostatnými akademickými subdisciplínami počítačovej lingvistiky.

\underbar{Zodpovedanie otázok} sa stalo aktívnou oblasťou výskumu,
pre ktorý boli vytvorené anotované korpusy a~ktorý odštartoval
vedecké súperenie. Idea spočíva v~posune od vyhľadávania pomocou
klávesnice (na ktoré prístroj odpovedá celým súborom potenciálne
relevantných dokumentov) k~variantu, v~ktorom používateľ kladie
konkrétnu otázku a~systém generuje jedinú odpoveď: „V~akom veku
vystúpil Neil Armstrong na Mesiac?“ – „38“. Pokiaľ to súvisí
s~už spomínanou základnou oblasťou vyhľadávania na webe,
zodpovedanie otázok je predovšetkým zastrešujúcim výrazom
výskumných otázok typu: Aké \emph{druhy} otázok by sa mali
rozlišovať a~ako by sa malo s~nimi zaobchádzať, ako sa môže súbor
dokumentov, ktorý potenciálne obsahuje odpoveď, analyzovať
a~porovnávať (dávajú tieto dokumenty konfliktnú odpoveď?) a~ako
môže byť špecifická informácia – odpoveď – spoľahlivo
extrahovaná z~dokumentu bez neoprávneného ignorovania kontextu. 

To na druhej strane súvisí s~úlohou získavania informácií, s~oblasťou, ktorá sa tešila veľkej popularite a~vplyvu v~období „štatistického obratu“ v~počítačovej lingvistike v~raných 90. rokoch 20. storočia. Extrahovanie informácií sa sústreďuje na identifikáciu špecifických informácií v~špecifických triedach dokumentov; akými by mohli byť napríklad detekcia kľúčových hráčov prevzatia podnikov, ktorí sú uvedení v~novinových článkoch. Druhý variant, na ktorom sa pracovalo, sú správy o~teroristických útokoch, v~prípade ktorých problémom zostáva zmapovanie textu do šablóny špecifikujúcej páchateľa, cieľ, čas a~miesto útoku, ako aj jeho dôsledky. Doménovo špecifická náplň šablóny je ústrednou charakteristikou extrahovania informácií, ktorá je aj z~tohto dôvodu ďalším príkladom „zákulisnej“ technológie, ktorá predstavuje dobre ohraničenú oblasť výskumu, ale z~praktických dôvodov musí byť vsadená do vhodného aplikačného prostredia. 


JBOWL (Java knižnica \emph{Bag-Of-Words}) softvérová knižnica bola vyvinutá v~Centre pre informačné technológie (FEI-CIT) v~Košiciach na podporu NLP Text Mining aplikácií. JBOWL je modulárny systém umožňujúci spravovanie textových dokumentov. Poskytuje funkcie a~prostriedky podporujúce spracovanie textov prirodzeného jazyka (napr. tokenizáciu, morfologickú analýzu, lematizáciu, viacznačnosť, syntaktickú analýzu založenú na sieťach ATN, identifikáciu klasterov a~fráz, meranie závažnosti termínov a~ich označovanie), objavuje znalosti a~ťaží z~neštruktúrovaných textových dokumentov. Okrem iného systém implementuje viaceré algoritmy kontrolovaného a~nekontrolovaného strojového učenia s~nastaviteľnými vstupnými parametrami a~metódami na hodnotenie kvality modelov Text Miningu.

\boxtext{V Centre pre informačné technológie v Košiciach bola vyvinutá softvérová knižnica, ktorá spravuje textové dokumenty}

Dve hraničné oblasti, ktoré niekedy hrajú rolu samostatnej aplikácie a~inokedy rolu podporného, skrytého komponentu, sú sumarizovanie a~generovanie textu. Sumarizovanie zjavne súvisí s~úlohou skracovania textu a~ponúka sa napríklad aj ako funkcia MS Wordu. Pracuje prevažne na základe štatistických metód, pričom najprv identifikuje „dôležité“ slová v~texte (napríklad slová, ktoré sú v~tomto texte vysoko frekventované, ale výrazne menej používané v~bežnom jazyku), a~následne určuje tie vety, ktoré obsahujú veľa „dôležitých“ slov. Tieto vety sú v~dokumente vyznačené alebo sú z~neho extrahované a~použité na tvorbu sumáru. V~tomto variante, ktorý je doteraz najpoužívanejší, sa sumarizovanie rovná extrahovaniu viet: text je redukovaný na podskupinu jeho viet. Všetky komerčné sumarizéry využívajú práve tento nápad. Alternatívny postup, ktorému sa venuje len časť výskumu, spočíva v~aktuálnej syntéze \emph{nových} viet, t.~j. buduje súhrn viet, ktoré sa nemusia ukázať v~takejto forme vo východiskovom texte. Takýto postup si však vyžaduje určité hlbšie porozumenie textu a~je oveľa menej priamočiary. Textový generátor ako celok vo väčšine prípadov nie je samostatnou aplikáciou, ale je včlenený do väčšieho softvérového prostredia, ako napríklad do klinického informačného systému, kde sa údaje o~pacientoch zhromažďujú, skladujú, spracúvajú, pričom generovanie správ je len jednou z~mnohých funkcií. 
