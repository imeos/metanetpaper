Tieto jazyky sa zoskupili na základe nasledujúcej päťbodovej škály: 
\begin{itemize}
\item Cluster 1: vynikajúca podpora LT
\item Cluster 2: dobrá podpora
\item Cluster 3: nevýrazná podpora
\item Cluster 4: čiastková podpora
\item Cluster 5: slabá alebo žiadna podpora
\end{itemize}

Podpora LT sa merala podľa nasledovných kritérií:
\begin{itemize}
\item Práca s rečou: Kvalita existujúcich technológií na rozpoznávanie reči, kvalita existujúcich technológií na vytváranie reči, záber domén, počet a veľkosť existujúcich rečových súborov, množstvo a pestrosť dostupných na reči založených aplikácií
\item Strojový preklad: Kvalita existujúcich technológií SP, počet pokrytých jazykových párov, pokrytie lingvistických fenoménov a domén, kvalita a veľkosť existujúcich paralelných súborov, množstvo a pestrosť dostupných SP aplikácií
\item Textová Analýza: Kvalita a pokrytie existujúcich technológií textovej analýzy (morfológie, syntaxu, sémantiky), pokrytie lingvistických fenomémov a domén, množstvo a pestrosť dostupných aplikácií, kvalita a veľkosť existujúcich (anotovaných) textových súborov, kvalita a pokrytie existujúcich lexikálnych zdrojov (napr. WordNet) a gramatík
\item Zdroje: Kvalita a veľkosť existujúcich textových súborov, rečových súborov a paralelných súborov, kvalita a pokrytie existujúcich lexikálnych zdrojov a gramatík
\end{itemize} 