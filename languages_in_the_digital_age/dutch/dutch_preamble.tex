%                                     MMMMMMMMM                                         
%                                                                             
%  MMO    MM   MMMMMM  MMMMMMM   MM    MMMMMMMM   MMD   MM  MMMMMMM MMMMMMM   
%  MMM   MMM   MM        MM     ?MMM              MMM$  MM  MM         MM     
%  MMMM 7MMM   MM        MM     MM8M    MMMMMMM   MMMMD MM  MM         MM     
%  MM MMMMMM   MMMMMM    MM    MM  MM             MM MMDMM  MMMMMM     MM     
%  MM  MM MM   MM        MM    MMMMMM             MM  MMMM  MM         MM     
%  MM     MM   MMMMMM    MM   MM    MM            MM   MMM  MMMMMMM    MM
%
%
%          - META-NET Language Whitepaper | Dutch Metadata -
% 
% ----------------------------------------------------------------------------

\usepackage{booktabs}
\usepackage{longtable}
\usepackage{tabulary}
\usepackage{tabularx}
\usepackage{rotating}
\usepackage{makecell}
\usepackage{multirow}
\usepackage{colortbl}
\usepackage{multicol,framed,lipsum}
\usepackage{polyglossia}
\setotherlanguages{dutch,english}

\title{Het Nederlands in het Digitale Tijdperk --- The Dutch Language in the Digital Age}

\spineTitle{The Dutch Language in the Digital Age --- Het Nederlands in het Digitale Tijdperk}

\subtitle{White Paper Series --- Witboekserie}

\author{
  Jan Odijk
}

\authoraffiliation{
  Jan Odijk~ {\small Universiteit Utrecht}
}

\editors{
  Georg Rehm, Hans Uszkoreit\\(redactie, \textcolor{grey1}{editors})
}

% Text in left column on backside of the cover
\SpineLText{\selectlanguage{english}%
  In everyday communication, Europe’s citizens, business partners and politicians are inevitably confronted with language barriers.
  Language technology has the potential to overcome these barriers and to provide innovative interfaces to technologies and knowledge.
  This white paper presents the state of language technology support for the Dutch language.
  It is part of a series that analyzes the available language resources and technologies for 30~European languages.
  The analysis was carried out by META-NET, a Network of Excellence funded by the European Commission.
  META-NET consists of 54 research centres in 33 countries, who cooperate with stakeholders from economy, government agencies, research organisations, non-governmental organisations, language communities and European universities.
  META-NET’s vision is high-quality language technology for all European languages.
}

% Text in right column on backside of the cover
\SpineRText{\selectlanguage{dutch}%
   Burgers, mensen in het bedrijfsleven en politici van Europa worden onvermijdelijk in de alledaagse communicatie geconfronteerd met taalbarrières. Taaltechnologie heeft het potentieel deze barrières te slechten en innovatieve interfaces te leveren naar technologieën en kennis. Dit witboek beschrijft de toestand van taaltechnologische ondersteuning voor het Nederlands. Het maakt deel uit van een serie die de beschikbare taalbronnen en technologieën analyseert voor 30 Europese talen. De analyse werd uitgevoerd door META-NET, een `Network of Excellence’ gefinancierd door de Europese Commissie. META-NET bestaat uit 54 onderzoekscentra in 33 landen, die samenwerken met belanghebbenden uit de economie, regeringsagentschappen, onderzoeksorganisaties, niet-gouvernementele organisaties, taalgemeenschappen en Europese universiteiten. De visie van META-NET is taaltechnologie van hoge kwaliteit voor alle Europese talen.
}

% Quotes from VIPs on backside of the cover
\quotes{%
\selectlanguage{dutch}``Als NOTaS zijn wij zeer enthousiast over dit indrukwekkende boek. Het laat duidelijk zien dat we nog een lange maar wel een begaanbare weg voor ons hebben." \\
  \textcolor{grey2}{--- Debbie Kenyon-Jackson (Voorzitter van NOTaS)}\\[3mm]
  ``Taaltechnologie is een onmisbaar wapen als een kleinere taal zoals het Nederlands zich wil handhaven in de oorlog tegen de hegemonie van het Engels.  Er zijn nog maar een paar veldslagen gestreden - weliswaar zonder duidelijke winnaar - maar de komende jaren worden we in elk geval onophoudelijk verder bestookt."  \\
  \textcolor{grey2}{--- Prof.dr.Dirk Van Compernolle (KU Leuven)}
}

% Funding notice left column
\FundingLNotice{\selectlanguage{dutch}
  De auteurs van dit document bedanken de auteurs van het taalwitboek voor het Duits  \cite{lwp-german} voor de toestemming om geselecteerd taalonafhankelijk materiaal uit hun witboek hier te hergebruiken.
  Verder wil de auteur Catia Cucchiarini (Nederlandse Taalunie), Walter Daelemans (Universiteit Antwerpen), Alice Dijkstra (NWO), Jean-Pierre Martens (Universiteit Gent), Jacomine Nortier (Universiteit Utrecht), Peter Spyns (Nederlandse Taalunie) en Remco van Veenendaal (TST-centrale) bedanken voor hun bijdragen aan het witboek.
  \bigskip

 De ontwikkeling van dit witboek is gefinancierd door het Zevende Kaderprogramma en het  ondersteuningsprogramma voor ICT-beleid van de Europese Commissie onder de contracten T4ME (Toewijzingsovereenkomst 249119), CESAR (Toewijzingsovereenkomst 271022), META-NET4U (Toewijzingsovereenkomst 270893) en META-NORD (Toewijzingsovereenkomst 270899).}

% Funding notice right column
\FundingRNotice{\selectlanguage{english}
  The authors are grateful to the authors of the White Paper on German \cite{lwp-german} for permission to re-use selected language-independent materials from their document.
  Furthermore, the author would like to thank Catia Cucchiarini (Dutch Language Union), Walter Daelemans (Antwerp University), Alice Dijkstra (NWO), Jean-Pierre Martens (Ghent University), Jacomine Nortier (Utrecht University), Peter Spyns (Dutch Language Union) and Remco van Veenendaal (HLT Agency) for their contributions to this white paper.
  \bigskip   \bigskip

  The development of this white paper has been funded by the Seventh
  Framework Programme and the ICT Policy Support Programme of the
  European Commission under the contracts T4ME (Grant Agreement
  249119), CESAR (Grant Agreement 271022), METANET4U (Grant Agreement
  270893) and META-NORD (Grant Agreement 270899).}
  
\hyphenation{al-go-rit-mi-sche ana-ly-se an-der-zijds ant-woord ant-woor-den Ask-Now au-to-ma-tic au-to-ma-ti-se-ren A-xen-do au-teurs-on-der-steu-nings-sy-ste-men be-dien-den be-doe-ling be-drijfs-do-mei-nen  be-drijfs-le-ven be-drijfs-spe-ci-fie-ke be-drij-ven be-grij-pen be-kij-ken be-lang-rij-ke be-na-de-ring be-oor-de-len be-schik-baar-heid  be-schrij-ven be-ta-ling be-trek-ke-lij-ke be-voor-deeld bij-ko-men-de bij-na bui-ten-ge-slo-ten com-mer-ciële com-mu-ni-ca-tie-no-den com-ple-xi-teit com-pu-ters con-fe-ren-ties copy-right-schen-din-gen crowd-sour-cing daad-wer-ke-lijk des-am-bi-gu-e-ring di-a-kri-ti-sche dichtst-bij-zijnde dien-sten Di-gi-ta-le di-gi-ta-le do-cu-men-ten do-mein-mo-del-le-ring do-mi-nan-te dy-na-mi-sche eind-ge-brui-kers-toe-pas-sin-gen e-lec-tro-ni-ca e-lek-tro-ni-ca ele-men-ten e-ner-zijds e-qui-va-lent erf-goed e-va-lu-a-tie fei-te-lij-ke fi-nan-cie-rings-toe-zeg-gin-gen Frank-rijk func-tio-na-li-teit ge-brui-kers-ac-cep-ta-tie ge-brui-kers-er-va-ring ge-brui-kers-vrien-de-lij-ke ge-meen-schap-pe-lijke ge-net-werk-te Ge-noot-schap  ge-or-ga-ni-seerd ge-pu-bli-ceerd ge-se-lec-teerd ge-stan-daar-di-seerd ge-stuur-de ge-za-men-lij-ke gram-ma-ti-ca-le hier-on-der Hier-on-der hoe-veel-heid in-ge-ni-eurs-we-ten-schap-pen In-ter-net-do-mei-nen in-te-res-sant ken-nis-ma-na-ge-ment kern-roe-pas-sin-gen ko-nink-lij-ke Ko-nink-lij-ke kwa-li-teit le-vert Lu-xem-burg mak-ke-lij-ker me-de-werk-ers meer-ta-lig-heid  ME-TA-VI-SI-ON ME-TA-RE-SEARCH ME-TA-SHA-RE mo-del-le-ren moe-der-taal-spre-kers moei-lijk mo-ge-lijk-he-den na-tuur-lij-ke na-tuur-lijk-heid nau-we-lijks na-vi-ga-tie-sy-steem Ne-der-land Ne-der-lands Ne-der-land-se Nij-me-gen nog-al nood-za-ke-lij-ker nood-za-ke-lij-ker-wijs  on-der-zoeks-or-ga-ni-sa-ties on-der-steund on-der-steu-nen on-der-steu-nings-pro-gram-ma on-der-zoeks-or-ga-ni-sa-tie ont-bre-kend ont-wik-ke-laars ont-wik-keld ont-wik-kel-de ont-wik-ke-ling oor-spron-ke-lij-ke or-ga-ni-sa-ties ou-de-ren per-soon-lij-ke pho-nes pu-blieke re-le-van-te rij-zen se-man-ti-sche spe-ci-fie-ke spre-ker-on-af-han-ke-lij-ke spre-kers sta-tis-ti-sche stra-te-gi-sche strij-den sym-bo-li-sche sy-ste-men taal-mo-del-len Taal-unie taal-tech-no-lo-gi-sche taal-ver-wer-king ta-bel-len tech-no-lo-gi-sche te-ge-lij-ker-tijd te-gen-woor-dig tekst-ana-ly-se teks-ten tekst-ver-wer-kers par-sing toe-pas-sin-gen toe-pas-sings-om-ge-ving toe-pas-sings-ge-bie-den toe-pas-sings-ka-der toe-pas-sings-ont-wik-ke-laars toe-wij-zings-over-een-komst     Toe-wij-zings-over-een-komst trans-pa-ran-ten uit-ge-voerd uit-vin-ding uit-zon-de-ring ver-an-de-rin-gen  ver-ba-zing-wek-kend ver-e-nigd ver-ge-lij-ken ver-ge-lij-king ver-te-gen-woor-di-ger ver-wer-ken ver-wer-king ver-wer-ven  ver-ze-ke-ren vir-tu-ele vi-sie-ont-wik-ke-lings-pro-ces Vlaan-de-ren voor-deel voor-uit-gang web-in-houd wer-kers werk-woord-con-struc-ties werk-woor-den woor-den woor-den-boek-af-dek-king werk-woor-denwoord-groe-pen woord-volg-or-de zaak-voe-ring zoek-mo-ge-lijk-he-den}
 