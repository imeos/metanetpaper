%                                     MMMMMMMMM                                         
%                                                                             
%  MMO    MM   MMMMMM  MMMMMMM   MM    MMMMMMMM   MMD   MM  MMMMMMM MMMMMMM   
%  MMM   MMM   MM        MM     ?MMM              MMM$  MM  MM         MM     
%  MMMM 7MMM   MM        MM     MM8M    MMMMMMM   MMMMD MM  MM         MM     
%  MM MMMMMM   MMMMMM    MM    MM  MM             MM MMDMM  MMMMMM     MM     
%  MM  MM MM   MM        MM    MMMMMM             MM  MMMM  MM         MM     
%  MM     MM   MMMMMM    MM   MM    MM            MM   MMM  MMMMMMM    MM
%
%
%          - META-NET Language Whitepaper | Italian Metadata -
% 
% ----------------------------------------------------------------------------

\usepackage{polyglossia}
\setotherlanguages{italian,english}


\title{La Lingua Italiana nell'Era Digitale --- The Italian Language in the Digital Age}

% Title for the spine of the cover
\spineTitle{The Italian Language in the Digital Age --- La Lingua Italiana nell'Era Digitale}

\subtitle{White Paper Series --- Collana Libri Bianchi}

\author{
  Nicoletta Calzolari\\
  Bernardo Magnini\\
  Claudia Soria\\
  Manuela Speranza
}
\authoraffiliation{
  Nicoletta Calzolari~ {\small CNR-ILC}\\
  Bernardo Magnini~ {\small FBK}\\
  Claudia Soria~ {\small CNR-ILC}\\
  Manuela Speranza~ {\small FBK}
}
\editors{
  Georg Rehm, Hans Uszkoreit\\(curatori, \textcolor{grey1}{editors})
}

% Text in left column on backside of the cover
\SpineLText{\selectlanguage{english}%
  In everyday communication, Europe’s citizens, business partners and politicians are inevitably confronted with language barriers.  
  Language technology has the potential to overcome these barriers and to provide innovative interfaces to technologies and knowledge. 
  This white paper presents the state of language technology support for the German language. 
  It is part of a series that analyzes the available language resources and technologies for 31~European languages. 
  The analysis was carried out by META-NET, a Network of Excellence funded by the European Commission.
  META-NET consists of 54 research centres in 33 countries, who cooperate with stakeholders from economy, government agencies, research organisations, non-governmental organisations, language communities and European universities. 
  META-NET’s vision is high-quality language technology for all European languages. 
}



% Text in right column on backside of the cover
\SpineRText{\selectlanguage{italian}%
Nella comunicazione quotidiana, i cittadini europei, i partner commerciali e i politici si trovano inevitabilmente di fronte a delle barriere linguistiche. La tecnologia linguistica ha il potenziale per superare queste barriere e fornire delle interfacce innovative alle tecnologie e alla conoscenza. Questo Libro Bianco presenta lo stato del supporto alla tecnologia del linguaggio per la lingua italiana. Fa parte di una serie che analizza le risorse linguistiche e le tecnologie disponibili per 31 lingue europee. L'analisi è stata condotta da META-NET, una rete di eccellenza finanziata dalla Commissione Europea. META-NET è costituito da 54 centri di ricerca in 33 paesi, che collaborano con esponenti del mondo economico, agenzie governative, organizzazioni di ricerca, organizzazioni non governative, le comunità di lingua e università europee. La visione di META-NET è quella di raggiungere una tecnologia linguistica di alta qualità per tutte le lingue europee.
}

% Quotes from VIPs on backside of the cover
\quotes{
  Lorem ipsum dolor sit amet, consectetur adipisicing elit, sed do eiusmod tempor incididunt ut labore et dolore magna aliqua. Ut enim ad minim veniam, quis nostrud exercitation ullamco laboris nisi ut aliquip ex ea commodo consequat. Duis aute irure dolor in reprehenderit in voluptate velit esse cillum dolore eu fugiat nulla pariatur. Excepteur sint occaecat cupidatat non proident, sunt in culpa qui officia deserunt mollit anim id est laborum.
}

% Funding notice left column
\FundingLNotice{\selectlanguage{italian}\vskip2mm
Gli autori di questo documento sono grati agli autori del libro bianco
sulla lingua tedesca per il permesso di riutilizzare dal loro
documento materiali selezionati indipendenti dal linguaggio \cite{lwpgerman}.

\bigskip

  Questo Libro Bianco \`{e} stato finanziato dal Settimo Programma Quadro e dal Programma ICT Policy Support della Commissione Europea nell'ambito dei contratti T4ME (Grant Agreement 249119), CESAR (Grant Agreement 271022), METANET4U (Grant Agreement 270893) e META-NORD (Grant Agreement 270899).}

% Funding notice right column
\FundingRNotice{\selectlanguage{english}\vskip2mm
  The authors of this document
  are grateful to the authors of the White Paper on German for
  permission to re-use selected language-independent materials from
  their document \cite{lwpgerman}.
  
  \bigskip
  
  The development of this White Paper has been funded by the Seventh
  Framework Programme and the ICT Policy Support Programme of the
  European Commission under the contracts T4ME (Grant Agreement
  249119), CESAR (Grant Agreement 271022), METANET4U (Grant Agreement
  270893) and META-NORD (Grant Agreement 270899).}
