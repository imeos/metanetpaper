%                                     MMMMMMMMM
%
%  MMA    MM   MMMMMM  MMMMMMM   MM    MMMMMMMM   MMA   MM  MMMMMMM MMMMMMM
%  MMMA AMMM   MM        MM     MMMM              MMMM  MM  MM        MM
%  MM MMM MM   MMMMMM    MM    IM  MI   MMMMMMM   MM MMxMM  MMMMMM    MM
%  MM  M  MM   MM        MM   .MMMMMM.            MM  MMMM  MM        MM
%  MM     MM   MMMMMM    MM   MM    MM            MM   MMM  MMMMMMM   MM
%
%
%          - META-NET Language Whitepaper | Italian Metadata -
%
% ----------------------------------------------------------------------------

\usepackage{polyglossia}
\setotherlanguages{italian,english,icelandic}


\title{La Lingua Italiana nell'Era Digitale --- The Italian Language in the Digital Age}

% Title for the spine of the cover
\spineTitle{The Italian Language in the Digital Age --- La Lingua Italiana nell'Era Digitale}

\subtitle{White Paper Series --- Collana Libri Bianchi}

\author{
  Nicoletta Calzolari\\
  Bernardo Magnini\\
  Claudia Soria\\
  Manuela Speranza
}

\authoraffiliation{
  Nicoletta Calzolari~ {\small CNR-ILC}\\
  Bernardo Magnini~ {\small FBK}\\
  Claudia Soria~ {\small CNR-ILC}\\
  Manuela Speranza~ {\small FBK}
}

\editors{
  Georg Rehm, Hans Uszkoreit\\(curatori, \textcolor{grey1}{editors})
}


% Text in left column on backside of the cover
\SpineLText{\selectlanguage{english}% 
In everyday communication,
Europe’s citizens, business partners and politicians are inevitably
confronted with language barriers.  Language technology has the
potential to overcome these barriers and to provide innovative
interfaces to technologies and knowledge.  This white paper presents
the state of language technology support for the Italian language.  It
is part of a series that analyses the available language resources and
technologies for 30~European languages.  The analysis was carried out
by META-NET, a Network of Excellence funded by the European
Commission.  META-NET consists of 54 research centres in 33 countries,
who cooperate with stakeholders from economy, government agencies,
research organisations and others.  META-NET’s vision is high-quality
language technology for all European languages.}

\SpineRText{\selectlanguage{italian}%
Nella comunicazione quotidiana, i cittadini europei, i partner
commerciali e i politici si trovano inevitabilmente di fronte a delle
barriere linguistiche. La tecnologia linguistica ha il potenziale per
superare queste barriere e fornire delle interfacce innovative alle
tecnologie e alla conoscenza. Questo Libro Bianco presenta lo stato
del supporto alla tecnologia del linguaggio per la lingua italiana. 
Fa parte di una serie che analizza le risore linguistiche e le
tecnologie disponibili per 30~lingue europee. L'analisi è stata
condotta da META-NET, una rete di eccellenza finanziata dalla
Commisione Europea. META-NET è costituito da 54 centri di ricerca in
33 paesi, che collaborano con esponenti del mondo economico, agenzie
governative, organizzazioni di ricerca e altri. 
La visione di META-NET è quella di raggiungere una tecnologia
linguistica di alta qualità per tutte le lingue europee.}

%. Fa parte di una serie che analizza le risorse linguistiche e le tecnologie disponibili per 30 lingue europee. L'analisi \`{e} stata condotta da META-NET, una rete di eccellenza finanziata dalla Commissione Europea. META-NET \`{e} costituito da 54 centri di ricerca in 33 paesi, che collaborano con esponenti del mondo economico, agenzie governative, organizzazioni di ricerca, organizzazioni non governative, le comunit\`{a}  di lingua e universit\`{a} europee. La visione di META-NET \`{e} quella di raggiungere una tecnologia linguistica di alta qualit\`{a}  per tutte le lingue europee.
%}
\vspace{-5mm}
% Quotes from VIPs on backside of the cover
\quotes{
\selectlanguage{italian}``%Nel mondo della cultura umanistica e del
%giornalismo si discute tanto, a volte anche impropriamente, sulla necessit di difendere l'italiano da anglismi spesso inevitabili e dai
%pericoli derivanti dall'uso di nuove tecnologie.
%, come gli sms. 
E non ci si rende abbastanza conto che, se in Italia non verranno sviluppate
le ricerche sulle tecnologie sulla lingua -- soprattutto il Trattamento
Automatico del Linguaggio -- la lingua italiana \`{e} destinata a
diventare sempre pi\`{u} marginale, fin quasi a scomparire. Se questa
\`{e} la cattiva notizia, la buona notizia \`{e} che il TAL, nel
mondo della ricerca italiano, gode di molte attenzioni.''\\
%, come questo Libro Bianco dimostra.''\\
   \textcolor{grey2}{--- Prof. Giordano Bruno Guerri (Presidente, Fondazione Il Vittoriale degli Italiani)}\\[3mm]
  \selectlanguage{english}
%``''
%The field of humanities and journalism is
%   strongly debating, at times improperly, the need to defend Italian''
%   against often unavoidable Anglicisms and against the threat
%   deriving from the use of new technologies, such as SMSs. 
%The community doesn't
%realize that the Italian language is destined to become ever more
%marginal, and finally disappear if research in new language technologies - in particular,
%research in Natural Language Processing -  is not pursued. This is the
%bad news. The good news is that Italian research in Natural Language Processing enjoys
%considerable attention, as this White Paper shows.
%''\\
%   \textcolor{grey2}{--- Professor Giordano Bruno Guerri (President, Fondazione Il Vittoriale degli Italiani)}\\[3mm]
}

% Funding notice left column
\FundingLNotice{\selectlanguage{italian}\vskip2mm
Gli autori di questo documento sono grati agli autori del Libro Bianco sulla lingua tedesca per aver consentito di riutilizzare alcuni materiali selezionati dal loro documento \cite{lwpgerman}.

\bigskip

\begin{spacing}{1.2}
  Questo Libro Bianco \`{e} stato finanziato dal Settimo Programma
  Quadro e dal Programma di sostegno alla politica in materia di TIC
  (tecnologie dell'informazione e delle comunicazioni) della
  Commissione Europea nell'ambito dei contratti T4ME (accordo di
  finanziamento 249\,119), CESAR (accordo di finanziamento 271\,022),
  METANET4U (accordo di finanziamento 270\,893) e META-NORD (accordo
  di finanziamento 270\,899).
\end{spacing}}

% Funding notice right column
\FundingRNotice{\selectlanguage{english}\vskip2mm
  The authors of this document
  are grateful to the authors of the White Paper on German for
  permission to re-use selected language-independent materials from
  their document \cite{lwpgerman}.
  
  \bigskip
\begin{spacing}{1.2}
  The development of this White Paper has been funded by the Seventh
  Framework Programme and the ICT Policy Support Programme of the
  European Commission under the contracts T4ME (Grant Agreement
  249\,119), CESAR (Grant Agreement 271\,022), METANET4U (Grant Agreement
  270\,893) and META-NORD (Grant Agreement 270\,899).
\end{spacing}}

