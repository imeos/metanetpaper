%                                     MMMMMMMMM        
%                                                                             
%  MMO    MM   MMMMMM  MMMMMMM   MM    MMMMMMMM   MMD   MM  MMMMMMM MMMMMMM   
%  MMM   MMM   MM        MM     ?MMM              MMM$  MM  MM         MM     
%  MMMM 7MMM   MM        MM     MM8M    MMMMMMM   MMMMD MM  MM         MM     
%  MM MMMMMM   MMMMMM    MM    MM  MM             MM MMDMM  MMMMMM     MM     
%  MM  MM MM   MM        MM    MMMMMM             MM  MMMM  MM         MM     
%  MM     MM   MMMMMM    MM   MM    MM            MM   MMM  MMMMMMM    MM
%
%
%          - META-NET Language Whitepaper | Greek Metadata -
% 
% ----------------------------------------------------------------------------

\usepackage{polyglossia}
\setotherlanguages{english,greek}


\title{Η Ελληνικη Γλωσσα στην Ψηφιακη Εποχη --- The Greek Language in the Digital Age}

\spineTitle{The Greek Language in the Digital Age --- Η Ελληνικη Γλωσσα στην Ψηφιακη Εποχη}

\subtitle{White Paper Series --- Σειρά Λευκών Βίβλων}

\author{
  Maria Gavrilidou\\
  Maria Koutsombogera\\
  Anastasios Patrikakos\\
  Stelios Piperidis
}

\authoraffiliation{
  Maria Gavrilidou~ {\small R.\,C.~“Athena”, ILSP}\\
  Maria Koutsombogera~ {\small R.\,C.~“Athena”, ILSP}\\
  Anastasios Patrikakos~ {\small R.\,C.~“Athena”} \\
  Stelios Piperidis~ {\small R.\,C.~“Athena”, ILSP}
}

\editors{
  Georg Rehm, Hans Uszkoreit\\(επιμελητές, \textcolor{grey1}{editors})
}

% Text in left column on backside of the cover
\SpineLText{\selectlanguage{english}%
  \begin{spacing}{1}
  In everyday communication, Europe’s citizens, business partners and politicians are inevitably confronted with language barriers.  
  Language technology has the potential to overcome these barriers and to provide innovative interfaces to technologies and knowledge. 
  This white paper presents the state of language technology support for the Greek language. 
  It is part of a series that analyses the available language resources and technologies for 30~European languages. 
  The analysis was carried out by META-NET, a Network of Excellence funded by the European Commission.
  META-NET consists of 54 research centres in 33 countries, who cooperate with stakeholders from economy, government agencies, research organisations, non-governmental organisations, language communities and European universities. 
  META-NET’s vision is high-quality language technology for all European languages. 
\end{spacing}
}

% Text in right column on backside of the cover
\SpineRText{\selectlanguage{greek}%
  \begin{spacing}{1}
Κατά την καθημερινή τους επικοινωνία οι ευρωπαίοι πολίτες, επιχειρηματίες και πολιτικοί έρχονται αντιμέτωποι με γλωσσικούς φραγμούς. Η γλωσσική τεχνολογία έχει τη δυνατότητα να ξεπεράσει αυτά τα εμπόδια παρέχοντας καινοτόμες διεπαφές τεχνολογίας και γνώσης. Η παρούσα Λευκή Βίβλος περιγράφει την υποστήριξη της γλωσσικής τεχνολογίας για τα Ελληνικά και αποτελεί μέρος μιας συλλογής που καταγράφει τη διαθεσιμότητα γλωσσικών πόρων και τεχνολογιών σε 30 ευρωπαϊκές γλώσσες. Την ανάλυση πραγματοποίησε το ΜΕΤΑ-ΝΕΤ, ένα Δίκτυο Αριστείας χρηματοδοτούμενο από την ΕΕ και αποτελούμενο από 54 ερευνητικά κέντρα σε 33 χώρες, τα οποία συνεργάζονται με οικονομικούς και κυβερνητικούς φορείς, ερευνητικούς οργανισμούς, μη κυβερνητικές οργανώσεις, γλωσσικές κοινότητες και ευρωπαϊκά πανεπιστήμια. Το όραμα του ΜΕΤΑ-ΝΕΤ είναι η ανάπτυξη γλωσσικής τεχνολογίας υψηλής ποιότητας για όλες τις ευρωπαϊκές γλώσσες.
\end{spacing}
}

% Quotes from VIPs on backside of the cover
\quotes{%
  \begin{spacing}{1}
``H ενίσχυση της γλωσσικής τεχνολογίας διασφαλίζει τη θέση της ελληνικής γλώσσας και του ελληνικού πολιτισμού στον ψηφιακό κόσμο, ενισχύοντας ταυτόχρονα τόσο την ανάπτυξη όσο και την επικοινωνία των πολιτών στην Κοινωνία της Πληροφορίας."\\[2mm]
\textcolor{grey2}{Γεώργιος Μπαμπινιώτης (Καθηγητής Γλωσσολογίας, Υπουργός Παιδείας, Δια Βίου Μάθησης και Θρησκευμάτων)}\\[5mm]
``Further support to language technologies safeguards the presence of Greek language and culture in the digital environment, while at the same time promoting development and fostering communication among citizens within the Information Society."\\[2mm]
\textcolor{grey2}{George Babiniotis (Prof.~of linguistics, Minister of Education, Lifelong Learning and Religious Affairs)}
\end{spacing}
}

% Funding notice left column
\FundingLNotice{\selectlanguage{greek} %
  Οι συντάκτες του κειμένου αυτού θα ήθελαν να ευχαριστήσουν τους συγγραφείς της γερμανικής Λευκής Βίβλου για την άδεια χρήσης επιλεγμένων εισαγωγικών χωρίων από το κείμενό τους \cite{lwpgerman}.

  \bigskip
  
  \begin{spacing}{1.2}
  Η κατάρτιση αυτής της Λευκής Βίβλου χρηματοδοτήθηκε από το 7ο Πρόγραμμα Πλαίσιο και το Πρόγραμμα “Υποστήριξη της Πολιτικής για τις ΤΠΕ” της Ευρωπαϊκής Επιτροπής, με τα Έργα Τ4ΜΕ (Αρ.~Σύμβασης: 249\,119), CESAR (Αρ.~Σύμβασης: 271\,022), METANET4U (Αρ.~Σύμβασης: 270\,893) και META-NORD (Αρ.~Σύμβασης: 270\,899).\end{spacing}%
}

\FundingRNotice{\selectlanguage{english}  %
  The authors of this document
  are grateful to the authors of the White Paper on German for
  permission to re-use selected language-independent materials from
  their document \cite{lwpgerman}.
  
  \bigskip
  
  \begin{spacing}{1.2}
  The development of this white paper has been funded by the Seventh
  Framework Programme and the ICT Policy Support Programme of the
  European Commission under the contracts T4ME (Grant Agreement
  249\,119), CESAR (Grant Agreement 271\,022), METANET4U (Grant Agreement
  270\,893) and META-NORD (Grant Agreement 270\,899).\end{spacing}
}
