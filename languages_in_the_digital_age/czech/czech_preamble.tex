%                                     MMMMMMMMM                                         
%                                                                             
%  MMO    MM   MMMMMM  MMMMMMM   MM    MMMMMMMM   MMD   MM  MMMMMMM MMMMMMM   
%  MMM   MMM   MM        MM     ?MMM              MMM$  MM  MM         MM     
%  MMMM 7MMM   MM        MM     MM8M    MMMMMMM   MMMMD MM  MM         MM     
%  MM MMMMMM   MMMMMM    MM    MM  MM             MM MMDMM  MMMMMM     MM     
%  MM  MM MM   MM        MM    MMMMMM             MM  MMMM  MM         MM     
%  MM     MM   MMMMMM    MM   MM    MM            MM   MMM  MMMMMMM    MM
%
%
%          - META-NET Language Whitepaper | Czech Metadata -
% 
% ----------------------------------------------------------------------------

\usepackage{polyglossia}
\setotherlanguages{czech,english}


\title{Čeština v digitálním věku --- The Czech Language in the Digital Age}

\subtitle{Série Bílé knihy --- White Paper Series} %done by BH

%Ondrej Bojar
%2 Silvie Cinkova
%3 Jan Hajic
%4 Barbora Hladka
%5 Vladislav Kubon
%6 Jiri Mirovsky
%7 Jarmila Panevova
%8 Nino Peterek
%9 Johanka Spoustova 

\author{
  Silvie Cinková \\
  Jan Hajič  \\
  Barbora Hladká  \\
  Vladislav Kuboň  \\
  Jiří Mírovský	 \\
  Jarmila Panevová \\
  Nino Peterek  \\
  Johanka Spoustová  \\
  Zdeněk Žabokrtský  \\
}

\authoraffiliation{
  Silvie Cinková~ {\small Charles University in Prague} \\
  Jan Hajič~ {\small Charles University in Prague}  \\
  Barbora Hladká~ {\small Charles University in Prague}  \\
  Vladislav Kuboň~ {\small Charles University in Prague}  \\
  Jiří Mírovský~ {\small Charles University in Prague}	 \\
  Jarmila Panevová~ {\small Charles University in Prague} \\
  Nino Peterek~ {\small Charles University in Prague}  \\
  Johanka Spoustová~{\small Charles~University~in~Prague}  \\
  Zdeněk~Žabokrtský~{\small Charles~University~in~Prague}  \\
 }

\editors{
Georg Rehm, Hans Uszkoreit\\(editoři, \textcolor{grey1}{editors}) %done by BH
}

% Text in left column on backside of the cover
\SpineLText{\selectlanguage{english}%
  In everyday communication, Europe’s citizens, business partners and politicians are inevitably confronted with language barriers.  
  Language technology has the potential to overcome these barriers and to provide innovative interfaces to technologies and knowledge. 
  This white paper presents the state of language technology support for the German language. 
  It is part of a series that analyzes the available language resources and technologies for 31~European languages. 
  The analysis was carried out by META-NET, a Network of Excellence funded by the European Commission.
  META-NET consists of 54 research centres in 33 countries, who cooperate with stakeholders from economy, government agencies, research organisations, non-governmental organisations, language communities and European universities. 
  META-NET’s vision is high-quality language technology for all European languages. 
}

% Text in right column on backside of the cover
\SpineRText{\selectlanguage{czech}%
  Obyvatelé Evropy, obchodní partneři a politici čelí v každodenní komunikaci jazykovým bariérám. Jazykové technologie mají potenciál tyto bariéry překonat a poskytnout rozhraní k technologiím a znalostem. Tato Bílá kniha prezentuje stav podpory jazykových technologií pro češtinu. Je částí série, která analyzuje dostupné jazykové zdroje pro 31 evropských jazyků. Analýzu provedl META-NET, síť excelence financovaná Evropskou komisí. META-NET soutřeďuje 54 výzkumných center ve 33 zemích, které spolupracují se zainteresovanými partnery z ekonomie, vládních agentur, výzkumných organizací, nevládních organizací, jazykových komunit a evropských univerzit. META-NET představuje vizi vysoce kvalitních jazykových technologií pro všechny evropské jazyky.}

% Quotes from VIPs on backside of the cover
\quotes{%
 The META-NET project brings a significant contribution to the technological support for languages of Europe and as such
will play an indispensable role in the development of multilingual
European culture and society. \\
  \textcolor{grey2}{--- Professor Ing. Ivan Wilhelm, CSc., Dr.h.c. mult. (Deputy minister for education, youth and sport, former rector of Charles University in Prague)}\\[3mm]  
Jazykové technologie jsou bezpochyby neoddělitelnou složkou moderního výzkumu v oblasti informačních věd. Technologie mají zásadní vliv na jejich budoucí rozvoj jak v základním výzkumu, tak i v aplikacích. \\
  \textcolor{grey2}{--- Prof. Ing. Vladimír Mařík, DrSc. (člen Rady pro výzkum, vývoj a inovace, odborného a poradního orgánu vlády České republiky, předseda výzkumné rady Technologické agentury ČR)}
}

% Funding notice left column
\FundingLNotice{\selectlanguage{czech}
  Autoři tohoto dokumentu děkují autorům Bílé knihy pro němčinu za povolení  použít vybrané jazykově nezávislé části z jejich dokumentu \cite{lwpgerman}. Zároveň děkujeme za milou spolupráci kolegům Evě Hajičové, Jirkovi Hanovi, Karlu Olivovi, Magdaleně Rysové, Magdě Ševčíkové, Ivanu Šmilauerovi a Danielu Zemanovi.\vfill
  \bigskip
  Práce na této Bílé knize byla financována 7. Rámcovým programem Evropské komise a Programem na podporu politiky informačních a komunikačních technologií (ICT Policy Support Programme of the European Commission) na základě smluv T4ME (grantové dohoda 249119), CESAR (grantová dohoda 271022), METANET4U (grantová dohoda 270893) a META-NORD (grantová dohoda 270899).}

% Funding notice right column
\FundingRNotice{\selectlanguage{english}
  The authors of this document are grateful to the authors of the white paper on German for permission to re-use selected language-independent materials from their document \cite{lwpgerman}. We also wish to thank our colleagues Eva Hajičová, Jirka Hana, Karel Oliva, Magdalena Rysová, Magda Ševčíková, Ivan Šmilauer, Daniel Zeman for their nice cooperation.\vfill
  \bigskip
  The development of this white paper has been funded by the Seventh
  Framework Programme and the ICT Policy Support Programme of the
  European Commission under the contracts T4ME (Grant Agreement
  249119), CESAR (Grant Agreement 271022), METANET4U (Grant Agreement
  270893) and META-NORD (Grant Agreement 270899).}
