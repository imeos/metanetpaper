%                                     MMMMMMMMM                                         
%                                                                             
%  MMO    MM   MMMMMM  MMMMMMM   MM    MMMMMMMM   MMD   MM  MMMMMMM MMMMMMM   
%  MMM   MMM   MM        MM     ?MMM              MMM$  MM  MM         MM     
%  MMMM 7MMM   MM        MM     MM8M    MMMMMMM   MMMMD MM  MM         MM     
%  MM MMMMMM   MMMMMM    MM    MM  MM             MM MMDMM  MMMMMM     MM     
%  MM  MM MM   MM        MM    MMMMMM             MM  MMMM  MM         MM     
%  MM     MM   MMMMMM    MM   MM    MM            MM   MMM  MMMMMMM    MM
%
%
%          - META-NET Language Whitepaper | Hungarian Metadata -
% 
% ----------------------------------------------------------------------------

\usepackage{polyglossia}
\usepackage{multicol,framed,lipsum}
\setotherlanguages{hungarian,english}

\hyphenation{kor-mány-za-ti pénz-ügyi ese-té-ben szer-keze-tű se-gít-het web-ol-da-lán leg-újabb leg-újabb szer-rint meg-ködiscretionaryny-í-té-sé-hez ta-nu-lást in-tui-tív szö-veg-szer-kesz-tő-ben szó-ra-koz-ta-tó-ipar-ba vá-la-szo-kat egy-faj-ta szö-ve-gek-re na-gyobb nyelv-ta-nu-lás-ra al-kal-ma-zást rend-sze-rek ma-gyar Hor-vát-or-szág-ban dia-lek-tus ma-gya-rok bir-tok-vi-szonyt Mi-nisz-té-rium la-ti-nul ma-gyar-or-szá-gi Ins-ti-tut ha-gyo-má-nyok Habs-burg ta-nul-ni gon-doz-ni ma-gyar-or-szá-gi al-kal-ma-zá-sok úgy-ne-ve-zett al-kal-ma-zá-sa sze-re-pe di-gi-tá-li-san fo-gya-ték-kal Fo-gya-té-kos esz-kö-zök komp-lex rö-vi-den egy-sze-rű-sí-tett ese-té-ben meg-ke-re-sé-se mo-du-lok egy-sze-rű-sí-tett esz-kö-zök-re szö-veg-szer-kesz-tők-re nin-cse-nek le-ve-let ért-he-tő ma-gyar-ra sze-re-pét ma-gyar sze-rint fej-lesz-té-se rész-le-ges kulcs-sza-vas fi-nom-han-go-lá-sa szö-ve-ge-ket Táv-köz-lé-si gye-re-kek se-gít-het-nek ügy-fél-kap-cso-la-tok mo-del-lek-nek sza-bad prog-ra-mok-ban rend-sze-rek kény-sze-rí-ti kér-dés-meg-vá-la-szo-lás Hold-ra kulcs-sze-rep-lőit egya-ránt rend-sze-rek ke-ret-prog-ra-mok egya-ránt in-ter-ope-ra-bi-li-tást kor-pusz bo-nyo-lul-tabb adat-cse-ré-lő prog-ra-mok te-rü-let-re fel-adat elem-zé-sét ala-csony esz-kö-zök Lu-xem-bourg al-kal-ma-zá-sok ki-fej-lesz-té-sé-hez esz-kö-zök mód-sze-rek szer-vez in-teg-rál-ni know-ledge popu-larity partici-pation know-ledge popu-lation Fe-renc exist Eng-lish analo-gously Alf-réd}


\title{A magyar nyelv a digitális korban --- The Hungarian Language in the Digital Age}

% Title for the spine of the cover
\spineTitle{The Hungarian Language in the Digital Age --- A magyar nyelv a digitális korban}

\subtitle{White Paper Series --- Fehér könyvek sorozat}

\author{             
  Simon Eszter \\
  Lendvai Piroska \\
  Németh Géza \\
  Olaszy Gábor \\
  Vicsi Klára \\
}

\authoraffiliation{
  Simon Eszter~ {\small MTA Nyelvtudományi Intézet}\\
  Lendvai Piroska~ {\small MTA Nyelvtudományi Intézet}\\
  Németh Géza~ {\small BME} \\
  Olaszy Gábor~ {\small BME}\\
  Vicsi Klára~ {\small BME}\\
}

\editors{
  Simon Eszter, Lendvai Piroska\\(szerkesztők, \textcolor{grey1}{editors})
}

% Text in left column on backside of the cover
\SpineLText{\selectlanguage{english}%
  In everyday communication, Europe’s citizens, business partners and politicians are inevitably confronted with language barriers. Language technology has the potential to overcome these barriers and to provide innovative interfaces to technologies and knowledge. This white paper presents the state of language technology support for the hungarian language. It is part of a series that analyses the available language resources and technologies for 31 European languages. The analysis was carried out by META-NET, a Network of Excellence funded by the European Commission. META-NET consists of 54 research centres in 33 countries, who cooperate with stakeholders from economy, government agencies, research organisations, non-governmental organisations, language communities and European universities. META-NET’s vision is high-quality language technology for all European languages. 
}

% Text in right column on backside of the cover
\SpineRText{\selectlanguage{hungarian}%
  A mindennapi kommunikáció Európa polgárai, mind az üzleti, mind a politikai szférában elkerülhetetlenül nyelvi akadályokba ütközik. A nyelvtechnológia hozzá tud járulni ezen akadályok legyőzéséhez, továbbá  kapcsolódási pontokat nyújt az innovatív technológiák és tudás felé. Ez a fehér könyv a magyar nyelvtechnológia helyzetét mutatja be, egyben egy sorozat részét képezi, amely az elérhető nyelvi erőforrásokról és technológiákról ad elemzést 31 európai nyelvre. A felmérést a META-NET, az Európai Bizottság által alapított hálózat végezte. A META-NET 33 ország 54 kutatóközpontjából áll, akik gazdasági döntéshozókkal, kormányzati szervekkel, kutatószervezetekkel, nyelvi közösségekkel és európai egyetemekkel dolgoznak együtt. A META-NET jövőképe: kiváló minőségű nyelvtechnológia minden európai nyelvre.
}

% Quotes from VIPs on backside of the cover
\quotes{%  
  \selectlanguage{english}{META-NET is making a significant contribution to innovation, research and development in Europe and to an effective implementation of the European idea.}\\
  \textcolor{grey2}{--- Valéria Csépe, Deputy General Secretary of Hungarian Academy of Sciences}\\[3mm]
  \selectlanguage{hungarian}{A META-NET jelentős mértékben hozzájárul az innovációhoz és a kutatás-fejlesztéshez, valamint az európai eszme hatékony megvalósításához.}\\
  \textcolor{grey2}{--- Csépe Valéria, főtitkárhelyettes, MTA}
}

% Funding notice left column
\FundingLNotice{\selectlanguage{hungarian}\vskip2mm
  A dokumentum szerzői köszönettel tartoznak a német fehér könyv szerzőinek azért, hogy engedélyezték a német változat egyes nyelvfüggetlen részeinek újrafelhasználását.
  
  \bigskip
  
  A fehér könyv megírását az Európai Bizottság 7. keretprogramja és ICT PSP programja támogatta a T4ME (szerződésszám: 249119), a CESAR (szerződésszám: 271022), a METANET4U (szerződésszám: 270893) és a META-NORD (szerződésszám: 270899) projekteken keresztül.}

% Funding notice right column
\FundingRNotice{\selectlanguage{english}\vskip2mm
  The authors of this document are grateful to the authors of the white paper on German for permission to re-use selected language-independent materials from their document. \cite{lwpgerman} 
  
  \bigskip
  
  The development of this white paper has been funded by the Seventh Framework Programme and the ICT Policy Support Programme of the European Commission under the contracts T4ME (Grant Agreement 249119), CESAR (Grant Agreement 271022), META\-NET4U (Grant Agreement 270893) and META-NORD (Grant Agreement 270899).}
