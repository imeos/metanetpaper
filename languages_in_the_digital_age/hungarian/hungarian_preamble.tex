%                                     MMMMMMMMM                                         
%                                                                             
%  MMO    MM   MMMMMM  MMMMMMM   MM    MMMMMMMM   MMD   MM  MMMMMMM MMMMMMM   
%  MMM   MMM   MM        MM     ?MMM              MMM$  MM  MM         MM     
%  MMMM 7MMM   MM        MM     MM8M    MMMMMMM   MMMMD MM  MM         MM     
%  MM MMMMMM   MMMMMM    MM    MM  MM             MM MMDMM  MMMMMM     MM     
%  MM  MM MM   MM        MM    MMMMMM             MM  MMMM  MM         MM     
%  MM     MM   MMMMMM    MM   MM    MM            MM   MMM  MMMMMMM    MM
%
%
%          - META-NET Language Whitepaper | Hungarian Metadata -
% 
% ----------------------------------------------------------------------------

\usepackage{polyglossia}
\usepackage{multicol,framed,lipsum}
\setotherlanguages{hungarian,english}


\title{A magyar nyelv a digitális korban --- The Hungarian Language in the Digital Age}

\subtitle{White Paper Series --- Fehér könyvek sorozat}

\author{             
  Simon Eszter \\
  Lendvai Piroska \\
  Németh Géza \\
  Olaszy Gábor \\
  Vicsi Klára \\
}

\authoraffiliation{
  Simon Eszter~ {\small MTA Nyelvtudományi Intézet}\\
  Lendvai Piroska~ {\small MTA Nyelvtudományi Intézet}\\
  Németh Géza~ {\small BME} \\
  Olaszy Gábor~ {\small BME}\\
  Vicsi Klára~ {\small BME}\\
}

\editors{
  Simon Eszter, Lendvai Piroska\\(szerkesztők, \textcolor{grey1}{editors})
}


\SpineLText{\selectlanguage{english}%
  In everyday communication, Europe’s citizens, business partners and politicians are inevitably confronted with language barriers.  
  Language technology has the potential to overcome these barriers and to provide innovative interfaces to technologies and knowledge. 
  This white paper presents the state of language technology support for the German language. 
  It is part of a series that analyzes the available language resources and technologies for 31~European languages. 
  The analysis was carried out by META-NET, a Network of Excellence funded by the European Commission.
  META-NET consists of 54 research centres in 33 countries, who cooperate with stakeholders from economy, government agencies, research organisations, non-governmental organisations, language communities and European universities. 
  META-NET’s vision is high-quality language technology for all European languages. 
}

\SpineRText{\selectlanguage{hungarian}%
  Im kommunikativen Miteinander stoßen Europas Bürger, die europäische
  Wirtschaft und auch die Politik schnell an sprachliche
  Grenzen. Moderne Sprachtechnologie besitzt das entscheidende
  Potenzial, Sprachgrenzen zu überwinden und innovative Schnittstellen
  zu Technologien und Wissen zu ermöglichen. Dieses Weißbuch stellt
  den Stand der sprachtechnologischen Unterstützung für die deutsche
  Sprache dar; es gehört zu einer Serie, die die vorhandenen
  Ressourcen und -technologien für 31~europäische Sprachen analysiert.
  % 
  Die Analyse wurde von META-NET erstellt, einem von der Europäischen
  Kommission geförderten Spitzenforschungsnetzwerk. META-NET besteht
  aus 54 Forschungszentren in 33 Ländern, die mit
  Interessensvertretern aus Wirtschaft, Verwaltung, NGOs,
  Sprachgemeinschaften und europäischen Universitäten
  zusammenarbeiten. Die Vision von META-NET ist hochqualitative
  Sprachtechnologie für alle Sprachen Europas.}

\quotes{%
  Excepteur sint occaecat cupidatat non proident, sunt in culpa qui officia deserunt mollit anim id est laborum. Lorem ipsum dolor sit amet, consectetur adipisicing elit, sed do eiusmod tempor labore et dolore magna aliqua. \\
  \textcolor{grey2}{--- Prof. Dr. John Doe (Member of the European Parliament and VIP)}\\[3mm]
  Excepteur sint occaecat cupidatat non proident, sunt in culpa qui officia deserunt mollit anim id est laborum. Lorem ipsum dolor sit amet, consectetur adipisicing elit, sed do eiusmod tempor labore et dolore magna aliqua. \\
  \textcolor{grey2}{--- Dr. Jane Doe (Member of the European Parliament and VIP)}
}

\FundingLNotice{\selectlanguage{hungarian}\vskip2mm
  Lorem ipsum dolor sit amet, consectetur adipisicing elit, sed do eiusmod tempor incididunt ut labore et dolore magna aliqua. Ut enim ad minim veniam, quis nostrud exercitation ullamco laboris nisi ut aliquip ex ea commodo.
  \bigskip
  A fehér könyv megírását az Európai Bizottság 7. keretprogramja és ICT PSP programja támogatta a T4ME (szerződésszám: 249119), a CESAR (szerződésszám: 271022), a METANET4U (szerződésszám: 270893) és a META-NORD (szerződésszám: 270899) projekteken keresztül.}

\FundingRNotice{\selectlanguage{english}
  The authors of this document are grateful to the authors of the White Paper on German for permission to re-use selected language-independent materials from their document. \cite{lwpgerman} 
  \bigskip
  The development of this white paper has been funded by the Seventh Framework Programme and the ICT Policy Support Programme of the European Commission under the contracts T4ME (Grant Agreement 249119), CESAR (Grant Agreement 271022), META\-NET4U (Grant Agreement 270893) and META-NORD (Grant Agreement 270899).}
