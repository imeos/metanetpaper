%                                     MMMMMMMMM                                         
%                                                                             
%  MMO    MM   MMMMMM  MMMMMMM   MM    MMMMMMMM   MMD   MM  MMMMMMM MMMMMMM   
%  MMM   MMM   MM        MM     ?MMM              MMM$  MM  MM         MM     
%  MMMM 7MMM   MM        MM     MM8M    MMMMMMM   MMMMD MM  MM         MM     
%  MM MMMMMM   MMMMMM    MM    MM  MM             MM MMDMM  MMMMMM     MM     
%  MM  MM MM   MM        MM    MMMMMM             MM  MMMM  MM         MM     
%  MM     MM   MMMMMM    MM   MM    MM            MM   MMM  MMMMMMM    MM
%
%
%            - META-NET Language Whitepaper | German paper -

\documentclass[]{../../metanetpaper}

\usepackage{covington}
\usepackage{booktabs}
\usepackage{longtable}
\usepackage{tabulary}
\usepackage{tabularx}
\usepackage{rotating}
\usepackage{makecell}
\usepackage{multirow}
\usepackage{colortbl}
\usepackage{polyglossia}
\setotherlanguages{german, english}

%!TEX TS-program = xelatex
\RequireXeTeX %Force XeTeX check

\title{Deutsch im Digitalen Zeitalter --- The German Language in the Digital Age}

\subtitle{White Paper Series --- Weißbuch-Serie}

\author{
  Aljoscha Burchardt \\
  Markus Egg \\
  Kathrin Eichler \\
  Brigitte Krenn \\
  Jörn Kreutel \\
  Annette Leßmöllmann \\
  Georg Rehm \\
  Manfred Stede \\
  Hans Uszkoreit \\
  Martin Volk
}
\authoraffiliation{
  Aljoscha Burchardt~ {\small DFKI}\\
  Markus Egg~ {\small Humboldt-Universität zu Berlin}\\
  Kathrin Eichler~ {\small DFKI} \\
  Brigitte Krenn~ {\small ÖFAI}\\
  Jörn Kreutel~ {\small FH Brandenburg}\\
  Annette Leßmöllmann~ {\small Hochschule Darmstadt}\\
  Georg Rehm~ {\small DFKI} \\
  Manfred Stede~ {\small Universität Potsdam}\\
  Hans Uszkoreit~ {\small Universität des Saarlandes, DFKI} \\
  Martin Volk~ {\small Universität Zürich}
}
\editors{
  Georg Rehm, Hans Uszkoreit (Hrsg., \textcolor{grey1}{eds.})
}

\begin{document}

\renewcommand*{\figureformat}{\sffamily\thefigure\autodot}

\maketitle
\onecolumn
%                                     MMMMMMMMM                                         
%                                                                             
%  MMO    MM   MMMMMM  MMMMMMM   MM    MMMMMMMM   MMD   MM  MMMMMMM MMMMMMM   
%  MMM   MMM   MM        MM     ?MMM              MMM$  MM  MM         MM     
%  MMMM 7MMM   MM        MM     MM8M    MMMMMMM   MMMMD MM  MM         MM     
%  MM MMMMMM   MMMMMM    MM    MM  MM             MM MMDMM  MMMMMM     MM     
%  MM  MM MM   MM        MM    MMMMMM             MM  MMMM  MM         MM     
%  MM     MM   MMMMMM    MM   MM    MM            MM   MMM  MMMMMMM    MM
%
%
%        - META-NET Language Whitepaper | German mocktitle -


\null
\vfill

\pagestyle{empty} 

{\small
  Die Ausarbeitung dieses Weißbuchs wurde mit Mitteln aus dem Siebten Rahmenprogramm und dem Programm zur Unter-stützung der Politik für Informations- und Kommunikationstechnologien der Europäischen Kommission im Rahmen der Verträge T4ME (Finanzhilfevereinbarung 249119), CESAR (Finanzhilfevereinbarung 271022), METANET4U (Finanzhilfevereinbarung 270893) und META-NORD (Finanzhilfevereinbarung 270899) finanziert.\\

  The development of this white paper has been funded by the Seventh Framework Programme and the ICT Policy Support Programme of the European Commission under contracts T4ME (Grant Agreement 249119), CESAR (Grant Agreement 271022), METANET4U (Grant Agreement 270893) and META-NORD (Grant Agreement 270899).
}

\clearpage


\pagenumbering{Roman} 
\setcounter{page}{5}
\pagestyle{scrheadings}

% --------------------------------------------------------------------------

\bsection{Vorwort --- Preface}
\begin{Parallel}[c]{79mm}{78mm}
\ParallelLText{
Dieses Weißbuch gehört zu einer Serie, die Wissen über Sprachtechnologie und deren Potenzial vermitteln soll, und sich insbesondere an Journalisten, Politiker, Sprachgemeinschaften und Lehrende richtet.

Die derzeitige Verfügbarkeit und Nutzung von Sprachtechnologie in Europa variiert stark je nach Sprache. Folglich müssen auch die notwendigen Maßnahmen für die weitere Unterstützung von Forschung und Entwicklung in diesem Bereich variieren. Die Maßnahmen hängen von zahlreichen Faktoren ab, z.\,B.~der Komplexität einer Sprache und der Anzahl ihrer Sprecher.

META-NET, ein von der Europäischen Kommission gefördertes Exzellenznetzwerk, hat die aktuellen Sprachressourcen und -technologien in der vorliegenden Weiß\-buch-Serie analysiert (siehe S.~\pageref{whitepaperseries}). Die Analyse umfasst die 23 europäischen Amtssprachen sowie weitere wichtige nationale und regionale Sprachen Europas. Die Ergebnisse zeigen, dass es bei allen Sprachen beträchtliche Defizite in der technologischen Unterstützung und signifikante Forschungslücken gibt. Die vorliegende ausführliche Expertenanalyse und Bewertung der aktuellen Situation dient dazu, die Wirksamkeit weiterer Forschungstätigkeit zu maximieren.

META-NET besteht aus 54 Forschungszentren aus 33 Ländern (Stand: November 2011, siehe S.~\pageref{metanetmembers}), die mit Interessensvertretern aus Wirtschaft (Softwareunternehmen, Technologieanbietern, Nutzer), Verwaltung, NGOs, Sprachgemeinschaften und europäischen Universitäten zusammenarbeiten. META-NET entwickelt gemeinsam mit diesen Gruppen eine übergreifende Technologievision und eine strategische Forschungsagenda für das mehrsprachige Europa 2020.}

\ParallelRText{This white paper is part of a series that promotes knowledge about language technology and its potential. It addresses journalists, politicians, language communities, educators and others. 

The availability and use of language technology in Europe varies between languages. Consequently, the actions that are required to further support research and development of language technologies also differs. The required actions depend on many factors, such as the complexity of a given language and the size of its community.

META-NET, a Network of Excellence funded by the European Commission, has conducted an analysis of current language resources and technologies in this white paper series (p.~\pageref{whitepaperseries}). The analysis focused on the 23 official European languages as well as other important national and regional languages in Europe. The results of this analysis suggest that there are tremendous deficits in technology support and significant research gaps for each language. The given detailed expert analysis and assessment of the current situation will help maximise the impact of additional research.

As of November 2011, META-NET consists of 54 research centres from 33 European countries (p.~\pageref{metanetmembers}). META-NET is working with stakeholders from economy (Software companies, technology providers, users), government agencies, research organisations, non-governmental organisations, language communities and European universities. Together with these communities, META-NET is creating a common technology vision and strategic research agenda for multilingual Europe 2020.}   \ParallelPar
\end{Parallel}

% --------------------------------------------------------------------------
\cleardoublepage

\tableofcontents
%\addtocontents{toc}{\protect\thispagestyle{empty}}

\cleardoublepage

\twocolumn

\setcounter{page}{1}
\pagenumbering{arabic} 
\pagestyle{scrheadings}


% --------------------------------------------------------------------------
\twocolumn[\ssection{Zusammenfassung}]

Europa ist in den vergangenen 60 Jahren eine politisch-wirtschaftliche Einheit geworden. Kulturell und sprachlich ist der europäische Raum reich und vielfältig. Von Portugiesisch bis Polnisch, von Italienisch bis Isländisch -- im kommunikativen Miteinander stoßen Europas Bürger, die europäische Wirtschaft und auch die Politik schnell an sprachliche Grenzen. Die Institutionen der EU geben jährlich rund eine Milliarde Euro für die Übersetzung von Texten und das Dolmetschen gesprochener Kommunikation aus, um am Grundsatz der Mehrsprachigkeit festzuhalten. Ist dieser Aufwand tatsächlich notwendig? Moderne Sprachtechnologie und Sprachforschung können einen wichtigen Beitrag leisten, um die Sprachgrenzen zu überwinden. Wenn Europäer sich miteinander unterhalten oder Verträge schließen wollen, kann moderne Sprachtechnologie in Verbindung mit intelligenten Endgeräten und Anwendungen ihnen hierbei künftig helfen -- selbst wenn sie keine gemeinsame Sprache sprechen. 

\boxtext{Sprachtechnologie baut Brücken für Europas Zukunft}

Die deutsche Wirtschaft profitiert überdurchschnittlich vom europäischen Binnenmarkt: 2010 gingen 60,3\% der deutschen Ausfuhren in die EU, weitere 10,8\% in andere europäische Länder. Doch die Geschäftstätigkeit stößt überall dort auf Schwierigkeiten, wo Sprachbarrieren dem Handel massiv im Wege stehen. Insbesondere kleine und mittelständische Unternehmen stehen vor einem mit vertretbarem Aufwand kaum lösbaren Problem. Die einzige (undenkbare) Alternative zum multilingualen Europa würde so aussehen, dass eine einzige Sprache die Vormachtstellung einnimmt und alle anderen Sprachen ersetzt. 

Eine Möglichkeit, Sprachbarrieren zu überwinden, ist das Erlernen von Fremdsprachen. Ohne technologische Unterstützung stellen die 23 Amts\-sprachen der Mitgliedstaaten der Europäischen Union und die rund 60 weiteren Sprachen Europas für die europäischen Bürger und ihre Wirtschaft, für politische Debatten sowie für den wissenschaftlichen Fortschritt jedoch ein unüberwindliches Hindernis dar. 

Die Lösung ist die Entwicklung von Schlüsseltechnologien: Sprachtechnologien bieten den europäischen Akteuren ungeheure Vorteile, nicht nur auf dem gemeinsamen europäischen Markt, sondern auch bei Handelsbeziehungen mit Dritt-, insbesondere Schwellenländern. Sprachtechnologielösungen werden letztendlich als einzigartige Brücke zwischen den Sprachen Europas dienen. Eine unverzichtbare Basis für die Entwicklung dieser Technologien ist eine systematische Analyse der linguistischen Besonderheiten aller europäischen Sprachen. Außerdem muss der derzeitige Stand der Unterstützung durch Sprachtechnologie analysiert werden. 

\boxtext{Sprachtechnologie als Schlüssel für die Zukunft} 
Die derzeit auf dem Markt erhältlichen Tools zur automatischen Übersetzung und Verarbeitung gesprochener Sprache werden den angestrebten Zielen nicht gerecht. Die dominanten Akteure sind privatwirtschaftliche Unternehmen mit Sitz in den Vereinigten Staaten. Bereits Ende der 1970er-Jahre erkannte die EU die tiefgreifende Bedeutung der Sprachtechnologie als Motor für die europäische Einheit und begann ihre ersten Forschungsprojekte zu finanzieren, beispielsweise EUROTRA. Gleichzeitig wurden nationale Projekte ins Leben gerufen. Diese lieferten zwar wertvolle Ergebnisse, führten aber niemals zu einer konzertierten europäischen Initiative. Im Gegensatz zu dieser hochgradig selektiven Förderung haben andere mehrsprachige Gesellschaften wie Indien (22 Amtssprachen) und Südafrika (11 Amtssprachen) langfristige nationale Programme für Sprachforschung und Technologieentwicklung aufgelegt. 

Die heute marktdominanten Sprach\-tech\-no\-logie-Ak\-teu\-re setzen auf ungenaue statistische Ansätze, bei denen keine tiefgreifende linguistische Analysen und Methoden zum Einsatz kommen. So werden Sätze automatisch übersetzt, indem ein neuer Satz mit Tausenden bereits früher vom Menschen übersetzter Sätze verglichen wird. Die Qualität des ausgegebenen Ergebnisses hängt vor allem vom Umfang und der Qualität der verfügbaren Daten ab. Bei der automatischen Übersetzung einfacher Sätze zwischen Sprachen mit ausreichendem Textmaterial kann man nützliche Ergebnisse erzielen. Doch die flachen, statistischen Verfahren sind zum Scheitern verurteilt bei Sprachen mit einem wesentlich geringeren Anteil an Referenzmaterial oder bei Sätzen von komplexer, nichtrepetitiver Struktur. Wenn wir Anwendungen schaffen möchten, die bei sämtlichen europäischen Sprachen ausgezeichnete Leistung bringen, ist eine Analyse der tieferliegenden strukturellen Eigenschaften der Sprachen die einzige erfolgversprechende Methode.

Aus diesem Grund fördert die Europäische Union Projekte wie EuroMatrix und EuroMatrixPlus (seit 2006) oder iTranslate4 (seit 2010). Hier werden durch Grundlagenforschung und angewandte Forschung Ressourcen für hochwertige Sprachtechnologielösungen für alle europäischen Sprachen bereitgestellt. Die europäische Forschung kann bereits zahlreiche Erfolge verbuchen. So nutzen die Übersetzungsdienstleister der Europäischen Union heute die Open-Source-Über\-set\-zungs\-soft\-ware Moses, die überwiegend im Rahmen europäischer Forschungsprojekte entwickelt wurde. Das vom deutschen Bundesministerium für Bildung und Forschung (BMBF) zwischen 1993 und 2000 geförderte Projekt Verbmobil katapultierte Deutschland in der Forschung im Bereich der Übersetzung gesprochener Sprache für einige Zeit in die internationale Führungsposition. Viele der zu dieser Zeit in Deutschland angesiedelten Forschungs- und Entwicklungslabore (z.\,B.~IBM und Philips) sind jedoch mittlerweile geschlossen oder umgezogen. Statt auf den Ergebnissen seiner Forschungsprojekte aufzubauen, hat Europa weiterhin eher isolierte Forschungsaktivitäten verfolgt, die auf dem Markt wenig wirkungsvoll waren. Der ökonomische Wert selbst der frühen Leistungen lässt sich an der Anzahl der Geschäftsausgliederungen erkennen. Ein Unternehmen wie das bereits 1984 gegründete Trados wurde z.\,B.~2005 an SDL mit Firmensitz in Großbritannien verkauft.

\boxtext{Sprachtechnologie trägt zur Einheit Europas bei}
Die bisher erlangten Erkenntnisse legen nahe, dass die heutige hybride Sprachtechnologie, die tiefe Verarbeitung mit statistischen Methoden kombiniert, die Kluft zwischen allen europäischen Sprachen überbrücken kann. Wie diese Weißbuch-Serie zeigt, bestehen jedoch im Hinblick auf die Einsatzfähigkeit von Sprachlösungen und den Stand der Forschung in den verschiedenen europäischen Mitgliedstaaten enorme Unterschiede. Auch wenn Deutschland zu den größeren EU-Sprachen gehört, ist dennoch weitergehende Forschung notwendig, um wirklich effektive Sprachtechnologielösungen für den alltäglichen Einsatz zu verwirklichen. Gleichzeitig stehen die Chancen gut, dass der deutschsprachige Teil Europas in diesem wichtigen Technologiebereich wieder eine international führende Position einnimmt. 

Die Vision von META-NET ist hochqualitative Sprachtechnologie für alle Sprachen Europas, die die politische und wirtschaftliche Einheit in kultureller Vielfalt vollendet. Mithilfe dieser Technologie können bestehende Schranken eingerissen und Brücken zwischen den Sprachen Europas gebaut werden. Hierfür müssen alle Interessengruppen aus Politik, Forschung, Wirtschaft und Gesellschaft ihre Kräfte bündeln.

Diese Weißbuch-Serie flankiert die anderen strategischen Aktivitäten von META-NET. Einen Überblick hierzu finden Sie im Anhang. Aktuelle Informationen, die neueste Version des META-NET Vision Papers \cite{Meta1} oder der strategischen Forschungsagenda (SRA) finden Sie auf unserer Website: http://www.meta-net.eu.


\clearpage
% --------------------------------------------------------------------------
\twocolumn[\ssection{Unsere Sprachen in Gefahr: Eine Heraus\-for\-de\-rung für die Sprachtechnologie}]

Wir sind Zeugen einer digitalen Revolution, die enorme Auswirkungen auf Kommunikation und Gesellschaft hat. Die jüngsten Entwicklungen in der digitalen In\-for\-ma\-tions- und Kommunikationstechnologie werden manchmal mit Gutenbergs Erfindung der Druckerpresse verglichen. Was kann uns diese Analogie über die Zukunft der europäischen Informationsgesellschaft und insbesondere der unserer Sprachen sagen?

Nach Gutenbergs Erfindung wurden echte Durchbrüche im Kom\-mu\-ni\-ka\-tions- und Wis\-sens\-aus\-tausch beispielsweie durch Luthers Bibelübersetzung    in die Landessprache  erzielt. In den folgenden Jahrhunderten wurden Kulturtechniken entwickelt, die die Sprachverarbeitung und den Wissensaustausch erleichterten:
\begin{itemize}
  \item Die orthografische und grammatikalische Standardisierung großer Sprachen ermöglichte die schnelle Verbreitung neue wissenschaftlicher und intellektueller Ideen.
  \item Die Entwicklung von Amtssprachen ermöglichte es den Bürgern, innerhalb oftmals politischer Grenzen miteinander zu kommunizieren.
  \item Sprachübergreifender Austausch wurde durch  Lehren und Übersetzen von Sprachen möglich.
  \item Die Schaffung redaktioneller und bibliografischer Richtlinien stellte die Qualität und Verfügbarkeit von Druckmaterial sicher.
  \item Die Entwicklung von Medien wie Buch, Zeitung, Radio und Fernsehen befriedigte ganz unterschiedliche Kommunikationsbedürfnisse.
\end{itemize}
In den vergangenen zwanzig Jahren trug die Informationstechnologie zu einer Automatisierung und Vereinfachung zahlreicher Prozesse bei:
\begin{itemize}
  \item Desktop-Publishing-Software ist an die Stelle von Schreibmaschine und Textsatz getreten.
  \item Microsoft PowerPoint ersetzte Overhead-Folien.
  \item Per E-Mail lassen sich Dokumente schneller versenden und empfangen als mit einem Faxgerät.
  \item Skype ermöglicht preiswerte Telefonanrufe über das Internet und virtuelle Konferenzen.
  \item Digitale Audio- und Video-Formate erleichtern den Austausch von Multimediainhalten.
  \item Suchmaschinen bieten Zugriff auf Webseiten per Stichwortsuche.
  \item Onlinedienste wie Google Translate erzeugen schnelle Grobübersetzungen.
  \item Soziale Medien (Facebook, Twitter) erleichtern Kommunika\-tion und  Informationsaustausch.
\end{itemize}
All diese Werkzeuge und Anwendungen sind zweifelsohne hilfreich, aber lange nicht ausreichend, um eine nachhaltige, mehrsprachige europäische Gesellschaft zu unterstützen, in der Informationen und Waren ungehindert fließen können.


\subsection{Sprachgrenzen bremsen die Europäische Informations\-ge\-sell\-schaft}
  
Wir können nicht genau vorhersagen, wie die Informationsgesellschaft der Zukunft aussehen wird. Es ist jedoch sehr wahrscheinlich, dass die Revolution in der Kommunikationstechnologie Menschen, die unterschiedliche Sprachen sprechen, auf eine ganz neue Weise zusammenbringen wird. Dadurch ist der Einzelne gezwungen, neue Sprachen zu lernen. Insbesondere sind aber  Entwickler gefragt  neue Technologieanwendungen zu schaffen, die das gegenseitige Verstehen und den Zugriff auf gemeinsam nutzbares Wissen sicherstellen. Im globalen Wirtschafts- und Informationsraum interagieren immer mehr Sprachen, Sprecher und Inhalte immer schneller miteinander dank neuartiger Medien. Die derzeitige Popularität von Sozialen Medien (Wikipedia, Facebook, Twitter, YouTube und Google+) ist nur die Spitze des Eisbergs.

Heute können wir in Sekundenschnelle gigabyteweise Textmengen um die Welt schicken, bevor wir überhaupt merken, dass sie in einer Sprache sind, die wir gar nicht verstehen. Einem jüngsten Bericht der Europäischen Kommission zufolge erwerben 57\% der Internetnutzer in Europa Waren und Dienstleistungen in Sprachen, die nicht ihre Muttersprache sind; Englisch ist die gängigste Fremdsprache, gefolgt von Französisch, Deutsch und Spanisch. 55\% der Nutzer lesen Inhalte in einer Fremdsprache, und 35\% verwenden eine fremde Sprache, um E-Mails zu schreiben oder Kommentare im Web zu posten \cite{EC1}. Vor ein paar Jahren mag Englisch noch die lingua franca des Webs gewesen sein – die meisten Webinhalte waren auf Englisch verfasst – doch das hat sich dramatisch gewandelt. Der Anteil an Onlineinhalten in anderen europäischen Sprachen (sowie asiatischen Sprachen und Sprachen aus Nahost) hat explosionsartig zugenommen.

Erstaunlicherweise hat diese allgegenwärtige, durch Sprachgrenzen bedingte digitale Kluft bisher kaum öffentliche Aufmerksamkeit erhalten.  Eine drängende Frage ergibt sich jedoch daraus: Welche europäischen Sprachen gewinnen in der vernetzten Informations- und Wissensgesellschaft an Bedeutung und welche sind dem Untergang geweiht?


\subsection{Unsere Sprachen in Gefahr}

Die Druckerpresse hat den Informationsaustausch in Europa mit vorangetrieben, gleichzeitig aber auch zum Aussterben vieler europäischer Sprachen geführt. Regionale Sprachen und Minderheitssprachen fanden sich kaum in Druckform. Sprachen wie Kornisch und Dalmatisch waren auf mündliche Übertragungsformen begrenzt und damit in ihrer Reichweite stark beschränkt. Wird das Internet die gleiche Wirkung auf unsere modernen Sprachen haben?

Europas rund 80 Sprachen sind eines unser reichsten und wichtigsten Kulturgüter und ein vitaler Teil dieses einzigartigen sozialen Modells \cite{EC2}. Sprachen wie Englisch und Spanisch werden auf dem aufstrebenden digitalen Marktplatz wohl überleben, doch zahlreiche europäische Sprachen könnten in einer vernetzten Gesellschaft schnell unbedeutend werden. Dies würde  Europas globales Standing schwächen und dem strategischen Ziel zuwiderlaufen, allen europäischen Bürgern ungeachtet ihrer Sprache gleiche Partizipationsmöglichkeiten zuzusichern. Einem Bericht der UNESCO über Mehrsprachigkeit zufolge ist Sprache ein wichtiges Medium, um grundlegende Rechte wie politische Meinungsäußerung, Bildung und gesellschaftliche Beteiligung wahrnehmen zu können. \cite{Unesco1}


\subsection{Sprachtechnologie ist eine Schlüsseltechnologie}

Früher konzentrierten sich Investitionen in den Spracherhalt vor allem auf Sprachausbildung und Übersetzungen. Einer Schätzung zufolge belief sich 2008 der europäische Markt für Übersetzungen, Dolmetschen, Softwarelokalisierung und Website-Globalisierung auf 8,4 Milliarden €, wobei von einem jährlichen Wachstum um 10\% ausgegangen wird \cite{EC3}. Diese Zahl deckt jedoch nur einen geringen Anteil unseres aktuellen und künftigen Kommunikationsbedarfs zwischen den verschiedenen Sprachen ab. Die einzige Lösung, den Sprachgebrauch im Europa von morgen in voller Breite und  Tiefe sicherzustellen, ist der Einsatz geeigneter Tech\-no\-lo\-gie -- so wie wir auch Technologie einsetzen, um  unseren Transport- und Energiebedarf zu decken.

Sprachtechnologie für alle Formen geschriebener und gesprochener Sprache ermöglicht es den Menschen ungeachtet von Sprachbarrieren und Computerfertigkeiten zusammenzuarbeiten, Geschäfte zu tätigen, Wissen auszutauschen und sich an sozialen und politischen Debatten zu beteiligen. Diese Technologie arbeitet oft unsichtbar in komplexen Softwaresystemen und unterstützt schon heute:
    \begin{itemize}
      \item Informationssuche mit Suchmaschinen
      \item Rechtschreib- und Grammatikprüfung
      \item Produktempfehlungen in Online-Shops
      \item Sprachanweisungen von Na\-vi\-ga\-tions\-sys\-te\-men
      \item Online-Übersetzung von Webseiten 
    \end{itemize}
Sprachtechnologie besteht aus einer Reihe von Kerntechnologien, die Einsatz in größerer Anwendungssoftware finden. Die META-NET Weißbücher zeigen auf, wie gut die Kerntechnologien bereits für die einzelnen europäischen Sprachen einsetzbar sind. 

Damit Europa seine Position an der vordersten Front der globalen Innovation behaupten kann, benötigt es eine auf alle europäischen Sprachen abgestimmte Sprachtechnologie, die robust und preiswert ist und sich in größere Softwareumgebungen einbetten lässt. Ohne Sprachtechnologie können wir in der Zukunft keine multimedialen und mehrsprachigen Endanwendungen für alle europäischen Nutzer realisieren.

\end{document}
