\documentclass[]{../metanetpaper}

%!TEX TS-program = xelatex
\RequireXeTeX %Force XeTeX check


\title{Languages in the European Information Society}
\date{\today}
\author{Dr. John Doe (DFKI), Prof. Dr. Felix Sasaki (DFKI),\\
        Pepe Pérez (Academia de \LaTeX), Tan Ah Kao (USEV)}


\begin{document}

	\maketitle

	\section*{Preface}
This white paper is part of a series that promotes knowledge about
language technology and its potential. It addresses educators, journalists, politicians, language communities and others. The availability and use of language technology in Europe varies between languages. Consequently, the actions that are required to further support research and development of language technologies also differ for each language. The required actions depend on many factors, such as the complexity of a given language and the size of its community.
META-NET, a Network of Excellence funded by the European Commission, has conducted an analysis of current language resources and technologies. This analysis focused on the 23 official European languages as well as other important national and regional languages in Europe. The results of this analysis suggest that there are many significant research gaps for each language. A more detailed expert analysis and assessment of the current situation will help maximise the impact of additional research and minimize any risks. META-NET consists of 47 research centres from 31 countries that are working with stakeholders from commercial businesses, government agencies, industry, research organisations, software companies, technology providers and European universities. Together, they are creating a common technology vision while developing astrategic research agenda that shows how language technology applications can address any research gaps by 2020.

\clearpage

	\tableofcontents

	% 1.
	\clearpage
	\section{Executive Summary --- Zusammenfassung}
	

Many European languages run the risk of becoming victims of the digital age because they are underrepresented and underresourced online. Huge regional market opportunities remain untapped today because of language barriers. If we do not take action now, many European citizens will become socially and economically disadvantaged because they speak their native language. Innovative language technology is an intermediary that will enable European citizens to participate in an egalitarian, inclusive and economically successful knowledge and information society. Multilingual language technology will be a gateway for instantaneous, cheap and effortless communication and interaction across language boundaries. Today, language services are primarily offered by commercial providers from the US. Google Translate, a free service, is just one example. The recent success of Watson, an IBM computer system that won an episode of the Jeopardy game show against human candidates, illustrates the immense potential of language technology. As Europeans, we have to ask ourselves several urgent questions:
\\
\begin {itemize}
\item Should our communications and knowledge infrastructure be dependent upon monopolistic companies?
\item Can we truly rely on language-related services that can be immediately switched off by others?
\item Are third parties from other continents willing to address our translation problems and other issues that relate to European multilingualism?
\item Can our European cultural background help shape the knowledge society by offering better, more secure, more precise, more innovative and more robust high-quality technology?
\end {itemize}
This whitepaper for the German language demonstrates that a language technology research and development community exists in Germany, Austria and Switzerland. However, many large companies have stopped their activities in language technology research and development, which is nowadays almost exclusively performed by small and medium enterprises that can hardly compete on the global market. While Germany used to be a European hub in this area in the past, we now have to ask ourselves if we want to actively compete in the global market for research and development in this
future technology or not. Although a number of technologies and resources for Standard German exist, there are fewer technologies and resources for the German language than for the English language. The existing tech nologies and resources are also poorer in quality. According to the assessment detailed in this report, immediate
action must occur before any breakthroughs for the German language can be achieved.

	\section{Risk for Our Languages and a Challenge for Language Technology}

We are witnesses to a digital revolution that is dramatically impacting communication and society.
	\margin{We are currently witnessing a digital revolution that is comparable to Gutenberg’s invention of the
	printing press.}
Recent developments in digitalinformation and communication technology are sometimes compared to Gutenberg’s invention of the printing press. What can this analogy tell us about the future of the European information society and our languages in particular?\\
After Gutenberg’s invention, real breakthroughs in communication and knowledge exchange were accomplished by efforts such as Luther’s translation of the Bible into vernacular language. In subsequent centuries, cultural techniques have been developed to better handle language processing and knowledge exchange:
\begin{itemize}
\item the orthographic and grammatical standardisation of major languages enabled the rapid dissemination of new scientific and intellectual ideas;
\item the development of official languages made it possible for citizens to communicate within certain (often political) boundaries;
\item the teaching and translation of languages enabled exchanges across languages;
\item the creation of editorial and bibliographic guidelines assured the quality and availability of printed material;
\item the creation of different media like newspapers, radio, television, books, and other formats satisfied different communication needs. In the past twenty years, information technology has helped to automate and facilitate many of the processes:
\item desktop publishing software has replaced typewriting and type setting;
\item Microsoft PowerPoint has replaced overhead projector transparencies;
\item e-mail send and receive documents faster than a fax machine;
\item Skype offers cheap Internet phone calls and hosts virtual meetings;
\item audio and video encoding formats make it easy to exchange multimedia content;
\item search engines provide keyword-based access to web pages;
\item online services like Google Translate produce quick, approximate translations;
\item social media platforms such as Facebook, Twitter, and Google+ facilitate communication, collaboration, and information sharing.
\end{itemize}
Although such tools and applications are helpful, they are not yet capable of supporting a sustainable, multilingual European society for all where information and goods can flow freely.
	
	\subsection{Language Borders Hinder the European Information Society}
We cannot predict exactly what the future information society will look like. But there is a strong likelihood that the revolution in 6 communication technology is bringing people speaking different languages together in new ways. This is putting pressure on individuals to learn new languages and especially on developers to create new technology applications to ensure mutual understanding and access to shareable knowledge.
	\margin{A global economy and information space confronts us with more languages, speakers and content.}
In a global economic and information space, more languages, speakers and content interact more quickly with new types of media. The current popularity of social media (Wikipedia, Facebook, Twitter, YouTube, and, recently, Google+) is only the tip of the iceberg. Today, we can transmit gigabytes of text around the world in a few seconds before we recognize that it is in a language we do not understand. According to a recent report from the European Commission, 57\% of Internet users in Europe purchase goods and services in languages that are not their native language. (English is the most common foreign language followed by French, German and Spanish.) 55\% of users read content in a foreign language while only 35\% use another language to write e-mails or post comments on the Web.i A few years ago, English might have been the lingua franca of the Web — the vast majority of content on the Web was in English — but the situation has now drastically changed. The amount of online content in other European (as well as Asian and Middle
Eastern) languages has exploded. Surprisingly, this ubiquitous digital divide due to language borders has not gained much public attention; yet, it raises a very pressing question: Which European languages will thrive in the networked
information and knowledge society, and which are doomed to disappear?

	\subsection{Our Languages at Risk}
	\includegraphics[scale=0.75]{../media/metanet-paper2.pdf} 
While the printing press helped step up the exchange of information in Europe, it also led to the extinction of many European languages. Regional and minority languages were rarely printed and languages such as Cornish and Dalmatian were limited to oral forms of transmission, which in turn restricted their scope of use. Will the Internet have the same impact on our languages? Europe’s approximately 60 languages are one of its richest and most important cultural assets, and a vital part of its unique social model.\footnote{European Commission, Multilingualism: an asset for Europe and a shared commitment, Brussels, 2008 (http://ec.europa.eu/education/languages/pdf/com/2008\_0566\_en.pdf).}
\margin{The wide variety of languages in Europe is one of its richest and most important cultural assets.} While languages such as English and Spanish are likely to survive in the emerging digital marketplace, many European languages could become irrelevant in a networked society. This would weaken Europe’s global standing, and run counter to the strategic goal of ensuring equal participation for every European citizen regardless of language. According to a UNESCO report on multilingualism, languages are an essential medium for the enjoyment of fundamental rights, such as political expression, education and participation in society.\footnote{UNESCO Director-General, Intersectoral mid-term strategy on languages and multilingualism, Paris, 2007 (http://unesdoc.unesco.org/images/0015/001503/150335e.pdf).}

	\subsection{Language Technology is a Key Enabling Technology}
In the past, investment efforts in language preservation focused on language education and translation. According to one estimate, the European market for translation, interpretation, software localisation and website globalisation was €8.4 billion in 2008 and is expected to grow by 10\% per annum.iv Yet this figure covers just a small proportion of current and future needs in communicating between languages. \margin {Language technology helps people collaborate, conduct business, share knowledge and participate in social and political debates across differ} The most compelling solution for ensuring the breadth and depth of language usage in Europe tomorrow is to use appropriate technology, just as we use technology to solve our transport, energy and disability needs among others. Digital language technology (targeting all forms of written text and spoken discourse) helps people collaborate, conduct business, share knowledge and participate in social and political debate regardless of language barriers and computer skills. It often operates
invisibly inside complex software systems to help us:
\begin{itemize}
\item find information with an Internet search engine;
\item check spelling and grammar in a word processor;
\item view product recommendations in an online shop;
\item hear the verbal instructions of a car navigation system;
\item translate web pages via an online service.
\end{itemize}
Language technology consists of a number of core applications that enable processes within a larger application framework. The purpose of the META-NET language white papers is to focus on how ready these core technologies are for each European language.\margin {Europe needs robust and affordable language technology for all European languages.}
To maintain our position in the frontline of global innovation, Europe will need language technology adapted to all European languages that is robust, affordable and tightly integrated within key software environments. Without language technology, we will not be able to achieve a really effective interactive, multimedia and multilingual user experience in the near future.
	
	\subsection{Opportunities for Language Technology}
In the world of print, the technology breakthrough was the rapid duplication of an image of a text (a page) using a suitably powered printing press. Human beings had to do the hard work of looking up, reading, translating, and summarizing knowledge. We had to wait until Edison to record spoken language – and again his technology simply made analogue copies. Digital language technology can now automate the very processes of translation, content production, and knowledge management for all European languages. It can also empower intuitive language/speech-based interfaces for household electronics, machinery, vehicles, computers and robots. Real-world commercial and industrial applications are still in the early stages of development, yet R\&D achievements are creating a genuine window of opportunity. For example, machine translation is already reasonably accurate in specific domains, and experimental applications provide multilingual information and knowledge management as well as content production in many European languages. As with most technologies, the first language applications such as voice-based user interfaces and dialogue systems were developed for highly specialised domains, and often exhibit limited performance. But there are huge market opportunities in the education and entertainment industries for integrating language technologies
into games, cultural heritage sites, edutainment packages, libraries, simulation environments and training programmes. Mobile information services, computer-assisted language learning software, eLearning environments, self-assessment tools and plagiarism detection software are just some of the application areas where language technology can play an important role.\margin {\includegraphics[scale=0.7]{../media/metanet-paper.pdf} } 
The popularity of social media applications like Twitter and Facebook suggest a further need for sophisticated language technologies that can monitor posts, summarise discussions, suggest opinion trends, detect emotional responses, identify copyright infringements or track misuse.\end{document}
